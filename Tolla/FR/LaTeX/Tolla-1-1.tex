\addchap{\RomanNumeralCaps 1.}

La famille Feraldi n’est pas princière, mais elle marche de pair avec bien de princes. Alexandre Feraldi, comte du saint empire, baron de Vignano, chevalier de l’ordre de Constantin, est un des soixante patriciens inscrits sur les tables du Capitole. Il n’a jamais voulu entrer dans l’armée pontificale, où son père était lieutenant-colonel. Une santé délicate, l’instruction sérieuse qu’il a reçue au collège de Nazareth, et, par dessus tout, la nécessité de rétablir les affaires de sa famille lui a fait embrasser l’étude des lois et de la jurisprudence. Le temps n’est plus où l’on trouvait dans chaque Romain l’étoffe d’un soldat, d’un laboureur et d’un jurisconsulte; mais les patriciens ont conservé le respect des trois arts glorieux qui firent la grandeur de leurs ancêtres. Le comte Feraldi, docteur en droit sans déroger, se maria en \oldstylenums{1816} à Catherine Mariani, fille du marquis de Grotta Ferrata. Vers la même époque, deux de ses cousins germains, du même nom que lui, épousèrent des princesses, une Odescalchi et une Barberini. Alexandre Feraldi ne fut pas in sensible à l’honneur de ces alliances, qui relevaient le nom de sa famille. Trois mois après, une succession inespérée, qui vint le surprendre pendant la grossesse de sa femme, le mit pour toujours au-dessus du besoin, en portant son revenu à vingt-cinq où trente mille francs. Jamais homme ne fut plus heureux que le comte Feraldi dans la première année de son mariage. Ce petit homme aimable, vif et sautillant, très brun, sans que sa physionomie présentât rien de noir, très fin et très subtil, avec beaucoup de franchise et d’ouverture de cœur, remplissait de sa joie et animait de sa gaieté le palais un peu délabré de ses ancêtres. Sa femme, assez belle, mais d’une beauté sèche et pour ainsi dire indigente, l’aimait éperdûment. Ses amis le plaisantaient quelquefois sur l’excès de son bonheur. «Où s’arrêtera, disait-on avec emphase, la fortune des Feraldi? Le Pactole coule dans leur jardin; les rejetons des familles princières viennent se greffer sur leur arbre généalogique. Nous te prédisons, ô trop heureux Alexandre, que ta femme avant deux mois accouchera d’un pape.»

Le \oldstylenums{1}$\textsuperscript{er}$ septembre \oldstylenums{1817}, la comtesse mit au monde une fille qui fut baptisée sous le nom de Vittoria. Un an plus tard, Vittoria eut un frère qu’on appela Victor. Le triomphant petit comte Alexandre n’avait pas trouvé de noms plus modestes pour ses enfants. C’était plaisir de l’entendre demander si son fils Victor avait pris le sein, et si sa fille Vittoria avait mangé sa bouillie. La comtesse et les gens de la maison appelaient tout bonnement le petit garçon Toto et la petite fille Tolla.

Le palais Feraldi est situé dans un des plus nobles quartiers de Rome, à deux pas de l’ambassade de France. Il n’est ni très grand ni très beau: il n’a ni la vétusté originale du palais de Venise, ni l’immensité du palais Doria, ni la majesté du palais Farnèse; mais il a un jardin. Tolla fut élevée au milieu des arbres et des fleurs. Une grande allée, abritée contre le vent du nord par une muraille de cyprès, était sa promenade d’hiver. A l’âge de sept où huit mois, elle fit la connaissance d’un vieux citronnier en fleurs qui devint son meilleur ami.

Elle tendait vers lui ses petits bras; elle arrachait à belles mains les longues fleurs et les gros boutons violacés, et elle les portait à sa bouche. Le médecin de la maison, le docteur Ely, permit que dès les premiers jours d’avril on la gardât une heure où deux au jardin, étendue en liberté sur un tapis, à l’ombre de son citronnier, où sous un chêne vert, autre ami vénérable. L’été venu-c’est au jardin qu’elle prit ses premiers bains, dans une eau que le soleil avait eu soin de chauffer. La liberté, le mouvement, le grand air et les parfums généreux qui s’exhalent des arbres, tout concourut à fortifier ce jeune corps: Tolla grandit avec les plantes qui l’environnaient, sans effort et sans douleur. Une promenade au jardin l’endormait en quelques minutes; en s’éveillant, elle souriait à la vie, à ses parents et à son jardin. Le travail des premières dents, si redouté des mères, se fit en elle sans qu’on s’en aperçût, et un beau matin la comtesse, qui la nourrissait, poussa un cri de surprise en se sentant mordue par deux petites perles bien aiguisées.

Tous les ans, au moins d’août, le comte s’embarquait pour Capri, où il possédait un beau vignoble. Tandis qu’il surveillait ses vendanges, la comtesse allait vivre à Lariccia, en bon air, dans une jolie villa où de mémoire d’homme personne n’avait pris les fièvres. Son mari venait bientôt l’y rejoindre. Ils y restaient avec leurs enfants jusqu’aux froids, et ne retournaient jamais à Rome avant d’avoir vu cueillir les olives.

Tolla passa à Lariccia les plus beaux jours de son enfance. Elle y était plus libre qu’à Rome, quoiqu’on l’eût placée sous la haute main du petit Menico, fils d’un fermier de son père. Menico, c’est-à-dire Dominique, avait cinq ans de plus que Tolla et six ans de plus que Toto, mais il n’abusa jamais de l’autorité que lui donnaient son âge et la confiance de la comtesse. Il ne savait rien refuser à Tolla. En dépit de toutes les recommandations de prudence et d’abstinence qu’on ne lui avait pas ménagées, il hissait lui-même sa petite élève sur tous les ânes du village, et il maraudait à son intention dans les jardins les mieux enclos. Plus d’une fois on surprit le mentor éclatant de rire à la vue de Tolla qui mordait à belles dents une lourde grappe de raisins jaunes, où qui se bar bouillait les joues avec une grosse figue violette. Les jardins, les bois, les ânes et Menico furent pendant douze ans les seuls précepteurs de Tolla. Sa mère lui apprit un peu de religion et de musique. Comme on ne la força jamais de se mettre au piano, elle y vint toujours volontiers. Ses petits doigts aimaient à courir sur les touches d’ivoire. Il se trouva qu’elle avait l’oreille juste, et même, ce qui est plus rare chez les enfants, le sentiment de la mesure. Le célèbre \emph{maestro} Terziani, qui l’entendit un jour par hasard, déclara que c’était grand dommage de ne lui point donner un maître, mais on le laissa dire.

La religion, et surtout ce catholicisme splendide qui règne à Rome, trouva chez elle une âme bien préparée. La pompe des cérémonies, les parfums de l’encens, l’or, le marbre, la musique sacrée, l’attirèrent invinciblement, comme ce citronnier fleuri auquel elle tendait les bras. Son imagination avide s’empara du premier aliment qui lui fut offert. Elle s’éprit d’une passion filiale pour la madone, cette dame vêtue de bleu et d’or qu’on lui disait si bonne, et qu’elle voyait si belle. L’enthousiasme puéril qu’elle conçut pour certaines images se changea peu à peu en dévotion. A force de prier dans la chambre de sa mère devant une \emph{sainte famille} de Sassoferrato, elle se lia tout particulièrement avec saint Joseph: elle lui envoyait des baisers, comme à un vieux et respectable parent de la maison. «Tu verras,» lui disait-elle, «comme je t’embrasserai, si je vais au ciel!» Cette âme aimante n’eut pas besoin d’apprendre la charité. A quatre ans, elle déchirait ses habits, parce qu’elle avait remarqué qu’on les donnait aux petits pauvres lorsqu’ils étaient déchirés. Elle émiettait son déjeuner aux oiseaux du jardin. «Ne sont ils pas notre prochain?» disait-elle. Je nourris mes frères ailés. «Sa charité s’étendait jusqu’aux morts. Un jour, sa mère la conduisit à l’église des Jésuites, où l’on prêchait pour les âmes du purgatoire. C’était dans l’octave de Saint-Ignace, un mois environ avant qu’elle eût accompli sa sixième année. Pendant tout le sermon, Toto n’eut d’yeux que pour la statue colossale en argent massif posée sur un globe de lapis-lazuli: il demanda plusieurs fois à sa mère si le bon Dieu était aussi riche que saint Ignace, et s’il avait en quelque endroit du monde une aussi belle statue. Tolla écouta le prédicateur. Quand la première quêteuse passa près d’elle, elle jeta dans la bourse une petite pièce de monnaie que sa mère lui avait donnée pour mais lorsqu’on vint quêter devant elle pour la secondé fois, comme elle n’avait plus d’argent, elle détaché vivement son petit bracelet de corail et le donna aux âmes du purgatoire. On ne s’en aperçut que le soir en la déshabillant.

«Tu n’aurais pas dû,» lui dit sa mère, «donner ton bracelet sans ma permission.»

Elle répliqua vivement: «Vous n’avez donc pas entendu, maman, comme ces pauvres âmes ont soif?»

A treize ans, Tolla savait lire et écrire, monter à cheval, grimper aux arbres, sauter les fossés, jouer du piano, aimer ses parents et prier Dieu. Son père s’aperçut qu’avec ses petits talents, sa parfaite ignorance et ses grandes qualités, elle ne ressemblait pas mal à un buisson d’aubépine en fleur. On résolut de la mettre en pension. L’établissement en vogue en ce temps-là était l’institut royal de Marie-Louise, à Lucques. Les élèves y accouraient du fond de l’Italie et même des pays d’outre-mer et d’outre-monts. Le bruit des concours annuels qui s’y faisaient et des récompenses qui y étaient décernées retentissait dans toute la péninsule de Naples à Venise. Le comte Feraldi espéra que l’amour de la gloire éveillerait chez sa fille le goût du travail, et que l’appât de ces couronnes tant enviées lui ferait regagner le temps perdu. Il la conduisit à la surintendante de l’institut royal, la comtesse Trebiliani.

Tolla, jetée sans transition dans les habitudes régulières et presque monastiques d’une grande communauté, n’eut pas le temps de regretter sa liberté, sa famille et les bois de Lariccia. Elle s’éprit pour l’étude d’une passion soudaine, mais où la curiosité avait plus de part que l’émulation. Elle se souciait médiocrement de paraître savante, mais elle conçut un incroyable désir de savoir. Toutes les facultés sérieuses de son esprit, brusquement éveillées, entrèrent en travail, et l’on crut reconnaître que l’oisiveté où elle avait vécu avait centuplé ses forces. Son esprit ressemblait à ces terres incultes du Nouveau-Monde qui n’attendent qu’une poignée de semence pour révéler leur inépuisable fécondité. Ignorante comme elle l’était, tout lui parut nouveau, tout piquait sa curiosité; elle ne dédaignait rien, rien ne lui semblait usé ni banal. Les histoires les plus insipides, les abrégés les plus nauséabonds avaient pour elle autant d’attraits que des romans. La géographie lui parut une science curieuse et attachante: en feuilletant un atlas, elle éprouvait les émotions d’un voyageur qui découvre des Amériques à chaque pas. Pour tout dire en un mot, rien ne la rebuta, pas même l’arithmétique; elle fut charmée de ces petits raisonnements secs et précis; elle saisit au premier coup d’œil tout ce qu’ils ont d’ingénieux dans leur simplicité, et je ne sais s’il s’est trouvé personne, depuis Pythagore, à qui la table de Pythagore ait fait autant de plaisir.

A la fin de l’année \oldstylenums{1831}, Tolla, sans avoir songé un seul instant à se couvrir de gloire suivant les intentions de son père, se trouva la première de sa classe et reçut la croix d’or, aux applaudissements de toute la cour. Elle maintint sa supériorité, sans y penser, jusqu’à l’âge de dix-sept ans. Dans l’automne de \oldstylenums{1834}, un décret du duc de Lucques supprima l’institut royal et rendit les élèves à leurs familles. Tolla parlait assez élégamment le français et l’anglais; elle avait amassé la petite somme de connaissances qu’un pensionnat peut offrir à une jeune fille; un excellent maître avait cultivé sa voix et changé en talent ce qui n’était chez elle que l’instinct de la musique; ses parents la trouvèrent parfaite, et son père glorieux se hâta de la conduire dans le monde.

Elle y fit une entrée triomphale, et Rome se souvient encore de sa présentation chez la marquise Trasimeni. Les mères de famille, intéressées à lui trouver des défauts, avaient armé leurs yeux de la curiosité la plus malveillante. Elle subit sans s’en douter ce formidable examen où tous les juges étaient prévenus contre elle: elle en sortit à son honneur. L’aréopage des femmes de quarante ans décida à l’unanimité qu’elle avait une petite figure française assez gentille. Les hommes la proclamèrent de prime-saut la plus jolie fille de Rome.

Sa beauté était de celles qui découragent les statuaires et leur font cruellement sentir l’impuissance de leur art. Ses mains, sa figure et ses épaules avaient la pâleur mate du marbre, et cependant le marbre le plus fidèle n’aurait jamais pu passer pour son image. Rien n’était plus facile que de rendre la finesse aristocratique de ce nez imperceptiblement arqué, la courbe fière des sourcils, l’ampleur un peu dédaigneuse des lèvres, le modèle délicat des joues, où deux imperceptibles fossettes se dessinaient par instants; mais David lui-même, le sculpteur de la vie, aurait été incapable d’exprimer le mouvement, la santé, et comme la joie secrète qui animait ces traits adorables. La jeunesse dans toute sa force éclatait à travers cette enveloppe délicate; la pâleur de son visage était saine et robuste. Elle ressemblait à ces lampes d’albâtre qu’une flamme intérieure fait doucement resplendir. Ses yeux châtains, mais qui paraissaient noirs, avaient le regard doux, étonné et un peu farouche d’une jeune biche qui écoute les échos lointains du cor. Sa chevelure longue, épaisse et soyeuse, s’entassait sur sa tête et débordait en deux boucles pesantes jusque sur ses épaules. Son corps mignon, souple, frêle et cependant vigoureux, ressemblait à ces statues antiques dont la vue n’inspire que de hautes pensées et de nobles désirs, quoiqu’elles se montrent sans voiles et qu’elles ne soient vêtues que de leur chaste beauté. Ses mains étaient petites, et son pied aurait été remarqué à Séville où à Paris.

Tolla fut d’autant plus admirée à Rome qu’elle n’avait pas une beauté romaine. Cette nation vigoureuse qui se baigne dans les eaux jaunes du Tibre a conservé, quoi qu’on dise, une assez bonne part de l’héritage de ses ancêtres. Les hommes ont toujours cet air mâle et sérieux, cette noble prestance et cette dignité extérieure qui distinguait jadis un Romain d’un Grec où d’un Gaulois; les femmes sont encore ces belles et massives créatures parmi lesquelles le vieux Caton choisissait la gardienne de son foyer et-la mère de ses enfants. Les jeunes Romaines, avec leur front bas, leur face brillante, leurs puissantes épaules, leurs bras charnus, leurs jambes épaisses, leurs pieds solides et leur large et opulente beauté, semblent si bien prédestinées aux devoirs de la famille, qu’il est difficile de voir en elles autre chose que des mères et des nourrices futures: elles ont la physionomie plantureuse et féconde de cette brave terre d’Italie qui a nourri sans s’épuiser tant de fortes générations. Leur regard, leur sourire, et jusqu’à leur coquetterie a quelque chose de tranquille, de positif et de convenu, comme le mariage et le ménage. Au milieu de cette foule un peu banale, Tolla surprenait l’admiration par une grâce plus âpre, par des mouvements plus vifs, par je ne sais quel charme bizarre et inusité. Son entrée produisit sur les regardants une impression analogue à celle que vous éprouveriez, si dans un boudoir tout imprégné de poudre à la maréchale quelque brise soudaine apportait les fraîches senteurs d’une forêt. Dès ce moment, tous les sourires parurent fades, excepté le sien, et toutes les beautés robustes au milieu desquelles elle glissait au bras de son père ne furent plus que des poupées majestueuses.

Elle avait choisi pour son début une toilette extrêmement simple, qui fut copiée dès le lendemain par toutes les brunes, et qui resta à la mode pendant deux où trois mois. C’était une robe de tarlatane avec un dessous de taffetas blanc, un camélia blanc au corsage, un large velours ponceau dans les cheveux, et une longue épée d’argent plantée horizontalement dans la natte, suivant la mode des filles de la campagne et des \emph{minintes} du Transtévère\footnote{~alias \emph{Trastevere}, un des rioni de Rome}. Cette coiffure rustique inspira au fameux improvisateur Benzio un sonnet qui se terminait ainsi:

\begin{quote}
«D’où viens-tu? De la cour imposante d’un roi où de la modeste chaumière d’un berger? Est-ce \emph{contessina\footnote{~petite comtesse}} que l’on te nomme? où faut-il t’appeler \emph{contadina\footnote{~paysanne}?}

Si tu es \emph{contessina}, tous les bergers vont s’armer contre la noblesse; si tu es \emph{contadina}, tous les comtes vont acheter des guêtres de cuir et des vestes de velours.»
\end{quote}

Tolla supporta sans aucune gaucherie le petit triomphe qui lui fut décerné. On sait combien il est difficile d’essuyer, sans perdre contenance, une averse de compliments. Cette épreuve, très rude en tout pays, est formidable en Italie, dans la patrie de l’hyperbole. Tolla s’entendit comparer à ce que les trois règnes de la nature renferment de plus exquis: on lui décerna à bout portant la qualification d’astre, de merveille et de divinité. Les femmes elles-mêmes prirent part à ce concert, toutes prêtes à la proclamer vaniteuse si elle acceptait les louanges, et sotte si elle les repoussait; mais elle trouva dans l’enjouement naturel de son esprit un refuge contre l’une et l’autre accusation: elle ne reçut ni ne rejeta les flatteries sous lesquelles on espérait l’accabler. Tantôt elle les accueillit en badinant t d’un ton qui voulait dire: «j’écoute par politesse les sottises que la politesse vous a inspirées»; tantôt elle les renvoya plaisamment à leurs auteurs, quand leurs auteurs étaient des femmes. Elle payait leurs louanges avec usure, et rendait des diamants pour des cristaux, des soleils pour des étoiles. Ces innocentes malices de la naïveté obtinrent les applaudissements muets, mais unanimes, de tous les hommes: il est si difficile de résister au charme de la jeunesse! C’est ainsi que la plus jolie fille de Rome, sans chercher l’esprit, sans faire des mots et sans médire de personne, gagna haut la main son brevet de fille d’esprit.

Si Tolla n’avait eu pour elle que son esprit et sa beauté, elle aurait trouvé un épouseur; mais comme elle avait une dot, il s’en présenta quarante. Le comte Feraldi ne se faisait pas faute de dire à qui voulait l’entendre: «Il y a vingt mille sequins où cent mille francs de bon argent dans un coffre de ma connaissance pour le brave garçon que choisira la plus jolie fille de Rome.» Tolla dansa pendant deux hivers avec toute la jeunesse des états pontificaux sans choisir personne. Ses parents ne la pressaient pas. «Prends ton temps,» lui disait son père. «Je conviens qu’il n’est pas facile de trouver un homme digne de toi: pour ma part, je n’en connais point.» La comtesse, à qui ses bonnes amies demandaient, par pure charité, pourquoi Tolla, avec sa beauté, son esprit et sa dot, était arrivée à l’âge de dix-neuf ans sans se marier, leur répondait sans malice aucune: «Nous ne sommes pas de ces parents qui grillent de se débarrasser de leurs filles.» Tolla dans le monde était l’orgueil de son père; Tolla dans sa famille était la vie et la bonne humeur de la maison. Entre un bal et une promenade à cheval avec son frère, qui venait de terminer ses études, elle partageait avec sa mère les travaux domestiques et les soins du ménage; elle revoyait les comptes du ministre, c’est-à-dire de l’intendant; elle traçait à sa femme de chambre, qui lui servait de lingère et de couturière, le dessin d’un col où d’une paire de manches; elle présidait à quelque arrangement nouveau dans son cher jardin, où elle travaillait en chantant à un bel ouvrage de tapisserie. Elle était présente partout, voyait tout, savait tout, disposait tout, commandait, souriait et plaisait à tout le monde. Cette petite personne mondaine, cette danseuse infatigable, cette écuyère intrépide qui sautait les barrières et les fossés, pratiquait au palais Feraldi toutes les gracieuses vertus d’une mère de famille.
