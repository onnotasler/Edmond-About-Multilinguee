\addchap{\RomanNumeralCaps 2.}

Le \oldstylenums{30} avril \oldstylenums{1837}, l'élite de la noblesse de Rome était réunie chez la
marquise Trasimeni. Les jeunes gens dansaient au piano dans le salon des
tapisseries; quelques mères de famille surveillaient nonchalamment les
plaisirs de leurs filles; les papas jouaient au whist dans le boudoir de
la marquise; le jardin, de plain pied avec l'appartement, était peuplé
d'une douzaine de fumeurs qui promenaient dans l'obscurité la lueur de
leurs cigares. On jouissait des premières douceurs du printemps et des
derniers plaisirs de l'hiver.

M\textsuperscript{me} Assunta Trasimeni avait alors la maison la plus
agréable et la moins bruyante de Rome. Les étrangers ne s'y faisaient
point présenter, où s'y ennuyaient mortellement, faute de pouvoir
comprendre le charme intime et la grâce silencieuse de ces réunions;
mais les Romains auraient regardé comme une calamité publique la
suppression des jeudis de la marquise. Ce haut salon, dont la voûte,
peinte à fresque par un élève de Jules Romain, portait quatre grandes
figures un peu effacées représentant Rome, Naples, Florence et Venise;
ces belles tapisseries du \RomanNumeralCaps 16\textsuperscript{e} siècle dont le temps
avait adouci et fondu les couleurs, ces meubles d'ébène
imperceptiblement fendillée, ce vieux lustre de cristal de roche, ce
piano de Vienne, dont les sons étaient amortis par les tentures, tout
respirait une bonhomie grandiose et un peu triste. Les domestiques,
enfants de la maison, vêtus de livrées héréditaires, présentaient si
cordialement les verres de limonade, que pas un des invités ne songeait
à regretter les réceptions fastueuses et la prodigalité banale de tel
prince où de tel banquier.

Le salon, les meubles, les habitudes douces et régulières de la maison,
tout encadrait merveilleusement la figure de la marquise. Elle touchait
à sa quarantième année; elle était grande, un peu maigre, et blonde avec
d'admirables yeux noirs. Sa beauté était faite de dignité, de
bienveillance et de tristesse. Elle portait invariablement une robe de
velours noir, et personne ne se souvenait de l'avoir vue autrement
vêtue, même dans sa jeunesse et du vivant de son mari. Quoique sa mère
lui eût laissé de beaux diamants, on ne lui vit jamais d'autres bijoux
qu'une petite bague d'or, presque usée, qui n'était pas un anneau de
mariage. Cette digne et sérieuse personne ne riait jamais; son sourire
avait je ne sais quoi de résigné. Elle n'aimait ni le jeu, ni la
conversation, ni la musique, excepté quelques vieux airs qu'elle jouait
sur son piano lorsqu'elle était seule; elle avait renoncé à la danse à
l'âge de dix-neuf ans, une année avant son mariage. Sa position et la
fortune de son mari l'avaient con damnée à recevoir et à aller dans le
monde; cependant ni dans le monde ni chez elle aucun homme ne lui avait
fait la cour. Une heure d'entretien lui avait toujours suffi pour
éteindre les passions que sa beauté avait allumées. L'amour le plus
intrépide aurait reculé devant le spectacle de ce cœur brisé, de cette
sensibilité éteinte, de cette âme pleine de ruines mystérieuses. Elle
n'aimait, après Dieu, que son fils Philippe, un beau jeune homme de
vingt ans, qui venait d'entrer dans la garde noble. Elle ne haïssait
personne: le seul homme dont elle évitât la rencontre était un ancien
ami de son mari, le colonel Coromila. Sa vie égale et monotone était
comme un tissu de prières et de bonnes actions. Toutes ses matinées se
passaient à l'église des Saints-Apôtres, sa paroisse; le soir, elle
allait dans les salons, comme une sœur de charité dans les mansardes,
pour soutenir les faibles et soulager les affligés. Elle excellait à
consoler les amours malheureux et à guérir ces secrètes blessures de
l'âme pour lesquelles le monde a si peu de pitié. Elle s'employait, avec
une prédilection visible, à marier les jeunes filles, et à aplanir les
obstacles que l'inégalité des fortunes élève entre ceux qui s'aiment. La
marquise avait détaché de son revenu une somme assez forte destinée à
doter annuellement quatre filles pauvres; mais, en dehors de cette
fondation pieuse, il lui arriva, dit-on, plus d'une fois de compléter la
dot d'une fille de noblesse. Ses petites soirées des jeudis ont fait en
une année plus de mariages que les grands bals du prince Torlonia n'en
feront en dix ans. Elle ne recevait cependant que de huit heures à
minuit. Sa santé ne lui permettait pas les longues veilles, et ce
n'était pas sans dessein qu'entre tous les jours de la semaine elle
avait choisi le jeudi. Les invités se retiraient à minuit moins un
quart, de peur d'empiéter sur le vendredi, jour de mortification, où les
théâtres font relâche dans toute l'Italie.

C'était un préjugé répandu dans Rome que toutes les unions contractées
sous les auspices de la marquise étaient nécessairement heureuses, et
lorsqu'on voulait désigner un mauvais ménage, on disait: Ils n'ont pas
été mariés par la Trasimeni. Quoique cette sainte femme fût un objet de
vénération pour tous et d'admiration pour quelques-uns, la curiosité
publique, qui ne perd jamais ses droits, cherchait encore, après plus de
vingt ans, le secret de sa tristesse; mais personne ne connaissait le
chagrin qui avait assombri une si belle vie. La comtesse Feraldi, son
amie d'enfance, se rappelait que la belle Assunta avait refusé deux où
trois fois la main du marquis Trasimeni, sans que rien pût expliquer
cette répugnance. Le jour du mariage, on avait eu beaucoup de peine à
lui faire quitter le noir pour prendre le costume traditionnel des
mariées. Elle avait dit à sa mère en partant pour l'église: J'entre dans
le mariage comme dans un couvent. De ces souvenirs très vagues, dont
l'authenticité même était fort contestée, quelques personnes avaient pu
conclure que la marquise portait le deuil d'un premier amour.

Au moment où commence cette histoire, M\textsuperscript{me} Trasimeni
était assise dans un coin du grand salon, entre la comtesse Feraldi et
une étrangère établie depuis plusieurs années à Rome, la générale
Fratief. Tout en causant, ces trois mères regardaient avec une
satisfaction visible un quadrille où leurs enfants étaient réunis.
Philippe où Pippo Trasimeni dansait avec Tolla, en face de Nadine
Fratief, toute fière d'avoir pour cavalier le lion des bals de Rome, le
roi de la jeunesse dorée, Lello Coromila, des princes Coromila-Borghi.

Pour un homme averti, les physionomies de ces quatre jeunes gens
auraient été un spectacle curieux. Lello Coromila paraissait causer très
vivement avec sa danseuse, qui semblait plaisanter et rire sans
arrière-pensée, avec tout l'abandon de la jeunesse. Pippo lutinait Tolla
pour obtenir une petite rose pâle qu'elle avait attachée à son corsage,
et Tolla, qui ne céda qu'à la dernière figure de la contredanse, était
très animée à la défense de son bien. Ni M\textsuperscript{me} Feraldi,
ni la générale, ni même la bonne marquise avec sa pénétration
maternelle, ne devinaient les sentiments cachés sous cette surface de
gaieté et d'indifférence; mais, à mieux surveiller les visages, elles
auraient reconnu que les yeux de Lello dévoraient Tolla, que Tolla,
confuse, inquiète et presque heureuse, se débattait contre un sentiment
nouveau pour elle, que Philippe, leur ami commun, les regardait l'un et
l'autre en homme qui voudrait les voir l'un à l'autre, --- et que
Nadine, malgré une expérience prématurée de l'art de feindre, laissait
percer dans ses yeux un peu d'amour, beaucoup d'ambition, et une de ces
haines concentrées dont les femmes seules sont capables.

Manuel où Lello Coromila était le fils cadet du prince Coromila-Borghi.
Les Coromila, si l'on en croit leur arbre généalogique, datent de la
guerre de Troie. L'histoire de leur famille remplit trois volumes
in-quarto, publiés à Parme en \oldstylenums{1780} par l'admirable imprimerie de Bodoni.
Le tome premier s'arrête à l'ère chrétienne, le second à l'an \oldstylenums{1000}; le
troisième, qui est presque entièrement authentique, contient la gloire
sérieuse de la famille. Ser Tita Coromila, grand amiral de la république
de Venise et père du doge Bartolomeo Coromila, remporta, à la fin du
XV\textsuperscript{e} siècle, la victoire navale de Naxie, qui arrêta
l'élan de la flotte turque et assura à Venise la domination de
l'archipel. Giuseppe Coromila était le chef de l'ambassade qui vint
complimenter le roi de France Henri \RomanNumeralCaps 4., à son avènement au trône. En mai
\oldstylenums{1797}, lorsque le gouvernement aristocratique de Venise abdiqua en faveur
du peuple, Lodovico Coromila quitta sa patrie et vint s'établir à Rome
avec sa famille. Les domaines de cette grande maison sont situés, partie
dans la Romagne, partie dans le royaume lombard-vénitien. Leur palais du
Corso c'est le plus magnifique de tous ceux qu'on admire à Rome; leur
villa d'Albano a des jardins aussi vastes et plus variés que ceux de
Versailles, et ils conservent à Venise quatre palais sur le grand canal.
Les trois branches de la famille réunissent entre elles une fortune
territoriale évaluée à près de cinquante millions; les Coromila-Borghi
possèdent un peu plus du quart de ce fabuleux patrimoine.

Tandis que l'héritier des doges s'avançait, pour la pastourelle,
au-devant de Nadine et de Tolla, la grosse générale Fratief couvait des
yeux les millions qu'elle voyait danser en sa personne, et répétait pour
la centième fois un panégyrique uniforme des perfections de Lello. Elle
s'obstinait à l'appeler le prince Lello, quoiqu'on lui eût redit à
satiété que Lello n'était et ne serait jamais prince. Le seul prince
Coromila-Borghi était son père, le vieux Luigi, après qui le titre
passait à l'aîné. Lello devait se résigner, comme son oncle le colonel,
à n'être jamais que le chevalier Coromila; mais la générale ne regardait
point les choses de si près. Chaque fois qu'il lui arrivait de se
méprendre, elle alléguait que chez elle, en Russie, tous les enfants
d'un prince sont princes, le prince eût-il une douzaine d'enfants.

La personne de Manuel Coromila, sans justifier le lyrisme maternel de la
générale, n'était point faite pour déplaire. Sa taille était haute, ses
épaules larges, son attitude prépondérante. Il avait véritablement une
physionomie romaine. Ses grands yeux à fleur de tête ne manquaient pas
d'un certain feu; son oreille rouge, teint fleuri, sa voix sonore
révélaient une santé excellente et une organisation robuste; sa barbe
noire, qui n'avait jamais été rasée, frisait légèrement sur ses joues;
ses cheveux presque bleus s'enlevaient vigoureusement sur un cou plus
blanc que celui d'une femme. Il avait les mains fortes et peu effilées;
mais elles étaient son si blanches, si grasses et si fermes, que leur
carrure inspirait la sympathie et la confiance. A tout prendre, Lello
était un fort beau jeune homme de vingt-deux ans.

De son esprit, la générale n'en disait mot: les choses de l'esprit
n'étaient pas du domaine de la générale. Elle s'extasiait sur sa grâce,
son élégance, sa gaieté, ses folies, sa piété. Lello était le
boute-en-train de la jeunesse romaine. Jusqu'à l'âge de vingt et un ans,
il avait vécu sous la surveillance sévère de son aïeul maternel; mais
depuis une année il s'était donné carrière. Il était l'organisateur de
tous les plaisirs, l'inventeur de tous les bons tours, le roi de tous
les bals, le conducteur de tous les \emph{cotillons}\footnote{Cotillon, alias le quadrille, est une danse de bal et de salon en vogue du début du XIX\textsuperscript{e} siècle à la Première Guerre mondiale.}. Du reste il entendait la messe tous les jours, récitait le rosaire en famille tous
les soirs, recevait les sacrements à tout le moins deux fois par mois,
et s'agenouillait sur le passage de la procession des quarante heures.

Il était bien rare que la générale, entraînée par sa préoccupation
dominante, ne mêlât point à son panégyrique l'éloge du palais Coromila,
de la galerie estimée deux millions, des écuries revêtues de marbre
blanc comme une église, des voitures, des livrées et des cent cinquante
serviteurs qui peuplaient la maison. Elle assaisonnait ces propos d'un
certain nombre de \emph{ah!} prononcés avec une aspiration gutturale
particulière aux gens du Nord. Dans sa bouche, cette exclamation était
je ne sais quoi de mitoyen entre \emph{ah!} et \emph{ach!}

Lorsqu'elle eut tout dit, elle passa, suivant sa coutume, à l'éloge de
sa fille, qu'elle appelait majestueusement «mademoiselle ma fille.» Elle
abusait de la patience inaltérable de la marquise et de
M\textsuperscript{me} Feraldi pour redire les perfections de Nadine, ses
talents, la dépense qu'on avait faite pour son éducation à Paris et à
Rome, les inquiétudes qu'elle avait données dans son enfance, la crainte
qu'on avait eue de la voir scrofuleuse comme presque toutes les jeunes
filles de l'aristocratie russe, les sirops amers qu'elle avait pris, les
beaux résultats qu'on avait obtenus, ses os raffermis, sa taille
redressée, les appareils de Valérius devenus inutiles, sa beauté de jour
en jour plus brillante, les succès qu'elle avait eus dans le monde, les
partis qu'elle avait refusés (le plus modeste était d'un million), les
triomphes qui l'attendaient à Pétersbourg, les bontés de l'empereur
Nicolas, qui la regardait comme sa fille adoptive et lui destinait le
\emph{chiffre} des demoiselles d'honneur, enfin la belle entrée qu'elle
ferait à la cour de Russie avec une robe traînante de velours ponceau,
un \emph{kokochnik}\footnote{une coiffure traditionnelle féminine russe} brodé d'or et de perles, et le chiffre en diamants
sur l'épaule gauche.

M\textsuperscript{me} Fratief parlait comme les autres crient. Elle
joignait à ce petit défaut l'habitude de se répéter souvent et
d'inventer quelquefois; mais il était convenu qu'elle avait bon cœur.
D'ailleurs sa qualité d'étrangère, le train qu'elle menait et le soin
qu'elle avait pris d'élever sa fille dans la religion romaine la
faisaient tolérer dans la plus haute société. On lui savait gré d'avoir
amené dans le giron de l'église la fille d'un général russe et dérobé au
schisme grec une âme de qualité. Le manège désespéré auquel elle se
livrait pour attirer l'attention de Manuel Coromila n'inquiétait
personne. On savait que Lello n'était pas encore à marier, et d'ailleurs
sa famille lui destinait une princesse. M\textsuperscript{me} Trasimeni
laissa donc à la générale tout le temps d'achever les deux portraits
qu'elle recommençait tous les soirs pour avoir le plaisir de les
enfermer dans le même cadre. Lorsqu'on fut au \emph{kokochnik} et au
chiffre en diamants, qui formaient la péroraison habituelle, la
marquise, après un petit compliment à l'adresse de Nadine, se tourna
vers Mm. Feraldi:

«Et Tolla?»

«A propos! c'est vrai,» ajouta la générale. «On dit que vous la
mariez; j'en serai bien heureuse.»

«Cela n'est pas encore fait,» reprit vivement M\textsuperscript{me}
Feraldi. «Tu sais, ma chère,» dit-elle à la marquise, «que dans les
premiers jours du mois dernier nous avons reçu deux lettres, l'une de
mon frère d'Ancône, l'autre de mon cousin de Forli, qui proposaient,
chacun de son côté, un mari pour Tolla. Le jeune homme de Forli a
vingt-quatre ans; il est fils unique, et il aura vingt mille francs de
rente.»

«Mais c'est magnifique, chère comtesse!» interrompit la générale,
«et j'espère bien que Tolla\ldots»

«Tolla a vu celui qu'on lui proposait. C'est un beau garçon, grand,
blond et parfaitement élevé. Elle l'a refusé net.»

«Sans dire pourquoi?»

«Elle a dit qu'il lui était antipathique. L'autre n'est pas encore
venu à Rome, et il ne viendra que si nous lui donnons des espérances. On
le dit fort bien de sa personne; il n'a pas trente ans. Il est plus
riche que notre prétendant de Forli. Nous nous sommes informés de sa
réputation: nous n'en avons appris que du bien. Il sait quelle est la
dot de Tolla, et il vient d'écrire à mon mari qu'il en était très
satisfait, qu'il se serait contenté de moitié. ‹Ce que je cherche,
disait-il en terminant, c'est une amie, une femme aimante, une bonne
mère de famille, une personne enfin qui sache me pardonner mes
innombrables défauts.›»

«Ah! c'est beau! c'est admirable! c'est sublime!» s'écria la
générale, «et dans un siècle comme le nôtre, où les jeunes gens sont
devenus plus égoïstes que les vieillards! Le digne jeune homme! j'espère
bien que Tolla ne le refusera pas!\ldots»

La générale en était là de ses exclamations, lorsqu'un murmure aussi
léger, aussi rapide, aussi dru et aussi précis que le bruit du vent dans
les feuilles sèches, se répandit dans le salon, dans le jardin, dans la
salle de jeu, dans tous les coins de la maison, et vint enfin bourdonner
autour de ce trio de mères de famille. Une nouvelle imprévue, et qui les
frappa toutes les trois comme un coup de foudre, arriva jusqu'à elles
sans qu'on pût savoir d'où elle était venue. C'était une de ces rumeurs
agiles et discrètes qui semblent se répandre d'elles-mêmes et par leur
propre force, et qui entrent dans toutes les oreilles sans qu'on les ait
vues sortir d'aucune bouche. Lorsqu'elle s'abattit sur le divan de la
marquise, des émotions bien diverses, mais également violentes, se
peignirent sur le visage des trois mères qui causaient en semble. La
générale rougit comme une apoplectique: le désappointement, la jalousie,
l'a varice déçue, l'ambition détrônée, la crainte du ridicule, la
résolution de combattre, la confiance dans ses forces, et au pis-aller
l'espoir de la vengeance, en un mot toutes les passions haineuses
passèrent avec la rapidité de l'éclair sur cette large figure
empourprée. M\textsuperscript{me} Feraldi, surprise par un coup de
bonheur auquel elle n'était point préparée, s'arrêta bouche béante,
aussi stupéfaite qu'un aveugle qui recouvrerait la vue devant un feu
d'artifice. La bonne marquise, qui avait vu naître Tolla, qui l'appelait
tendrement «ma fille,» „et qui n'avait consenti à recevoir un Coromila
dans sa maison que sur les instances de Philippe, réprima un mouvement
de surprise douloureuse et fit rentrer deux grosses larmes, lorsqu'elle
entendit murmurer cette terrible nouvelle: «Savez-vous? Lello aime
Tolla!»

La comtesse et la générale, en femmes du monde, furent promptes à cacher
leur émotion. La générale surtout escamota si vivement son dépit, que
l'œil d'une ennemie n'en aurait rien vu. La conversation se prolongea
sans incident jusqu'à onze heures trois quarts, et l'on ne s'entretint
que de la pluie et des sermons de l'abbé Fortunati, qui faisait
merveille aux Saints-Apôtres. Tolla conduisit le \emph{cotillon} avec
Lello. M. Feraldi, qui bouillait d'impatience en attendant l'heure du
départ, gagna cinquante-deux fiches à son cousin le cardinal Pezzato.
Tout le monde se retira à l'heure ordinaire, et la générale, en
remerciant la maîtresse de la maison, suivant l'usage établi en Russie,
assura qu'elle n'avait jamais passé une soirée plus délicieuse.

En arrivant au grand escalier, Tolla voulut prendre le bras de son père;
mais, sur un signe du comte, elle partit en avant avec Toto. Elle trouva
sous le vestibule un colosse hâlé qui l'enveloppa maternellement dans
une lourde pelisse. C'était son ancien pédagogue de Lariccia, le fidèle
Menico. «Il pleut un peu,» lui dit-il, «et quoique la maison ne soit
pas loin, Amarella m'a envoyé. Mais qu'avez-vous, mademoiselle? Il vous
est arrivé quelque chose!»

«Tu crois, mon Menico?»

«J'en suis sûr, mademoiselle. Il y a deux choses au monde que je
connais bien, c'est le ciel et votre visage. Ici et là, je sais quand
l'orage doit venir.»

«J'ai donc la figure à l'orage?»

«Non, mais il me semble que vous êtes à la fois heureuse et fâchée.
Est-ce vrai, mademoiselle? Peut-être; mais pourquoi veux-tu que je te
dise mes secrets, mon pauvre Dominique? Ce sont choses où tu ne peux
rien.»

«Pardonnez-moi, mademoiselle, je puis toujours faire finir celui qui
voudrait vous fâcher. Venez, que je vous débarrasse de votre manteau:
nous sommes arrivés.»

Le comte et la comtesse accouraient sur les pas de leurs enfants après
une conférence d'une minute. Toto se retira discrètement, sans faire
allusion à ce qu'il avait entendu dans la soirée. Le comte embrassa sa
fille et sa femme, et rentra chez lui. Menico alla se coucher à
l'écurie, où un palefrenier lui prêtait la moitié de son lit.
M\textsuperscript{me} Feraldi reconduisit Tolla dans sa petite chambre,
la fit asseoir sur le seul canapé qui s'y trouvât, s'y jeta vivement à
côté d'elle, l'embrassa avec effusion et lui dit: «Raconte-moi tout!
Il t'aime?»

«Je le crois.»

«Depuis quand?»

«Qui sait? Peut-être depuis le commencement de l'hiver.»

«Te l'a-t-il dit?»

«Jamais. La seule preuve d'amour qu'il m'ait donnée pendant six
mois, c'est de m'inviter à danser de préférence à toutes les autres. On
me l'enviait assez. La Russe a fait des pieds et des mains pour obtenir
un cotillon avec lui; elle n'y est jamais parvenue. Moi, je ne regardais
cette préférence que comme un hommage rendu à la sagacité avec laquelle
j'exécutais les nouvelles figures que nous inventions; mais ces
demoiselles avaient de meilleurs yeux que moi: il y a longtemps qu'elles
ont remarqué le plaisir qu'il éprouve à me faire danser, l'empressement
avec lequel il me cherche en entrant dans un salon, sa joie dès qu'il
m'aperçoit, son désappointement si je n'y suis pas. D'ailleurs il a
parlé.»

«A qui?»

«A ses amis. Il n'a jamais osé me dire qu'il m'aimait, mais il a eu
l'imprudence de le laisser voir aux cinq où six étourdis qui composent
sa cœur. Ceux-là l'ont appris à d'autres; ils se sont mis à me
persécuter de cet amour, ils ont prétendu que je le partageais, et je ne
danse pas avec l'un d'entre eux sans qu'il me dise: ‹Lello vous aime›.»

«Lello vous aime!» répéta M\textsuperscript{me} Feraldi en serrant
sa fille dans ses bras. «Et que leur répondais-tu?»

«Moi? La première fois que Pippo Trasimeni s'amusa à me dire que
j'étais aimée et que j'aimais, je lui répondis avec vivacité: ---
‹Comment, m'estimez-vous assez peu pour croire que je m'amuserais à
faire l'amour par passetemps?› ‹Je ne dis pas cela,› reprit-il. ---
‹Pardonnez-moi, vous le dites. Le caractère de M. Coromila est connu; on
sait que depuis la mort de son grand-père il a fréquenté des jeunes gens
de toute sorte, au lieu de s'en tenir à ceux qui vous ressemblent,
Philippe. On répète partout qu'il se joue de la chose du monde la plus
sérieuse, l'amour; qu'il est un de ces hommes qui n'ont d'autre
occupation au monde que de tromper notre sexe, et qu'une liaison avec
lui ne saurait amener rien de bon›.»

«Et Philippe t'a répondu?\ldots»

«Rien.»

«Il te donnait raison.»

«Oui; mais le jeudi suivant je le retrouvai chez sa mère, et elle me
dit: ‹Lello vaut mieux que vous ne pensez; il ne parle que de vous et il
vous aime à la folie.› C'est la seule fois qu'on m'ait dit du bien de
Lello.»

«Et qui est-ce qui t'en a dit du mal?»

«Toutes les femmes. Voici plus de quatre mois que les filles de mon
âge se servent de son nom pour me persécuter. L'une vient me dire:
‹Enfin, vous êtes amoureuse, et c'est Lello qui a fait ce miracle-là!›
Une autre me félicite d'avoir fixé le plus volage des hommes. Me Fratief
n'a-t-elle pas eu le front de me dire un jour à brûle-pourpoint:
‹Franchement, ma chère, comptez-vous vous faire épouser par Lello?› ---
Une question si impertinente, venant d'une fille qui n'est pas mon amie
et que je connais à peine, me saisit tellement que je restai un instant
sans parole; mais je revins à moi, et je lui répondis que j'étais
incapable de m'intéresser à une personne qui n'aurait pas les vues les
plus honnêtes. Elle répliqua vivement: ‹Ne vous fiez pas à Lello; il en
a trompé plus d'une, et il change d'amour deux fois par mois.› Je
l'entendais décrier partout comme un homme léger; mais je ne savais
comment concilier l'effronterie dont on l'accusait avec le respect qu'il
témoignait pour moi. Jamais il n'a pris une de ces libertés que les
jeunes gens se permettent au bal; jamais il ne m'a serré la main en
valsant. Quand nos regards se rencontraient, il était plus prompt que
moi à détourner les yeux. Quelquefois j'enrageais de penser qu'il
affichait devant les autres un si grand amour pour moi, sans m'en avoir
donné la moindre marque. Puis, songeant au respect qu'il me témoignait,
j'en étais touchée. Peut-être est-ce là ce qui a pris mon cœur.»

«Tu l'aimais! Pourquoi ne m'en as-tu rien dit?»

«Je l'aimais peut-être; mais comme il ne m'avait pas donné de
marques visibles de son amour, je n'osais pas m'avouer le mien à moi
même. Il me semblait que c'était une folie d'aimer sans savoir que
j'étais payée de retour, sinon par les bavardages des effrontés qu'il
avait autour de lui. C'est alors que vous avez fait cette petite maladie
qui vous a retenue trois semaines à la maison, et moi avec vous. Trois
semaines sans le voir! La privation que je ressentis me donna la mesure
de mon amour. Pendant cette longue séparation, on dansa trois fois chez
la Trasimeni et deux fois à l'ambassade de France. Ces jours-là je
restai à ma fenêtre jusqu'à la fin de la soirée, pour avoir le plaisir
d'entendre sa voix lorsqu'il sortirait avec ses amis. J'avais soin de me
cacher dans l'ombre de mes rideaux: je serais morte de honte, s'il avait
pu seulement soupçonner ma faiblesse. Quelquefois je l'entendais parler
de moi avec ses camarades. Un soir, tandis que ses amis chantaient à
tue-tête une grosse chanson dont le refrain était:
\begin{quote}
L'acqua fa male,\\
Il vino fa cantare,
\end{quote}
je reconnus sa belle voix qui fredonnait cette chanson des pêcheurs de
Sainte-Lucie:
\begin{quote}
Io ti voglio ben assai, \\
Ma tu non pensi a me!
\end{quote}
et il lança en s'éloignant un soupir grave et puissant qui semblait
sortir du fond de son cœur. Peut-être, s'il avait osé me déclarer sa
passion, aurais-je su y résister et la combattre par le dédain; mais
cette extrême timidité, si rare chez un homme, me subjugua.

«Mais ce soir qu'a-t-il fait? qu'a-t-il dit? Il s'est donc trahi?»

«Mon Dieu! non. Ce soir, Philippe m'a demandé cette fleur que j'avais à
mon corsage; e la lui ai donnée. Après la contredanse, Lello a entraîné
son ami dans le jardin, et lorsqu'ils sont rentrés, Philippe n'avait
plus la fleur à sa boutonnière. Je devinais bien le chemin qu'elle avait
pris, mais j'eus l'air de ne rien savoir, et je demandai à Philippe ce
qu'il en avait fait; il me répondit: ‹Manuel m'a tant prié de la lui
donner, qu'il a bien fallu en faire, le sacrifice.› Je feignis d'être
piquée, mais j'aurais voulu sauter au cou de ce bon Philippe.
Malheureusement on les avait suivis au jardin, on les avait écoutés, on
a parlé, et voilà comment vous avez tout appris.»

«Mieux vaut tard que jamais, ajouta la comtesse, trop heureuse pour
formuler un reproche. Maintenant, terrible enfant, écoute-moi. Tu aimes.
Si nous t'abandonnons à tes inspirations, cet amour ne te donnera que
des chagrins: j'en attends quelque chose de mieux. Me promets-tu de
suivre mes conseils et ceux de ton père?

«Oui, ma mère.»

«Si Lello t'écrit, tu nous montreras ses lettres?»

«Oui, ma bonne mère.»

«Tu ne lui répondras rien sans nous consulter?»

«Rien.»

«Toutes les fois que tu le rencontreras dans le monde, tu me
répéteras ses paroles et les tiennes?»

«Je le promets.»

«Et moi, je te promets que tu seras avant an la femme de Lello.
Bonne nuit, madame Coromila!»

La comtesse courut retrouver le comte, qu'une préoccupation violente
tenait éveillé. Ils passèrent la nuit à débattre un plan de campagne
dont le résultat devait être le bonheur de leur fille et la grandeur de
la maison Feraldi.
