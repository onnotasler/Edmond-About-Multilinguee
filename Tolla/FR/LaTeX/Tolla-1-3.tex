\addchap{\RomanNumeralCaps 3.}

Tandis que Tolla se confessait à sa mère, M\textsuperscript{me} Fratief
se faisait raconter par Nadine l'événement de la soirée et les amours de
Lello. Elle lui reprocha amèrement de ne l'avoir pas tenue au courant de
ce qui se passait. Si Nadine n'en avait rien dit, c'est qu'elle avait
une confiance limitée dans le bon sens de sa mère; elle raisonnait comme
ces chasseurs qui aiment mieux chasser sans chien qu'avec un chien mal
dressé. M\textsuperscript{me} Fratief, née Redzinska, était veuve du
général Fratief, aide de camp de l'empereur Alexandre. Après la campagne
de France, Fratief, qui n'était plus jeune et que les plaisirs faciles
de Paris avaient vieilli autant que la guerre, fut nommé gouverneur de
Varsovie. Il vit, au premier bal qui lui fut donné par la ville, la
célèbre Sophie Redzinska, dont la beauté opulente lui rendit six mois de
jeunesse. Il l'épousa sans dot et malgré les remontrances de la cour,
qui se scandalisait de voir un général illustre, un ami de Souvarof et
un favori du maître s'abaisser jusqu'à une Polonaise. Le vieux soldat,
aiguillonné par un dernier amour sut donner à ses faiblesses une couleur
politique et persuader à l'empereur qu'une telle mésalliance rallierait
la noblesse de Varsovie. Après une année de mariage, il mourut, comme le
roi Louis XII, au milieu de son bonheur domestique. La générale resta
veuve à vingt ans avec une fille de trois mois. Son mari laissait pour
tout héritage une année de solde, quarante mille francs environ. Fils
d'un petit marchand de la troisième guilde, il avait poussé sa fortune,
franchi tous les grades de l'armée et escaladé tous les degrés de la
noblesse, sans songer à s'enrichir. Mm. Fratief, qu'on appelait à
Varsovie la Belle et la Bête, avait si bien mis à profit la courte durée
de son règne, elle avait regardé de si haut ses compatriotes et ses
anciens amis, protégé si dédaigneusement sa famille et gouverné sa bonne
ville d'un air si impertinent, qu'elle fit en peu de temps une ample
provision d'ennemis. Toutes les autorités de la ville assistèrent par
devoir aux funérailles du général, mais sa veuve ne reçut pas quatre
visites. Le patriotisme polonais saisissait l'occasion de faire pièce à
la Russie sans danger. La belle Sophie tira vanité de cette haine
universelle, qui témoignait de son importance et du pouvoir qu'elle
avait eu. Elle s'exila comme en triomphe d'une ville qui la repoussait,
et partit pour Pétersbourg avec sa fille, ses quarante mille francs, sa
beauté, ses diamants, son orgueil, sa sottise et ses espérances.
Arrivée, elle vit avec surprise que la cour n'était pas venue au-devant
de sa chaise de poste. Elle demanda une audience de l'empereur; elle
l'obtint, et elle courut au palais d'hiver, prête à verser ses chagrins,
ses inimitiés et toutes ses confidences dans le cœur. paternel
d'Alexandre. L'empereur la reçut à son tour d'inscription, entre un
gouverneur de province et un savant étranger; il lui débita avec bonté
un petit compliment de condoléance, et promit de lui assurer, à elle et
à sa fille, une existence honorable. Au sortir de cette audience, Sophie
courut annoncer aux cinq où six personnes qu'elle connaissait dans la
ville que l'empereur l'avait reçue comme un père, qu'il avait pleuré en
parlant de son fidèle Fratief, et qu'il avait fini par lui dire en
propres termes: «Désormais, madame, vous faites partie de ma famille;
j'adopte votre chère petite Nadine, je me charge de sa fortune et de la
vôtre. Mon palais et mon cœur. Vous seront toujours ouverts: frappez, et
l'on vous ouvrira; demandez, et vous recevrez.»

Huit jours après, elle reçut deux brevets de quinze cents roubles
argent, où de six mille francs de pension, l'un pour elle, l'autre pour
sa fille. C'est ce que la loi de l'empire accorde à toutes les veuves où
orphelines des aides de camp généraux. Chacune de ces deux pensions
cessait de plein droit le jour du mariage de la titulaire. Sophie
s'imagina qu'on lui faisait une injustice parce qu'on ne faisait point
d'injustice en sa faveur; mais elle avait trop de vanité pour se
plaindre. Elle loua sur le canal Catherine un appartement de quatre
mille francs, et commanda un mobilier de vingt mille. A ceux qui
connaissaient le chiffre de sa fortune et la modicité de sa pension,
elle donnait à entendre qu'elle avait dans l'amitié de l'empereur des
ressources inépuisables. On la vit pendant trois ans à toutes les
réunions de la cour, où le nom de son mari lui donnait les grandes et
petites entrées. Sa beauté lui attira quelques déclarations et une où
deux demandes en mariage, qu'elle repoussa, attendant mieux. Le
grand-duc Michel la distingua pendant un mois où deux; il fut
promptement rebuté, non par sa pruderie, mais par sa sottise. Elle
s'essaya sans succès dans le rôle des grandes coquettes: elle avait la
figure sans l'esprit de l'emploi. Ses agaceries ne servirent qu'à la
compromettre. Trop froide pour faire des sottises gratuites, trop
maladroite pour en faire de profitables, elle ne sut ni se donner ni se
vendre, et elle garda, sans savoir pourquoi, une vertu à laquelle on ne
crut guère et dont personne ne lui sut gré. Après trois ans de ce
ménage, elle disparut subitement: ses ressources étaient épuisées. Son
mobilier et ses diamants indemnisèrent à peine ses créanciers. Elle
partit pour l'Allemagne, où elle vécut d'épargne et de jeu, courant les
eaux, cherchant un mari, grossissant la liste des conquêtes qu'elle
croyait avoir faites, et usant sur les grands chemins les restes de sa
beauté, qui passa vite. En 1828, elle vint à Paris, et elle songea à
l'éducation de Nadine, qui avait onze ans. Elle se logea rue de
l'Université, et meubla péniblement un très petit coin d'un très grand
hôtel. Pour se faire admettre dans les salons du faubourg Saint-Germain,
elle s'avisa de conduire sa fille au catéchisme de Saint-Thomas-d'Aquin.
Nadine y fit sa première communion. Si on l'avait su à Pétersbourg, la
mère et la fille au raient infailliblement perdu leur pension. Cette
imprudence ne leur servit de rien, et personne à Paris ne leur en tint
compte: la générale, à force de vanteries et de mensonges évidents,
avait obtenu de passer pour une aventurière. L'éducation de Nadine fut
un prodige d'économie mal entendue. Toutes ses leçons furent payées deux
francs l'une dans l'autre. Une grande fille noirâtre, la plus disgraciée
des élèves du Conservatoire, lui enseigna l'art de martyriser un piano.
On lui déterra la plus rousse et la plus piteuse des maîtresses
d'anglais, une image vivante de la misère, qui au rait pu poser pour la
statue de l'Irlande. Ce fut un surnuméraire des bureaux de la préfecture
qui lui apprit la langue et la littérature françaises, l'histoire, la
géographie, l'arithmétique, la physique, et un peu de métaphysique. Son
maître de danse est mort l'an dernier à l'hospice de La Rochefoucauld:
il était le dernier de sa profession qui eût conservé l'usage de la
pochette. Grâce au zèle de ces pauvres gens, que la générale appelait
les premiers maîtres de Paris, Nadine oublia complètement le russe, le
polonais et l'allemand, qu'elle avait sus dans son enfance; elle écrivit
assez, correctement le français, sauf les participes, et elle déchiffra
les premiers chapitres du \emph{Vicar of Wakefield;} elle sut danser
toutes les contredanses et en jouer une. Dans les intervalles de ses
leçons, elle se donna à elle-même un supplément de connaissances
positives en dévorant le fonds d'un petit cabinet de lecture de la rue
de Poitiers. Les romanciers à la mode de 1830 à 1834 furent les vrais
maîtres de son esprit. Les appareils orthopédiques de Valerius et les
trapèzes du gymnase Amoros furent les précepteurs de sa beauté.

Nadine avait dix-sept ans, une jolie figure et la taille droite, lorsque
sa mère, désespérant de la produire à Paris, se décida à la conduire en
Italie. Un vieil émigré français entré au service de la Russie comme les
Modène et les La Ribeaupierre, le marquis de Certeux, gouverneur de la
résidence impériale de Gatchina, lui envoya une lettre de recommandation
pour sa sœur, M\textsuperscript{me} la chanoinesse de Certeux, qui la
présenta à toute l'aristocratie romaine. Nadine eut du succès: elle
était grande, grasse et blanche; on l'invita partout, on la fit danser,
mais personne ne songea à demander sa main. La générale, qui était femme
à prendre les épouseurs au collet, fit le guet pendant trois ans autour
de sa fille sans pouvoir appréhender au corps le moindre millionnaire.
Pour comble de douleur, elle fut forcée de reconnaître que la beauté de
Nadine n'était pas dorée au feu, et qu'elle passerait bientôt. Cette
fille de vingt ans luttait sans succès contre un embonpoint toujours
croissant; ses corsets étaient des œuvres d'art qui attestaient les
progrès de la mécanique au XIX\textsuperscript{e} siècle; l'émail de ses
dents se fendait, et sa mère, qui la coiffait elle-même, lui avait déjà
arraché quelques cheveux blancs. M\textsuperscript{me} Fratief, qui
avait reporté sur sa fille toutes ses espérances, et qui ne comptait
plus que sur elle pour échapper à la médiocrité et à ses douze mille
francs de pension, s'endetta pour la faire belle. Nadine, dont le linge
aurait fait sourire la plus modeste bourgeoise, portait des robes de
velours d'Afrique et de taffetas chiné que Palmyre lui envoyait de
Paris. Ces frais de toilette furent d'abord à l'adresse de tous les
jeunes Romains qui avaient cinquante mille francs de rente et au-dessus;
mais du jour où Manuel Coromila, après la mort de son grandpère, fit son
entrée dans le monde, la fille et la mère ne pensèrent plus qu'à lui. Il
remarqua Nadine et s'en occupa quinze jours; il n'en fallait pas
davantage pour qu'on fondât sur lui les espérances les plus sérieuses.

Cette revue rétrospective servira peut-être à expliquer pourquoi le 30
avril 1837 M\textsuperscript{me} Fratief et sa fille regardaient Tolla
comme un joueur malheureux regarde la carte qui doit achever sa ruine.
Elles cherchèrent ensemble quel serait le moyen le plus sûr de reprendre
le cœur. qu'on leur avait dérobé.

Pour Lello, il rentra au palais Coromila en rêvant à un bon tour qu'il
voulait jouer à un de ses amis. Il s'agissait de semer des pétards sous
les pas d'un pauvre garçon qui courtisait une petite mercière et qui
trahissait l'amitié en gardant le secret de ses amours. Rome a des
habitudes de petite ville; les boutiques s'y ferment de bonne heure, et
les jeunes gens y font des farces par désœuvrement. Le fils des doges
s'assura en rentrant qu'on lui avait apporté une petite boîte de poudre
fulminante; puis il baisa la rose de Tolla, se regarda dans la glace,
fredonna in air du Barbier, se laissa déshabiller par son valet de
chambre, et se mit au lit en pensant à Tolla, à la mercière, à un cheval
qu'il voulait acheter, et à la bonne figure que ferait son ami
pataugeant à travers un feu d'artifice. Il dormit à franc étrier jusqu'à
huit heures du matin. La marquise passa la nuit en prières. Tolla rêva
qu'un certain citronnier de sa connaissance se couvrait, par exception,
de fleurs d'oranger.

Le lendemain, comme Lello s'apprêtait à employer sa poudre fulminante,
quelques grains égarés entre la boîte et le couvercle s'allumèrent par
le frottement, et tout lui sauta au visage. Le bruit se répandit dans
Rome qu'il avait les sourcils brûlés, trois où quatre énormes ampoules,
et qu'il garderait la chambre pen dant une semaine où deux.
M\textsuperscript{me} Feraldi s'empressa d'envoyer chercher de ses
nouvelles: il faut, pensait-elle, que je rassure ma pauvre Tolla. Le
même jour, Nadine dit à sa mère:

«Victoire! Il s'est blessé à la figure. Elle ne le verra pas de quinze
jours. Maintenant, ma bonne petite mère, veux-tu m'en croire? Envoie
François savoir de ses nouvelles.»

«Y songes-tu? Nous le connaissons à peine; il n'est jamais venu nous
voir.»

«Précisément. Quand il saura que nous nous sommes inquiétées de sa
santé, il nous devra une visite.»

Le courrier, l'intendant, le valet de chambre et le cuisinier de la
générale, François, surnommé Cocomero où le Melon, était un vigoureux
Napolitain. Lorsqu'il revint du palais Coromila, il avait l'œil droit
entouré d'une auréole bleue. Il s'était rencontré avec Menico sous le
vestibule; il avait voulu prendre le pas, l'antipathie avait agi, et
Menico lui avait montré le poing d'un peu trop près. Chacun des deux
combattants garda scrupuleusement le secret de ses prouesses. Menico,
qui n'était à Rome que pour quelques jours, craignait qu'on ne le
renvoyât garder ses buffles; Cocomero avait trop d'amour-propre pour
avouer une défaite. Il attribua à un coup d'air la couleur anormale de
son orbite. Pendant les dix jours que Manuel resta à la maison, la
générale et la comtesse y envoyèrent Cocomero et Menico tous les ma
tins; mais Cocomero avait trop de prudence pour s'exposer à un second
coup d'air. Il descendait en droite ligne de ces guerriers napolitains
qui répondirent à leur général: «Vous voulez que nous allions là-bas;
nous ne demanderions pas mieux, mais\ldots{} c'est que\ldots{}
là--bas\ldots{} il y a le canon!»

La première fois que Lello reparut dans le monde, il oublia de faire
danser Nadine, mais il fut plus empressé que jamais auprès de Tolla.
Tolla s'était intéressée à sa santé! A la dernière figure du cotillon,
il lui dit en tremblant un peu:

«Si je pensais que madame votre mère fût disposée à me le permettre,
j'irais la remercier de l'intérêt qu'elle m'a témoigné après ce ridicule
accident; mais,» ajouta-t-il en la regardant fixement, «je crains de
n'être point agréé.» Tolla sentit le rouge lui monter au visage. Elle
répondit en balbutiant que sa visite leur aurait fait honneur, que sa
personne ne pouvait qu'être agréable à tous ceux qui avaient la bonne
fortune de l'approcher, «D'ailleurs,» dit-elle en terminant, «tous ceux
qui viennent à la maison nous font une grâce.»

Cette invitation, qui pourrait nous paraître d'une politesse exagérée,
n'était en Italie que strictement convenable. Nous n'avons qu'une faible
idée de tous les raffinements inventés par la courtoisie italienne. Si
l'on frappe à la porte de votre chambre, vous répondez brutalement:
«Entrez!» Un Italien, sans savoir quelle est la personne qui frappe,
répond en un seul mot: «Que votre seigneurie me fasse la faveur
d'entrer, \emph{favorisca!}» C'est ainsi que la réponse de Tolla doit
être interprétée.

L'italien: S'il vous plaît!

Tolla et la famille entière attendirent avec la plus vive anxiété cette
visite de Lello. Il ne vint pas. Il était dans une situation d'esprit
que toutes les femmes refuseront de comprendre, mais qui inspirerait de
la sympathie et peut être de la compassion à beaucoup de jeunes gens.

Il aimait, et sans recourir à un long examen de conscience, il voyait
clairement que son cœur. était pris. Il aimait une personne moins riche
que lui et d'une condition un peu inférieure à la sienne. Il pouvait
prétendre à la main d'une princesse et à une dot de deux où trois
millions. Épouser Tolla, c'était renoncer à l'appui de quelque grande
alliance et retrancher de son revenu possible et probable environ cent
mille francs de rente: considération misérable sans doute! mais les
Italiens sont des esprits positifs. L'histoire romaine en est la preuve.

Il aimait, et sans recourir à un long examen de conscience, il voyait
clairement que son cœur était pris.

Il aimait une personne moins riche que lui et d'une condition un peu
inférieur a l sienne. Il pouvait prétendre a la main d'une princesse et
a une dot de deux où trois millions. Épouser Tolla, c'était renoncer a
l'appui de quelque grande alliance et retrancher de son revenue possible
et probable environ cent mille francs de rente: considération misérable
sans doute! Mail les Italiens sont des esprits positifs. L'histoire
romaine en est la preuve.

Il aimait; malheureusement il n'était pas sûr que sa famille consentît à
un tel mariage. Il dépendait de son père, vieillard inflexible. Ce vieux
Louis Coromila était aveugle et paralytique, mais du fond de son
fauteuil il conduisait toute sa maison et faisait trembler ses fils
comme au temps où le chef de famille avait droit de vie et de mort sur
ses enfants. Après la mort de son père, Lello aurait encore, sinon à
redouter, du moins à ménager ses deux oncles, le cardinal et le colonel.
Il ne se souciait pas d'être déshérité au profit de son frère.

Si Tolla avait été une ouvrière où une petite bourgeoise, Lello se fût
abandonné sans résistance au penchant qui l'entraînait vers elle; mais
avant de séduire une fille noble qui a un père de cinquante ans, un
frère de dix-neuf et un cousin cardinal, l'amoureux le plus imprudent y
regarde à deux fois. D'ailleurs Lello voulait garder aux yeux du monde
et à ses propres yeux la qualité d'honnête homme. Il se disait: «Je ne
veux ni la séduire, ni la compromettre, ni l'empêcher de se marier. Je
l'aime cependant. Eh bien! je l'aimerai à distance, sans le lui dire.»
Mais il ne pouvait empêcher ses yeux de parler, ni les yeux de Tolla de
répondre, ni leurs cœurs de s'attacher secrètement l'un à l'autre. Il
avait beau se promettre de laisser à Tolla toute sa liberté, afin de
conserver toute la sienne: il s'apercevait tous les jours qu'il avait
obtenu plus qu'il ne désirait et qu'il s'était engagé plus qu'il
n'aurait voulu. Il croyait avoir remporté une grande victoire sur
lui-même lorsqu'il avait tenu devant Tolla les discours les plus
passionnés, sans lui dire: «Je vous aime.» Il se faisait comme un devoir
religieux d'éviter cette formule, dont il prodiguait l'équivalent à
toute heure. Il disait en rentrant chez lui: «J'ai sauvé deux âmes.» Il
n'avait sauvé que trois mots.

Quelquefois, voyant l'abandon et la naïveté de Tolla, qui laissait
éclater l'amour dans tous ses regards, il se sentait pris de défiance.
La défiance est une terrible vertu en Italie. Je connais un sculpteur
romain qui a marché pendant cinq ans avec une paire de pistolets dans
les poches de son pantalon: il se défiait de quelqu'un. Lello se défiait
par moments de sa chère Tolla. Il était bien jeune, mais le soupçon naît
plus tôt chez les riches que chez les pauvres, sans doute parce qu'ils
ont plus de choses à garder. Cet enfant de vingt-deux ans avait entendu
parler des petits manèges que les mères emploient pour marier leurs
filles et des ruses que les filles inventent elles-mêmes pour entrer en
possession d'un mari. Il avait pu voir de ses yeux comment les Nadine
Fratief et leurs pareilles cherchent un homme, sans lanterne, et il se
demandait quelquefois si l'amour que Tolla lui laissait deviner n'était
pas un piège vulgaire destiné à prendre les cœurs. Sa vanité se
révoltait à l'idée qu'il pouvait être dupe; mais la présence de Tolla et
le long regard de ses yeux limpides dissipait bientôt tous ces méchants
soupçons.

Ces alternatives de défiance et d'abandon, de calcul et de
désintéressement, donnaient à sa conduite toutes les apparences de la
coquetterie.

Pendant un mois, il rencontra Tolla presque tous les soirs sans lui
parler de la permission qu'il avait demandée et obtenue. La gêne que
cette idée lui causait le rendit plus froid et plus réservé. Nadine, qui
ne perdait pas un seul de ses mouvements, jugea que ce grand amour avait
baissé de quelques degrés. Le monde se demanda s'il n'avait pas été trop
prompt à accueillir la nouvelle de la passion de Lello. La bonne
marquise espera que ses craintes auraient tort. Un soir, Pippo dit à son
ami:«Eh bien! beau ténébreux; nous avons donc été mal reçu au palais
Feraldi?»

«Moi! je n'y suis pas allé.»

«En ce cas, j'ai tort. Tu n'a pas été mal reçu; tu n'as pas été reçu du
tout.»

«Voilà ce qui te trompe: j'ai été mieux que reçu, j'ai été invité; mais
je n'y suis pas allé.»

«A d'autres! C'est bien toi qui refuserais une invitation pareille!
Pourquoi ne me dis-tu pas qu'un habitant du purgatoire a refusé d'entrer
au paradis? Avoue franchement que tu as trouvé la porte fermée. Tu n'es
pas le seul. Il y a peu d'élus.»

En ce moment, l'orchestre essayait les premières mesures de la Dernière
Pensée de Weber. Manuel n'eut que le temps de dire à Philippe: «Viens
demain à deux heures au palais Feraldi, tu m'y trouveras.» Et il courut
valser avec Tolla.

La première fois qu'elle s'arrêta pour se reposer, il lui dit: «Je n'ai
pas osé porter à ma dame votre mère les remerciements que je lui dois.»

Tolla aurait voulu pouvoir arrêter son cœur, qui bondissait: elle devina
que sa poitrine devait avoir ces mouvements qu'on simule au théâtre pour
indiquer une émotion violente, et elle en fut honteuse. Elle répondit:
«J'avais parlé à ma mère de l'honneur que vous vouliez nous faire; mais
en voyant que vous ne veniez pas, j'ai cru que vous aviez oublié ce que
vous m'aviez dit.» Manuel répliqua vivement:

«Je puis donc venir? Votre mère me le permet?

«Et pourquoi vous le défendrait-elle? Elle vous recevra avec le plus
grand plaisir.»

«Ainsi demain, dans la journée, je pour rais?\ldots»

«Demain, si vous voulez.»

Le lendemain, Tolla et sa mère reçurent cette visite tant désirée. Le
premier abord fut froid et embarrassé. Lorsqu'on rencontre à deux heures
de l'après-midi une personne qu'on n'a jamais vue qu'aux bougies, il
semble qu'on fasse une nouvelle connaissance. M\textsuperscript{me}
Feraldi soutint un peu la conversation. On parla du choléra, qui, après
avoir ravagé le midi de la France, avait gagné l'Italie. L'arrivée de
Pippo ramena quelque gaieté; il conta les nouvelles de la ville et un
trait assez curieux de M\textsuperscript{me} Fratief. En sa qualité de
dame patronnesse d'une œuvre de bienfaisance, elle avait quêté des
vêtements pour ses pauvres. La princesse Prosperi lui avait donné, entre
autres choses, une pèlerine cardinale en pou de soie glacé. Or, en
traversant le Corso, la femme de chambre de la princesse prétendait
avoir reconnu cette pèlerine, déguisée par une large dentelle, sur les
épaules de Nadine.

Lello s'amusa beaucoup aux dépens de la générale, et rit de manière à
montrer ses dents. Quand ses yeux rencontraient ceux de Tolla, ils ne se
détournaient point, et ils parlaient assez haut. Tolla, de son côté,
laissa deviner qu'elle n'était point ingrate. D'amour, on ne dit pas un
mot, et quelques efforts que fît Pippo pour faire parler son ami, Manuel
sortit sans s'être déclaré.

Il prit l'habitude de venir dans la maison; bientôt même il fit ses
visites le soir, comme les amis intimes. Il se tenait toujours sur là
défensive; mais l'amour le gagnait insensiblement, grâce au vide de son
esprit et à l'oisiveté de sa vie. Ses habitudes étaient celles de tous
les jeunes Romains de distinction. Il se levait à huit heures, restait
dans sa chambre à prendre le chocolat, à faire sa toilette et à ne rien
faire, jusqu'à onze heures. A onze heures, il entendait la messe; à
midi, il s'établissait dans le cabinet de son père jusqu'à deux heures.
Il dînait à fond, puis rentrait chez lui pour faire la sieste, si
toutefois il n'aimait mieux aller s'installer dans la boutique du
tailleur, rendez-vous des jeunes gens à la mode et centre du mouvement
intellectuel. A cinq heures et demie, il montait à cheval et faisait un
temps de galop jusqu'à la villa Borghèse. A sept heures, il commençait
une petite promenade à pied, le cigare à la bouche; il faisait acte de
présence au cabinet de lecture et au café. A huit heures, il venait
retrouver son père, réciter le chapelet en famille et lire à haute voix
une méditation. A neuf heures, il s'habillait, faisait une courte visite
à Tolla, et se montrait dans le monde. A onze heures, il soupait; à
minuit, il se reposait des fatigues de la journée et prenait des forces
pour le lendemain.

Après deux mois de visites assidues, Lello était plus épris que jamais,
mais il ne s'était pas expliqué sur ses intentions. On touchait à
l'époque où le comte avait l'habitude de partir pour Capri. Les progrès
du choléra, les cordons sanitaires et les difficultés du voyage
l'empêchèrent de partir. Il décida que ses vendanges se feraient sans
lui, et que la famille entière se réfugierait à Lariccia le surlendemain
de l'Assomption. Cette résolution fut arrêtée le 1\textsuperscript{er}
août. Les parents de Tolla auraient voulu savoir avant de partir ce
qu'ils pouvaient attendre de Lello. Ils souffraient, à la fin, d'une si
longue incertitude, et la comtesse avait surpris quelques larmes dans
les yeux de sa fille. D'ailleurs M\textsuperscript{me} Fratief avait
fait suivre Coromila par François, et elle allait répétant partout que
M\textsuperscript{lle} Feraldi recevait des visites clandestines. Enfin
le frère de la comtesse avait écrit d'Ancône pour annoncer que son jeune
prétendant perdait patience, et demandait un oui où un non.

On tint en l'absence de Tolla un conseil de famille où Toto fut admis.
Toto était un jeune homme rempli de prudence et de réflexion. C'était
lui qui avait dissuadé ses parents de rompre dès le mois de mai avec le
jeune homme d'Ancône. Lorsqu'on chercha en commun le meilleur moyen de
forcer Manuel à prendre un parti, M. Feraldi proposa de lui parler
lui-même, et de le prier de suspendre ses visites où de les expliquer.
Toto rejeta vivement cette proposition: elle avait un caractère
comminatoire qui pouvait effaroucher Lello. La comtesse voulut se
charger de sonder le terrain: son fils repoussa cet expédient, qui
sentait l'intrigue et pourrait éveiller la défiance.

«Il faut,» dit-il, «que ce soit Tolla qui le force à se prononcer.»

«Elle n'y consentira jamais,» dit le comte.

«Elle a trop de dignité,» ajouta la comtesse.

«Sans doute,» reprit Toto, «si nous lui proposions d'entrer dans un
petit complot dont le but est son bonheur, elle nous renverrait bien
loin; mais forçons-la de servir nos calculs sans les connaître: elle ne
travaillera bien que si elle n'est pas dans le secret.» Là-dessus, il
exposa son plan, qui fut adopté sans discussion.

Une heure après, M\textsuperscript{me} Feraldi fit voir à Tolla la
lettre de son oncle d'Ancône. Elle lui rappela qu'on avait consenti à
suspendre les négociations d'un mariage fort avantageux dès qu'elle
avait avoué son amour pour Coromila; qu'on avait perdu du temps et
encouru le blâme de plus d'une personne en recevant tous les jours celui
dont elle se croyait aimée; qu'après deux mois de cette périlleuse
expérience, on ne savait pas encore si Lello songeait à demander sa
main; que si telle était son intention, il en aurait déjà parlé à coup
sûr, sinon à la comtesse, du moins à sa fille; que, puis-qu'il n'en
avait rien dit, il y aurait de la folie à repousser un mariage
magnifique sans avoir même pour consolation la certitude d'être aimée.

«Ses yeux me l'ont assez dit,» interrompit Tolla.

Sa mère lui remontra doucement que tous les regards du monde ne valent
pas une parole, que cet échange de regards pouvait la mener loin,
qu'elle aurait vingt ans au 1\textsuperscript{er} septembre; que si elle
perdait une année où deux à se laisser regarder tendrement par Coromila,
sa réputation en souffrirait; qu'elle deviendrait difficile à marier et
peut-être malheureuse pour toute sa vie. La perspective de cet avenir
imaginaire émut en passant la bonne comtesse, qui versa de vraies
larmes. Il n'en fallut pas davantage pour persuader à Tolla que ses
parents souffraient cruellement du doute où elle les laissait plongés.
Elle pleura à son tour, et elle écouta avec résignation l'ultimatum de
sa mère.

«Mon enfant, il faut en finir,» lui dit la comtesse. «Tu es libre
d'accepter où de repousser le parti que ton oncle nous propose; mais
nous ne pouvons pas en conscience prolonger indéfiniment l'incertitude
d'un galant homme qui a demandé ta main. Nous partirons le 17 pour
Lariccia; prends jusqu'au courrier du 16 pour te décider. Réfléchis,
pèse, examine: ton avenir ne dépend que de toi-même, car je ne pense pas
qu'en quinze jours M. Coromila prenne une détermination.»

Le dernier mot était la flèche du Parthe.

Tolla fit tout au monde pour que son amant fût informé de sa situation.
Lorsqu'il la connut, il ne se départit point de sa réserve accoutumée.
Un soir, M\textsuperscript{me} Feraldi leur fournit l'occasion de
s'entretenir longtemps ensemble. Lello ne s'occupa qu'à démontrer que si
jamais il aimait, il serait le plus constant des hommes.

«Cependant,» remarqua Tolla, «on en cite plus d'une que vous avez
oubliée.»

«Moi! Je me fais fort de vous prouver en dix minutes que si j'ai oublié
telle et telle personne, la faute en est tout entière à leur
coquetterie, et je n'ai fait que suivre l'exemple qu'elles m'avaient
donné.»

«Quoi! votre passion de la place du Peuple?\ldots»

«C'est elle qui m'a congédié.»

«Et vos amours de la place de Venise?»

«Fallait-il rester fidèle à une personne qui me recevait tous les matins
et qui écrivait tous les soirs à un autre?»

«Soit; mais celle qui vient de partir pour Frascati?»

«Oui, parlons un peu de l'habitante de Frascati! une comédienne du plus
grand talent, qui serrait la main de son voisin de droite tandis qu'elle
me disait à l'oreille: ‹Je te serai fidèle!› D'ailleurs j'espère que
vous me ferez l'honneur de ne pas donner le nom de passion à ces
caprices dont le plus long a duré un mois. Quand j'aimerai, je le sens,
ce sera pour la vie.»

Tolla ne répliqua rien. Elle baissait la tête et semblait tristement
préoccupée.

«Qu'avez-vous? demanda Lello.»

Elle répondit qu'elle était triste parce qu'on voulait son consentement
pour décider son mariage avec le comte Morandi, d'Ancône. «Nous partons
mercredi pour Lariccia, et l'on me demande un oui où un non pour mardi.
Je ne peux me décider à dire oui. Je vois bien cependant que la raison
me défend de refuser un parti si avantageux. Il y a longtemps que je
diffère cette réponse de jour en jour. Mes parents perdent patience, ma
mère pleure, mon frère me presse. Tous les jours de poste il faut que je
livre une bataille, que j'entende des reproches, que je voie des larmes:
je n'en puis plus, et je suis au désespoir.»

Elle attendait avec anxiété la réponse de Lello. Il était assis devant
elle. La pauvre fille avait les yeux baissés, sans oser regarder celui
qui tenait sa vie dans ses mains.

«Quel jour avons-nous aujourd'hui? demanda-t-il d'un ton cavalier.»

«Vendredi.»

«Eh bien! vous n'avez plus à souffrir que pour deux courriers. Moi, je
n'épouserais jamais une personne qui n'aurait pas mon La pauvre cœur.»

Tolla trouva juste la force de répondre d'une voix étouffée: «Ni moi non
plus, si j'étais libre de suivre mes sentiments.» L'entrée de la
comtesse lui permit de cacher ses larmes. Manuel prit congé sans rien
voir, et sortit d'un pas délibéré. De sa vie il n'avait été plus
irrésolu.

Du latin triduum («trois jours»).

Tolla resta désespérée. Pour la première fois depuis deux mois, elle
douta sérieusement de l'amour de Lello. Dans sa douleur, elle se souvint
de demander assistance à saint Joseph, pour qui sa dévotion ne s'était
jamais refroidie. Elle commença dès le lendemain un \emph{triduo},
c'est-à-dire un tiers de neuvaine, suppliant son bon vieux saint de lui
apprendre à quel mari Dieu la destinait. «Si dans trois jours,» se
dit-elle, «Lello n'a pas parlé, c'est que le ciel me condamnera à
épouser l'autre.» Sa mère lui permit de passer la plus grande partie de
ces trois jours à l'église, dans la compagnie d'une vieille tante, et
Dieu sait si elle pria du fond du cœur.

Ses parents la laissaient faire, mais ils n'espéraient plus rien. Ils
croyaient fermement que tout finirait par une bonne lettre à Ancône.
Personne ne pouvait croire que Manuel saurait se décider dans ces trois
jours, lorsque la peur de la perdre et la douleur qu'elle avait laissé
voir ne lui avaient pas arraché une parole.

«C'était un beau rêve,» dit le comte; «mais nous voilà réveillés. Il
épousera la princesse que ses parents lui destinent.»

«Pourvu que Tolla ne tombe pas malade!» soupira la comtesse.

«Tout n'est pas perdu,» dit Toto. «C'est demain dimanche. Philippe
Trasimeni ne sera pas de service: invitez-le à passer la soirée avec
nous.»

Philippe savait que Lello venait tous les jours au palais Feraldi, et il
le croyait engagé envers Tolla. Il fut grandement surpris lorsque Toto
lui dit devant la famille assemblée: «Toi qui as passé l'été dernier à
Ancône, tu dois connaître Morandi. Conte-nous tout ce que tu en sais,
car il va probablement épouser ma sœur.»

Le pauvre Pippo tombait des nues. Il commença l'éloge de Morandi, qu'il
connaissait pour un galant homme, d'une excellente famille de patriotes
italiens; mais il était tellement abasourdi, qu'il n'entendait pas ses
propres paroles. Tolla, pâle et tremblante, les entendait encore bien
moins. Lello entra. Philippe, plus troublé que jamais, sortit comme un
fou, courut chez lui, monta à cheval, et fit quatre lieues au galop pour
remettre un peu d'ordre dans ses idées.

Manuel devina à l'émotion de Tolla que la conversation qu'il avait
interrompue ne lui était pas agréable. Il n'osa questionner personne,
mais il sortit au bout d'un quart d'heure et courut à la poursuite de
Pippo. Il le chercha toute la soirée sans le rejoindre, et pour de
bonnes raisons. Il rentra au palais Coromila, se mit au lit et passa la
première nuit blanche dont il ait gardé le souvenir. Le lundi, à six
heures du matin, il frappait à la porte de Philippe.

Le bon Philippe, tout en galopant sur la route d'Ostie, avait deviné une
partie de la vérité. Le trouble de Manuel et les premières questions
qu'il lui fit achevèrent de l'éclairer. Il comprit que Lello et Tolla
s'aimaient passionnément, mais que la timidité de l'une et
l'irrésolution de l'autre allaient peut-être les séparer pour toujours.
En conséquence son plan fut bientôt fait.

«Que veux-tu savoir?» demanda-t-il à son ami. «Quand Tolla épouse
Morandi? Bientôt assurément, car elle lui fera écrire demain qu'elle
l'accepte pour mari, et Morandi n'est pas assez sot pour faire attendre
la plus belle, la plus spirituelle et la meilleure fille qui soit au
monde. Morandi a du bonheur, et si je n'aimais Tolla comme un frère, je
donnerais dix ans de ma vie pour être à la place de Morandi. Quant à la
pauvre fille, je crois qu'elle donnerait sa place pour rien à celle qui
vou drait la prendre. Sais-tu qu'elle résiste depuis un mois à toute sa
famille? Mais le curieux de l'histoire, c'est qu'ils ont compté sur moi
pour lui arracher ce malheureux oui. Il paraît que sa résistance vient
d'une inclination qu'elle a prise pour quelqu'un que tu connais. Si tu
rencontres ce monsieur-là, prie-le, au nom de la comtesse et au nom du
bon sens, d'être désormais plus rare dans la maison Feraldi. Lors qu'on
ne veut pas le bonheur pour soi, il ne faut pas écorner la part des
autres.»

Tandis que Pippo parlait à Manuel, Tolla, levée au petit jour, priait
ardemment à l'église des Saints-Apôtres. C'était la fête de la madone et
le dernier jour de son \emph{triduo}.

Cosimo où le Marchand de Fer du Petit-Montrouge

En revenant de la messe, elle trouva sa cousine Agate et sa cousine
Philomène grands atours, qui l'embrassèrent comme à la tâche. Ces deux
excellentes Romaines étaient l'Héraclite et le Démocrite de leur sexe.
Agate avait le rire éclatant d'une trompette. Philomène se distinguait
de sa sœur par une sensibilité diluvienne. Elles étaient allées
l'avant-veille à l'amphithéâtre d'Auguste, où l'on joue en plein jour et
en plein air des drames et des vaudevilles. Philomène était encore tout
émue par le souvenir d'une pièce en sept actes intitulée: \emph{Cosimo o
commerciante il ferro del Piccolo Monte-Rosso}, qui faisait alors les
délices de Rome. Ayate, dans ce drame larmoyant, avait amplement trouvé
de quoi rire. Ni l'une ni l'autre ne regrettait les douze sous et demi
qu'elle avait payés pour sa chaise, et depuis deux jours elles
racontaient à toute la ville, l'une combien elle avait été heureuse de
rire, l'autre comme elle s'était régalée de pleurer. Elles commençaient
en duo le récit de leurs émotions contradictoires, lorsque Philippe
entra fort agité. Tolla bondit sur sa chaise, mais Agate la retint par
le bras.

«Figure-toi, ma chère, que le premier acte se passe devant un café, mais
un café si ressemblant, avec des tables vertes et des chaises de paille,
que c'est à mourir de rire. Un grand seigneur parisien entre dans ce
café du Piccolo Monte-Rosso pour prendre un verre d'eau de-vie. Il cause
avec le garçon et lui demande les nouvelles du quartier. Le garçon,
c'était Andréa, tu sais, Andréa qui est si drôle!»

«Alors,» poursuivit Philomène, «arrive un homme enveloppé dans un
manteau\ldots»

«En plein été, quoique les arbres soient couverts de feuilles?»

«Cet homme barbare a la férocité de dé poser cruellement par terre un
pauvre petit enfant nouveau-né dont les cris lamentables appellent en
vain sa malheureuse mère. Mais voici le digne Cosimo qui arrive avec sa
chère femme!»

«Et un melon\ldots»

«Pour respirer l'air frais de la campagne et prendre sa nourriture sur
l'herbe tendre.»

Pendant que Philomène s'apitoyait sur l'enfant abandonné recueilli par
Cosimo, la comtesse s'entretenait avec Pippo sur le balcon. Tolla aurait
donné ses deux cousines, seulement pour entrevoir la physionomie de sa
mère; mais la grosse personne d'Agate éclipsait totalement
M\textsuperscript{me} Feraldi.

«Au second acte, poursuivit Philomène, on voit un homme où plutôt un
tigre qui chasse de sa maison une malheureuse femme trop pauvre pour
payer son loyer. ‹Je pars,› lui dit-elle, ‹mais souviens-toi, cœur. de
fer, que celui qui chasse un pauvre de sa maison chasse la bénédiction
de Dieu.› Il faut voir comme on a applaudi la pauvre femme! on l'a
rapperlée douze fois.»

«Oui, et elle a ri au public, en faisant chaque fois une belle
révérence.»

«Mais quand l'homme cruel a défendu à ses domestiques de laisser mendier
les pauvres dans la cour de sa maison, tout le monde a crié en même
temps: ‹Ouh! ouh!› Si l'on avait eu des pierres, on lui en aurait jeté.
Au troisième acte, la pauvre femme vient tomber pâle et mourante-à la
porte de Cosimo. On lui apporte un petit verre d'eau-de--vie.»

«Il y a cinq petits verres d'eau-de-vie dans la pièce. Et un beau jeune
homme de vingt ans lui demande poliment si elle ne veut pas se reposer.
A sa vue, elle pousse un cri, et elle reconnaît l'enfant qu'on lui a
pris vingt ans auparavant pour l'exposer au Piccolo Monte-Rosso. Elle
l'embrasse\ldots»

«Pardon, elle ne l'embrasse pas. Le cardinal-vicaire ne permet pas que
les femmes embrassent les hommes sur le théâtre. Et puis, tu vas bien
rire, figure-toi, ma Tolla, qu'au moment où la vieille femme doit crier
au bon jeune homme: Tu es mon fils! toutes les cloches du voisinage se
sont mises à sonner en même temps, et comme le théâtre est en plein air,
et qu'il était impossible de s'entendre, la vieille femme s'est assise,
le jeune homme a pris une chaise, et ils ont causé en riant jusqu'à ce
que les cloches eussent fini.»

«Oui, mais quel beau moment, lorsqu'à la fin du septième acte Cosimo
s'est avancé sur les bords de la scène, et qu'il a dit au public: ‹Ceci
vous prouve qu'il y a un Dieu qui punit les coupables et récompense les
innocents!› Quels applaudissements! quelles larmes! Pour moi, j'en suis
encore bouleversée.»

Le supplice de Tolla ne dura pas plus d'une heure. Lorsque les deux
cousines se retirèrent, l'une en s'essuyant les yeux, l'autre en se
tenant les côtes, elle courut au balcon: Pippo était parti sans passer
par le salon. M\textsuperscript{me} Feraldi, assise sur le bord d'une
caisse de fleurs, paraissait enfoncée dans une réflexion profonde.

«Eh bien! mère?» murmura Tolla du ton dont un condamné demande des
nouvelles de son recours en grâce.

«Philippe vient de sa part. Il demande ta main.»

Tolla chancela et s'appuya à la muraille. Elle avait le vertige. Sa mère
la soutint et la ramena dans le salon.

«Écoute,» lui dit-elle. «Il a beaucoup pleuré devant Pippo; il t'aime,
et tu seras sa femme; mais il ne peut, quant à présent, que donner. sa
parole de t'épouser. Son frère aîné s'est amouraché d'une petite
Vénitienne, en dépit du prince, du cardinal et du chevalier. Cette
affaire a soulevé de grands orages dans la famille, et tant qu'elle ne
sera pas terminée, Lello ne veut point parler de son mariage: il exige
même que la parole qu'il nous donne aujourd'hui demeure un secret pour
quelque temps. Je me contenterai volontiers de sa promesse: il n'y
manquera pas, j'en suis sûre. Si tu veux t'en contenter comme moi, et si
tu consens à tenir la chose secrète, nous pourrons écrire à Ancône. Ton
oncle répondra à Morandi que tu ne peux pas l'épouser, qu'il te
coûterait trop de quitter Rome et d'aller vivre si loin de nous.»

Tolla resta muette de joie. Tout ce qu'elle avait compris dans le
discours de sa mère, c'est qu'elle était aimée et qu'elle serait la
femme de Lello. L'horizon s'éclaira vivement autour d'elle: les objets
les plus sombres prirent des couleurs éclatantes: elle éprouvait
l'éblouissement du bonheur. Elle saisit sa mère dans ses bras et
l'accabla de caresses. En ce moment, Menico ouvrait timidement la porte:
elle courut à lui et lui sauta au cou.

Menico avait rencontré le Napolitain de la Fratief, qui rôdait autour du
palais, et il avait engagé avec lui une conversation où il s'était foulé
le poignet droit. Il allait demander à M\textsuperscript{me} Feraldi une
compresse d'eau-de-vie camphrée, lorsque le plus mignon, le plus frais
et le plus brûlant de tous les baisers vint s'abattre au milieu de son
visage.

«Mon cher Menico!» lui cria-t-elle, «mon frère nourricier! que tu es
bon! que tu es beau! Je t'aime! je suis heureuse!»

«Moi aussi, mademoiselle,» hurla Menico en sanglotant, «je suis bien
heureux, vous m'avez embrassé; c'est la première fois depuis 1830.
J'avais le poignet foulé, mais maintenant je n'ai plus mal. Ma bonne
demoiselle! vous aimez donc quelqu'un, puisque vous m'embrassez?»

«Oui, j'aime, je suis aimée, je me marie... bientôt; pas tout de suite,
entends-tu? C'est un secret, ne le dis à personne, mais bientôt\ldots{}
Tu seras de la noce, mon Menico; nous nous marierons à Lariccia; tes
buffles auront congé ce jour-là. Je veux que nous dansions ensemble!»

«Menico savait fort bien avec qui se mariait Tolla. Depuis quinze jours,
il partageait les angoisses de sa chère maîtresse. Cependant il se
souvint de jouer l'ignorance, et il ne prononça pas le nom de Coromila.
Dans l'excès de sa joie, cet homme inculte ne se départit pas un instant
de la réserve et de la prudence italienne; mais tandis que la comtesse
prenait soin de son poignet enflé, il se promit de commencer une
neuvaine à l'intention de ce mariage et de veiller comme un dogue au
salut de Lello.»

Lello vint à neuf heures du soir. Il eut une assez longue conférence
avec le comte et la comtesse, à qui il demanda solennellement la main de
leur fille. M. Feraldi lui fit observer qu'il ne pouvait pas se marier
sans le consentement de ses parents. «Je le sais,» répondit-il, «et
quand la loi me le permettrait, je ne le voudrais pas; mais ce
consentement, je prends sur moi de l'obtenir, et je vous prie de ne vous
en point mettre en peine.» A cette assurance formelle, le comte ne
répondit rien: il savait d'ailleurs que le vieux Luigi Coromila était
condamné unanimement par les médecins, et que Lello serait libre avant
une année. Cependant, pour plus de prudence, et de peur que la question
de la dot n'indisposât la famille de Lello contre ce mariage, le comte,
sur le conseil de son fils, doubla la somme qu'il destinait à Tolla, et
lui assura la propriété de ses vignes de Capri, estimées deux cent mille
francs. Lorsque tout fut conclu, on appela Tolla. Elle reçut enfin de la
bouche de Lello l'assurance de son amour. Elle mit sa main dans la
sienne et le baisa sur les lèvres. Ils étaient fiancés.
