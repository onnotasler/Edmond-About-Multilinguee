\addchap{\RomanNumeralCaps 4.}

M\textsuperscript{me} Fratief et sa fille ignorèrent ce qui s'était
passé au palais Feraldi. Nadine, prévoyant que le départ pour Lariccia
précipiterait la marche des événements, avait aposté Cocomero sur la
place des Saints-Apôtres, pour surveiller le camp ennemi. Elle poussa un
cri de colère lorsqu'elle vit revenir son espion sur un brancard, la
figure en sang et le crâne sensiblement déformé. L'état de son visage
expliquait la foulure de Dominique.

Cocomero était un pur Napolitain du quai de Sainte-Lucie, court, trapu,
rougeaud, goulu, fainéant, poltron, hébété et fripon comme Polichinelle
en personne. Sa grosse face plate, élargie par une énorme paire de
favoris roux, était toute barbouillée de mauvaises passions; ses petits
yeux gris clair trahissaient à certains moments une férocité porcine.
Depuis la place des Saints-Apôtres jusqu'à la via Frattina, où logeaient
ses maîtresses, il répéta entre ses dents la plus terrible malédiction
que l'on connaisse à Rome: \emph{«Accidente!»} ce qui veut dire en bon
français: «Puisses-tu mourir d'accident, sans confession, damné!» Dans
un pays où l'on croit au mauvais œil comme à la sainte Trinité, une
malédiction de cette importance équivaut à mille soufflets, et les
Romains du Transtévère répondent à un accidente par un coup de couteau;
mais Dominique était loin, et Cocomero sacrait tout à son aise, sans
aucun respect pour la police ecclésiastique de Rome, qui fait coller aux
portes de toutes les boutiques un petit écriteau avec ces mots:
\emph{«Blasphémateurs, souvenez-vous que Dieu vous entend!»}

La générale, après quelques exclamations modérées, qu'on entendit d'une
lieue à la ronde, s'empressa de soigner son domestique. Elle avait
appris un peu de médecine, pour faire croire qu'elle était née dans un
château, et elle traînait partout avec elle un gros cahier manuscrit,
plein de recettes, de secrets merveilleux, de remèdes de famille, de
\emph{gouttes} infaillibles, et même de paroles magiques. La pièce la
plus remarquable de ce recueil était une certaine recette pour purifier
le sang, en coupant les quatre pattes d'un lézard vert pendant la pleine
lune, et en prenant une \emph{purge} le lendemain. Cocomero se laissa
soigner sans mot dire, et il s'ingéra une bonne dose de certain
vulnéraire de ménage dont la saveur alcoolique lui agréait fort; mais il
se refusa obstinément à nommer l'auteur de ses maux. «C'est moi,»
disait-il, «qui me suis fait mal. J'ai trébuché sur une pierre; ma tête
a donné contre une borne; je suis un maladroit, mais je ne suis pas un
poltron.» Il ajouta sournoisement: «Si un homme m'avait fait autant de
mal que je viens de m'en faire moi-même, il ne s'en vanterait pas
longtemps, fût-il aussi fort que Néron!» Néron est encore le héros
favori du petit peuple de Rome et de Naples.

«Tais-toi! dit la générale. Et la justice?»

«La justice, madame? On ne me condamnerait pas sans témoins, n'est-il
pas vrai?»

«Sans doute.»

«Eh bien! il n'est pas facile de trouver des témoins contre un homme qui
a donné un coup de couteau. Les témoins sont personnes prudentes qui se
disent: ‹Celui-là n'a pas peur. Il a tué un homme; donc il est capable
d'en tuer deux: ne nous brouillons pas avec lui.›»

«Oui, mais un condamné à mort ne se venge pas de ses témoins.»

«Mais, reprit Cocomero d'un petit air dévot, le saint père est galant
homme; il ne veut pas la mort du pécheur; il répugne à verser le sang
chrétien, et ceux qui ont commis l'imprudence de tuer un homme en sont
quittes pour les galères à perpétuité.»

«A perpétuité! N'est-ce pas pire que la mort?»

«Faites excuse, madame. Lorsqu'on a quelque protection, un bon maître,
par exemple, ou une bonne maîtresse, on peut espérer pour les prochaines
fêtes de Pâques une commutation de peine: vingt ans de fers. C'est
encore bien sévère, n'est-il pas vrai, madame? Mais, au bout d'un an ou
de six mois, la même protection agissant toujours, les vingt ans seront
réduits à dix, les dix à cinq, les cinq à trois. Or le plaisir de tuer
un ennemi ne vaut-il pas bien trois ans de galères?»

«Et l'enfer, malheureux?» objecta Nadine.

«Et la confession, mademoiselle!» répondit Cocomero, un peu scandalisé.

C'est dans ces sentiments que le digne Napolitain se coucha le soir de
l'Assomption, tandis que ses maîtresses se dépitaient de ne rien savoir,
que Lello échangeait son premier baiser avec Tolla, et que Philippe
Trasimeni, enchanté du succès de sa négociation et du bonheur de ses
amis, courait raconter toute l'histoire à sa mère.

La marquise était loin de s'attendre à semblable nouvelle. Il y avait
trois mois et demi que la rumeur publique lui avait appris la passion de
Lello, et elle ne croyait pas qu'un Coromila fût capable d'aimer si
longtemps. De puis cet éclat, les deux amants, soumis à un espionnage
formidable, s'étaient étudiés à tromper tous les yeux; le comte et la
comtesse, craignant le ridicule qui s'attache aux ambitions déçues,
avaient caché leur projet à leurs meilleurs amis, et Pippo, qui
connaissait l'antipathie de sa mère pour les Coromila, n'avait voulu lui
raconter sa campagne qu'après la victoire. D'ailleurs la marquise avait
cessé d'aller dans le monde depuis l'invasion du choléra. Elle s'était
liguée contre le fléau avec le docteur Ely et l'abbé Fortunati. Le
docteur avait fait le voyage de Paris en 1832 pour observer l'effet
des-divers traitements qui y furent essayés; l'abbé enrôla parmi les
fidèles de sa paroisse et les admirateurs de son éloquence une vingtaine
d'infirmiers volontaires; la marquise dé pensa trente mille francs,
toutes ses économies, pour transformer en hôpital une maison qui lui
appartenait. Tous ces soins s'emparèrent si bien de son esprit, qu'elle
n'eut plus le loisir de songer à autre chose, et elle avait presque
oublié qu'il y eût des mariages en ce monde, lorsque son fils vint lui
annoncer triomphalement qu'il mariait Lello à Tolla.

Pour un marquis et pour un garde-noble, Pippo avait l'esprit un peu bien
libéral. Il prisait médiocrement les avantages de la naissance et de la
fortune, sous prétexte qu'il était riche et noble depuis sa plus tendre
enfance, et il prétendait que les seules gens qui fassent cas des titres
et de la richesse sont ceux qui ont pris la peine d'acheter leurs titres
et de gagner leur argent. S'il méprisait toutes les distinctions
sociales, en revanche il estimait fort la noblesse des sentiments, et il
s'amusait quelquefois, au grand scandale de ses camarades, à bouleverser
l'ordre hiérarchique de l'aristocratie romaine, donnant la couronne
fermée à ceux qui pensaient en princes, et reléguant dans la bourgeoisie
tout prince convaincu de penser en bourgeois. Sur le livre d'or de
Pippo, Tolla Feraldi était inscrite parmi les reines, Lello parmi les
princes: Dominique, le piqueur de buffles, n'était rien moins que le
chevalier Meņico. On devine aisément que l'inventeur de ce beau système
n'était pas un chaud partisan des mariages à la mode, et qu'il
n'admirait guère cette loi des convenances qui veut qu'un prince épouse
une princesse, et qu'un millionnaire épouse un million.

«Victoire!» cria-t-il à sa mère; «Rome se convertit à mes idées. Une
grande famille va donner l'exemple: la foule suivra. Tu sais que
l'héritier présomptif du prince Coromila-Borghi est à Venise, aux pieds
d'une adorable petite bourgeoise qu'il jure d'épouser à la barbe de ses
ancêtres. Eh bien! ce n'est pas tout; son frère cadet, notre Lello,
qu'ils voulaient marier à une princesse, a demandé aujourd'hui même la
main de Tolla.»

La marquise écouta avec une douleur sourde la narration détaillée que
lui fit Pippo. Une ou deux fois elle fut sur le point d'interrompre un
récit dont chaque mot éveillait en elle de douloureux souvenirs;
cependant elle se contint jusqu'au bout. Lorsque son fils, après avoir
tout dit, lui demanda ses applaudissements, elle secoua tristement la
tête.

«Pauvre Tolla! Pourquoi as-tu mis son bonheur aux prises avec l'orgueil
des Coromila?»

«L'orgueil des Coromila se fait vieux. Le père n'a pas six mois à vivre;
le cardinal est condamné par tous les médecins: reste le chevalier.»

La marquise se leva pour aller regarder å la fenêtre. Philippe
poursuivit:

«Le chevalier ne m'inquiète nullement.»

«Ah!»

«Nullement. Il appartient à l'espèce d'hommes la plus inoffensive: c'est
un égoïste. Y a-t-il rien de plus aimable qu'un homme qui ne s'occupe
jamais des autres? Je ne voudrais pas lui ressembler: non, l'égoïsme est
une vertu sociale dont je ne suis point jaloux; mais quoi que je voie
plus d'une personne (et tu es du nombre) prévenue contre le chevalier,
je me déclare incapable de le craindre ou de le haïr. Je l'ai rencontré
ce matin; il fumait son cigare au sortir de la messe, et suivait tout
doucement le Corso, en poussant son ventre devant lui. Ses gros yeux
indifférents erraient au hasard de balcon en balcon, de voiture en
voiture; il semblait se soucier de la gloire des Coromila comme de la
fumée qu'il abandonnait au vent. S'il pensait sérieusement à quelque
chose, c'était assurément au déjeuner qu'il avait fait ou au dîner qu'il
allait faire. Il avait l'air d'un homme de bon sens et de bon appétit,
qui n'a point de remords et qui n'aurait garde de s'en préparer, de peur
de mal dormir. Je l'ai regardé marcher, d'un pas pesant et satisfait,
jusqu'au palais de ses pères, et j'ai crié (en moi-même): ‹Vivent les
égoïstes!› Ce gros homme ne prendra jamais la peine de contrecarrer ma
petite providence! Est-ce bravement raisonné cela? Embrasse-moi, et
adieu; je suis de service ce soir.»

Il embrassa tendrement sa mère, pirouetta sur ses talons, et courut
mettre son uniforme. La marquise se demanda longtemps si elle irait voir
M\textsuperscript{me} Feraldi. Elle croyait connaître assez la famille
Coromila pour pouvoir prédire que le mariage ne se ferait jamais, et son
amitié pour Tolla lui commandait de la détromper. D'un autre côté, le
soin qu'on avait pris de se cacher d'elle, la crainte de paraître
malveillante ou jalouse, et surtout la perspective du récit douloureux
par lequel il faudrait appuyer son opinion, la firent hésiter jusqu'au
soir. A la fin, le dévouement prit le dessus. «Je leur raconterai tout,»
pensa-t-elle. «De cette façon mes souffrances n'auront pas été stériles,
et le malheur de ma vie sera le salut de Tolla.»

Elle se présenta à dix heures au palais Feraldi. Menico, le bras en
écharpe, lui répondit que la comtesse n'était pas rentrée: Lello n'était
pas encore parti. Elle revint le lendemain dans la matinée. Cette fois,
M\textsuperscript{me} Feraldi et sa fille étaient véritablement sorties
pour entendre une messe d'actions de grâces à la Trinité-des-Monts. La
marquise alla voir ses malades, et se consulta, chemin faisant, pour
savoir si elle n'écrirait pas à M\textsuperscript{me} Feraldi; mais il
lui répugnait de confier au papier le secret qu'elle n'avait encore
partagé qu'avec son confesseur. Elle rencontra fort à point l'abbé
Fortunati, et lui demanda son avis. L'abbé était un orateur et un homme
d'action, mais une âme scrupuleuse et timorée, peu capable de donner un
conseil. Il lui répondit d'agir suivant sa conscience et de s'en
remettre à la bonté de Dieu. La pauvre femme, livrée à elle-même,
n'imagina qu'un seul expédient pour sortir d'incertitude. Elle résolut
de retourner le soir au palais Feraldi pour parler à la comtesse. «Si je
trouve encore la porte fermée,» se dit-elle, «c'est que le ciel ne
voudra pas que je les avertisse. Qui sait si Lello n'aura pas assez
d'amour et de persévéré rance pour surmonter tous les obstacles que je
prévois?»

En rentrant chez elle, elle trouva la carte de la comtesse avec le mot
adieu écrit au crayon. A neuf heures du soir, elle vit les portes du
palais fermées; aucune des fenêtres qui donnent sur la place n'était
éclairée. Le portier lui annonça que toute la famille partait le
lendemain au petit jour pour Lariccia, et qu'on venait de se mettre au
lit. Elle retournait à la maison, lorsqu'elle reconnut dans l'obscurité
le beau Lello, courant comme s'il avait des ailes. Il entra dans le
palais, et au bout de dix minutes il n'était pas sorti. «Allons, pensa
la marquise,» «c'est sans doute la volonté de Dieu!»

Cette soirée fut pour les deux amants la fête de l'amour permis. Lello
trouva la famille réunie au jardin, sous les citronniers, autour d'une
table antique où l'on avait servi des sorbets à la rose. Le ciel était
sans nuages, et la lune répandait sur les larges allées sa chaste et
honnête lumière. La brise du sud, humide et tiède, remuait mollement le
feuillage, et animait tout le jardin d'une vie douce et indolente. Tous
les bruits du dehors s'étaient apaisés, et la petite cloche d'un couvent
voisin interrompait seule d'heure en heure cet épais silence qui pèse
sur les nuits de Rome. Tous les domestiques, Menico excepté, dormaient
sur une terrasse; les oiseaux, bercés par la brise, dormaient sur les
branches; les bas-reliefs en cadrés dans la façade du palais, les
statues du péristyle et les Hermès du jardin semblaient fermer les yeux.
Lello s'arrêta sur les marches du palais, et chanta d'une voix pure et
sonore le premier couplet d'une romance que Philippe avait écrite pour
lui:

\begin{quote}
Le ciel est bleu, la mer tranquille;\\
Les Romains couchés par la ville,\\
La tête au pied d’un mur, dorment profondément;\\
Et la brise du soir, sur les jardins errante,\\
Porte des orangers la senteur enivrante\\
u cœur de ton amant.
\end{quote}

Tolla se leva précipitamment, et courut se jeter dans ses bras. Elle le
conduisit à ses parents en voltigeant autour de lui, comme une ombre
légère, dans son peignoir de mousseline blanche. En présence du comte,
de la comtesse et de Toto, Manuel lui mit au doigt son anneau de
fiancée. C'était un petit cercle d'or entouré de turquoises, qu'il avait
commandé le matin même dans la via Condotti à l'un de ces artistes en
boutique qui sont les premiers bijoutiers du monde. Il prit la main de
Tolla, comme pour juger de l'effet de son petit présent, et il la baisa
longuement. Tolla, par un mouvement de naïveté sauvage qui fit un peu
rougir sa mère, reprit vivement sur sa main le baiser qu'il y avait mis.
Toute la soirée se passa dans ces enfantillages qui sont peut-être les
plaisirs les plus vifs de l'amour. Les parents de Tolla, témoins muets,
mais non pas indifférents, de cette scène charmante, ne songeaient point
à contraindre les sentiments de leur fille: ils voulaient attacher
Lello, et ils savaient que rien n'attache comme le bonheur. Les deux
enfants couraient en liberté dans les allées, ou s'arrêtaient pour
écouter le silence, ou marchaient lentement, appuyés l'un sur l'autre,
en babillant comme deux pinsons sur la même branche par un beau jour de
printemps. Ils se racontèrent plus de vingt fois, sans se lasser ni l'un
ni l'autre, les commencements de leur amour et l'histoire de leurs cœurs
pendant les six mois qui venaient de s'écouler.

Les projets vinrent ensuite, et Dieu sait com bien de châteaux en
Espagne ils construisirent et renversèrent pour avoir le plaisir de les
re bâtir.

«Nous passerons tous nos hivers à Venise,» disait Lello. «Je n'y connais
personne; nous ne serons pas condamnés à aller dans le monde. Nous
vivrons pour nous, cachés dans mon vieux palais, que je veux faire
rajeunir.»

«Non,» répondait Tolla, «il faut le laisser comme il est. Les murs
sont-ils bien noirs? Aussi noirs et aussi curieusement fouillés qu'une
dentelle de Chantilly.

Tant mieux, je ne veux pas qu'on y touche. Ma chambre a-t-elle des
vitraux coloriés comme une chapelle? Est-elle tendue de cuir gaufré et
doré? Je l'aime comme elle est. Ai-je un grand lit d'ébène à colonnes
torses avec des rideaux de damas du temps de Véronèse? Il faut les
laisser. Je ne veux pas qu'on cache sous un tapis le pavé de mosaïque.

Il faudra pourtant bien un tapis pour les enfants. Comment
pourraient-ils se rouler sur ces dures mosaïques?

«Vous avez raison, mais je ne supporte pas un tapis neuf. Il faudra
trouver dans le garde-meuble quelque vieillerie splendide, un présent du
roi de France à notre aïeul le doge, ou un tapis de Smyrne rapporté par
notre ancêtre l'amiral. Ils me sauront gré du soin que je prends de
leurs reliques, et les vieux portraits de la galerie souriront en me
voyant passer.

Pour la promenade,» reprenait Lello, «je ferai faire une grande gondole
noire aussi triste qu'un catafalque; mais l'intérieur sera garni de
satin blanc, comme le nid d'un cygne. Ceux qui nous verront glisser sur
le Grand-Canal nous prendront pour des officiers autrichiens qui vont
commander l'exercice; ils ne devineront pas le bonheur qui se cache sous
cette tenture de deuil.»

«Il faudra que Menico apprenne à manier la rame vénitienne; je ne veux
pas qu'un valet étranger soit en tiers dans nos secrets d'amour.»

«L'été, nous habiterons notre villa d'Albano. Le parc est si grand, que
nous ferons notre promenade du matin, à cheval, sans sortir de chez
nous.»

«Non, votre parc est public, et nos regards seraient épiés par trop de
monde.»

«Je le fermerai.»

«Je vous le défends! Que deviendraient les pauvres gens qui ont pris
l'habitude de s'y promener comme des princes, et les petits paysans qui
viennent vous voler vos oranges? D'ailleurs je ne vois pas pourquoi je
serais toujours chez vous quand vous ne parlez pas de venir chez moi.
Nous passerons notre été à Lariccia.»

«Et le parc fermé, où le trouverons-nous?»

«Vous serez quitte pour faire entourer de murs le petit bois de quarante
arpents.»

«Vous oubliez que Lariccia n'est pas à Permettez-vous que j'appelle Toto
pour lui demander s'il veut nous donner Lariccia?»

«Eh bien! nous n'irons pas à Lariccia. Je vous emporterai dans l'île de
Tibère et la mienne, et vous habiterez, malgré vous, mon repaire de
Capri. Je parie que vous n'avez pas seulement vu Capri, ignorant que
vous êtes? Ah! c'est un beau pays. J'y suis allée une fois, quand
j'étais petite, et je m'en souviens comme d'hier. Lorsqu'on est dans le
golfe de Naples, on voit une belle montagne blanche, grise, rousse, de
toutes couleurs, debout au milieu de l'eau. Tous les rivages de l'île
paraissent droits comme des murs, et l'on cherche des yeux une échelle
de corde pour aborder; mais il y a une jolie petite marine où l'on
débarque sans danger au milieu des pêcheurs en caleçon blanc et en
bonnet rouge. Pour arriver à mes vignes et à mon château, il faut gravir
un escalier d'une lieue; mais vous avez de bonnes jambes, n'est-ce pas?
La maison est une tour carrée, blanche comme la neige, un toit en
terrasse et des fenêtres si étroites que le soleil n'ose pas entrer chez
nous. Les vignerons habitent à l'entour, dans des cabanes tapissées de
pampres roux et de raisins noirs. Nous avons deux grands palmiers devant
notre porte: leur ombre grêle se dessine en bleu sur les murs de la
maison. Quand j'étais enfant, je les prenais pour des géants, avec leurs
panaches. Vous verrez les mûriers que mon grand-père a plantés, et le
gros figuier qui est sous ma fenêtre, tout peuplé de nids de
tourterelles! Aimez-vous le vin de Capri? Non pas le rouge: il ressemble
trop à du vin; mais le blanc, qui exhale ce joli parfum de violette? On
en récolte beaucoup sur mes terres, et mon crû est le plus renommé de
tout le pays. La bonne vie, Lello! et comme nous serons heureux ensemble
sur notre rocher, loin de Rome et du monde entier, au milieu de nos
braves paysans. Ils nous aimeront: vous apporterez beaucoup d'argent
pour les faire riches; moi, je doterai toutes les filles sur mes
économies. Croyez-vous qu'une fois que nous serons là, vous avec moi,
moi avec vous, et nos enfants autour de nous, nous aurons le courage de
nous exiler à Venise pour tout un hiver? Venise doit être triste au mois
de novembre: il y pleut à torrents; les brouillards des lagunes me font
peur; on ne connaît pas les brouillards dans notre chère Capri!»

«Je t'aime, Tolla! nous resterons à Capri toute notre vie.»

«L'hiver et l'été, n'est-il pas vrai? Dieu me garde peut-être encore
quinze années de beauté: je ne veux être belle que pour toi.»

«Tu es un ange! Rome ne méritait pas de te connaître. Est-ce que la
ville entière ne devrait pas être à tes genoux? Je m'indigne quand je
pense qu'il y a des jeunes gens assez aveugles pour admirer une Bettina
Negri ou une Nadine Fratief. Et ces petites sottes qui ont pu espérer
qu'elles te voleraient mon cœur! Elles seront bien punies lorsqu'elles
nous verront passer au Corso dans la même voiture, ou galoper côte à
côte dans les avenues de la villa Borghèse, ou valser ensemble à
l'ambassade de France!»

«En ce temps-là, je ne serai plus obligée de baisser les yeux quand vous
paraîtrez dans un salon pour vous regarder à la dérobée. J'entrerai
fièrement, au bras de mon Lello, les yeux attachés sur ses yeux. C'est
ma mère qui sera heureuse de se montrer partout avec nous! Je ne ferai
pas plus de toilette qu'à présent; non, je ne veux pas avoir l'air d'une
parvenue. D'ailleurs le blanc me va bien, et puis je n'ai jamais aimé
les bijoux.»

«Les bijoux ne serviraient qu'à cacher quelque chose de votre beauté.
Vous n'en porterez jamais. J'excepte cependant les diamants de ma mère.
Elle m'a légué une rivière d'un grand prix, mais d'une admirable
simplicité. Ne voudrez-vous point porter ces pauvres diamants pour
l'amour de celle qui n'est plus?»

«Je ferai ce que vous voudrez, Lello. Vous serez mon maître, et vous
aurez le droit de me mettre un collier.»

«Nous irons à tous les bals, nous serons de toutes les fêtes;
j'inviterai Rome à venir dans notre palais assister à notre bonheur. Je
voudrais pouvoir vous montrer au monde entier. Nous voyagerons; nous
irons en France.»

«Quand vous aurez appris le français, mon bien-aimé paresseux! En
attendant, je vais voyager seule, demain matin, sur la route de
Lariccia.»

«Grâce à ce bienheureux choléra, que le ciel confonde!»

Tolla lui posa deux doigts sur la bouche:

«Chut! et point de paroles de mauvais augure! Promettez-moi seulement de
veiller sur vous, d'éviter soigneusement le danger, d'appeler le docteur
Ely au moindre symptôme, d'exécuter aveuglément ses ordonnances, en un
mot de conserver votre vie comme une chose qui m'appartient.»

«Ne craignez rien, Tolla; je suis sûr de né point mourir de cette
horrible maladie.»

«Sûr? Et pourquoi?»

«Parce que je mourrai d'amour et d'ennui le jour de votre départ.»

«Non, monsieur; le jour de mon départ, vous m'écrirez une longue lettre,
et vous n'aurez pas le temps de mourir.»

«Oui, certes, je vous écrirai, et par tous les courriers, c'est-à-dire
tous les deux jours. Longuement? C'est ce que je ne sais pas encore. Je
n'ai pas été jusqu'ici grand barbouilleur de papier, et je pense qu'en
amour un baiser en dit plus long qu'une lettre de quatre pages.»

«L'amour est un grand maître: il vous apprendra l'art d'écrire.
Souvenez-vous seulement que je vous répondrai avec une exactitude
judaïque: lettre contre lettre, et page pour page. Mais chut! on nous
appelle. Voyez donc quelle heure il est?»

Lello regarda sa montre et répondit avec stupéfaction: Minuit! Il
croyait causer depuis une demi-heure.

«Déjà!» dit tristement Tolla.

«Mais est-ce que vous avez envie de dormir?»

«Non? Et vous?»

«Moi! il me semble que nous sommes en plein midi, que le ciel est peuplé
de soleils, et que c'est offenser Dieu que de s'ailler coucher à l'heure
qu'il est.»

«Mais mon père et ma mère, qui n'ont ni vos vingt-deux ans ni votre
amour, ont besoin de quelques heures de repos. Adieu, Lello.»

Lello se pencha sur elle pour la baiser au front. Elle s'enfuit en lui
criant: Non, pas ici; devant ma mère!

Le comte, la comtesse et Toto embrassèrent Manuel Coromila, comme s'il
eût déjà fait partie de la famille. Tolla lui tendit les joues, puis
elle lui prit la tête dans ses deux mains, et l'embrassa à son tour.
Tout le monde le reconduisit à travers les appartements jusqu'à la porte
du palais.

«Adieu, frère,» lui dit Toto.

«Venez nous voir à Lariccia,» dit le comte.

«Soignez-vous bien,» ajouta la comtesse.

«Vivez pour que je vive,» murmura Tolla. En ce moment, on entendit un
sanglot qui semblait sortir d'un instrument de cuivre. Dominique, caché
derrière une colonne de marbre cipolin, prenait sa part des émotions de
la famille.
