\addchap{\RomanNumeralCaps 5.}

Le lendemain, à six heures du matin, l'heureux Lello dormait à poings
fermés, lorsque Tolla et ses parents s'embarquèrent dans une grande
chaise de poste qui faisait de temps immémorial le voyage de Lariccia.
La comtesse et Tolla occupaient le fond de la voiture, le comte et son
fils étaient fort à l'aise sur le devant; les domestiques pendaient en
grappes à l'entour. Le cuisinier, le marmiton et le palefrenier
s'accrochaient de leur mieux au siège du cocher; le camérier du comte,
Amarella et Menico s'empilaient sur le banc de derrière, et le soleil
oblique du matin chauffait vigoureusement tous ces visages hâlés.

M\textsuperscript{lle} Amarella était cette éternelle Romaine que tous
les peintres rapportent dans leurs cartons: grande, belle, large, lourde
et médiocrement faite, avec une physionomie fière et stupide qui ne
déparait point sa figure. Son vrai nom était Maria, mais elle devait à
son humeur aigrelette le sobriquet d'Amarella. Ses parents, pauvres
journaliers de Lariccia, lui avaient fait apprendre à coudre; mais
c'était elle qui s'était élevée d'elle-même à la dignité de femme de
chambre. La nature, qui s'amuse quelquefois à donner à une couturière
des qua lités d'homme d'état, l'avait douée d'une certaine ambition et
d'une remarquable persévérance. Ce qu'elle avait dépensé de ruse pour
entrer chez le comte et pour supplanter sa devancière passe toute
croyance. M\textsuperscript{me} Feraldi racontait avec admiration
comment Amarella, peu de temps après son entrée dans la maison, avait eu
envie d'un vieux châle en crêpe de Chine, autour duquel elle avait
tourné deux ans et demi, et qu'elle s'était fait donner à la fin sans
l'avoir demandé une seule fois. Cette patiente fille poursuivait depuis
le printemps un nouveau projet qu'elle n'avait encore laissé entrevoir à
personne: elle voulait se marier, et elle avait jeté son dévolu sur
l'excellent Menico. Le jeune piqueur de buffles avait une beauté mâle et
robuste, faite pour séduire une âme paysanne; mais ce qui attirait
surtout Amarella, c'était la candeur de ce grand enfant, en qui elle
devinait des trésors de tendresse, de dévouement et d'obéissance
aveugle. Elle espérait trouver en lui l'idéal de toutes les femmes: un
mari qui ferait trembler tout le monde, et qui tremblerait devant elle.
Son plan était tracé à l'avance: Menico reviendrait à Rome au mois de
novembre; il succéderait au portier du palais Feraldi, qu'on saurait
bien faire chasser. Le mariage se ferait en même temps que celui de
mademoiselle, peut-être dans six mois, dans un an au plus tard; le comte
donnerait une dot; le seigneur Lello, dans l'ivresse de son bonheur, en
offrirait sans doute une seconde. Amarella, pour ne point se séparer de
son mari, resterait au service de la comtesse. Elle organisait sa vie à
l'avance, montait sa maison, prenait une bonne d'enfants et un petit
domestique pour faire les courses, et menait le même train que le
concierge d'un prince ou le suisse d'un cardinal.

Cependant Menico, la tête appuyée sur l'épaule du camérier, ronflait à
l'unisson des roues de la voiture. Sa femme en espérance le pinça
familièrement pour le réveiller.

«\emph{Aô!} Menico, Menicuccio, Cuccio!» lui cria-t-elle en épuisant
tous les diminutifs de son nom, «nous voici à Tavolato, et les fiasques
sont sur la table.»

Tavolato est un cabaret situé sur la route de Lariccia, à deux lieues
environ de la porte de Saint-Jean de Latran. Les promeneurs s'y
arrêtent, comme à Ponte-Molle, pour vider quelques bouteilles de vin
d'Orvieto.

Maîtres et valets descendirent sous une sorte de hangar construit avec
des branchages de lauriers-roses. Le cabaretier apporta un pain bis, un
fromage de lait de jument et une douzaine de flacons de verre blanc, au
large ventre, au col effilé, bouchés à la mode antique par une goutte
d'huile et une feuille de vigne, et remplis d'un petit vin blanc, léger,
sucré, limpide et joyeux. Tolla s'amusa à déboucher les bouteilles et à
enlever avec un petit paquet d'étoupes la goutte d'huile qui ferme le
goulot et protégé le vin contre le contact de l'air; puis elle remplit
tous les verres, excepté le sien, et l'on but en chœur à sa santé. Les
douze flacons se vidèrent comme par enchante ment, et Menico en prit sa
bonne part, quoiqu'il ne bût que de la main gauche. Il trouva même le
temps d'engloutir une livre de pain, tandis que Tolla émiettait sa part
à une nichée de poussins, accourus avec leur mère sur les pas du
cabaretier.

Lorsqu'on remonta en voiture, Menico était de si belle humeur,
qu'Amarella crut le moment propice à l'exécution de ses petits projets.

«Il me semble,» lui dit-elle, «que tu ne détestes pas l'Orvieto?»

«Les prêtres ne défendent pas d'aimer le bon vin,» répondit
sentencieusement Dominique.

«En buvais-tu beaucoup à Lariccia?»

«Autant que j'en voulais boire.»

«Comment l'entends-tu?»

«Quand mademoiselle est à Lariccia, elle m'en fait donner tous les
soirs.»

«Mais quand mademoiselle n'y est pas?»

«Quand mademoiselle n'y est pas, je n'ai pas soif.»

Amarella partit d'un grand éclat de rire. Elle affectait une grosse
gaieté, quand elle ne savait que dire et qu'elle voulait montrer ses
dents.

«Tu es un brave garçon d'aimer ainsi mademoiselle, mais je crois qu'elle
te le rend bien.»

«Est-ce qu'elle t'a jamais parlé de moi?»

«Très souvent. Elle dit que tu serais capable de tuer un homme pour
elle.»

«Un homme! Je tuerais un cardinal!» Amarella fit un signe de croix.

«Mais,» reprit-elle, «tu dois bien t'ennuyer pendant l'hiver, quand
mademoiselle est à Rome et que tu restes seul avec tes vilains buffles?»

«Quelquefois, mais je trouve toujours le moyen de me faire envoyer à la
ville une ou deux fois dans un hiver.»

«Sais-tu qu'ils sont très laids, tes buffles, avec leur peau galeuse,
leur grosse tête et leur dos bossu?»

«Oui; mais moi, quand je galope derrière eux, la lance à la main, dans
une grande plaine nue, en serrant mon cheval entre mes guêtres, il me
semble que je suis beau comme un Romain d'autrefois.»

«Mais lorsque tu reviens de Rome et que tu as vu tant de palais et
d'églises, comment peux-tu encore regarder ce grand désert brûlé par le
soleil, sans herbe, sans arbres, sans maisons, où l'on ne rencontre que
des aqueducs écroulés et de vieilles ruines de brique? Moi, je trouve
cela affreux.»

«Horrible!» ajouta le camérier, qui se piquait d'avoir du goût.

«C'est que vous avez vécu longtemps à la ville, répondit sincèrement
Menico; moi, qui ne sais rien et qui ai passé toute ma vie dans cette
grande solitude qui s'étend autour de Rome, j'aime ces plaines brûlées,
ce soleil ardent, ces ruines rouges, et jusqu'au chant des cigales, dont
les ailes grises viennent quelquefois me fouetter la figure. Quand je
suis triste, il me plaît de voir que tout est triste autour de moi.»

«Et quand tu es gai?»

«Alors c'est autre chose. Je vois des fleurs sur toute la terre, et les
masures rouges deviennent plus belles que des églises le jour de Pâques.
Comprends-tu?»

«Tu regrettais donc tes herbages et tes masures pendant les quatre mois
que tu as passés à Rome?»

«Non.»

«Pourquoi? J'étais auprès de mademoiselle.»

«Et si mademoiselle t'appelait à Rome pour toute la vie, y
viendrais-tu?»

«De grand cœur. Allons, mon Menico, tu mourras citoyen de la grande
ville.»

«Peut-être.»

«Et tes enfants seront de petits Romains.»

«Quels enfants? Je ne me marierai jamais.»

Amarella se remit à rire, mais du bout des dents.

«Jamais! C'est tard. Et pourquoi?»

«Je n'ai pas le temps.»

«Explique-moi cela, je t'en supplie.»

«Rien de plus simple. Si j'épousais une femme, je lui obéirais, n'est-ce
pas?»

«Probablement.»

«Eh bien! on ne peut pas servir deux maîtres à la fois.»

Tandis que Dominique confessait si naïvement son adoration pour sa
maîtresse, la voiture roulait sur la voie Appienne; le Monte Cavo se
rapprochait rapidement, et Tolla, avant de s'engager dans la route qui
mène aux jardins et aux parcs d'Albano, jetait un dernier coup d'œil à
ces prairies desséchées qui entourent la ville d'une ceinture de
tristesse et de désolation. Lorsqu'on suit cette route pendant l'été, on
est tenté de croire que la terre d'Italie, partout si belle et si
féconde, a été marquée d'un fer rouge autour de Rome, soit pour expier
les crimes des empereurs, soit pour effacer les scandales des papes. La
route ne traverse que des terrains nus, hérissés d'herbes flétries,
divisés par quelques barrières de bois mal équarri, et animés de loin en
loin par la présence d'un bouvier à cheval qui chasse une vingtaine de
bœufs blancs et de buffles noirs. On rencontre de temps en temps un
petit temple dépouillé de ses marbres, un tombeau en ruine, ou les
restes d'une villa où les éperviers font leur nid. Mais Tolla prêtait à
cette solitude morte la vie, la jeunesse et l'amour qui abondaient dans
son âme. La joie dont elle était pleine débordait sur tous les objets
environnants, ressuscitait les ruines et faisait reverdir la terre. Elle
comprit alors pour la première fois cette fiction des poètes qui prétend
que l'amour fait naître les fleurs sous ses pas.

La famille Feraldi traversa à dix heures la grande rue de Lariccia. Vers
le même moment, Lello s'habillait pour aller voir Philippe Trasimeni: il
avait dormi sans débrider jusqu'à neuf heures et demie.

«Qui t'amène si matin?» demanda Pippo en le voyant entrer.

«Le bonheur, mon ami! J'ai passé une soirée comme les saints n'en ont
pas souvent en paradis.»

«Bravo! Et comme je suis le seul à qui tu puisses sans indiscrétion
faire part de ta félicité, tu m'apportes le trop plein de ton âme?
Verse, mon ami, verse.»

«Ce ne'st pas tout. J'ai un conseil à te demander.»

«Demandez et vous recevrez. C'est parole d'Évangile.»

«Mon cher Pippo, elle est partie.»

«Je le sais, mais si c'est sur moi que tu comptes pour la faire
revenir\ldots»

«Non. J'irai la voir un de ces jours; je l'ai promis à son père. Nous
prendrons rendez-vous à Albano. Voudras-tu être du voyage?»

«De grand cœur; aujourd'hui, demain, pourvu que je ne sois pas de
service.»

«Non, plus tard: je ne veux pas faire d'imprudence; mais, en attendant,
il faut\ldots{} Ne te moque pas de moi; j'ai promis de lui écrire.»

«Eh bien?»

«Par tous les courriers.»

«Après?»

«A dater d'aujourd'hui.»

«Où est le mal?»

«Si j'avais déjà reçu une lettre d'elle, je ne serais pas en peine: je
lui répondrais paragraphe par paragraphe; mais tu sais combien j'ai peu
l'habitude d'écrire, et je voudrais\ldots»

«Quoi? me prendre pour secrétaire?» demanda Philippe en riant aux
éclats. «Grand merci! Je te ferai des vers tant que tu voudras, parce
que tu n'en voudras pas tous les deux jours, et parce que je tiens pour
démontré que tu n'es pas capable d'en faire; mais comme tout homme qui a
appris à écrire est capable de faire de la prose, j'espère bien que tu
sauras te passer de moi.»

«Sans doute, et si tu attendais les demandes pour faire les réponses, tu
saurais que je ne veux de toi qu'un simple conseil. Je prendrai le style
familier, n'est-ce pas? Je lui parlerai un peu de tout, de l'état
sanitaire, des bals, de ce qui me sera arrivé dans la journée, de\ldots»

«En deux mots, mon cher, parle-lui d'elle et de toi. C'est le texte
invariable de toutes les lettres d'amour, depuis l'antiquité la plus
reculée.»

«Et puis-je me permettre de la tutoyer? Je lui ai dit tu, hier au soir,
dans la chaleur du discours; mais peut-être dans une lettre le vous
serait-il plus de saison?»

«Mon cher Lello, le vous est une invention des Romains de la décadence.
Ce vous sans équivalait dans l'origine à un long compliment, ainsi
conçu: ‹Homme, tu as tant de vertu, de puissance et de gloire, que tu
n'es pas un seul homme, mais dix ou douze hommes réunis en faisceau.
Agréez mon respectueux hommage.› Tous les peuples qui pensent qu'un
homme en vaut un autre, que le maître n'est pas à son domestique comme
la dizaine est à l'unité, ont gardé le tu. Les premiers chrétiens se
tutoyaient, les apôtres tutoyaient le Sauveur, tandis qu'un pair
d'Angleterre dit vous à son chien, doute pour indiquer qu'il le respecte
autant qu'une meute entière. Décide maintenant si tu dois dire vous à ta
maîtresse.»

«Non, par Bacchus! Tu es un homme de bon conseil. Adieu, merci; je vais
écrire.»

Il courut au palais Coromila, s'enferma à double tour dans sa chambre,
de peur de surprise, et écrivit en moins de trois heures la lettre
suivante:

\begin{quote}
Ma chère Vittoria,

Il n'y a pas à dire, il faut que ce soit moi qui écrive le premier. Eh
bien! soit, puisque cette lettre m'en attirera une de ta main.

Je me suis demandé si je devais t'écrire en \emph{vous} ou en \emph{tu;}
mais il m'a semblé que le \emph{tu} convenait mieux entre deux personnes
qui s'aiment. Va donc pour le \emph{tu}.

Ce soir, c'est le jour de la comtesse Sutri. Il faudra y aller danser,
etc. (etc. ne veut pas dire: faire l'amour); mais avec qui dansera-t-on?
Avec personne, ou avec des laides, comme la B\ldots{} ou la M\ldots{} Si
l'on joue, je jouerai, et, moyennant un petit sacrifice de huit ou dix
écus, j'assurerai ta tranquillité et la mienne, car tu n'auras pas de
reproches à me faire. Baste! Dans ma lettre de samedi, je te rendrai
compte de tout.

On meurt toujours assez gaillardement. Du reste, rien de nouveau depuis
hier. On dit qu'il y a eu un cas de choléra dans les environs de
Lariccia. Je voudrais que cela fût vrai: la peur qui a chassé monsieur
ton père nous le ramènerait incontinent. On parle de deux cas à
Frascati.

A propos de Frascati, j'espère que tu ne fréquenteras pas ce pays-là. Il
s'y trouve en ce moment un certain petit homme brun foncé, qui arrive
d'Ancône et qui a naguère témoigné pour toi une vive sympathie. Son nom
commence par un m et finit par un i. Je ne voudrais pas que le voisinage
fît naître quelque petit amour, qui ferait écrire quelques petites
lettres, qui feraient\ldots{} Mais, allons! je crois que je puis me fier
à toi.

Adresse ta réponse à Manuel Miracolo. J'avais d'abord pensé à Romilaco;
mais le pseudonyme serait trop transparent. Je crois que les gens de la
poste ne reconnaîtront pas Coromila dans Miracolo.

Adieu, il est tard; on m'attend dans le cabinet de mon père. Je te
laisse: tu peux croire avec quel regret! Mes respects à ta mère et à ton
père; j'embrasse Toto. Je ne te presse pas de me répondre sans retard:
je suis sûr que la recommandation serait inutile, et c'est dans cet
espoir que je me dis pour la vie ton très affectionné et sincère.

--- LELLO.
\end{quote}

Les Feraldi dévorèrent en famille cette singulière lettre d'amour, où la
pauvreté d'esprit engendrait la froideur, et où la gaucherie se cachait
de son mieux sous un air cavalier. Lecture faite, le père haussa les
épaules, et dit en souriant: «Bavardage d'amoureux!» La mère répéta avec
une complaisance visible les deux derniers mots: \emph{«affezionatissimo
vero!»} Le frère garda ses impressions pour lui; il savait de longue
main que Lello n'était pas un aigle; il avait tremblé à l'idée de cette
correspondance, qui pourrait refroidir le cœur de son futur beau-frère
en épuisant ce qu'il avait d'esprit; il savait que les hommes de tout
âge sont de grands écoliers qui pardonnent rarement à ceux ou à celles
qui leur ont donné des \emph{pensums}, mais, à tout prendre, il n'était
pas mécontent du premier \emph{pensum} de Lello.

Tolla était au comble de la joie. Elle ne jugeait point la lettre de son
Lello, et comment l'aurait-elle jugée? Elle la baisait, elle la serrait
sur son cœur, elle lui parlait, elle l'approchait de son oreille, comme
si le papier avait pu lui répondre. Tout lui semblait admirable dans
cette chère petite lettre: le papier était d'un beau blanc, l'encre d'un
beau bleu, la cire d'une odeur exquise, et le style à l'avenant. Si
quelqu'un s'étonne qu'une fille spirituelle, instruite et délicate
puisse se tromper à ce point et baiser avec enthousiasme une lettre
assez sotte et presque impertinente, je répondrai que c'était sa
première lettre d'amour, et qu'une première lettre d'amour est toujours
jugée avec indulgence, fût-elle adressée à une duchesse et écrite par un
commis-voyageur. Tolla lui renvoya, sans chercher ses mots, une lettre
de douze pages, qui était moins une réponse qu'un \emph{post-scriptum}
ajouté à leur longue conversation du jardin. C'était un récit détaillé
de tous les sentiments qui avaient traversé son cœur durant deux longues
journées, la suite de ses pensées d'amour, qui s'enchaînaient l'une à
l'autre comme les anneaux d'un collier d'or. La route lui avait parlé de
Lello; elle avait entendu son nom dans le bruit des roues de la voiture:
arrivée, elle avait parlé de lui à tout ce qui l'entourait, à la maison,
au jardin, aux meubles de sa petite chambre, aux vieux arbres,
confidents de ses premiers secrets. Le lendemain matin, en attendant
l'arrivée de la poste, elle avait poussé jusqu'à Albano, seule, à
cheval, par le petit sentier du ravin, pour donner un coup d'œil à la
villa Coromila. Elle avait trouvé la porte ouverte à deux battants,
comme si la maison eût attendu sa future maîtresse. Jamais le parc ne
lui avait paru si beau. Les grands chênes avaient l'air de se ranger au
bord des avenues, comme de fidèles serviteurs, pour lui rendre hommage.
Elle les avait passés en revue en les saluant de la main. Elle avait
rencontré une vieille femme qui ramas sait du bois mort; elle lui avait
donné de quoi se chauffer tout l'hiver. Deux bambins qui tentaient
l'escalade d'un poirier s'étaient enfuis à son approche; elle avait
cueilli des poires pour les leur jeter. Elle avait découvert, au fond du
parc, à une demi-lieue de la maison, une charmante retraite; c'était un
massif de grands buis, de troènes et de lauriers. Il fallait absolument
y construire un cabinet de travail. C'était là qu'elle enseignerait le
français à son roi fainéant: cette partie du jardin prendrait désormais
le nom d'académie de France.

La lettre se terminait par une page entière d'un délicieux radotage
d'amour, intraduisible dans une langue aussi précise que la nôtre.
C'étaient des superlatifs impossibles, un mélange bizarre d'adjectifs
entrelacés, un chaste et pur dévergondage de style, une prose poétique
aussi fraîche que la rosée du printemps, aussi sonore que le bruit des
baisers, un hymne à la créature où le Créateur n'était pas oublié,
l'aveu virginal d'une passion sans tache et d'un bonheur sans remords.

Le croira-t-on? lorsqu'elle relut sa lettre, elle la trouva froide. Elle
aurait voulu pouvoir écrire comme Lello.

Voici la réponse qu'elle reçut.

\begin{quote}
Rome, 19 août 1837.

Ma chère Tolla,

La poste ne donne pas encore de lettres. J'en suis donc à attendre ta
réponse à ma lettre du 17 courant; mais, pour gagner du temps, je
commence toujours à t'écrire. Si ta lettre m'arrive ensuite, je t'en
accuserai réception.

Il y a un vieux proverbe qui dit: Le diable est plus laid en peinture
qu'en réalité. J'espérais qu'il en serait de même de ton absence, et je
croyais pouvoir m'y faire; mais je vois bien que le proverbe a menti,
car je suis comme un poisson hors de l'eau. J'ai passé hier devant ta
maison, et je me suis senti tout mélancolique en voyant les volets
fermés. J'ai pensé à nos causeries, à nos promenades, etc. tout cela est
suspendu! Pour combien de temps? Pour un mois. En vérité, c'est un peu
bien long; mais il faut s'y résigner, d'autant plus que ce mois de
prudence portera ses fruits dans l'avenir.

J'espérais aller te voir lundi; mais, si tu veux bien le permettre, nous
remettrons la partie à jeudi. D'abord je serai plus libre, et je pourrai
rester plus longtemps; puis nous ne saurions avoir trop de prudence, et
je crains d'éveiller les soupçons.

Je voudrais te dire une infinité de choses; mais il vaut mieux les
réserver pour notre première conversation, qui sera, je te le promets,
longue et bonne.

Passons à la soirée de la comtesse Sutri. J'y suis allé sur les neuf
heures et demie. J'ai fait un whist avec mon oncle le colonel. J'ai
perdu une douzaine de fiches à dix sous, et j'ai quitté le jeu vers onze
heures. J'ai passé dans le grand salon et je suis tombé au milieu d'une
contredanse. Les danseuses étaient la B\ldots, la L\ldots, la D\ldots,
et mademoiselle la fille de M\textsuperscript{me} Fratief. Je restai
spectateur indifférent. La générale accourut à moi, dès qu'elle
m'aperçut, en criant: Ah! cher prince! Il faut que je vous raconte ce
qui nous arrive: une histoire épouvantable! L'Anglais qui demeure dans
notre maison, au-dessus de nous, prétend qu'on lui a volé un fusil; il a
fait venir la police: on a eu l'indélicatesse de fouiller la chambre de
mon domestique. J'ai eu beau dire que Cocomero était un honnête homme,
que mes gens n'étaient pas capables d'une mauvaise action: vos sbires
sont des malotrus. Ils ont retourné le lit de ce pauvre garçon, qui
pleurait comme un enfant de se voir injustement soupçonné; mais ils
n'ont rien trouvé: j'en étais bien sûre. Croyez-vous que je ferais bien
de me plaindre au cardinal-vicaire? Enfin des jérémiades dont je suis
encore assourdi.. A ce moment j'entendis les premières mesures d'une
certaine valse de ma connaissance et de la tienne; mais comme j'aurais
été forcé de danser avec la chère Nadine, je fis la sourde oreille. Mon
indifférence fut funeste à la valse: le piano s'arrêta, et l'on ne dansa
plus. M\textsuperscript{me} Fratief partit avec sa fille: elle comptait
sur moi pour la reconduire; mais je me contentai de lui faire un profond
salut et de dire à son intention la \emph{prière pour les voyageurs.}
Ai-je bien fait, mon maître?

Et maintenant, parlons un peu du choléra.

Le fléau a complètement disparu dans le Borgo; il règne à la place
Montanara et à la via Margutta, et il commence à faire son chemin dans
le Corso. J'ai un peu de peur; mais à force de précautions, j'espère
échapper. Ne crains rien, et si par accident le courrier arrive un jour
sans t'apporter de lettre, ne va pas te figurer pour cela que je suis
mort.

Je termine ici la première partie de ma lettre: si je reçois la tienne
après dîner, j'ajouterai un \emph{post-scriptum.} Mes respects à tes
parents; embrasse ton frère pour moi. Je suis avec tendresse ton très
affectionné.

P.-S. J'ai reçu ta lettre, et je te laisse à penser si elle m'a été
agréable.

--- LELLO.
\end{quote}

Cette correspondance se prolongea, sans incident notable, jusqu'aux
derniers jours de septembre. Tolla écrivait des lettres adorables, et
adorait aveuglément les lettres médiocres de Lello. Toto, en observateur
froid et judicieux, relevait à part lui dans les lettres du jeune
Coromila tous les passages qui pouvaient l'éclairer sur l'état de son
cœur ou sur la solidité de son caractère. Il remarqua bientôt dans le
style une fatigue sensible. Le 22 août, Lello, charmé d'avoir pu écrire
une longue lettre, s'écriait avec enthousiasme:

\begin{quote}
Comment! je suis au bout de ma feuille de papier! allons, je vais écrire
en travers. Eh bien! non, j'ajouterai une feuille. De cette façon
j'écrirai deux fois plus qu'à l'ordinaire. Te souviens-tu qu'un certain
soir je m'accusais de n'être pas grand barbouilleur de papier? Le fait
est que cela a toujours été mon défaut; mais quand j'écris à toi, je ne
sais à quoi cela tient, je ne m'épuise jamais, et je trouve toujours du
110uveau à te dire. Qui m'expliquera cette énigme?
\end{quote}

Le 14 septembre cette fécondité était bien épuisée. Il écrivait:

\begin{quote}
Sais-tu que c'est un supplice terrible que d'improviser une lettre de
but en blanc, sans avoir à quoi répondre? Le langage de l'amour est
fécond, j'en conviens, mais dans la conversation, et non dans la
correspondance. Si tu étais ici, je saurais que dire; mais si je t'écris
que je t'aime, c'est chose dite et redite; que je te suis fidèle, c'est
chose trop évidente; que je désire ton retour, c'est un sujet tellement
rebattu qu'il ne me reste plus qu'à jurer comme un païen en voyant que
tu ne reviens pas. Que dire? mon Dieu! que dire?

Je te dirai premièrement que le choléra\ldots{} „ Le choléra, comme on
l'a déjà vu, tenait une grande place dans cette correspondance
amoureuse, et les lettres de Lello pourront ser vir un jour à l'histoire
du choléra de 1837. Lello racontait toutes les phases du fléau en
observateur exact, et toutes les émotions qu'il en ressentait, en
psychologue sans vanité. Il avait cette naïveté des peuples du Midi, qui
ne rougissent ni de leurs terreurs ni de leurs larmes.

Le choléra, écrivait-il le 24 août, continue sa moisson de chrétiens; on
dit qu'hier nous allions un peu mieux: on a vu moins de communions et
d'enterrements que les jours passés. Je te confesse que j'ai grand'peur,
non que je sois malade, je me sens comme un taureau, mais d'entendre
dire: «Un tel jouait hier à l'écarté, on l'enterre aujourd'hui; une
telle était hier à la promenade, elle sera ce soir au cimetière;» tout
cela m'a jeté dans une sombre mélancolie. La pensée de ma Tolla me
soutient, mais quelquefois elle ajoute à ma tristesse. Je me dis:
Serai-je vivant demain pour recevoir sa lettre? la reverrai-je jamais?
que de viendra-t-elle si je meurs? Et la mélancolie est si forte,
qu'elle m'arrache des larmes. N'y pensons plus; gai! Gai!

Oui, gai! gai! cela est facile à dire; mais il faudrait pouvoir être
gai. Une centaine de morts par jour, et des personnes de connaissance:
la princesse Massimi, la princesse Chigi, et tant d'autres!
\end{quote}

Une semblable correspondance n'était pas faite pour rassurer la famille
Feraldi. La peur du mal donna à la pauvre comtesse une légère
indisposition. Dès que Manuel en fut informé, il écrivit à Tolla:

\begin{quote}
J'ai appris avec déplaisir que ta mère avait des douleurs d'entrailles.
Pour l'amour de Dieu, dis-lui de se soigner, et à la moindre diarrhée
fais-lui faire de la pulpe de tamarin pour tisane et de l'eau de riz
pour lavement. C'est l'ordonnance du docteur Ely.

Ce matin j'ai été pris d'une peur affreuse: j'avais des coliques. J'ai
cru sans hésiter à une attaque de choléra, et j'ai demandé de l'eau de
riz; mais tandis qu'elle se faisait, mon mal s'est passé, et j'ai envoyé
tous les remèdes au diable.
\end{quote}

De tels détails insérés dans une lettre d'amour n'ont rien de choquant
en Italie, et Tolla remercia avec effusion son cher Lello de l'intérêt
qu'il prenait à la santé de la comtesse.

Toto, qui observait en même temps sa sœur et Coromila, s'aperçut que de
jour en jour cette excellente fille s'attachait davantage à son amant,
par toutes les craintes qu'il lui avait données et les dangers qu'il
avait courus.

Quelquefois, pour faire trêve aux pressentiments sinistres, Lello
parlait de ses espérances et de ses projets pour l'avenir. Tantôt il
offrait à Dieu ses ennuis présents, et lui demandait en échange un
bonheur parfait; tantôt il énumérait un à un les plaisirs qu'il se
promettait pour l'hiver prochain. Toto aurait voulu qu'il comptât un peu
plus sur lui-même, au lieu de s'en remettre à la Providence. «Patience!»
écrivait Lello (Toto l'aurait voulu moins patient), «offrons nos
tribulations à Dieu, et en échange du sacrifice qu'il nous impose, il
nous donnera une parfaite félicité. Je me repais déjà de la pensée de
ces jours où nous serons heureux ensemble, où ensemble nous remercierons
Dieu de nous avoir assistés dans nos besoins et ré compensés de nos
souffrances. O douce idée!!!»

«Voilà des rêveries bien creuses et des espérances bien vagues,» pensait
le sage Toto Feraldi. «Je songe,» écrivait Lello, «je songe à l'hiver
prochain, aux visites que je te ferai dans ta loge à l'opéra, aux
réunions choisies où nous verrons sans oublier la prudence» (trop de
prudence! pensait Toto), «aux cotillons, aux contredanses, aux petites
jalousies qui naîtront dans ton cœur ou dans le mien, aux journées
pluvieuses que nous passerons chez toi, et à tant d'autres belles choses
dont l'énumération serait trop longue.»

«Il ne parle pas du mariage!» murmurait intérieurement le frère de
Tolla.

Un jour, Tolla lut en pleurant de joie ce passage d'une lettre de Lello:

\begin{quote}
Tu peux imaginer ou plutôt tu dois savoir comme un amant s'attache à
tout ce qui vient de la personne aimée; mais ce que tu n'imagineras
jamais, c'est l'attachement que j'ai pour tes lettres. Sache que j'ai
commandé à Castellani une cassette de noyer poli, avec une magnifique
serrure qui s'ouvrira avec une clé d'or suspendue à un anneau d'or: le
tout me coûtera une vingtaine d'écus, et pourquoi? pour serrer tes
lettres, qu'un jour, s'il plaît à Dieu, nous relirons ensemble.
\end{quote}

Toto ne fit aucune objection aux larmes de sa sœur, mais il eût mieux
aimé ne pas savoir le prix de la cassette.

Depuis le départ de la famille Feraldi, Lello promettait de faire le
voyage d'Albano. Tolla, avertie la veille, monterait à cheval avec sa
mère, et l'on se rencontrerait par hasard aux environs du tombeau des
Horaces. Malgré les instances de Tolla et l'empressement de Pippo, qui
devait être de la partie, ce voyage resta six semaines à l'état de
projet. Lello avait peur d'éveiller les soupçons. Il était surveillé par
trois ou quatre personnes, et il croyait avoir cent espions à ses
troussés. M\textsuperscript{me} Fratief et sa fille lui tendirent
plusieurs piégés dans l'espoir de lui faire avouer sa correspondance
avec les Feraldi; mais il prit si habilement ses mesures, il sut si bien
faire l'ignorant, \emph{l'Indien,} comme on dit à Rome, qu'elles
n'obtinrent aucune preuve contre lui. Ces petits complots le mirent en
fureur. Il écrivait à Tolla: «Cette Nadine! j'ai envie de lui faire la
cour, de la rendre folle de moi, et de lui infliger une mystification
qui la forcera d'entrer au couvent, pour le moins! Mais non, tu n'aurais
qu'à prendre de la jalousie, et puis on jaserait sur moi.» Ses amis et
les anciens compagnons de ses plaisirs le savaient amoureux: il n'était
plus de leurs parties; mais il se gardait de prononcer devant eux le nom
de Tolla. Un jour, son valet de chambre lui remit, en présence de sept
ou huit jeunes gens, une lettre de Lariccia. Tous ces jeunes fous lui
crièrent à la fois: «De qui? de qui?» Il répondit, en mettant la lettre
dans sa poche: «C'est d'un abbé!» Il racontait à sa maîtresse, avec une
satisfaction visible, ces petits succès de dissimulation: cacher son
bonheur est un plaisir italien. Il se cachait aussi de sa famille, mais
pour des causes différentes: il avait peur de ses oncles et de son père.

\begin{quote}
Je voudrais t'écrire plus longuement, disait il un jour à Tolla; mais je
suis entouré d'espions, mon père me fait appeler à chaque instant, et
lorsque je monte chez lui, je n'aime point à laisser sur mon bureau ma
lettre commencée. Je jette tout dans un tiroir et je prends la clé dans
ma poche. Au moment où je t'écris, je suis enfermé à double tour dans ma
chambre, quoiqu'il n'y entre pas un chat; mais on ne saurait trop
prendre de précautions.
\end{quote}

«Pauvre garçon!» disait Tolla.

«Poltron!» pensait Toto.

Les derniers jours de septembre parurent bien longs à toute la maison
Feraldi. Lello promettait toujours de venir et ne venait jamais. Il
alléguait deux grandes affaires dont il attendait le dénouement. «Quand
vous saurez ce qui m'a retenu,» écrivait-il à la comtesse, «vous ne
regretterez pas le temps perdu. Notre bonheur avance à grands pas, et le
jour où nous nous verrons à Albano, je vous porterai de bonnes
nouvelles.» Pippo Trasimeni avait écrit de son côté qu'il lui tardait
fort de venir serrer la main à Tolla, mais que Lello se faisait trop
tirer l'oreille. Il fondait une sorte d'association de charité, et les
convocations, les assemblées, les quêtes et les circulaires prenaient le
plus clair de son temps. Il avait l'air de traiter encore une autre
affaire avec son oncle le chevalier et son frère aîné, qui était revenu
de Venise; mais aucun ami de la famille n'était dans le secret, excepté
un Français, monsignor Rouquette, secrétaire particulier du
cardinal-vicaire.

Le 29 septembre, à huit heures du soir, on relisait en commun la
correspondance de Lello dans la chambre du comte, autour d'un petit feu
clairet où Toto jetait de temps à autre une poignée de sarments. La
famille entière, sans excepter Tolla, était en proie à une sorte de
malaise qui ressemblait beaucoup à de la tristesse. Le comte relevait
tout haut les expressions ambiguës, les phrases équivoques et les
symptômes d'indifférence épars dans toutes ces lettres. La comtesse et
Tolla prenaient la défense de Lello. Toto ne donnait point son avis, il
aurait eu trop à dire; mais il offrait de partir pour Rome et d'aller
voir par lui-même ce qu'on pouvait encore espérer. La comtesse ne
voulait pas exposer son fils à ce voyage, tant qu'il serait question du
choléra; mais ne pouvait-on pas envoyer un homme intelligent et dévoué,
par exemple Menico? Si l'on apprenait que Lello avait cédé à l'influence
de sa famille, de ses amis ou d'une maîtresse, on verrait à se pourvoir
ailleurs. Tolla trouverait des maris à choisir. Elle n'avait que vingt
ans et un mois; sa beauté était dans tout son éclat, sa réputation
intacte: Lello, en évitant de se compromettre, ne l'avait point
compromise. Morandi d'Ancône était venu pour l'automne à Frascati, chez
la vieille duchesse Pisani. Peut-être serait-il disposé à reprendre les
négociations?

Tolla se récriait à cette seule idée. Elle jurait d'épouser le cloître
ou Lello.

Ces débats furent interrompus par l'arrivée du valet de chambre de
Lello, qui apportait une longue lettre de son maître. Menico, qui
revenait des champs, fut chargé de conduire le messager à la cuisine et
de lui faire fête. Tolla déchira vivement l'enveloppe, et lut à haute
voix la lettre suivante:

\begin{quote}
Grandes nouvelles, ma chère Tolla, et bonnes nouvelles! Je commence à
croire que Dieu nous protégé et que notre bonheur est assuré. Te Deum
laudamus!

Sache d'abord que, moi qui ne songe jamais à rien, j'ai eu l'idée de
fonder un grand hospice pour les orphelins du choléra. Cette idée, il
fallait la mettre à exécution sans argent, sans local, sans rien! J'ai
donc surmonté ma timidité naturelle; je me suis fait actif, remuant et
presque effronté. J'ai parlé à trois ou quatre cardinaux; ils ont soumis
mon projet au saint-père, qui l'a approuvé des deux mains. J'ai formé un
comité, nous avons organisé des quêtes dans toutes les églises et même
dans les maisons. Tu te demandes comment un paresseux tel que moi a pu
prendre tant de peine? Tu ne t'étonneras plus de rien quand tu sauras
que c'était à ton intention. Et comment? On m'avait prédit que cette
bonne œuvre attirerait la bénédiction du ciel sur mes fils (entends-tu?
mes fils!), et que si je parvenais à mener à fin cette entreprise,
j'obtiendrais la chose que je désire le plus ardemment. Figure-toi si je
m'y suis mis de tout mon cœur! Et j'ai réussi!\ldots{}
\end{quote}

«Qu'il est bon!» murmura Tolla en s'essuyant les yeux.

«Je n'ai jamais dit qu'il fût méchant,» répondit le comte.

«Oui, fais amende honorable,» répliqua la comtesse.

«Achevons vite,» dit Toto. «Ce n'est pas là cette grande nouvelle qu'il
nous promet.»

Tolla continua.

\begin{quote}
La récompense ne s'est pas fait attendre. Tu sais que mon frère s'était
amouraché à Venise de la fille d'un petit banquier qui n'est pas même
noble. Il jurait de l'épouser, et cette fantaisie mettait mon père au
désespoir. Il dicta à mon oncle le colonel une lettre sévère à laquelle
mon frère fit une réponse fort impertinente, disant que si l'on ne lui
permettait pas le mariage public, il trouverait assez de prêtres pour le
marier secrètement: qu'il avait donné sa parole, et qu'il faisait plus
de cas de son honneur personnel que de la vanité de la famille; enfin
qu'il ne s'effrayait point des menaces, puisqu'on ne pouvait le
déshériter de son majorat. Je fus scandalisé, comme tout le monde, du
langage de mon frère, et je devinai aisément que s'il persistait à
mécontenter la famille, je ne pourrais obtenir de longtemps ce
bienheureux consentement auquel nous aspirons. Le cardinal et le colonel
me surent gré des sentiments que je témoignais, et ils redoublèrent pour
moi les marques de leur amitié. Monsignor Rouquette, cet ami du colonel,
dont l'esprit et la gaieté sont si célèbres dans Rome, vint un jour me
voir. C'était dans la dernière quinzaine du mois d'août, peu de temps
après ton départ. Il me félicita des bons sentiments où il me voyait, et
me dit en confidence que la conduite de mon frère pouvait me faire le
plus grand tort. Je feignis de ne pas comprendre le sens de ses paroles.
«Votre frère,» me répondit-il, «était destiné de tout temps à une grande
alliance, et nous espérions lui voir épouser la fille d'un très riche
pair d'Angleterre. S'il avait répondu à l'attente de ses parents et de
ses amis, vous, son cadet, qui ne porterez point le titre de prince,
vous auriez pu vous marier, suivant votre penchant que je ne connais
pas, soit dans une famille princière, soit dans une famille de simple
noblesse, soit avec une riche héritière, soit avec une fille sans dot;
mais, si votre aîné se mésallie, vous comprenez que toute l'ambition de
la famille se reportera sur vous, et que le prince votre père y
regardera à deux fois avant de vous accorder son consentement. Il ne
souffrira jamais que cette immense fortune que lui ont léguée ses
ancêtres se disperse après sa mort. Or notez que si vous et votre frère
vous alliez épouser deux dots de trois ou quatre cent mille francs, pour
peu que vos enfants suivissent cet exemple, la branche des
Coromila-Borghi serait dans la misère à la troisième génération.»

Je fus frappé de la sagesse de ce raisonnement, et je déplorai amèrement
la folie de mon frère, qui portait un si rude coup à nos chères
espérances. Je serrai les mains de cet excellent monsignor, et je le
suppliai d'user de toute son influence sur mon frère pour l'amener à des
idées plus raisonnables. «Vous pouvez m'y aider,» me dit-il en souriant.
«Et comment, s'il vous plaît? Est-ce au cadet à conseiller son aîné?»
«Oui, quand le cadet est l'aîné par la sagesse.» «Et qui vous dit que je
sois plus sage que mon frère?» «J'en suis sûr, et je vous connais. Vous
êtes assez désintéressé pour épouser une personne sans fortune, mais
vous êtes trop gentilhomme et vous avez l'âme trop grande pour vous
allier à une bourgeoise.»

J'avouai, en rougissant de l'éloge, qu'il avait dit la vérité. Il reprit
vivement:

«Je ne vous demande pas d'envoyer un sermon à votre frère, vous n'avez
ni l'âge ni la tournure d'un prédicateur; mais qui vous empêcherait de
lui écrire qu'on se raille de lui dans tous les salons de Rome, que les
jeunes gens racontent en riant qu'il est enchaîné aux pieds d'une
Omphale bourgeoise, qu'on tourne en ridicule sa constance et ses
soupirs, qu'on assure qu'il n'ose pas quitter Venise, parce que sa
maîtresse le lui a défendu, qu'il n'a pas le droit de sortir de la ville
pour plus de vingt-quatre heures, et qu'il mourrait foudroyé d'un
regard, s'il se hasardait à mettre le pied sur la terre ferme? Ajoutez,
et c'est chose vraie, que de tous les adorateurs de sa maîtresse, il est
le seul qu'elle traite aussi sévèrement. Arrangez tout cela comme il
vous plaira; vous êtes homme d'esprit, et je n'ai rien à vous
conseiller.»

J'écrivis en sa présence une longue lettre de quatre pages, assez bien
tournée: je le dis sans vanité. Mon père me félicita chaudement, et mon
oncle le colonel me dit en m'embrassant: «Je me souviendrai de ce que tu
viens de faire, et quand tu auras besoin de mon appui ou de ma bourse,
compte sur moi.» Je lui répondis hardiment que bientôt peut-être
j'aurais besoin de son appui. «Je te devine, répondit-il en souriant. Eh
bien! je ne m'en dédis pas: compte sur moi!»

Deux jours après le départ de ma lettre, monsignor Rouquette se mit en
route pour Venise. Il vit mon frère, lui prêta de l'argent, l'invita à
quelques parties: ce brave monsignor est un bon vivant dans la force du
terme. Mon frère trouva tant de plaisir dans sa compagnie, qu'il
consentit à le suivre dans un petit voyage à Trévise. Cette promenade
devait durer quatre jours: elle se prolongea plus d'une semaine. Chemin
faisant, mon frère reçut plusieurs lettres anonymes qui n'étaient pas à
l'honneur de sa maîtresse. Un ami sincère, qu'il avait chargé de le
tenir au courant des moindres événements, lui apprit qu'elle allait
beaucoup dans le monde, qu'elle était gaie et de belle humeur, mais
qu'il ne la croyait coupable que d'un peu de légèreté. Monsignor
Rouquette profita d'une boutade de mon frère pour l'emmener à Padoue.
Les lettres anonymes les y suivirent. Mon frère écrivit à sa maîtresse,
sous l'inspiration de monsignor, une lettre fort sèche où il lui
reprochait sa con duite. Elle ne répondit pas, ou la réponse se perdit
en chemin. Les deux voyageurs poussèrent jusqu'à Ferrare. Monsignor
conduisit mon frère dans un café où il entendit par hasard une
conversation qui roulait sur sa maîtresse: on l'accusait de traiter fort
bien un colonel autrichien. Précisément ce colonel était la bête noire
de mon frère, et peu s'en fallut qu'il ne repartît pour Venise, afin de
le provoquer; mais monsignor lui fit entendre le langage de la religion,
lui prêcha le pardon des injures, et le conduisit tout doucement de
Ferrare à Bologne, de Bologne à Florence, de Florence à Rome, où nos
conseils, notre amitié, les remontrances de mon père et les
plaisanteries de mon oncle ont achevé ce grand ouvrage.

«Et cette pauvre Vénitienne?» vas-tu dire, car je connais ton cœur.
Cette pauvre Vénitienne épouse dans huit jours le colonel autrichien que
mon frère avait en horreur. Avoue que monsignor Rouquette est un
admirable homme: il assure d'un seul coup le bonheur de ma famille, le
nôtre et celui d'un colonel autrichien!

Mon frère a pris en grippe les beautés italiennes; il aspire à se marier
en Angleterre; il rêve cils blancs et cheveux roux. sont transportés de
joie, et mon oncle le colonel m'a répété ce matin même qu'il n'avait
rien à me refuser.

Je patienterai encore un mois ou deux, pour ne point brusquer les choses
et pour préparer mon père à ma demande, puis je prendrai mon courage, à
deux mains, et j'irai lui dire: «Mon père, si vous m'aimez, souffrez que
j'épouse Tolla!»

En attendant, j'ai invité Pippo et mon ami monsignor Rouquette à une
promenade qui est irrévocablement fixée au 5 octobre. Nous serons à
trois heures précises à la hauteur de la route Torlonia. Si mon étoile
me permet d'y rencontrer la plus belle fille de Rome, il n'y aura pas
sur la terre un homme plus heureux que ton fidèle.

--- LELLO
\end{quote}

Après cette lecture, Tolla et sa mère témoignèrent une satisfaction si
complète, que ni le comte ni Toto n'osèrent la troubler par leurs
réflexions. Tolla attendit le 5 octobre avec une impatience fébrile.
Elle eut ces mouvements vifs, ces traits, ces boutades, ces éclats de
voix, ces fusées d'esprit, ces rires brillants et sonores qui sont comme
les pétillements du bonheur. Le grand jour arriva enfin. A dix heures du
matin, sa mère la trouva devant une glace, en amazone, manchettes plates
et col chevalière; elle essayait un adorable petit chapeau Louis XIII.
Elle se mit à table sans dîner, comme les enfants à qui l'on a promis de
les conduire au spectacle. Elle pressa la toilette de sa mère et
s'impatienta contre Toto, qui n'était pas prêt à deux heures. On partit
en fin. Lorsqu'on aperçut au loin le tourbillon de poussière qui
enveloppait la voiture de Lello, elle craignit d'être étouffée par les
palpitations de son cœur.

La voiture s'arrêta. Lello poussa un petit cri de surprise qui ne
manquait pas de vraisemblance. Il descendit, suivi de Pippo et de
monsignor Rouquette en habit de ville avec les bas violets. Pippo serra
cordialement la main de Tolla, du comte et de Toto, puis il s'empara de
la comtesse et ne la quitta plus. Monsignor Rouquette salua
gracieusement tout le monde, et s'entretint avec le comte, qu'il avait
rencontré quelquefois chez le cardinal-vicaire. Toto se rapprocha de sa
mère et de Philippe Trasimeni, pour que Lello fût seul avec Tolla.

Tolla se demandait si elle aurait assez d'empire sur elle-même pour
causer avec son amant sans lui sauter au cou. Comment pourrai-je, se
disait-elle, entendre sa voix, essuyer ses regards, m'enivrer de ses
paroles brûlantes, sans que mon visage, mon geste et tout mon être
trahissent mon bonheur?

Elle tomba du haut de son attente lorsqu'elle vit devant elle un jeune
homme poli, guindé, compassé, souriant comme une gravure de modes et
froid comme un compliment. Il lui parla plus de dix minutes sans sortir
des trivialités de salon. La pauvre fille ne pouvait en croire ses
oreilles. Elle se demanda un instant si elle rêvait. Enfin elle
interrompit brusquement les fadeurs dont elle était excédée; elle
regarda son amant jusqu'au fond des yeux, et lui dit sans dissimuler sa
colère:

«C'est là ce que tu as à me dire? Voilà les secrets de ton cœur que tu
n'osais pas confier au papier et que tu gardais pour notre première
entrevue! Tu m'as fait attendre six semaines pour me dire ces belles
choses-là! Que crains-tu? qu'attends-tu? Quand oseras-tu m'aimer en
face? Va! tu ne m'aimes point! Ton cœur est plus froid que le marbre. Je
comprends maintenant pourquoi tu n'as pas voulu venir plus tôt: tu
craignais l'instinct infaillible de l'amour vrai. Tu savais qu'au
premier mot de ta bouche je devinerais ta froideur, ma folie et ton
indignité!»

Elle salua Lello et ses amis, lâcha la bride à son cheval et se lança
dans la route Torlonia. Ses parents prirent congé et la rejoignirent en
un temps de galop. Manuel Coromila, confondu, altéré, remonta en voiture
sans rien comprendre à cette brusque sortie. Il avait étudié pendant
huit jours le compliment qu'il ferait à sa maîtresse. Il avait préparé
un petit mélange de respect, de tendresse, de prudence, dont il ne
doutait pas que Tolla ne fût charmée; mais il avait compté sans la
passion. En rentrant à la maison, Tolla courut à sa chambre et écrivit à
Lello:

\begin{quote}
Pardonne-moi; j'ai été cruelle: je ne savais pas; mais Tu as vais ce que
je disais. Tu m'aimes, j'en suis sûre, puisque je vis; mais ton abord
froid et souriant m'a glacée: ton visage était comme un soleil d'hiver.
J'aurais dû comprendre que tu avais tes raisons pour te montrer ainsi.
Peut-être la présence de tes amis? Non, puisque c'est toi qui les avais
amenés. N'importe, tu avais tes raisons. Je ne les connais elles sont
bonnes et je les approuve. ta manière d'aimer, et moi la mienne; ne
cherchons pas quelle est la meilleure: aimons-nous.
\end{quote}

Manuel avait amené Pippo par timidité, pour ne pas se trouver seul,
après un si long temps, devant la famille Feraldi; il avait amené
monsignor Rouquette par poltronnerie. Son nouvel ami avait témoigné le
désir d'être de la partie, et il n'avait pas osé lui dire non. La
présence de ces deux témoins, dont l'un s'était imposé et dont il
s'était imposé l'autre, le condamnait à dissimuler son amour sous des
formules de simple politesse. Lello avait cette pudeur, plus commune
chez les hommes que chez les femmes, qui n'admet pas un tiers dans les
épanchements de l'amour.

La contrariété qu'il éprouva de voir sa délicatesse si mal appréciée le
rendit maussade jusqu'au soir. Il se coucha de bonne heure. Les
tempéraments sanguins ont cela de particulier, que la colère les porte
quelquefois au sommeil. Le lendemain, il se leva à neuf heures, et
écrivit tout d'un trait la lettre suivante:

\begin{quote}
Rome, 6 octobre 1837.

Ma chère Tolla,

Tu dois comprendre combien il m'a été doux de te revoir et pénible de te
quitter; mais ce que tu ne saurais imaginer, c'est combien je suis resté
abasourdi de toute cette entrevue. Tu voudras savoir pourquoi? Eh bien!
je vais te le dire, dans l'espoir que tu profiteras de mes doux
reproches pour te corriger à l'avenir.

Il y avait tantôt deux mois que nous aspirions à cette bienheureuse
rencontre. Elle avait toujours été contrariée: elle s'arrange enfin.
Nous arrivons, nous nous voyons, et la première fois que tu ouvres la
bouche, c'est pour me reprocher mon indifférence! Tu me dis que je ne
suis pas capable d'aimer, que je suis de glace pour toi, au moment même
où je souffrais, Dieu sait combien! d'être condamné à te parler avec
cette froideur au milieu de tant d'yeux qui nous épiaient. J'enrageais
comme un chien de te voir et de ne pouvoir te dire un mot de tant de
choses que j'avais sur les lèvres. Tu doutes que je t'aime et tu me le
dis en face, tandis que je perds la tête, tandis que tu es ma seule
pensée; tandis que je crois t'aimer autant que tu m'aimes, sinon plus,
il faut que je t'entende dire que je ne t'aime pas et que je suis de
glace! Tu voudrais que je fisse l'amour comme un collégien, à grand
renfort de soupirs et de grimaces; cet amour-là est bon pour les
nigauds: n'espère pas le trouver en moi.

J'aime, mais comme on doit aimer, en gardant mon amour au fond du cœur
et en ne le laissant voir qu'à celle que j'aime. Quand tu me connaîtras
bien, tu verras que tes soupçons étaient injustes, et tu ne voudras plus
m'infliger de si pénibles reproches. J'en aurais aussi, moi, des
soupçons, si je voulais; mais je connais ton cœur, je compte sur toi, je
vis tranquille: pourquoi n'en fais-tu pas autant? Oui, ma chère Tolla,
si tu m'aimes, comme j'en suis bien convaincu, ne m'accuse plus de
froideur: tu me ferais de la peine.

Liberté sainte, où es-tu? Pourquoi n'étais tu pas au milieu de nous?
J'aurais voulu, entre autres choses, t'interroger sur un certain alinéa
d'une de tes lettres qui demande des éclaircissements; mais que faire?
c'était à chaque instant ou monsignor Rouquette ou Pippo qui tournait
les yeux de notre côté.

Tu m'as dit, et je l'ai encore sur le cœur, que je n'avais pas voulu
venir plus tôt. Pour quoi accables-tu un opprimé?

Je voudrais non-seulement aller à toi, mais rester auprès de toi, vivre
avec toi, sans te quitter une minute; mais où veux-tu que je prenne du
temps, lorsque je suis forcé d'être toute la journée à la maison auprès
de mon père? Il est aveugle, Tolla, et tu dois comprendre combien mes
soins lui sont nécessaires. Je n'ai à moi que l'après-midi. Disposesen
comme tu voudras, et si tu me fournis un moyen d'aller à Albano et de
revenir en quatre heures, je suis prêt à en profiter.

Hier je suis rentré un peu tard, mais ce pauvre papa ne m'a rien dit.
Presse donc votre retour à Rome!

Ma santé n'a pas souffert depuis hier. J'ai l'estomac barbouillé, mais
cela se passera. Je voudrais bien engraisser un peu: je ne sais si j'y
parviendrai.

Depuis hier soir, je me suis frappé le front plus de quarante fois en me
disant: J'avais encore ceci et cela à lui dire! Mais quand je songe aux
témoins qui nous observaient, je reconnais que j'ai mieux fait de
réserver tout cela pour ton retour.

Tu me pardonneras cette longue semonce, car tu reconnaîtras que c'est
mon cœur qui parle. Fasse le ciel que mes remontrances produisent
l'effet que je désire, et que tu cesses d'aggraver par tes reproches la
douleur que j'éprouve de vivre loin de toi! Ne doute jamais de l'amour,
du tendre amour de ton très affectueux et fidèle.

--- LELLO
\end{quote}

Cette lettre passa, comme toutes les autres, sous les yeux de la famille
de Tolla. M\textsuperscript{me} Feraldi fut d'avis de proposer une
nouvelle entrevue. Toto pensa qu'il valait mieux retourner à Rome. ---
Je n'espère rien, dit-il, des entrevues qui auront pour témoin monsignor
Rouquette, et quant à laisser Manuel aux mains de l'habile homme qui a
si bien rompu le mariage de son frère, c'est une imprudence que je ne
vous conseille pas. Avez-vous remarqué la figure de ce digne monsignor?

«Je ne l'ai pas regardé,» dit Tolla.

«Il a une laideur agréable,» dit la comtesse.

«Les lèvres minces,» dit le comte.

«Et l'œil mauvais,» ajouta Toto. «Ou je me trompe fort, ou ce galant
homme, cet ami intime du vieux colonel Coromila a commencé contre nous
une petite campagne. Nous sommes en force pour nous défendre, mais à une
condition: c'est que nous nous transporterons sans tarder sur le champ
de bataille. Si l'on m'en croit, nous partirons demain. Le choléra n'est
plus à craindre; l'automne tire à sa fin, nous faisons du feu: rien ne
nous retient plus à Lariccia, et tout nous rappelle à Rome.»

«Il a raison,» dit le comte.

«Quel bonheur!» dit Tolla. «Je le verrai demain!»

«Nous emmènerons Menico,» dit la comtesse. «J'ai appris que Tobie, le
portier, s'enivrait et battait sa femme: Menico le remplacera.»

«Tant mieux!» s'écria Toto. «C'est plus qu'un domestique, c'est un ami
intelligent et dévoué.»

«Et brave!»

«Et vigoureux! Les espions des Coromila n'auront pas beau jeu avec lui.»

«Et prudent! Jamais une querelle. Il a des bras à assommer un bœuf, et
il n'a pas donné un coup de poing de sa vie.»

«Te souviens-tu, Tolla, du jour où il avait volé pour toi les abricots
du voisin Giuseppe? Le jardinier voulait le battre: il se contenta de
relever ses manches, et le jardinier l'envoya prudemment à tous les
diables.»

Cet éloge de Dominique fut interrompu comme par un coup de foudre.

On entendit dans la cour de la villa des cris si aigus, que tout le
monde se leva en sursaut. Au même instant, Amarella pâle, les yeux
hagards, et violemment émue pour la première fois de sa vie, vint
annoncer que le cheval de Menico était rentré seul, au galop, la bride
sur le cou. Menico était le meilleur cavalier de Lariccia: que son
cheval l'eût désarçonné, on ne pouvait le croire. Aurait-il été victime
d'un guet-apens? On ne lui connaissait point d'ennemis. Toto sortit en
courant, suivi de tous les hommes de la maison et d'Amarella. Ils
n'avaient pas fait vingt pas dans le village, qu'ils rencontrèrent un
groupe de paysans qui rapportaient sur un brancard le corps de
Dominique. Une balle lui avait traversé la tête d'une tempe à l'autre.

Le barbier accourut au bout de quelques minutes. C'était un petit homme
jovial. Il déclara qu'il n'y avait rien à faire pour le blessé qu'une
bonne bière en bois de sapin: il avait le cerveau traversé de part en
part, et il serait froid dans une heure. «Pauvre Menico!» ajouta-t-il
d'un air guilleret, «je voudrais pouvoir te guérir; mais que veux-tu? Je
ne suis pas le bon Dieu!»

Le corps fut déposé dans une des chambres du rez-de-chaussée. Toto et
Tolla refusèrent de le quitter, et voulurent passer la nuit en prières
avec le curé de la paroisse. Amarella disparut après la consultation du
barbier.

Le frère et la sœur prièrent ardemment pour la vie de Dominique, ou du
moins, puisque tout espoir était perdu, pour le salut de son âme. L'idée
qu'il allait comparaître devant son juge sans avoir eu un moment de
connaissance faisait frémir la bonne Tolla. «Si du moins,» disait-elle,
«Dieu lui permettait de recevoir les secours de la religion et de
détester ses fautes!»

«Son pouls bat toujours,» disait Toto, «mais si faiblement qu'on le sent
à peine. Pauvre Menico! c'était notre ami le plus ancien.»

«Nous avons perdu le bon génie de la maison. Je m'attends à tout
désormais. Lello ne m'aime plus!»

A quatre heures du matin, le blessé n'avait pas repris ses sens;
cependant son pouls battait encore. Tolla, pâle et les cheveux épars,
agenouillée devant ce grabat, ressemblait à ces statues de la prière que
le sculpteur a prosternées devant les tombeaux des rois. Son frère
s'était assoupi; elle-même était plongée dans une sorte de torpeur. Elle
n'entendit pas le bruit d'une voiture qui s'arrêtait devant la porte, et
elle se leva brusquement sur ses pieds, croyant rêver, lorsqu'elle vit
entrer Amarella suivie du docteur Ely. Amarella avait fait six lieues en
trois heures sur le cheval de Menico.

Le comte et la comtesse arrivèrent au bout de quelques minutes. En leur
présence, le docteur reconnut l'entrée et la sortie de la balle, situées
toutes deux à six centimètres au-dessus de la commissure externe des
deux yeux; mais la balle, au lieu de traverser le cerveau, avait
circonvenu les os en sous-parcourant la peau du crâne, et l'état du
blessé, quoique grave, n'était point désespéré. Lorsque le pansement fut
opéré et l'appareil placé, Menico revint à lui. Son premier regard fut
pour Tolla, le second pour le curé.

«Aurai-je le temps de me confesser?» demanda-t-il d'une voix éteinte.

«Oui, mon garçon,» répondit le docteur; «j'espère même que tu auras le
temps de vivre.»

Tous les assistants se retirèrent dans la chambre voisine. Au bout d'un
quart d'heure, on les fit rentrer. Le prêtre s'en alla chercher le saint
viatique à tout événement. Le blessé paraissait jouir de toutes ses
facultés intellectuelles; seulement il était faible et abattu.

Le docteur s'arrêta un instant avec le comte à la porte de la chambre,
et ils échangèrent à voix basse les paroles suivantes:

«Savez-vous,» demanda le docteur, «comment cela est arrivé?»

«Non, cher docteur: on l'a trouvé sur la route d'Albano.»

«Avait-il des ennemis?»

«Nous ne lui en connaissons pas.»

«Son père, ses frères, ne sont en guerre avec personne?»

«Il est fils unique, et son père est mort il y a dix ans.»

«S'il connaît son assassin, pensez-vous qu'il soit disposé à le nommer?»

«J'en doute. Vous savez le peu de respect qu'ils ont tous pour la
justice.»

«Oui, ils aiment mieux se venger que se plaindre, et ils croiraient
commettre une lâcheté en invoquant le secours des lois.»

«Cependant je vais essayer de le faire parler. Il ne faut pas que ce
crime reste impuni.»

«Essayez. Il est très faible; il n'aura pas la force de mentir.»

«D'ailleurs il vient de recevoir l'absolution: il n'osera pas commettre
un péché.» Cette conversation ne fut entendue d'aucun de ceux qui
entouraient Menico; mais il arrive souvent que les malades ont l'ouïe
d'une sensibilité prodigieuse, et les yeux de Menico brillèrent d'un
éclat singulier à ces paroles du docteur: «Ils aiment mieux se venger
que se plaindre.»

«Docteur,» observa le comte en approchant, «ce n'est pas nous qui ferons
l'interrogatoire. La femme de chambre de ma fille ne nous a pas attendus
pour le commencer.»

Amarella disait à Menico: «Eh bien! mon pauvre garçon, tu as donc des
ennemis?»

«Tu vois bien que non, puisque tout le monde pleure autour de moi.»

«Si je savais quel est le méchant qui t'a tiré un coup de fusil!»

«On ne m'a pas tiré de coup de fusil. C'est moi qui suis tombé sur les
cailloux.»

«Mais comment serais-tu tombé sur les deux tempes en même temps?»

«Cela n'est pas plus difficile que de dormir sur les deux oreilles.»

«Mais, malheureux, tu avais une balle dans le corps!»

«Est-ce que j'avais une balle dans le corps?»

«Oui, tu avais une balle dans le corps.» Il répondit en riant doucement:
C'est que j'aurai bu après quelqu'un de malpropre.»

«Nous ne saurons rien,» dit le comte.

«Il a le cerveau aussi sain que vous et moi,» ajouta le docteur.
«Maintenant je réponds de sa vie.»

Amarella poussa un cri de joie.

«De quoi te mêles-tu?» lui demanda naïvement Menico. «Mademoiselle
Tolla, je suis content de ne pas mourir avant votre mariage. Monsieur le
comte, j'ai une grâce à vous demander. Quand je serai guéri,
voudrez-vous permettre que j'aille vous servir à Rome?»

«C'est une affaire arrangée depuis hier,» dit Tolla.

«Certes, ajouta son père, je ne veux pas te laisser ici, exposé aux
coups du brigand qui a voulu t'assassiner!»

«Merci, monsieur le comte. Vous m'avez bien compris.»

«Docteur,» demanda Toto, «ne pourriez-vous nous prêter quelqu'un de vos
élèves qui achèverait ce que vous avez si heureusement commencé?»

«C'est bien mon intention. Je tiendrai compagnie à ce jeune médecin et à
mon bon Dominique jusqu'à ce que la guérison soit parfaite. Mon père, ma
mère et ma sœur partent avec vous ce matin pour Rome.»
