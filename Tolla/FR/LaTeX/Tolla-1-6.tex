\addchap{\RomanNumeralCaps 6.}

Pour la première fois de sa vie, Tolla quitta la campagne sans regret.
Elle se plaignit de la lenteur des chevaux: il lui tardait d'être à
Rome. Du plus loin qu'elle aperçut le dôme Saint-Pierre, elle battit des
mains par un mouvement de joie enfantine qui fit sourire le docteur.

Cependant, si elle avait été en état d'analyser ses sentiments et de se
rendre compte de l'état de son cœur, elle aurait reconnu que son bonheur
était plus mélangé et sa joie moins tranquille qu'à l'époque de son
départ pour Lariccia. Au mois d'août, elle ne craignait que pour la vie
de Lello, et cette crainte était tempérée par une confiance aveugle dans
la bonté de Dieu: elle aurait cru calomnier la Providence en supposant
que le fléau pût frapper son amant. Mais cette malheureuse entrevue, la
contenance embarrassée de Lello, la présence de monsignor Rouquette, la
dernière lettre qu'elle avait reçue, les observations que cette pièce
singulière avait suggérées au comte et à Toto, enfin le coup mystérieux
qui venait de frapper le plus humble et le plus dévoué de ses amis,
toutes ces circonstances accumulées jetaient dans son âme un trouble
secret dont elle essayait en vain de se défendre. Elle devinait que ce
qu'elle avait à craindre, ce n'était plus un de ces malheurs soudains
qui viennent directement de la main de Dieu, mais plutôt quelqu'un de
ces coups invisibles que dirige la haine ou l'ambition des hommes. Au
demeurant, la perspective de piégés à déjouer, de résistances à vaincre,
d'obstacles à surmonter, en un mot d'une guerre à soutenir, ne lui
faisait pas peur. Elle avait appris dès l'enfance à franchir les
barrières, et à ne craindre ni fatigue, ni danger. Cette éducation
virile avait aguerri son esprit. «Nous verrons bien,» disait-elle, «si
un amour honnête ne sera pas assez fort, avec l'aide de Dieu, pour
triompher de la haine et de l'intrigue.»

En entrant à Rome, la comtesse reconnut monsignor Rouquette, qui
descendait de voiture devant le musée de Saint-Jean-de-Latran. Elle le
montra au docteur Ely.

«Monsignor Rouquette!» dit le docteur.

«Le connaissez-vous?»

«C'est un de mes malades; mais, comme il se porte mieux que moi, nous ne
nous voyons pas souvent.»

«Que dit-on de lui par la ville?»

«On dit que c'est un galant homme et un homme d'esprit, qui pourra, si
Dieu le veut, devenir plus tard un saint homme.»

«Voilà tout ce qu'on dit?»

«Tout,» répondit prudemment le docteur.

«Alors, cher docteur, dites-moi ce qu'on en pense, car Rome est la ville
du monde où ce qu'on pense ressemble le moins à ce qu'on dit.»

«On pense que monsignor Rouquette n'est ni jeune ni vieux, ni beau ni
laid, ni blond ni brun, ni grand ni petit, ni riche ni pauvre, prêtre ni
laïque, ni honnête ni fripon, ni\ldots{} Mais pourquoi me forcez-vous à
me compromettre?»

«Parlez, mon ami,» dit vivement Tolla. «Cet homme, que j'ai vu il y a
trois jours pour la première fois, est venu se jeter au travers de mon
bonheur, pour me servir ou pour me perdre. Apprenez-moi, si vous le
connaissez, ce que je dois craindre ou espérer.»

«Tout, mon cher petit ange, selon qu'il ni sera pour vous ou contre
vous. Vous savez que j'ai la mauvaise habitude de juger les gens sur la
physionomie: ce monsignor-là possède une des figures les plus
significatives qu'il m'ait été donné d'observer; une vraie tête d'étude.
Le front est haut et large, le crâne vaste, le cerveau développé, les
yeux petits, ronds et enfoncés; les prunelles d'un bleu aigre et
transparent, comme chez les bêtes fauves; les narines ouvertes, mobiles
et palpitantes, signe in faillible de passions ardentes et de grands
appétits; les lèvres fines, si toutefois il a des lèvres; des dents à
tout mordre; un menton court, ramassé, trapu et profondément entaillé
par une fossette; le front plissé, les pommettes couperosées et une
large patte d'oie épanouie sur chaque tempe. Devinez à quoi je pense en
voyant cette figure travaillée, tourmentée et crevassée par un feu
intérieur? A la solfatare de Naples. Je faire un volcan mal éteint, et,
Dieu me pardonne! je crois voir la fumée sortir des rides de son front.»

«Bravo, docteur!» interrompit le comte. «On dirait, à vous entendre, que
son éminence le cardinal-vicaire a un secrétaire intime venu en droite
ligne de l'enfer.»

«Je ne sais pas s'il en vient, mais je vous réponds qu'il y va. M.
Rouquette est un homme vigoureux de corps et d'esprit, qui, pour son
malheur et pour celui des autres, est né dans une étable de village ou
dans une mansarde de Paris avec des instincts de prince. Le monde n'a
jamais manqué de ces hommes d'action que le sort jette sur le pavé, sans
argent, sans naissance, et sans aucun autre instrument d'action que leur
intelligence et leur volonté. Ils deviennent, selon les circonstances,
illustres ou infâmes; ils font beaucoup de mal ou beaucoup de bien, mais
ils ne meurent pas sans avoir fait quelque chose. Soit qu'ils
détroussent les passants comme Cartouche, soit qu'ils dévalisent les
peuples comme Law, soit qu'ils renversent les trônes comme Marat, soit
qu'ils fondent des dynasties, ils ont entre eux une étroite parenté, et
ils appartiennent tous à la grande famille des aventuriers. Rouquette
est un des cadets de la famille. Au temps des petites guerres du moyen
âge, il aurait commandé une troupe de routiers; pendant les luttes de
Louis XIV, il aurait obtenu des lettres de marque et commandé un
corsaire; au siècle suivant, il aurait inventé quelques mines du
Mississippi ou tenu les cartes dans quelque tripot; sous la république
française, il eût été l'orateur de son carrefour et le président de sa
section. En 1837, découragé de vivre dans un pays où la paix, la loi, la
troupe de ligne et la gendarmerie ont fermé à jamais l'ère des
aventures, il est venu à Rome: il aspire aux dignités ecclésiastiques,
les seules qui soient accessibles à un homme d'esprit sans naissance et
sans fortune. Il choisit dans le sacré collège les deux hommes qui ont
le plus de chances d'arriver à la papauté: il se fait secrétaire du
cardinal-vicaire, il s'insinue dans la confiance du cardinal Coromila.
Sans renoncer aux douceurs de la vie laïque, il porte l'habit
ecclésiastique, il obtient le titre de monsignor et le droit de porter
les bas violets: prêt à entrer dans les ordres au premier évêché vacant,
ou à jeter la soutane aux orties, dès qu'il trouvera une dot à épouser.
Habile à tout, capable de tout, obéissant aux événements jusqu'à ce
qu'il puisse leur commander, commandant à ses passions jusqu'à ce qu'il
soit assez riche pour leur obéir, il a déjà gagné assez de crédit pour
que rien ne lui soit impossible, pas même le bien. Si quelque intérêt
proche ou lointain le porte à assurer votre bonheur, comptez sur lui,
vous serez heureuse; mais s'il s'avisait de parier que je mourrai dans
l'année, ma foi! je commencerais par faire mon testament. Tout cela
entre nous!» ajouta le bon docteur en appuyant l'index sur ses lèvres.
«Mais ne me dira-t-on pas, à moi qui ai ouvert à cette belle enfant les
portes de la vie, quel danger elle craint et quel bonheur elle espère?»

La comtesse lui raconta en quelques mots l'histoire des amours de Tolla.

«Je ne vois pas apparaître monsignor Rouquette,» dit le docteur.

«Maman a oublié de vous dire que la seule fois que Lello est venu nous
voir à la campagne, monsignor Rouquette était avec lui.»

\emph{«Diamine!»} dit le docteur. «C'était son juron favori.
\emph{Diamine} est un blasphème anodin qui remplace \emph{diavolo!}
comme en français \emph{jarnicoton} remplace \emph{jarnidieu}.» «C'est
ce Rouquette qui a rompu le mariage de Coromila l'aîné avec une
Vénitienne.»

«Nous le savons.»

«Dans quel intérêt a-t-il fait cela? Pour complaire au prince et au
cardinal. Le chevalier ne compte pas. Or le prince et le cardinal s'en
iront prochainement rejoindre leurs ancêtres: je ne leur donne pas six
mois, et Rouquette est sur le point de perdre un de ses Eh bien! mon
petit ange, votre affaire ne me paraît pas mauvaise. Quand les deux
vieux Coromila n'y seront plus, Rouquette n'aura plus aucune raison de
contrarier votre mariage. Ayez seulement six mois de patience et de
prudence, et recommandez au beau Lello d'étouffer son feu sans
l'éteindre.»

Les conseils du docteur furent scrupuleusement suivis. Lello n'avait pas
besoin qu'on lui recommandât la prudence. Mme Feraldi se chargea du soin
d'organiser le bonheur de ses deux enfants. Lello venait tous les soirs
à \emph{l'Ave Maria} passer une heure auprès de sa maîtresse; il courait
ensuite dire le chapelet avec sa famille; il s'habillait et allait dans
le monde, où il revoyait Tolla. Les jours où Tolla ne sortait pas, il
savait, sans se faire remarquer, prélever une heure ou deux sur sa
soirée pour causer avec elle.

Ils avaient adopté, dans le salon du palais Feraldi, une embrasure de
fenêtre grande comme une de ces chambres que les architectes nous
construisent à Paris; ils en avaient fait leur salon particulier, leur
domaine inviolable, et comme le sanctuaire de leur amour. face l'un de
l'autre, le coude appuyé sur la fenêtre, ils recommençaient tous les
soirs l'éternelle conversation que le genre humain répète depuis tant de
siècles sans la trouver monotone. Quelquefois, à bout de paroles, ils
gardaient le silence, ce silence des amants qui est le plus doux des
langages. Quelquefois, penchés l'un vers l'autre, la main dans la main
et les larmes bien près des yeux, ils disaient et redisaient ensemble
deux mots où se concentraient toutes leurs pensées et toutes leurs
espérances:

«\emph{Lello mio!}»

«\emph{Tolla mia!}»

«Mon Lello!» «Ma Tolla!» Il est bien vrai que l'italien est par
excellence la langue de l'amour. La voix se repose doucement sur la
première syllabe de \emph{mia}, et donne au mot ainsi prolongé toute la
suavité d'une caresse.

Lello et Tolla se querellaient quelquefois et ne s'en aimaient que
mieux. Ces querelles, toujours suivies du baiser de paix, sont
l'assaisonnement du bonheur. Ils s'étaient promis l'un à l'autre que
jamais, quels que fussent leurs griefs, ils ne se sépareraient le soir
sans être réconciliés.

«Je ne veux pas,» disait Tolla, «que tu t'en dormes sur une mauvaise
parole.»

«Enfant!» répondait Lello, «est-ce que je dormirais?»

Tolla avait l'âme trop sincèrement pieuse pour ne pas songer au salut de
son amant. D'ailleurs un instinct secret l'avertissait peut-être qu'il
n'oublierait pas ses devoirs envers elle, tant qu'il se souviendrait de
ses devoirs envers Dieu. En plaidant la cause du ciel, elle plaidait la
sienne.

Lello n'avait jamais négligé ces obligations de piété extérieure que les
lois de Rome rappellent et imposent au besoin à tous les sujets du pape,
et que les jeunes gens les plus dissipés accomplissent sans marchander.
Il faisait beaucoup plus, au moins en apparence, que la religion la plus
austère ne commande, et il usait des sacrements jusqu'à l'abus; mais
Tolla eut fort à faire pour lui rendre les sentiments religieux qu'il
professait et qu'il n'avait plus. Elle le tançait doucement, et le
priait de mettre ses idées d'accord avec sa conduite. «Tu es,» lui
disait-elle, «un mauvais chrétien d'une espèce comme singulière. Les
autres pensent bien et agissent mal; toi, tu penses mal et tu agis bien.
Je ne te dirai donc pas, mes confrères les prédicateurs: Conformez votre
conduite à votre foi; mais plutôt tâchez de croire à ce que vous
pratiquez.»

en attendant!

Comme l'impiété de Lello n'avait rien de systématique et qu'elle tenait
moins du scepticisme que du libertinage, elle guérit. Tolla eut la joie
de convertir son amant, de détruire l'effet des mauvaises compagnies, et
de dissiper au souffle de l'amour les fumées dont il avait le cerveau
obscurci. Les deux amants prièrent ensemble, et la prière devint le plus
cher plaisir de Tolla. Lello voulut qu'ils eussent le même confesseur.
«Il mettra,» disait-il, «un lien de plus entre nous; nos péchés mêmes
seront ensemble.» Tolla accepta le confesseur de Lello. Jamais le jeune
Coromila n'avait été aussi amoureux: il jouissait de son bonheur
provisoire sans songer au combat qu'il faudrait livrer pour le rendre
définitif. Si parfois au milieu d'un doux entretien l'image de son père,
de ses oncles, de ce formidable tribunal de famille, se présentait à son
esprit, il fermait les yeux pour ne pas voir. Lorsque Toto revint à
Rome, dans les premiers jours de décembre, avec Menico parfaitement
guéri, il fut émerveillé de l'harmonie qui régnait entre les deux
amants. Tolla s'était fait peindre en miniature pour se donner à Lello.
Derrière l'ivoire du portrait, elle avait écrit de sa main:
\emph{Aspettando!} De son côté, Manuel avait passé quarante ou cinquante
heures dans l'atelier de M. Schnetz, qui lui avait peint un portrait
magnifique, grand comme nature et plus beau. L'artiste avait
merveilleusement interprété la beauté de Lello, et mis en relief tout ce
qu'il y avait de romain dans sa physionomie. Les deux portraits furent
terminés en même temps, quoique les deux amants ne se fussent pas
entendus, et le jour où Lello apporta le sien à Tolla, croyant la
surprendre, Tolla tira de sa poche sa miniature, encadrée dans un petit
cercle d'or.

Quand ils se rencontraient dans le monde, ils s'y conduisaient avec la
plus grande réserve; ils dansaient rarement ensemble et ne regardaient
qu'à la dérobée. Dans les premiers jours qui suivirent le retour de
Tolla, Manuel se trahit un peu malgré toute sa prudence. Il était d'une
gaieté folle, et la joie lui sortait par les yeux: sa contenance fut
remarquée, et Tolla le pria de veiller sur lui. Alors il s'observa si
bien, il fut si froid, si sérieux et si guindé, que toute la ville se
demanda ce qu'il avait. Tolla revint à la charge et ne lui ménagea pas
les leçons. Enfin, après quelques oscillations, il trouva son équilibre,
et ne ressembla plus ni à une victime ni à un triomphateur.

Mme Fratief et sa fille épiaient avec une persévérance toute féminine
les moindres mouvements de Lello. A leur grand regret, elles étaient
réduites à le surveiller elles-mêmes. Elles avaient perdu leur digne
espion, ce pauvre Cocomero. Il avait quitté la maison le 6 octobre, de
lui-même, et sans qu'on pût savoir quelle mouche l'avait piqué. Nadine
supposait qu'il était retourné à Naples: depuis quelque temps, il
paraissait atteint d'une mélancolie qui ressemblait beaucoup au mal du
pays. La générale inclinait à croire qu'il s'était enrôlé dans
l'honorable corporation des sbires, où l'on ne manquerait pas
d'apprécier ses talents. En attendant qu'il daignât donner de ses
nouvelles, on l'avait remplacé à la maison par un grand lourdaud du
Transtévère, et la générale le remplaçait de son mieux à la ville. Elle
ne rencontrait jamais Lello dans le monde sans lui dire: «Attention!
j'ai l'œil sur vous!» Lello, dûment averti, se surveillait sévèrement,
et prenait la générale en horreur.

Elle s'avisa que Lello n'aimait peut-être Tolla que par amour-propre et
à force d'entendre dire qu'elle était la plus jolie fille de Rome. «Nous
sommes bien sottes,» pensa-t-elle, «de lui avoir laissé faire-cette
réputation-là!» La première fois qu'elle rencontra Tolla, elle lui cria:
«Eh! mon Dieu! ma toute belle, qu'avez-vous? Vous êtes toute défaite!»
Le lendemain, dans une autre maison, elle dit à Mme Feraldi: «Chère
comtesse, pensez donc à la santé de Tolla; elle ne se ressemble plus
depuis quelque temps!» Elle allait répétant à qui voulait l'entendre:
«Est-ce que la plus jolie fille de Rome est malade? Elle se fane de jour
en jour, et ses parents n'ont pas l'air de s'en douter. Savez-vous qui
est son médecin?» Cinq ou six mères de famille, qui avaient des filles à
marier, furent frappées de la justesse des observations de la générale.
Elles virent avec les yeux de la foi que Tolla avait les bras maigres et
la figure fatiguée; elles le dirent sur les toits, et bientôt il ne fut
bruit que du dépérissement de Tolla.

Tolla avait non-seulement cet éclat de santé que les femmes rapportent
de la campagne au commencement de l'hiver, mais encore ce je ne sais
quoi de radieux, de vivace et de bruyant que la bonheur ajoute à la
beauté. Il aurait fallu que Lello fût aveugle pour la croire enlaidie.
Il se contenta de sourire tranquillement le jour où il entendit quelques
bonnes âmes chuchoter autour de lui:

«Regardez donc la Feraldi. Est-elle passée!»

«Pauvre fille! Jaune comme un fruit dans une armoire.»

«Les yeux battus.»

«Les lèvres molles.»

«Il lui reste sa physionomie.»

«Oui; si on lui ôtait cela, elle serait presque laide.»

Mlle Nadine, de son côté, avait dressé une batterie contre la mère de
Tolla. Elle allait disant d'un petit air ingénu qui ne lui seyait pas
mal: «Savez-vous que Tolla est bien heureuse d'avoir une mère comme la
sienne? Cette Mme Feraldi a tant d'esprit que je l'admire. Ce n'est pas
ma pauvre bonne mère qui saura jamais attirer un jeune homme à la
maison, le flatter, le séduire, l'engager, le compromettre, et le
conduire, les yeux bandés, jusqu'à la porte de l'église! Après tout, ma
bonne mère, je t'aime comme tu es, avec ta naïveté sublime. Nous sommes
des sauvages du Nord; mais mieux vaut la barbarie qu'une civilisation
trop avancée. N'envions pas le savoir-faire des habiles, et gardons la
blancheur de nos neiges natales.» Nadine et sa mère, à force de
fréquenter l'église des Saints-Apôtres, acquirent la certitude que Lello
venait tous les soirs au palais Feraldi. La générale se chargea d'en
répandre la nouvelle avec un commentaire de sa façon. «Que vous semble,
disait-elle à toutes les femmes de sa connaissance, d'une mère qui
protégé de pareils rendez-vous? Quand le prince est entré, la grande
porte se ferme, et le concierge, une espèce de brute, n'ouvrirait pas
pour un million. Moi, si un jeune homme était admis à faire sa cour à
mademoiselle ma fille, je laisserais ma porte ouverte à tout le monde.
On ne se cache que pour mal faire. La petite est vraiment à plaindre;
elle aime ce garçon; on l'enfermé avec lui; le moyen qu'elle se défende?
Cependant il est possible que cela tourne à bien. Si le prince
s'avançait si loin, si loin, qu'il lui fût impossible de reculer! On
ferait parler l'honneur, l'amour, la reconnaissance; ne pourrait-on même
pas le contraindre? Toutes les fautes ne sont pas des maladresses, et il
y a souvent plus d'habileté dans un quart d'heure d'oubli que dans dix
années de vertu.» Ces calomnies furent colportées bruyamment dans tous
les salons de Rome. On les fit sonner très haut, dans l'espoir qu'elles
arriveraient aux oreilles de la famille Coromila. Elles furent
recueillies précieusement par trois personnes.

La première était Rouquette, qui s'en ré jouit.

La seconde était le frère de Lello, qui s'en effraya.

La troisième était le colonel, qui s'en amusa.

Le pauvre cardinal n'eut pas le temps d'apprendre ce qu'on disait de son
neveu. Il mourut comme un saint la veille de l'Épiphanie. Rouquette,
devenu le commensal et le confident du colonel, remercia intérieurement
les alliés inconnus qui secondaient si bien ses projets. Le vieux
prince, relégué par ses infirmités au fond de son palais, n'apprenait
que les nouvelles qu'on jugeait à propos de laisser arriver jusqu'à lui.
Son fils aîné voulait tout lui dire; il craignait que Lello ne fût
véritablement livré aux mains d'une famille d'intrigants; mais Rouquette
et le colonel le détournèrent de ce dessein.

«Qu'espérez-vous de l'intervention du prince?» lui demanda Rouquette.

«Mon père lui défendra de retourner chez cette fille.»

«Obéira-t-il? Oui. Mon père a beau être vieux, infirme, aveugle, plus
semblable à un mort qu'à un vivant; sa volonté est inflexible, et Lello
tremble encore devant lui. Il obéira.»

«Soit; je suppose qu'il se montre plus soumis que vous ne l'avez été en
pareille circonstance: le prince n'est malheureusement pas éternel. Si
Lello consent à oublier pour quelque temps qu'il est majeur et maître de
sa personne, il s'en ressouviendra à la mort de son père, et vous ne
saurez plus par quel frein le retenir. Gardez-vous d'élever la volonté
du prince entre lui et celle qu'il aime; le jour où la mort renverserait
la barrière, votre prisonnier vous échapperait, et pour toujours.»

«Il a raison,» ajouta le colonel. «D'ailleurs ton projet nous attirerait
des scènes de famille, des larmes, des prières et un débordement de
rhétorique dont je bâille à l'avance. Nous agirons quand il en sera
temps; rien ne presse.»

Mme Fratief, qui était pressée, dit un jour à la chanoinesse de Certeux:
«Chère madame! on ne parle dans Rome que de l'esprit d'un de vos
compatriotes, monsignor\ldots{} monsignor\ldots{} Ach! J'ai perdu son
nom. Ce monsignor qui a empêché un prince Coromila de se mésallier à
Venise\ldots»

«Monsignor Rouquette?»

«Précisément. Monsignor de Rouquette. Vous qui recevez la fine fleur de
la société romaine, dites-moi donc, chère madame, si monsignor de
Rouquette a autant d'esprit qu'on veut bien lui en prêter?»

«Vous n'avez jamais causé avec lui?»

«Je n'ai jamais pu le joindre, et notez que j'en meurs d'envie.»

«Si vous étiez assez aimable pour venir prendre le thé ce soir avec moi,
je vous servirais monsignor Rouquette entre la première et la deuxième
tasse.»

«Ah! chère madame, vous êtes ma bonne étoile. Figurez-vous que Nadine et
moi nous importunons le ciel depuis quinze jours pour qu'il nous envoie
monsignor de Rouquette.»

Nadine ajouta d'un petit ton dévot: «Ceci nous prouve, maman, que pour
obtenir de Dieu ce qu'on désire, il faut recourir à l'intervention des
saints.»

Lorsque Rouquette fut en présence de la générale, il devina aux premiers
mots un auxiliaire intéressé et compromettant. Il résolut de s'en amuser
et de s'en servir.

Elle crut être fort habile en commençant par, le féliciter de la cure
merveilleuse qu'il avait faite sur le frère de Lello: de l'aîné au
cadet, la transition serait aisée; mais Rouquette se défendit
énergiquement contre les éloges qu'elle prétendait lui faire accepter.
«Ce n'est pas moi,» dit-il, «qui ai guéri le fils aîné du prince
Coromila; tout l'honneur de la cure appartient à Dieu et au bon naturel
du malade. La famille Coromila ne périra point par les mésalliances.»

«Ah! monsignor, vous me rassurez. On disait que le prince Lello était en
grand danger.»

«Je vous assure, madame, qu'il se porte le mieux du monde.»

«L'air des jardins Feraldi est dangereux le soir, et les pauvres cœurs y
prennent la fièvre.»

«Dieu a fait l'homme plus robuste que la femme, et il arrive que l'un
reste en santé, tandis que l'autre tombe malade.»

«L'église a bien raison de défendre les jugements téméraires. L'homme
est si prompt à accuser son prochain! On parle quelquefois de serments
échangés, de promesses de mariage, d'anneaux passés au doigt, de
portraits donnés et reçus, quand il n'y a peut-être rien de vrai que
quelques baisers.»

«Le monde est encore plus méchant que vous ne croyez, madame. On va
souvent jusqu'à inventer des histoires de mariage secret.»

«Vraiment!»

«De promenade nocturne en tête à tête.»

«A pied?»

«Mieux, madame; en voiture.»

«Je n'avais jamais entendu conter pareille chose!»

«Avez-vous entendu parler d'un père et d'une mère complices d'un mariage
clandestin et forcés de cacher la grossesse de leur fille?»

«On dit cela ?»

«Souvent, madame, tant il y a de méchanceté en ce monde! Mais les hommes
de bon sens laissent tomber ces calomnies.»

«Je ne les laisserai pas à terre,» pensa la générale.

«Elle les ramassera,» se dit Rouquette.

La chanoinesse vint se mêler à la conversation. «Vous parliez mariage?»
demanda-t-elle à Rouquette.

«Hélas! madame, répondit-il, de quoi parlerait-on dans un pays où
l'amour, et par conséquent le mariage, est le seul intérêt de la vie
après le salut?»

«On dit que votre compagnon de voyage épouse la fille d'un lord
catholique?»

«On l'espère. Si les négociations réussissent, le mariage se fera à
Londres au mois de mai.»

«Est-ce à Londres aussi, demanda en sou riant la chanoinesse, que vous
comptez marier Lello?»

«Qui sait?\ldots{} Certes, si j'étais à sa place, je chercherais une
femme partout, excepté à Rome.»

«Pourquoi? Vous pouvez parler hardiment: tous les Romains sont partis,
et ce n'est ni la générale ni moi qui irons vous dénoncer.»

«Oh! madame, je n'ai rien contre les Romains ni contre les Romaines;
mais à mes yeux Rome est le pays du monde où les hommes mariés ont le
moins d'avenir. A Paris, à Pétersbourg, à Londres, l'homme qui se marie
épouse toute une armée de protecteurs, d'amis, de partisans, qui
s'engagent par contrat à le faire parvenir. A Rome, il épouse une femme
et rien de plus. Il y a tels mariages qui vous donnent en France la
croix et une place de préfet, en Angleterre la députation, en
Russie\ldots»

«En Russie, ajouta vivement la générale, une clé de chambellan, la
noblesse de deuxième classe, des croix, des pensions, des places, la
faveur, la fortune, et tout.

«Vous voyez bien, mesdames, que Rome est le patrimoine des célibataires,
et que les hommes mariés doivent chercher fortune ailleurs.

«La France,» dit la générale, «est un pays sans avenir. Ces messieurs de
1830 ont tout mis sens dessus dessous, les lois et les pavés. Qu'est-ce
qu'un député? Un homme qui n'a pas même d'uniforme! On parle des pairs
de France: ont-ils seulement le droit de bâtonner leurs gens?
L'aristocratie est tombée bien bas, de puis la suppression du droit
d'aînesse.»

«Le droit d'aînesse s'est conservé en Angleterre. L'Angleterre est
encore bonne.»

«Oui; mais combien trouvez-vous de familles catholiques dans la noblesse
anglaise? On les compte, cher monsignor, on les compte. Vous avez eu le
bonheur de découvrir un beau parti dans cette petite élite du royaume,
raison de plus pour n'y en pas chercher un second.»

«Reste donc la Russie. Par malheur, elle est schismatique.»

«Schismatique, monsignor! La Russie n'est pas schismatique. Jamais on
n'a dit que la Russie fût schismatique. Il y a des schismatiques en
Russie, j'en conviens, mais beaucoup moins qu'on ne pense. Est-ce que
toute la Pologne, sans aller plus loin, n'est pas catholique? L'empereur
est le plus tolérant des hommes; il est le père de tous ses sujets, sans
distinction: on ne l'a jamais accusé de favoriser les schismatiques. Que
mademoiselle ma fille arrive demain en Russie, soit avec sa mère, soit
avec son mari, sera-t-elle moins bien reçue à la cour, parce qu'elle est
catholique? Dites, madame la chanoinesse, si le marquis votre frère a dû
se faire schismatique pour arriver aux premières dignités de l'empire?»

«On m'a conté,» reprit modestement Rouquette, q«u'en Russie les filles
ne recevaient que le quatorzième de l'héritage de leurs parents.»

«Distinguons, cher monsignor. En effet, elles n'héritent que du
quatorzième lorsqu'elles ont des frères; mais une fille unique, comme
Nadine par exemple et tant d'autres héritières, ne partage le bien de
ses parents avec personne.»

«Au reste nous avons à Rome des jeunes gens assez riches pour prendre
une fille sans dot.»

«Bien, monsignor! Vous êtes un homme antique. Vous ne donnez pas, vous,
dans le travers ridicule des hommes d'aujourd'hui! Je ne connais rien
d'impatientant comme cette question: ‹Qu'a-t-elle!› Eh! mes chers
messieurs, ma fille a ce qu'elle a: épousez-la pour elle, ou je la
garde. Je vous dirai le lendemain du mariage si elle est sans un sou ou
si elle a dix millions.»

A ce chiffre de dix millions, Rouquette prit un air si respectueux que
la générale se persuada qu'il était dupe. «Décidément, madame,» dit-il
en terminant, «je crois que si je m'appelais Manuel Coromila, je
choisirais ma femme en Russie. Par malheur je ne suis rien, qu'un homme
de bon conseil.»

«Il va travailler Lello!» se dit la générale ivre d'espérance.

«Elle court perdre les Feraldi,» pensa Rouquette en la voyant sortir.

Huit jours après, il n'était bruit que du mariage secret de Lello et de
Tolla. On citait le jour, l'heure, la chapelle, le prêtre et les
témoins. Ces détails d'une précision inquiétante émurent le frère de
Manuel: il lui de manda s'ils étaient vrais, et ne voulut croire ses
dénégations que lorsqu'elles furent confirmées par Rouquette.

Tolla n'ignora pas longtemps les calomnies que la Fratief avait mises en
circulation. Un matin que Mme Feraldi réunissait chez elle quelques
jeunes filles de la société et quelques amis de Toto pour répéter
ensemble une mazurka, les deux cousines de Tolla vinrent la féliciter de
son mariage.

«Quel mariage?» demanda-t-elle en rougis sang jusqu'aux yeux.

«C'est bien mal à toi, Tolla, de n'en avoir rien dit à tes bonnes
cousines!»

«Ah! ah! ah! qu'elle est étonnante avec son air étonné!»

«Nous n'aurions pas dû être les dernières à apprendre ton bonheur.»

«Figure-toi que j'arrive dimanche dans une maison: la première chose
qu'on me dit, c'est que tu es la femme de Lello. Moi, je me mets à rire
et je trouve la plaisanterie assez Je sors, je rencontre Bettina Negri
et sa mère à la porte d'une église; elles m'arrêtent pour me dire: ‹Eh
bien! vous avez un nouveau cousin!› ‹Bah! est-ce que ma tante Feraldi
est accouchée?› ‹Non, mais Tolla s'est mariée avec Lello.› Enfin hier
maman reçoit la plus drôle de lettre du monde. On lui écrit de Forli:
‹Votre nièce est mariée, nous le savons; il n'est pas question d'autre
chose dans la ville: contez-nous donc les détails de l'aventure!›»

Tolla resta muette d'étonnement: après avoir pris tant de soin pour
cacher son amour, elle se voyait la fable de la ville et de la province.

Toto vit d'un coup d'oeil que tous les témoins de cette scène avaient
déjà entendu parler de ce prétendu mariage, et qu'ils y croyaient. Il se
hâta de répondre pour sa soeur: «On vous a trompées, mes chères
cousines, et si l'on répète devant vous cette sotte invention de nos
ennemis, vous pourrez répondre haute ment que Tolla n'est pas mariée.»

Tolla ajouta avec une indignation mal contenue: «Et qu'elle n'est pas
fille à accepter la honte d'une semblable union, et qu'elle méprise un
bonheur clandestin, et qu'elle ne voudrait pas d'un roi même à ce prix,
et qu'elle ne s'avilira jamais au point d'accepter la main d'un homme
qui craindrait de l'épouser à la lumière du soleil et à la face de
tous!»

Les deux cousines s'excusèrent à qui mieux.

«Pardon,» dit Philomène, «je ne voulais pas je chagriner; mais comme
tout le monde parle de ce mariage, je croyais\ldots{} Pardon\ldots»

«Mais es-tu simple,» dit Agate, «de pleurer pour si peu de chose! Et
quand cela serait vrai! Les mariages secrets sont aussi bons que les
autres, du moment où le prêtre y a passé, et ils sont bien plus
amusants!»

Le soir, Lello vint avec Philippe. Ils trouvèrent Tolla tout en larmes,
et elle leur ra conta ce qu'elle avait appris.

«C'est une invention de la Fratief,» dit Lello. «Il y a huit jours que
cela court la ville. Mon frère m'en a parlé.»

«Et qu'as-tu répondu?» demanda Tolla.

«J'ai répondu que la voix publique avait menti, et que je n'aurais pas
fait un tel pas sans consulter mes parents.»

«Tu ne lui as rien dit de nos engagements? Il serait peut-être temps
d'en instruire ta famille.»

«Mon cher amour, mon père est plus mal que jamais depuis la mort du
cardinal. Si par hasard on l'avait prévenu contre nos projets, la
déclaration que j'ai à lui faire pourrait lui porter un coup terrible.
Ne vaut-il pas mieux attendre que sa santé soit raffermie, si tant est
qu'il puisse guérir?»

«Attendons,» dit Tolla. «Je me boucherai les oreilles pour ne pas
entendre les calomnies de nos ennemis.»

«Faites mieux,» ajouta Pippo. «On vous accuse d'être mariés secrètement.
A votre place, je voudrais donner raison à ces chers accusateurs.
Voulez-vous que vous trouve un prêtre? Je serai votre témoin avec
quelque ami sûr et discret. Supposé que la chose transpire, personne n'y
croira. La nouvelle est usée: elle date de huit jours. D'ailleurs est-ce
qu'on croit jamais la vérité?»

«Qu'en penses-tu, Tolla?» demanda Manuel.

Tolla lui répondit d'une voix ferme et décidée: «Mon ami, hier peut-être
j'aurais dit oui. Après la scène de ce matin, je me mépriserais
moi-même, si j'étais capable d'accepter. Nous attendrons.»

Manuel et Philippe restèrent au palais Feraldi jusqu'à minuit. Le
lendemain, on racontait dans Rome que Tolla et Lello étaient sortis
ensemble à la brune. Une personne digne de foi les avait reconnus dans
les allées du Pincio, appuyés tendrement l'un sur l'autre. Un second
témoin les avait rencontrés en carrosse à cent pas de la porte du
Peuple; un troisième les avait surpris dans une petite voiture basse sur
la route qui mène à l'église Saint-Paul; un quatrième les avait aperçus
à cheval sur la route d'Albano. Un autre ne les avait pas vus, mais il
avait fait parler le cocher qui les conduisait tous les soirs. Ces
témoignages, qui auraient dû se détruire, se confirmaient l'un l'autre.
On aimait mieux croire à l'ubiquité de Tolla qu'à son innocence. Une
ligue redoutable se forma contre elle. Toutes les mères qui l'avaient
enviée, toutes les filles qui l'avaient jalousée, tous les jeunes gens
qui l'avaient désirée, s'enrégimentaient sous les ordres de la Fratief.
Les amis qui pouvaient la défendre, comme la marquise, Pippo, le docteur
Ely, étaient accablés par le nombre. La pauvre fille apprenait tous les
jours quelque nouvelle calomnie: elle s'en consolait en la racontant à
Lello, qui promettait de lui payer en bonheur tout ce qu'elle avait à
souffrir.

Dans les derniers jours de janvier, les consolations de son amant lui
manquèrent. Le vieux prince entrait dans son agonie, qui dura près de
trois semaines. Lello, cloué au chevet de son père, trouvait à peine le
temps d'écrire tous les jours un billet à Tolla. Elle n'avait plus
personne à qui confier ses ennuis: pouvait-elle apprendre à sa mère
toutes ces calomnies, ou sa mère était plus maltraitée qu'elle-même?

Elle s'associait à la douleur de Lello, et quoiqu'elle n'eût jamais vu
le prince Coromila, elle le pleurait comme un père. Elle ne songea pas
un seul instant que la mort de ce vieillard assurait son mariage. Le
prince mourut. Tolla fut trois ou quatre jours sans aller dans le monde:
elle se sentait incapable de retenir ses larmes. Le monde murmura. Si on
l'avait vue sourire et valser; on aurait poussé les hauts cris; on
aurait dit qu'elle triomphait de la mort du prince.

Lello, toujours prudent, lui écrivit le lendemain des funérailles de son
père: «J'apprends qu'hier soir on a remarqué ton absence au théâtre. Que
cela te serve de leçon pour l'avenir.»

C'était Mme Fratief qui avait pris la peine de courir de loge en loge à
la recherche de Tolla: «Avez-vous vu Tolla? Non. Comment n'est-elle pas
ici? elle qui adore la musique de Bellini! J'avais quelque chose à lui
dire. Je vais passer chez elle après le spectacle; mais j'y pense! je ne
la trouverais pas. Elle a quelqu'un à consoler.»

On savait cependant que Lello passait la soirée en famille.

Pour excuser sa douleur, Tolla dit qu'elle était malade. Cela n'était
qu'un demi-mensonge: la pauvre fille succombait à l'excès de ses ennuis.
Ses ennemis la prirent au mot, et glosèrent sur sa maladie.

La jeune Nadine disait ingénument à toutes les filles de son âge:
«Tâchez donc de savoir quelle est la maladie de Tolla. Ma mère la sait,
mais elle ne veut pas me le dire. Il paraît que c'est une maladie que
les jeunes filles n'ont jamais, dont on ne meurt pas, mais qui dure bien
des mois.»

En apprenant cette nouvelle invention, Tolla guérit de colère: elle
sentit ses forces doublées; tout son être s'exalta, toute son énergie se
tendit. Elle retourna dans le monde, courut les théâtres, les bals, les
soirées, dansa des nuits entières, fatigua ses valseurs, soupa à quatre
heures du matin, but du vin de Champagne, oublia sa pelisse en sortant
du bal, commit imprudence sur imprudence, et prouva une santé de fer.

Sa réputation n'y gagna rien. Les uns disaient: «C'est pour mieux cacher
\emph{son état}.» «Mais,» s'écriait la marquise Trasimeni, «elle a une
taille à prendre dans la main! Croyez-vous qu'elle puisse laisser
\emph{son état} à la maison?»

D'autres allaient chuchotant: «Elle ne se ménage pas assez pour une
fille qui relève de maladie.» Un plaisant remarquait la coïncidence de
la mort du prince et de la retraite momentanée de Tolla. «Les Coromila
se conservent bien, disait-on. S'il en meurt un, vite il en naît un
autre. Coromila est mort, vive Coromila!»

Mme Fratief, en voyant valser Tolla, disait charitablement à ses
voisines: «La malheureuse! elle veut donc tuer deux personnes à la
fois!»

Cependant Lello s'était laissé conduire à la villa d'Albano, où ce qui
restait de la famille se retira pendant quinze jours pour cacher sa
douleur et pour l'oublier. On chassait, on faisait de grandes cavalcades
et de longs repas. Rouquette organisa savamment cette vie oisive,
décente et plantureuse. Lello eut le temps, non pas d'envier, mais
d'entrevoir les douceurs de la vie de garçon. Cependant le voisinage de
Lariccia, les souvenirs de l'été dernier, peut-être même l'oisiveté, la
solitude et la bonne chère ravivèrent son amour pour Tolla Un soir, en
sortant de table, il lui écrivit: «Je te l'ai dit cent fois, mais je
veux te l'écrire, parce que les écrits restent: je t'aimerai toujours,
et je saurai mourir plutôt que d'oublier un ange tel que toi. Dieu voit
mon cœur, et en sa présence je te jure une fidélité éternelle.»

«Comme il m'aime!» s'écria Tolla lorsqu'on lut cette lettre en famille.

«Voilà un écrit précieux,» ajouta Toto. «Ne le perds pas, ma fille. Si,
après un pareil serment, il refusait de t'épouser, le pape l'y
forcerait.»

Les Coromila revinrent à Rome au commencement de mars, et Lello reprit
sa place à la fenêtre du palais Feraldi. Après un mois d'un bonheur
presque parfait, malgré le déchaînement de la haine et de la calomnie,
il se montra triste et préoccupé.

«Qu'as-tu?» lui demanda Tolla en le regardant jusqu'au fond de l'âme.

«Rien. Des ennuis de famille.»

«Tu as tout déclaré à tes parents?»

«Non.»

«Ils t'ont parlé de moi?»

«Non.»

«Quels ennuis peux-tu avoir? Tu es majeur, libre, maître absolu de tes
actions, riche\ldots{} Moins que tu ne penses.»

«Tant mieux! Je voudrais que tu n'eusses rien; je serais sûre d'habiter
notre petit domaine de Capri. Te souviens-tu de Capri? Voyons si tu as
profité de mes leçons de géographie! Capri est bornée au nord par
l'amour, à l'est par la fidélité, à l'ouest par beaucoup
d'enfants\ldots{} Ton père t'a donc déshérité?»

«Peu s'en faut.»

«Quel bonheur!»

«Il a laissé un fidéicommis à mon oncle.»

«Le joli mot! Il veut dire?\ldots»

«Que par suite d'un ordre secret de mon père, dont le testament ne dit
pas un mot, et dont l'exécution est confiée à mon oncle, mon frère aîné
sera cinq fois plus riche que moi.»

«Ainsi, mon pauvre ami, tu n'auras peut-être pas plus de deux millions?»

«Peut-être.»

«Alors, viens à Capri; je te promets pour cent millions de bonheur!»

Lello mentait, et l'argent n'était pour rien dans sa tristesse. Son père
n'avait fait ni fidéicommis, ni substitution; il avait légué au
chevalier une terre magnifique qui devait naturellement se partager
entre les deux frères après la mort de leur oncle.

La vraie cause du chagrin, de l'embarras ou du remords de Lello, la
voici. Le fils aîné du vieux Louis Coromila, de venu prince depuis la
mort de son père, avait terminé les négociations relatives à son
mariage; son départ était fixé au 30. avril. Il devait s'embarquer à
Civita-Vecchia pour Marseille, traverser la France, séjourner à Paris,
arriver à Londres pour les fêtes du couronnement de la reine Victoria,
et revenir avec sa femme par la France, la Belgique, l'Allemagne et la
Lombardie. Tous les jours on travaillait devant Lello à compléter, à
préciser et à embellir ce séduisant itinéraire. Le chevalier et
Rouquette ne s'occupaient pas d'autre chose, tandis que le jeune prince
enrégimentait sa suite et commandait sa livrée. Toutes les tables de la
maison étaient couvertes de grandes cartes routières; on voyait des
guides étalés sur tous les meubles. A chaque repas, Rouquette s'étendait
complaisamment sur la description des plaisirs de Paris. Le chevalier
répliquait par le tableau des magnificences de la cour de Londres. Le
prince, quoiqu'il dût se faire habiller à Paris, commanda à Rome son
habit de cour, dont Lello rêva plus de trois nuits. Rouquette était du
voyage; il eut aussi de longues conférences avec son tailleur. Ni le
chevalier ni le prince ne firent aucune proposition à Lello; mais on
démontrait devant lui que cette longue odyssée ne durerait pas beaucoup
plus de deux mois. Le chevalier plaisantait légèrement sur l'esprit
casanier, sur les animaux à coquille et sur les souriceaux qui n'osent
sortir de leur trou. Le prince se promettait de savourer bien mieux les
douceurs de la vie domestique après un temps de voyages et d'aventures.

Ces plaidoiries indirectes se prolongèrent jusqu'aux premiers jours
d'avril. Peut-être la famille aurait-elle perdu son procès, si Tolla
avait eu un grain de coquetterie; mais le bon heur de Lello était trop
pur et trop égal pour qu'il s'effrayât d'une absence de deux mois.

Sur ces entrefaites, Morandi fit écrire à la comtesse qu'il avait vu sa
fille à Lariccia vers le milieu du mois de septembre, qu'il l'avait
trouvée plus belle que tous les portraits qu'on lui en avait faits, et
que si Tolla n'avait refusé sa main que par crainte de quitter Rome, il
était prêt à déserter Ancône pour la capitale.

Victor Feraldi voulait qu'on fît lire cette lettre à Lello; Tolla s'y
opposa formellement. «Une semblable confidence,» dit-elle, aurait l'air
d'une menace.» Cependant la jalousie serait venue fort à point pour
aiguillonner l'amour de Lello, et pour ramener son esprit, qui s'égarait
à chaque instant vers la France et l'Angleterre.

Tolla s'en doutait si peu, qu'elle employait une partie de ses soirées à
lui apprendre le français. Les progrès n'étaient pas rapides: le
professeur et l'élève s'embrouillaient à qui mieux dans la conjugaison
du verbe \emph{aimer}. Quelquefois, pour faire trêve à la grammaire,
elle ouvrait un livre français, le lui mettait sous les yeux, et le
contraignait doucement à épeler, à lire et à traduire. A la fin de la
leçon, l'écolier reconnaissant embrassait son dictionnaire. Un soir, ils
lurent ensemble la fable des Deux Pigeons. Quand Manuel eut achevé
laborieusement le mot à mot, Tolla lui ôta le livre des mains et
traduisit la fable entière en vers libres ou plutôt en prose cadencée;
sa voix, sonore et brillante, avait je ne sais quoi de doux, de tendre
et de profond. Lello regardait voler ses paroles harmonieuses; il
croyait voir cette filleule des fées qui n'ouvrait jamais la bouche sans
laisser tomber des perles et des émeraudes. Lorsque Tolla lui prit la
main en traduisant ces beaux vers:

\begin{quote}
Amants, heureux amans, voulez-vous voyager?\\
Que ce soit aux rives prochaines!\\
Soyez-vous l’un à l’autre un monde toujours beau,\\
Toujours divers, toujours nouveau:\\
Tenez-vous lieu de tout; comptez pour rien le reste!\\
Il baissa la tête et fondit en larmes.
\end{quote}

Le matin même, en sortant de la messe, son oncle lui avait dit: «J'ai un
remords».

«Vous, mon oncle!»

«Oui. Je suis un mauvais parent. Ton frère va partir pour Londres, et je
reste à Rome, au lieu de l'accompagner. Je sacrifie mes devoirs à mes
habitudes.»

«Votre conscience est trop scrupuleuse. Est-ce que mon frère a besoin
qu'on le mène par la main? N'est-il pas assez grand pour se conduire
lui-même?»

«Oui, parbleu! S'il allait là-bas pour son plaisir, je resterais ici
pour le mien, et je me contenterais de lui souhaiter un bon voyage; mais
il part pour se marier, et je rougis de penser que l'héritier de la plus
grande maison d'Italie s'en ira à l'église sans un père, sans un oncle,
sans un frère, et seul de sa famille, comme un enfant trouvé. Si j'avais
seulement dix ans de moins, je ferais mes malles.»

«Mais, mon cher oncle, vous vous portez bien, Dieu merci! et vous n'êtes
aucunement cassé. D'ailleurs Londres n'est pas si loin, et l'on peut
voyager à petites journées.»

«Eh! crois-tu bonnement que ce soit le voyage qui m'épouvante? Non, non;
je n'ai pas peur d'une ou deux traversées sur un bon bateau, et de
quelques centaines de lieues en chaise de poste. La belle affaire pour
un homme bâti comme moi! Ce qui me tuerait, mon ami, ce sont les
plaisirs.»

«Les plaisirs!»

«Oui, les plaisirs. Tu es né à Rome, et tu n'as jamais quitté cette
terre de bénédiction; tu ne peux donc pas te faire une idée de la vie
dévorante qu'on mène à Londres et à Paris. Déjeuner en ville, dîner en
ville, spectacle le soir, bal après le spectacle, rentrer chez soi rompu
de fatigue et trouver sur sa table tout un volume d'invitations pour le
lendemain; s'habiller trois fois par jour, s'exténuer en visites, se
ruiner en compliments; attirer sur soi les regards de tout un peuple,
être l'événement du jour, le favori de la mode, la curiosité de la
saison; s'observer, se surveiller, poser enfin comme un acteur sur la
scène ou un prédicateur en chaire: est-ce une vie pour un homme de mon
âge, et ne vois-tu pas que je succomberais au bout d'un mois?»

«Mais, mon oncle, un bon dîner ne vous fait pas peur; vous allez au
théâtre tous les soirs; on ne donne pas un bal sans vous inviter, et
vous ne vous en portez pas plus mal.»

«Pauvre garçon! Est-ce qu'on dîne à Rome? On y prend de la nourriture.
Tu ne soupçonneras jamais toutes les sorcelleries de ces cuisiniers
français, leurs terribles friandises qui séduisent les yeux, captivent
l'odorat et centuplent l'appétit; la gaieté diabolique qui pétille au
milieu de ces repas, le fracas des bouchons qui sautent au plancher, le
cliquetis des verres entassés pêle-mêle devant chaque assiette, l'éclat
des cristaux, la lumière éblouissante des bougies, la variété
désespérante des vins: c'est un enfer, te dis-je, et j'en reviendrais
brûlé jusqu'aux os. Vive la bonne grosse cuisine italienne que nous
mangeons sans bruit dans la vieille argenterie de nos pères! Vivent nos
théâtres simples et tranquilles, où l'on ne va que pour entendre de la
musique et pour causer dans l'ombre avec ses amis! Ce maudit opéra de
Paris est une fournaise tumultueuse où les plus jolies femmes du monde
vont étaler leurs épaules nues sous un lustre pire que le soleil. Et les
bals, bonté divine! qu'ils ressemblent peu à nos jolies petites soirées
égayées par la contredanse, le whist et la limonade! Figure-toi un
formidable pêle-mêle de luxe, d'élégance et de coquetterie, une musique
insensée, des toilettes scandaleuses, une liberté inouïe, des escaliers
encombrés de fleurs, des buffets chargés de viandes, des soupers à
ressusciter les morts et à tuer les vivants! C'est un spectacle à voir
une fois; je l'ai vu, je n'en suis pas mort, mais on ne m'y reprendra
plus! Cependant Dieu m'est témoin que je voudrais pouvoir accompagner
ton frère.»

Cette appétissante satire des plaisirs de Paris produisit tout l'effet
qu'on en espérait: Manuel offrit de partir avec son frère. Le mot ne fut
pas plus tôt lâché, que le colonel, sans lui laisser le temps de se
reconnaître, courut avec lui annoncer la nouvelle à toute la maison. Le
hasard ou la prévoyance de Rouquette fit qu'il y eut ce jour-là vingt
personnes à dîner. Tout le monde but au prochain voyage des deux frères.
Lello était venu au palais Feraldi pour apprendre à Tolla ce que toute
la ville devait savoir le lendemain; mais la fable des deux pigeons lui
coupa la parole, et il pleura en songeant qu'il s'était condamné à
partir et qu'on lui avait fermé toute retraite.

Il se coucha mécontent de lui-même, incertain de ce qu'il dirait à Tolla
et fort en peine de se justifier à ses propres yeux. A force de
chercher, il s'avisa de prier Mme Feraldi de tout conter à sa fille. «Le
coup sera moins rude,» se dit-il, «s'il ne vient pas de moi.» Pour faire
sa paix avec sa conscience, il se promit qu'une fois hors de Rome il
trouverait le courage de demander le consentement de son oncle. Vingt
fois il avait eu la bouche ouverte pour lui tout déclarer, et une sotte
timidité l'avait toujours arrêté devant le nom de Tolla. «C'est la
présence de mon oncle qui me trouble,» pensa-t-il; «je serai plus hardi
en face d'un encrier.» Il s'endormit fort tard et rêva qu'il était un
pigeon battu par l'orage. Il fut réveillé à neuf heures du matin par la
visite de Rouquette.

«C'est vous?» lui dit-il en se frottant les yeux. «Je suis bien aise de
vous voir. Connaissez-vous la fable des deux pigeons?»

«Je la sais par cœur. C'est un délicieux roman de trois pages. La morale
surtout en est admirable.»

«Vous trouvez?»

«Sans doute, et je vous recommande de la méditer. Cette fable prouve,
mieux qu'un sermon, que deux frères ne doivent pas voyager l'un sans
l'autre.»

«Deux amants?»

«Deux frères!»

«J'avais entendu dire qu'il s'agissait de deux amants.»

«Qui est-ce qui vous a fait cette plaisanterie? Il n'y a pas plus
d'amour dans la fable que dans la barrette du cardinal-vicaire. Écoutez
plutôt:»

\begin{quote}
L’autre lui dit: Qu’allez-vous faire?\\
Voulez-vous quitter votre frère?
\end{quote}

Et plus loin:

\begin{quote}
…Hélas! dirai-je, il pleut:\\
Mon frère a-t-il tout ce qu’il veut,\\
Bon souper, bon gîte, et le reste?
\end{quote}

Mon \emph{frère,} entendez-vous? D'ailleurs qui est-ce qui dirait
\emph{et le reste,} sinon un frère? Et le frère répond:

\begin{quote}
Je reviendrai dans peu conter de point en point:\\
Mes aventures à mon frère.
\end{quote}

«Croyez-vous, en bonne foi, que, s'il s'agissait de deux amants, les
Français feraient apprendre ces vers aux petites filles? Au reste, La
Fontaine connaît trop bien le cœur humain pour vouloir que deux amants
demeurent cousus l'un à l'autre. Il sait que l'amour le mieux constitué
ne résisterait pas à ce régime, et mourrait d'ennui au bout de quelques
mois. L'absence, qui tue l'amitié et tous les sentiments tièdes, exalte
les passions violentes. Quelle est la femme qui a donné au monde le plus
éclatant exemple de fidélité? Pénélope, dont le mari a fait une absence
de vingt ans. Lucrèce a repoussé l'amour de Sextus parce que son mari
était au camp; elle l'aurait peut-être écouté, si elle avait eu Collatin
sur ses talons. C'est en amitié que les absents ont tort: en amour, ils
ont toujours raison. La petite fleur qui dit plus je vous vois, plus je
vous aime, est un oracle en amitié; c'est une sotte en amour.»

Fortifié par ces beaux raisonnements, Manuel vint à trois heures au
palais Feraldi. On venait de quitter la table. Le comte, la comtesse et
Toto prenaient le café au salon. Tolla s'habillait pour faire des
visites. Il promena sur ses auditeurs un sourire embarrassé.

«Je suis bien aise,» dit-il, «que Tolla ne soit pas ici. C'est à vous
que je viens demander assistance.»

«Et contre qui?» dit le comte.

«Contre elle. Si vous ne venez pas à mon aide, elle m'arrachera les deux
yeux tout au moins.»

«Mon cher client, l'affaire n'est pas de ma compétence. Défendez vos
yeux vous-même, si vous tenez à les garder.»

«Si j'y tiens, c'est qu'ils me servent à voir Tolla.»

«Voici bientôt un an qu'elle vous les arrache tous les jours,» reprit la
comtesse, «et vous n'êtes pas seulement borgne.»

Toto ajouta: «Avec tous les yeux qu'elle ta arrachés, on aurait de quoi
paver la queue d'un paon. Voyons, confesse-toi: qu'as-tu fait?»

«Rien encore; mais je médite une escapade.»

«Renonce à ton escapade, et je réponds de tes yeux.»

«Impossible, mon ami; j'ai donné ma parole. Il s'agit d'un voyage.»

«A Albano?»

«Plus loin; mais il est convenu que nous courrons la poste, et que notre
absence ne durera pas longtemps.»

«Huit jours?»

«Davantage. Enfin, puisque j'ai commencé ce diable d'aveu, sachez que
mon oncle, bien malgré moi, pour que mon frère ne soit pas seul à ce
mariage, a voulu, ne pouvant pas quitter Rome, où il a ses habitudes, me
faire partir pour Londres, et il m'a été impossible de refuser. Vous
comprenez que si Tolla\ldots»

Il n'eut pas le temps d'achever sa phrase. Toto, le comte et la comtesse
s'étaient dressés comme par ressort autour de lui.

«Vous êtes faible, Lello Coromila,» dit sévèrement le comte.

«Lâche cœur!» cria Toto.

«Elle en mourra!» dit la comtesse.

«Écoutez-moi,» reprit-il d'une voix émue. «Je vous jure que j'aime
Tolla, et que je l'épouserai. Maintenant écoutez-moi. Mon oncle et mon
frère, qui sont toute ma famille, désirent absolument que je fasse ce
voyage. Je souffre plus que vous ne sauriez croire à la seule pensée de
quitter Rome; mais je voudrais concilier tous mes devoirs. Si je
témoigne de la com plaisance à mes parents, je puis compter qu'ils me
paieront de retour. J'assiste au mariage de mon frère pour que bientôt
il assiste au mien.»

«Monsignor Rouquette n'est-il pas de la partie?» demanda le comte. «Il a
obtenu du cardinal-vicaire un congé de trois mois.»

«Cela vous prouve,» répliqua vivement Manuel, «que notre absence ne sera
pas longue: trois mois au plus, peut-être deux.»

«Combien de temps,» demanda Toto, «a duré son voyage à Venise?»

«Je t'assure, mon ami, que l'on calomnie ce pauvre Rouquette. Depuis six
mois que je l'étudie sans qu'il s'en doute, j'ai appris à lui rendre
justice. Il m'aime, et il se rangera plu tôt avec nous contre les miens
qu'avec ma famille contre nous.»

«Puisque vous avez foi en M. Rouquette,» dit la comtesse avec amertume,
«asseyons-nous. Vous avez vu comme la nouvelle de ce départ nous a
agréablement surpris: jugez par nous de l'effet qu'elle va produire sur
Tolla.»

«Chère comtesse, je souffrirai plus qu'elle. Aidez-moi à adoucir la
violence du coup. Je sens que je n'ai plus de courage.»

«Il doit t'en rester assez,» dit Toto, «car tu n'en dépenses guère au
palais Coromila.»

«Eh bien, oui! je suis faible, je suis lâche; j'ai peur de mon oncle,
quoiqu'il soit le meilleur des hommes; j'ai peur de mon frère, j'ai peur
de tout. Accable-moi, tu le peux, je te le permets, je ne me défendrai
pas: il y a des moments où je me méprise moi-même! Mais que veux-tu?
j'ai promis de partir, ma parole est donnée, la ville entière le sait.
Hier, à dîner, devant moi, ils ont annoncé mon départ à plus de vingt
personnes! Tout cela empêche-t-il que je n'aime ta sœur et que je ne
l'épouse à mon retour? La sotte promesse que mon oncle m'a arrachée
viole-t-elle les serments que je vous ai faits?»

Manuel s'arrêta brusquement; il avait en tendu la voix de Tolla, qui
descendait en chantant le grand escalier du palais.

La pauvre fille ouvrit la porte, courut à Lello, et s'arrêta tout
interdite à la moitié du chemin. Elle vit son père terriblement pâle, sa
mère agitée d'un tremblement nerveux, les yeux de son frère pleins de
larmes, la figure de son amant bouleversée. Ils se taisaient tous et
n'osaient ni se regarder ni la regarder. Son cœur se serra; elle se
laissa tomber sur une chaise sans essayer de rompre ce morne silence.
Trois longues minutes s'écoulèrent, durant lesquelles on n'entendit que
les sanglots de Mme Feraldi. Enfin Tolla n'y tint plus.

«Qu'est-il arrivé?» demanda-t-elle; «ma mère, mon père, mon frère,
Lello, qu'avez-vous? Parlez, je vous en prie. J'aurai du courage!
répondez-moi. Maman, je t'en supplie. Ah! vous me ferez mourir! Par
pitié dites-moi ce qui m'arrive!»

«Pauvre enfant!» répondit sa mère, «tu le sauras trop tôt!»

Elle ne demanda rien de plus; elle courut dans la chambre voisine et
fondit en larmes sans savoir encore pourquoi. Ce premier moment passé,
elle reprit possession d'elle-même et rentra résolument au salon.

«J'ai pleuré,» dit-elle. «Vous voyez que je suis calme. Maintenant je
veux savoir ce que je suis condamnée à souffrir.»

Au premier mot de \emph{départ,} elle s'évanouit. Sa mère et Toto la
portèrent dans sa chambre. Le comte la suivit, oubliant Manuel, qui
s'enfuit tout éperdu. En passant devant la loge du concierge, il appela
Menico, lui mit deux écus dans la main et le supplia de lui apporter des
nouvelles de sa maîtresse. Il attendit deux heures dans une anxiété
mortelle. Enfin Dominique parut: il était plus pâle qu'à l'ordinaire,
mais il avait toujours son air calme et indolent.

«Parle vite!» lui cria Manuel. «Comment va-t-elle?»

«Mieux, excellence. Elle a eu de grosses convulsions; maintenant elle
dort: vous l'avez pas tuée tout à fait.» Il ajouta, en posant les deux
écus sur la cheminée: «Voici votre argent. Vous allez voyager, vous en
aurez besoin. Madame vous fait dire que vous pouvez venir au palais
demain soir.»

Le lendemain, en entrant dans ce salon où il avait passé de si douces
heures, Manuel fut saisi d'un frisson étrange. Personne ne se leva pour
venir au-devant de lui. Tolla était trop faible pour courir comme
autrefois à sa rencontre. Le comte et Toto s'étaient habillés comme pour
une cérémonie. On avait enlevé tous les rideaux qui cachaient les vieux
portraits de la famille, et Manuel pouvait compter autour de lui dix
générations de Feraldi. Le comte lui montra de la main le fauteuil qui
l'attendait, puis il commença d'une voix ferme et triste:

«Manuel Coromila, vous voyez que nous sommes ici en conseil de famille.
J'ai convoqué mes ancêtres à cette réunion solennelle: je voudrais
pouvoir convoquer aussi les vôtres. Vous allez quitter Rome pour
longtemps, je dis longtemps, parce qu'il ne faut pas plus d'un mois pour
changer le cœur d'un homme de votre âge. Ce départ, ce n'est pas vous
qui l'avez voulu: il vous a été imposé par votre oncle et votre frère.
Je sais pourquoi. L'ambition de vos parents ne veut pas que vous
épousiez ma fille, et l'on compte sur les plaisirs de Paris et de
Londres pour vous la faire oublier. Vous étiez libre de rester: vous
avez consenti à partir. Vous étiez libre de déclarer ouvertement votre
amour pour Vittoria, depuis tantôt deux mois que vous n'avez plus de
père: vous vous êtes obstiné dans votre prudence et votre timidité. Je
ne vous accuse pas. Je ne vous reproche ni les partis que vous nous avez
fait rejeter, ni l'amour incurable que vous avez mis au cœur de ma
fille, ni les calomnies que vos assiduités ont attirés sur nous, ni la
scène d'hier et la douleur dont vous avez rempli ma maison; mais je
pense que c'en est assez et que nous avons assez souffert. Je vois bien
que vous n'aimez plus ou que vous aimez moins, ou que vous n'aimez
pas\^{}assez pour que l'amour vous donne du courage. Votre constance ne
tient plus qu'à un fil, et, sans toutes ces promesses et tous ces
serments qui vous sont échappés, la pauvre tout serait déjà oubliée. Eh
bien! soyez heureux; rien ne vous retient plus: je vous rends votre
parole.»
