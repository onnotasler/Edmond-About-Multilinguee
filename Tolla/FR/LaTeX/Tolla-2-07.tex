\addchap{\RomanNumeralCaps 7.}

Manuel avait écouté avec résignation les reproches du comte Feraldi, mais la conclusion le mit hors de lui. Il s'était attendu à des paroles sévères, non à cette dédaigneuse restitution de sa liberté. Il pâlit de colère, et balbutia d'abord quelques paroles inarticulées.

«Calme-toi,» lui dit Toto; «tu n'as ici que des amis.»

Il reprit avec violence: «Des amis! Monsieur le comte, si je ne m'étais pas accoutumé à vous regarder comme un second père, je n'endurerais pas si patiemment un tel outrage. Vous me croyez capable de violer mes serments!»

«Non.»

«Pardonnez-moi. Lorsqu'on dit à un homme: ‹Je vous rends votre parole,› c'est qu'on le juge assez méprisable pour la reprendre. Je m'appelle Coromila, et l'histoire de Venise, qui est celle de mes ancêtres, ne leur a jamais imputé ni un mensonge ni une trahison. Qui vous a permis de croire que je valais moins qu'eux et que je méditais de les déshonorer tous en ma personne? J'ai promis d'épouser votre fille; j'ai fait mieux, je l'ai juré; je ne l'ai pas juré une fois, mais cinquante, et sur tout ce qu'il y a de plus sacré; je l'ai juré par écrit, vous en possédez les preuves, et vous avez les mains pleines de mes serments! Et vous m'estimez assez peu pour me dire de sang-froid: ‹Soyez libre;› je vous accorde que vous n'avez rien promis, rien écrit, rien juré! Décidons à l'amiable que toutes vos lettres sont des faux, toutes vos promesses des mensonges, tous vos serments des parjures! Monsieur le comte, si l'on parle de la sorte aux hommes qu'on estime, que restera-t-il donc pour exprimer le mépris?»

\enlargethispage{\baselineskip}

«Manuel,» reprit le comte, «vous m'avez mal compris, ou plutôt j'ai mal parlé. À Dieu ne plaise que j'élève un doute sur votre honneur, qui m'est aussi cher que le mien. Voici ce que j'ai voulu dire. Lorsque vous avez demandé la main de ma fille, il y a huit ou neuf mois, vous étiez encore dans la dépendance d'un père. En engageant votre personne et votre fortune, vous disposiez en quelque sorte de biens qui ne vous appartenaient pas. Il est possible, et jusqu'à un certain point raisonnable, que le changement survenu dans votre condition, la teneur du testament de votre père, les intérêts nouveaux qui vous condamnent à ménager certaines personnes, les dispositions de votre famille, qui ne s'était pas prononcée en ce temps-là et qui depuis s'est montrée contraire à nos projets, enfin le temps, qui use toute chose, même les passions qui se croyaient éternelles, il est possible, dis-je, que l'un de ces motifs vous engage, non pas à violer, mais à regretter vos promesses. S'il en était ainsi, si vous n'aimiez plus ma fille que par scrupule, et si vous ne l'épousiez plus que par devoir, mon devoir à moi, dans son intérêt comme dans le vôtre, serait de tout rompre. Si au contraire je me suis trompé, si la prudence, qui est un défaut de mon âge, m'a aveuglé, prouvez-moi mon erreur et guérissez mes craintes: reprenez ces anciens serments qui vous sont échappés dans la première ferveur de votre amour, et donnez-moi en échange une pro messe sérieuse et irrévocable, faite de sang-froid, dans la pleine possession de vous-même, en présence de tous les obstacles que vous savez, et à la veille d'un voyage où l'on vous entraîne pour vous arracher à nous.»

Pendant ce discours du comte, Manuel sentait peser sur lui les regards de toute la famille. Après un accès de hardiesse dont il ne se serait jamais cru capable, sa timidité naturelle avait repris le dessus. Immobile et morne, il comptait machinalement les fleurs du tapis, dont le dessin se grava pour toujours dans sa mémoire. Il n'osait regarder personne en face, pas même la comtesse et sa fille, dont les yeux le cherchaient pour l'encourager. Il fit un effort pour regarder Tolla, et il leva les yeux jusqu'à ses mains, qui pendaient, à demi fermées, sur ses genoux. Ces petites mains pâles et amaigries parlaient plus éloquemment que le comte Feraldi. Elles rappelaient à Lello tant de chastes baisers, tant de douces étreintes! L'index de la main droite s'était levé si souvent en signe de menace amicale et souriante! Que de fois il s'était appuyé sur les lèvres de Lello pour lui imposer silence! La main gauche portait cette bague de turquoises qu'il y avait mise lui-même, dans une des plus belles heures de sa vie, et qu'il avait promis de remplacer par un anneau de mariage. La maigreur de ces pauvres petites mains, qui avaient perdu leurs jolies fossettes, résumait une longue histoire de larmes, de soucis, d'incertitudes, de patience, de résignation, de calomnies noblement pardonnées, de prières à mains jointes pour les calomniateurs. La main droite, négligemment renversée et entr'ouverte comme pour recevoir une main amie, semblait se tourner vers lui et lui dire: ‹Tu ne me veux plus!› Manuel entendit ce langage muet, tout en écoutant les paroles du comte. Ces deux discours, l'un ferme et précis, l'autre vague et confus, arrivaient en semble à son âme, comme le chant et l'accompagnement d'une même mélodie. Il se leva de son siège s'agenouilla devant Tolla, prit sa main dans la sienne, leva hardiment les yeux sur toute la famille, et dit d'une voix franche et résolue:

«Je jure\ldots{}»

«Arrêtez,» interrompit le comte. «Avant de vous lier par ce nouveau serment, songez qu'il doit être irrévocable. Si vous engagez à ma fille cette liberté que je viens de vous rendre, aucun prétexte, aucune raison ne pourra plus vous délier, pas même l'opposition la plus formelle de vos parents.»

«Monsieur le comte, je ferai tous mes efforts pour que mon bonheur soit approuvé de ma famille; mais si mes parents s'obstinent dans une injuste et tyrannique opposition, je me souviendrai que Dieu m'a fait libre. Et maintenant, par ce Dieu qui a comblé votre fille des plus adorables vertus, par ce Dieu qui m'a inspiré pour elle l'amour le plus pur, par ce Dieu miséricordieux avec qui elle m'a réconcilié, par ce Dieu terrible qui n'a jamais laissé le parjure impuni, je jure de n'avoir pas d'autre femme que Vittoria Feraldi.»

Tolla se pencha vers lui pour l'embrasser; mais la joie fut plus forte qu'elle, elle s'évanouit. Lorsqu'elle revint à elle, elle se cramponna instinctivement au bras de Lello: «Pourquoi t'en vas-tu? lui dit-elle à l'oreille\ldots{}»

«Maudit voyage! j'ai consenti sans savoir ce que je disais; je dégagerai ma parole.»

«Ne pars pas! Tu vois comme je suis faible. Qui sait si tu me retrouverais à ton retour?»

Manuel pleura un peu, promit beaucoup, et sortit réconcilié avec les Feraldi et avec lui-même.

En rentrant au palais Coromila, il trouva le tailleur, le brodeur et le passementier qui venaient prendre ses ordres pour un habit de Il eut honte d'annoncer à ces ouvriers qu'il avait changé d'avis et qu'il ne voyageait plus. Il les laissa prendre leurs mesures, discuta avec eux la coupe, la broderie, les galons, et ne s'ennuya pas à cet entretien. Rouquette survint, approuva son goût, et lui prédit qu'il ferait oublier Brummel à l'Angleterre. Le colonel entra ensuite, et lui dit: «Toi qui te connais en chevaux, tu m'achèteras en arrivant à Londres une jument pur-sang pour la selle, et un joli attelage de calèche. Tu t'en serviras durant ton séjour en Angleterre, et tu me les feras expédier le jour de ton départ.» Malgré la perspective d'une commission si agréable, Manuel prit son courage à deux mains; il essaya de dire qu'il n'était pas encore parti, et qu'il avait peur de s'embarquer dans un voyage si coûteux. Son frère se présenta fort à point pour répliquer qu'il se chargeait de toute la dépense. Que répondre à de si bonnes raisons? Tolla elle-même renonça à réfuter les arguments du tailleur et du frère, de Rouquette et du colonel. Lello aimait trop le plaisir pour sacrifier un si beau voyage. Tolla aimait trop Lello pour ne pas le lui pardonner.

\enlargethispage{3em}

Pour conjurer les mille dangers qu'elle prévoyait, elle ne ménagea point les recommandations à Lello, qui ne lui ménagea point les promesses Elle employa toutes les soirées du mois d'avril à demander et à obtenir des serments, sans parvenir à se rassurer. Elle fit jurer à Manuel que son absence ne durerait pas plus de deux mois. «Mais,» pensa-t-elle en frémissant, «si dans ces deux mois quelque autre femme! \ldots{}» Manuel fit serment de fuir toutes les occasions d'infidélité. ‹Malheureux!› se dit-elle; ‹il aura beau fuir, les occasions viendront à lui; il est si beau!› Elle chercha comment elle pourrait l'enlaidir pour deux mois. Elle s'ayisa de lui faire couper ses jolies moustaches noires. Le jour où Manuel se présenta devant elle avec la lèvre rasée, elle le trouva si étrange et si laid qu'elle se crut sauvée. Elle lui fit promettre, séance tenante, qu'il ne mettrait pas ses moustaches avant de rentrer à Rome. Pour être sûre que Rouquette ne lui volerait pas l'estime de son amant, elle fit jurer à Lello que, quoi qu'on pût lui dire contre elle, il suspendrait son jugement jusqu'au retour. «Et moi,» dit-elle, «quoi qu'on fasse, quoi qu'on dise, quelques preuves qu'on m'apporte, je ne me croirai abandonnée que si tu viens me l'apprendre toi-même.» Un matin, après avoir communié ensemble, ils s'agenouillèrent côte à côte devant l'autel de la Vierge. Tolla fit v\oe{}u d'entrer dans un cloître, si Dieu ne lui permettait pas d'être à Lello. Lello fit v\oe{}u de se retirer dans un ermitage à Capri, si quelque malheur ou quelque trahison l'empêchait d'épouser Tolla. Chacun d'eux appela la mort sur sa tête, s'il manquait jamais à ses serments. Au milieu de ces protestations, le mois d'avril passa vite.

Lorsque Rome apprit le prochain départ de Manuel, l'avis unanime fut que les Feraldi avaient perdu la partie. On alla jusqu'à dire que Lello se marierait en France. Les mieux informés nommaient la fille qu'il devait épouser. La générale, alarmée par ces faux bruits, craignit d'avoir fait la guerre à ses frais pour quelque famille du faubourg Saint-Germain. Pour sortir de peine, elle invita Rouquette à dîner; mais Rouquette, occupé de mille affaires et peu soucieux de ménager des alliés désormais inutiles, se tira de cette invitation par une réponse évasive. M\textsuperscript{me} Fratief et sa fille se dépitaient de ne rien savoir. Pendant un long mois, on les vit piétiner tous les salons de Rome, le nez au vent, l'oreille au guet, flairant l'air, aspirant le moindre bruit, interrogeant les visages, quêtant les nouvelles, plaignant tout haut la pauvre Tolla, maudissant tout bas monsignor Rouquette, et poursuivant l'introuvable Lello, qui passait toutes ses soirées au palais Feraldi.

La marquise Trasimeni n'était pas à Rome. Le docteur Ely, à la suite d'un gros rhume, l'avait envoyée à Florence dans les derniers jours de mars. Philippe avait pris un congé d'un mois pour accompagner sa mère. Il revint seul le 25 avril, et la première nouvelle qu'il apprit, c'est que Manuel partait dans quatre jours.

Il poussa un cri de surprise et de colère.

«Et Tolla?» se dit-il. «Est-ce que je serais un sot? Moi qui viens encore de prêcher à ma mère que ses soupçons avaient tort et que ses craintes étaient folles, me suis-je laissé berner par ce vieil ivrogne de colonel? Nous verrons bien!» Il ne fit qu'un bond jusqu'au palais Coromila. Manuel le reçut au milieu du pêle-mêle de ses bagages. Rouquette, assis sur une malle, lui offrit en ricanant un cigare de la Havane.

«Ah! monsieur,» dit Rouquette, «que vous arrivez à propos! Nous nous plaignions tout à l'heure d'être obligés de partir sans prendre congé de vous.»

«J'arrive tout botté, et voilà sur mon habit la poussière de Florence. Vous voyez monsignor, que je n'ai pas perdu de temps.»

«Croyez-vous? Il me semble que vous êtes resté un siècle dans cette belle Toscane.»

«Un mois, monsignor; pas davantage. Je vous remercie d'avoir trouvé le temps long.»

«Il s'est passé tant de choses en votre absence! Monsieur, si l'homme était sage, il ne s'éloignerait jamais de ses amis.»

«Vous parlez d'or, monsignor; mais ne savez-vous pas qu'il y a de mauvais génies qui font métier de séparer ceux qui s'aiment?»

«C'est ce que l'église appelle des esprits infernaux.»

«Oui, monsignor, infernaux. Si jamais j'en tiens un par les oreilles!»

«Monsieur,» reprit Rouquette d'une voix douce, «ces esprits-là ont le bras long et les oreilles courtes. On rencontre leur bras avant d'arriver à leurs oreilles.»

«À qui diable en avez-vous,» interrompit Manuel, «avec vos oreilles d'esprits infernaux? Est-ce que Philippe est devenu théologien? Aide-moi un peu à fermer ceci. Appuie hardiment: ‹le genou!› bon; voilà qui est fait. Que je suis aise, mon Pippo, que tu sois arrivé à temps!»

«C'est ce que je disais,» ajouta Rouquette: «monsieur arrive à temps.»

«Peut-être plus à temps qu'on ne pense, monsignor.»

«Mais je dis tout à fait à temps, pour aider votre ami à fermer ses malles. Je vais voir si mon valet de chambre s'occupe des miennes. Monsieur le marquis Trasimeni, vous devez avoir bien des choses à dire après une si longue absence. Tâchez, s'il est possible, de réparer le temps perdu. Au plaisir!»

«Ah! tu me défies,» pensa Philippe. «Eh bien! ma revanche! Il est trop tard pour empêcher Lello de partir: l'homme qui s'est donné la satisfaction de remplir toutes ces malles ne consentira jamais à les défaire. France, en Angleterre, au bout du monde, si bon lui semble; mais il ne faut pas qu'on puisse profiter de son absence pour égorger ma pauvre Tolla. Il me reste quatre jours pour lui assurer un refuge contre toutes les calomnies, pour compromettre Manuel aux yeux du monde entier, pour rendre toute rupture impossible, pour berner à mon tour ce digne colonel, et pour lier les mains à monsignor Rouquette, qui a les bras si longs. Quatre jours, c'est peu, mais c'est assez: les plus longues batailles n'ont pas duré plus de vingt-quatre heures: en avant!»

«À quoi rêves-tu?» lui demanda Manuel. «Tu as aujourd'hui une physionomie étrange.»

Philippe répondit avec un abandon bien joué: «Tu le demandes, frère? Je songe à ce voyage, qui va peut-être bouleverser tout mon avenir.»

«Et qu'y a-t-il de commun, s'il te plaît, entre ton avenir et mes voyages?»

«Tu le sauras un jour; mais parle-moi de Tolla. J'ai bien souvent pensé à elle durant ce long mois que j'ai vécu loin d'elle. Tout est rompu entre vous, n'est-il pas vrai?»

\enlargethispage{\baselineskip}

«Rompu! Es-tu fou?»

«Avoue-le-moi franchement, je ne t'en voudrai pas. Je comprends tes raisons: ton oncle, ton frère, monsignor Rouquette, ton nom, ta fortune\ldots{} J'ai fait bien des réflexions en un mois, et mes idées ont changé. D'ailleurs tu ne la rendrais pas heureuse. Qu'a-t-elle dit quand tu lui as annoncé ton escapade?»

«Elle a pleuré, elle a été un peu malade, puis elle m'a pardonné.»

«Adorable fille! Il y a vingt ans que je la connais, que je l'aime; nous avons été élevés ensemble. Eh bien! mon ami, depuis que j'ai l'âge de raison, je me demande s'il y a un homme qui mérite une telle femme! Tu reviendras dans six mois?»

«Dans deux mois.»

«Six!»

«Deux! te dis-je.»

«Mettons cinq. Pendant ces six mois, restera-t-elle dans sa famille, ou va-t-elle s'en fermer dans un couvent?»

«À quoi bon le couvent? Elle vivra, comme toujours, auprès de sa mère.»

«Tu as raison: pas de couvent, j'y perdrais trop D'ailleurs le colonel n'entendrait pas raison sur ce chapitre.»

«Et pourquoi?»

«Parbleu! crois-tu que ton oncle t'envoie à Paris et à Londres pour hâter ton mariage avec elle? Il prévoit tout ce qui peut advenir en six mois; il vous applique à tous deux la médecine des grands parents, aussi vieille qu'Aristote: à l'amant, le grand air et la poussière des chemins; à l'amante, le tourbillon des valses, le bourdonnement des danseurs et la poussière des salons. Et si la guérison se fait trop attendre, si l'amant traverse la mer sans écouter les sirènes, le fleuve sans regarder les ondines, et la forêt sans causer avec les dryades; la jeune fille est assez impertinente pour aimer obstinément celui qu'on veut qu'elle oublie, alors aux grands maux les grands remèdes! Un parent vénérable, un ami de la famille, un homme d'église au besoin, dresse un piège à la pauvre enfant sans défiance; on tend une bonne calomnie sur son passage, on fait faire à sa réputation une culbute dont elle ne se relèvera jamais: ‹cela vous apprendra, mademoiselle, à marcher droit!› Rappelle-toi Venise et les amours de ton frère. Crois-tu que ce mariage eût été aussi facile à rompre, si le maladroit, avant de partir, avait enfermé sa maîtresse dans un couvent? Le couvent, mon ami, est la seule forteresse où la réputation d'une fille soit à l'abri, parce que les hommes n'y pénètrent jamais. La vertu est robuste, elle se conserve partout, dans le monde, dans les bals et dans la valse à deux temps; la réputation est comme une robe blanche qu'il faut serrer dans un tiroir, si l'on ne veut pas qu'elle soit éclaboussée par un rustre ou déchirée par un faquin. Que Tolla reste dans le monde, je réponds de sa vertu, je ne réponds pas de sa robe blanche.»

«Et tu ne veux pas que je l'enferme dans un couvent!»

«D'abord consentirait-elle?»

«J'en réponds.»

«Ses parents?»

«Je m'en charge.»

«Et la permission des autorités ecclésiastiques?»

«Le cardinal Pezzato l'obtiendra.»

«Mais ton oncle?»

«Il apprendra l'affaire lorsqu'elle sera faite.»

«Et monsignor Rouquette?»

«Je suis plus fin que lui.»

«Tu serais homme à garder un secret pendant quatre jours?»

«Je ne suis donc pas Romain?»

«Comme tu prends feu pour le couvent! Cependant, mon ami, à juger froidement les choses, il n'y a pas péril en la demeure. Que crains-tu?»

«Tout!»

«Non, tu ne crains rien du c\oe{}ur de Tolla, trop heureux garçon! Le seul danger, c'est qu'un Rouquette à Paris, une Fratief à Rome lui impute à crime quelques distractions innocentes. Que t'importe? Tu fermeras l'oreille et tu laisseras dire. Qu'est-ce qu'ils pourraient inventer de nouveau après ce que nous avons entendu? Quelle créance accorderais-tu à leurs paroles, toi qui as vu comment ces artistes travaillent la calomnie? Si l'on t'écrivait dans un mois qu'on a rencontré Tolla, à dix heures du soir, en voiture, avec un jeune homme sur la route d'Albano; si monsignor Rouquette déposait sur ton bureau une liasse de lettres anonymes; si ton oncle t'écrivait que tu es la fable de Rome, comme tu l'as jadis écrit à ton frère, ne renverrais-tu pas loin de toi ces vieux mensonges, si usés qu'ils montrent la corde?»

«Oui; mais si véritablement Tolla se lais sait étourdir par ce tourbillon du monde?»

«Sois tranquille, je veillerai sur elle, et jamais le c\oe{}ur d'une femme n'aura un gardien plus jaloux.»

«Mais\ldots{}»

«Tu ne me connais pas, Manuel.»

«J'aime Tolla, depuis l'enfance, d'une amitié passionnée. Sans toi, je l'aurais peut-être aimée d'amour. Juge de ce que je deviendrais si je voyais qu'elle te trahît pour un indigne!»

«Cependant\ldots{}»

«Toi parti, je m'attache à sa personne, je me fais son garde du corps, je l'accompagne dans tous les bals, je ne la quitte pas plus que son ombre. Le soir, à l'heure où tu lui faisais ta visite quotidienne, j'irai la voir, je m'asscoicrai à ta place, nous parlerons de toi, et quelquefois nous pleurerons ensemble. Les larmes sont moins amères lorsqu'elles sont essuyées par l'amitié.»

«C'est fort joli, mais\ldots{}»

«Entends-tu d'ici les bonnes langues? Elle aime Philippe! Elle épouse Philippe! Philippe a supplanté son ami! Je ne poserai pas sur son front un baiser fraternel sans que le bruit en retentisse dans toute l'Italie. Que nous rirons de bon c\oe{}ur!»

«Mais, par tous les saints!\ldots{}» interrompit violemment Lello.

«Encore un mot. Le couvent a du bon, je te l'accorde; mais jusqu'à quel point as-tu le droit d'emprisonner celle qui t'aime?»

«Je me soucie bien du droit!» cria Manuel. «Droit ou non, je dis qu'elle ira au couvent, et qu'elle y restera jusqu'à mon retour, et qu'elle n'y recevra personne, excepté sa mère et notre confesseur. Je ne suis pas jaloux; mais, puis que tu te charges de l'être à ma place, tu vas voir comme je saurai profiter de tes conseils! Quel est le couvent le plus sévère?»

«\emph{Les Sepolte vive\footnote{
L'italien: \emph{les enterrées vives}
}}»

«C'est trop dur; un autre?»

«Saint-Antoine-Abbé.»

«Y reçoit-on des pensionnaires?»

«Oui.»

«Elle ira à Saint-Antoine-Abbé.»

«Mais, mon cher Lello, que veux-tu que je devienne? Tu pars pour Londres, tu en fermes Tolla: quels amis me laisses-tu?»

«Tu en trouveras d'autres: on en a toujours assez. Où ai-je fourré mon chapeau? Le voici. Mes gants? Dans ma poche. Mon ami, je ne te renvoie pas: je cours chez elle, chez sa mère, chez son oncle, chez le cardinal-vicaire, chez l'abbé La Marmora et chez la supérieure du couvent.»

«Moi, je rentre à la maison: nous ferons route ensemble jusqu'aux Saints-Apôtres.»

Chemin faisant, Manuel se disait avec une vivacité fébrile: «Ah! maître Philippe! vous l'aimez, et vous n'en savez rien! Et elle ne s'en doute pas! Mais moi, j'ai l'\oe{}il bon, Dieu merci! J'allais m'embarquer dans un joli voyage! Heureusement le couvent arrange tout.»

Philippe cachait sous un visage abattu la joie la plus triomphante: «Il est jaloux, donc il aime encore. Comme il a dévoré l'hameçon! Ses yeux lançaient des éclairs: il doit m'avoir en horreur. Tolla sera heureuse: le couvent sauve tout; il ferme la bouche au colonel, à Rouquette, à la Fratief et au monde. Il rend toute défection impossible. Quand Manuel aura enfermé sa maîtresse dans un cloître, il sera bien forcé de venir l'y reprendre.»

Le lendemain, Philippe déjeunait dans sa chambre lorsqu'il vit entrer Dominique. Il lui offrit une chaise et un grand verre de vin de Marsalla, brillant comme la topaze et chaud comme le soleil. Dominique, en homme bien appris, accepta le vin et refusa la chaise.

«C'est \emph{elle} qu'il t'envoie?» demanda Philippe.

«Non, \emph{ser} Pippo; je viens de ma part. Savez-vous qu'il a la cruauté de l'enfermer au couvent?»

«Elle a consenti?»

«Est-ce qu'elle peut rien lui refuser? Madame pleure, mais nos hommes sont contents. Notre oncle le cardinal est allé hier soir à Saint-Antoine: il a tout conté à la supérieure, la permission sera signée aujourd'hui; mais on exige que mademoiselle cache son amour à toutes les s\oe{}urs et à toutes les pensionnaires, et qu'elle ne laisse deviner à personne le pourquoi de sa retraite. Pauvre fille! Être obligée de resserrer ses sentiments, d'étouffer ses soupirs et de dévorer ses larmes! Et Dieu sait combien de temps elle va rester là toute seule à ronger son c\oe{}ur! Croyez-vous qu'on me permettrait d'entrer au couvent avec elle? Je ne compte pas, moi; je ne suis pas un homme; je suis le chien de la maison, qui lèche la main des maîtres et qui aboie aux ennemis.

«Impossible, mon pauvre chien; tu ressembles trop à un beau garçon. Il faudrait trouver une fille dévouée qui consentît à se renfermer pour quelques mois.»

«Hélas! \emph{ser} Pippo, les gens dévoués sont rares. Après vous et moi, j'ai beau chercher, je n'en vois plus.»

«Comment! parmi toutes les femmes de la maison il n'y en a pas une?»

«Je n'en connais pas. Songez donc, mon sieur: deux mois de prison, peut-être trois, ou même davantage; cent jours peut-être sans voir personne: quelle perspective pour une femme!»

\enlargethispage{\baselineskip}

«Comment appelles-tu cette grande fille qui a couru chercher le médecin quand tu avais la tête cassée?»

«Amarella. Elle n'a pas beaucoup de c\oe{}ur, allez. C'est une fille qui a ses idées.»

«Peste! tu es difficile, si tu trouves qu'elle n'a pas prouvé assez de dévouement.»

\enlargethispage{\baselineskip}

«Non, monsieur. Ce qu'elle a fait, ce n'est pas pour mademoiselle; c'est pour moi.»

«Qu'importe? Si elle consent à entrer au couvent, je m'inquiète bien si c'est pour l'amour de toi ou pour l'amour de Tolla! Ce qu'il faut, entends-tu? c'est que ta maîtresse ne soit pas seule; elle périrait d'ennui, d'amour et de silence. Va trouver cette fille. Tu as quelque crédit sur elle?»

«Je le pense, \emph{ser} Pippo; mais je n'ai jamais essayé, parce qu'elle a ses idées, et moi les miennes.»

«Laisse-moi tes idées en repos. Va trou ver cette fille, dis-lui ce que tu voudras, pro mets-lui ce qu'il faudra, arrange-toi comme tu pourras, mais décide-la à entrer au couvent: il s'agit du salut de mademoiselle.»

«Je cours, monsieur. Jusqu'ici je n'avais trompé personne, mais le salut de mademoiselle ayant tout!»

Le 29 avril, à dix heures du soir, Tolla et sa femme de chambre entrèrent au couvent de Saint-Antoine-Abbé. Elles y furent conduites par le comte, la comtesse, Victor, Manuel, Philippe, l'abbé La Marmora et Menico. La supérieure reçut Tolla des mains de sa mère. Elle l'embrassa tendrement et lui fit une petite exhortation maternelle sur les nouveaux devoirs qu'elle aurait à remplir, les privations auxquelles elle se condamnait, le passage de la vie tumultueuse des salons à la vie austère du cloître, et les avantages spirituels et temporels que Dieu lui réservait en échange d'un si vertueux sacrifice. Tolla dit adieu à tout le monde. Lorsqu'elle serra la main de Manuel, deux grandes larmes descendirent lentement le long de ses joues pâles; elle se pencha vers lui et lui dit à l'oreille:

«Me voici où tu as voulu; j'y resterai jusqu'à ce que tu viennes me reprendre: ne me fais pas attendre trop longtemps.»

Menico pleurait à la dérobée. Amarella lui demanda tout bas: «Est-ce pour moi, ces larmes?»

«Et pour qui donc? répondit-il en rougissant un peu de son mensonge.»

Lorsque la supérieure eut emmené sa nouvelle pensionnaire, les parents et les amis de Tolla restèrent quelques instants à écouter le grondement lugubre des portes qui se fermaient sur elle. Ce grand parloir sombre et froid n'était éclairé que par une lampe de cuisine dont la fumée montait en tourbillons jusqu'au plancher. Personne n'osait prendre la parole.

Menico s'approcha de Manuel et lui dit à haute voix:

«Adieu, excellence; je vous souhaite un bon voyage et \emph{beaucoup de plaisir.}»

«Ma pauvre fille!» murmura la comtesse en étouffant un sanglot.

«Madame la comtesse,» reprit Lello, «c'est ici que j'ai voulu prendre congé de vous et de votre famille. C'est ici que je vous donne rendez-vous dans deux mois pour conduire votre fille à l'autel.»

À la même heure, et tandis que Lello s'engageait irrévocablement à épouser Tolla, Rouquette et le chevalier soupaient joyeusement ensemble. Ces deux vases d'élection, l'un vaste et large comme un tonneau, l'autre sec et noueux comme un sarment de vigne, avaient déjà vidé six bouteilles de lacrima-christi rouge, le plus capiteux de tous les vins d'Italie. Le colonel s'enfonçait tout doucement dans cette ivresse tranquille et béate qui est le privilège des buveurs endurcis. L'excès du vin produisait en lui une félicité sans éclat, une torpeur sans malaise, un délicieux anéantissement. Sa grosse figure, aussi puissamment modelée que le masque antique de Vitellius, se couvrait par ses couches égales d'un coloris radieux; sa tête se renversait en arrière; ses jambes mollissaient sous lui jusqu'au moment où, tous les ressorts venant à se détendre, il passait sans secousse du fauteuil au tapis et de la veille au sommeil. Rouquette, les yeux écarquillés, la figure plaquée de rouge, avait une ivresse agitée et capricante. Il élevait la voix, se démenait sur son siège, et se ressuscitait lui-même par soubresauts, d'ailleurs maître de lui jusqu'au dernier moment, fidèle à l'habitude de peser ses paroles, et toujours éveillé aux affaires.

«Mon cher Rouquette,» disait le colonel en grasseyant, «vous êtes un grand homme.»

«Hé! hé!»

«Vous irez loin, si vous n'êtes jamais pendu.» Rouquette sauta comme un baril de poudre. «Rasseyez-vous donc, vous m'éblouissez. Est-ce que vous ne pourriez pas empêcher vos yeux de tourner dans leurs cages comme des écureuils? Que disions-nous? J'y suis. Vous avez sauvé une fois la famille Coromila. Une grande famille, Rouquette! Je tiens à mon nom, sans en avoir l'air; je ne le donnerais pas pour cent mille bouteilles de ce vin-là. Reste à sauver le petit. Il est bien empêtré, mon cher Rouquette.»

«Soyez tranquille, excellence: je l'emmène!»

«Oui, mais il reviendra.»

«Il reviendra tellement changé, que sa maîtresse ne le reconnaîtra plus.»

«Ne croyez pas cela, Rouquette. J'ai passé par là, tel que vous me voyez. Eh bien! celle que j'ai \ldots{} comment dit-on? trahie? oui; celle que j'ai trahie me reconnaît toujours. Ayez bien soin du petit.»

«Comme de moi-même, excellence.»

«S'il avait envie de faire quelques folies, mon ami, laissez-le faire. Cela le distraira. Je paierai tout. Nous ne regardons pas à l'argent dans la famille.»

«Nous y voici,» pensa Rouquette, qui tressaillit au mot \emph{d'argent.} «Excellence, j'ai déjà éprouvé votre générosité.»

«Oui, oui. Ces vingt mille francs qu'on vous a donnés après l'affaire de Venise! Vous en verrez bien d'autres. C'est une mine d'or que cette maison-ci. Piochez, Rouquette, piochez! Pendant que vous travaillerez là-bas, nous nous occuperons, nous, de la petite fille. Nous lui ferons une réputation. Que faut-il pour faire la réputation d'une femme? Des paroles, et rien de plus. J'en achèterai: je ne regarde pas à l'argent. Il faut que Tolla Feraldi soit citée dans toutes les familles de l'Italie comme un exemple à ne pas suivre. Quand tout le monde dira que c'est une fille perdue, Manuel n'osera plus la vouloir. Buvez donc, Rouquette. Vous n'êtes pas de ma force. Je suis un Romain de la vieille roche, moi. J'aurais fait un bel empereur. Toi, mon garçon, tu ne seras jamais qu'un pape. Si tu guéris le petit, je te donnerai tout ce que tu voudras. Veux-tu quarante mille francs, dis? Quarante. Réponds vite, avant que je ne m'endorme.»

Un domestique entra sur la pointe du pied.

«Que veux-tu?» murmura le colonel. «Va te coucher! Tu vois bien que tu dors.»

«Une lettre très pressée pour monseigneur.»

«Donne-la-lui et va te coucher. Je te défends de ronfler en ma présence.»

Rouquette déchira l'enveloppe d'une main avinée. «Du marquis Trasimeni, dit-il en bégayant.»

«Trasimeni! Voilà plus de quinze ans qu'il dort! Chut! C'était mon ami. Si je ne craignais pas de l'éveiller, je te conterais une bonne histoire. Sais-tu avec qui il s'est marié, Trasimeni?»

Rouquette n'était plus à la conversation. Il s'était levé, il s'appuyait au mur, auprès d'un candélabre, et épelait en se frottant les yeux la lettre suivante:

\enlargethispage{2.1em}

\begin{quote}

Monsignor,

Il me semble qu'il y a un siècle que je vous ai vu. Il s'est passé tant de choses depuis notre dernière rencontre! Mon ami Lello a conduit M\textsuperscript{lle} Vittoria Feraldi, au couvent de Saint-Antoine-Abbé, afin de mettre son honneur en sûreté et de faire connaître à toute la ville de Rome qu'il était décidé à la prendre pour femme. Je m'étonne que vous n'ayez rien su de cette affaire, pour laquelle le cardinal-vicaire a donné sa signature. On peut donc avoir le bras très long et l'oreille très courte? Je vous cherche depuis une heure pour vous apprendre une nouvelle aussi intéressante. Impossible d'arriver jusqu'à vous: il y a de mauvais génies qui font métier de séparer ceux qui s'aiment.

\hspace*\fill---Philippe Trasimeni\end{quote}

Rouquette poussa un cri aigre, revint à la table, avala une carafe d'eau et relut sa lettre pour la seconde fois. Il n'en fallut pas davantage pour le dégriser. «Colonel!» cria-t-il. Le colonel avait disparu sous la nappe. Rouquette tira violemment la table en renversant les flacons et les verres: il découvrit une masse aussi imposante, mais aussi immobile que les lions de basalte qui décorent l'entrée du Capitole. Il essaya dé le secouer: peine inutile! Il lui jeta quelques gouttes d'eau sur le visage: le formidable dormeur, pour toute réponse, lui détacha un coup de poing qui l'aurait assommé, s'il ne s'était retiré à temps. «Lourde brute!» murmura le pauvre Rouquette. «Et il y a cinquante ans qu'il apprend à boire! Que faire? Nous partons demain matin à cinq heures; il est minuit. Cinq heures pour arracher cette fille de son couvent! Ah! si j'étais pape! Tu me le paieras, Philippe Trasimeni! Si nous la laissons là, tout m'échappe, Manuel, l'argent, l'avenir, les Coromila! Comment le cardinal-vicaire a-t-il signé? Est-ce qu'il sait tout? Est-ce qu'il se cache de moi? N'est-il pas un peu parent des Feraldi? S'il m'échappait comme le reste? Tout s'ébranle, toute craque, toute croule sur ma tête. Travaillez donc comme un man\oe{}uvre à bâtir votre fortune pour que l'espièglerie d'un gamin la jette à bas! Voilà la justice céleste! Il faut que je parle à Manuel! C'est lui qui a fait la sottise, c'est à lui de la réparer.»

Il sortit, en trébuchant un peu, de la salle à manger, et courut à l'appartement de Lello. Le domestique qui lui avait apporté la lettre courut après lui et l'arrêta avec cette fermeté polie que les valets savent opposer à un maître qui a trop bu. Rouquette, exaspéré par un tel contre-temps, voulut jeter ce respectueux obstacle par la fenêtre. Le valet menaça d'appeler main-forte, et déclara qu'il ne laisserait point troubler le repos du chevalier Manuel. Rouquette changea de tactique et demanda à voir le prince. Un valet de chambre et quatre laquais, attirés par tout ce bruit, lui répondirent que le prince avait défendu qu'on entrât chez lui avant quatre heures sous aucun prétexte.

«C'est bien, reprit-il,» laissez-moi. «Je vais tâcher d'évei\-ller le colonel.» Tous ces hommes jurèrent qu'on les mettrait en morceaux avant de secouer le bras du colonel. «Alors ouvrez-moi la porte,» cria-t-il, «je veux sortir!» Ces braves gens se demandèrent s'il était prudent de lâcher dans la ville un si incorrigible réveille-matin. C'est après une résistance héroïque, des pourparlers interminables et des recommandations à exaspérer un saint, qu'ils tirèrent les verrous et l'abandonnèrent sur le Corso à la grâce de Dieu.

\enlargethispage{\baselineskip}

Rouquette erra quelques instants à l'aventure sans savoir à quelle porte frapper à une heure si ridiculement indue. Il regardait d'un \oe{}il hébété les maisons énormes qui bordent le Corso, lorsqu'il lut au coin d'une des rues qui viennent y aboutir: \emph{via Frattina.} Il se souvint qu'il était à deux pas de la générale, et, sans écouter l'avis officieux des horloges du quartier qui sonnaient unanimement deux heures du matin, il courut frapper à sa porte. Comme il arrive en pareil cas, les coups de marteau réveillèrent d'abord les gens d'en face, puis les maisons voisines, puis le locataire du troisième, puis l'Anglais du second, puis le marchand du rez-de-chaussée, avant d'être entendus chez M\textsuperscript{me} Fratief, qui logeait au premier. Lorsque son domestique se décida enfin à ouvrir un volet pour parlementer, Rouquette essuyait les feux croisés de quatorze bourgeois flanqués de quatorze chandelles, qui lui lançaient quatorze questions à la fois. Force-lui fut de décliner son nom au milieu de ce curieux auditoire, qui se demanda depuis quand les monsignori faisaient leurs visites à deux heures du matin. La porte s'ouvrit enfin. La générale, réveillée en sursaut par une heureuse nouvelle, accourut en si grande hâte, qu'elle oublia de mettre ses dents. Rouquette, aussi pressé qu'elle pour le moins, ne prit pas le temps d'excuser la rareté de ses visites et tous les péchés d'omission qu'il avait sur la conscience. Il alla droit au fait, annonça qu'il venait, de la part de Lello, prendre congé de ces dames. L'affaire était en bon chemin, Lello semblait fort décidé à ne prendre sa femme ni en France ni en Angleterre: il reviendrait à Rome dans deux mois; d'ici là, la belle Nadine et sa mère recevraient de ses nouvelles. Malheureusement Tolla, conseillée par sa mère ou par quelque autre intrigante, était allée se jeter dans un couvent; toute la ville de Rome l'apprendrait dans quelques heures, et le parti Feraldi, profitant du départ de Lello, ne manquerait pas de dire que c'était lui qui l'avait cloîtrée: calomnie dangereuse qu'il fallait démentir à tout prix en forçant cette petite folle à rentrer dans le monde. Tant qu'elle serait à Saint-Antoine-Abbé, personne n'aurait prise sur elle, et elle aurait prise sur Lello. Elle se poserait en victime et ameuterait tous les pleurards de l'Italie. «Si j'avais une journée à moi, dit-il, je saurais bien l'arracher de sa retraite; mais je pars à cinq heures du matin pour Civita-Vecchia, à trois heures du soir pour la France, et les bateaux à vapeur n'ont pas l'habitude d'attendre. Agissez, il y va de votre intérêt. Dites tout ce qu'il vous plaira, que ce n'est pas Lello qui l'a cloîtrée, mais la police; qu'on l'a mise au couvent par correction: si cela prend, elle sortira pour prouver qu'elle est libre, et une fois sortie on ne lui permettra plus de rentrer. Rendez-lui le séjour du couvent insupportable; si elle a quelque servante avec elle, prenez-lui sa servante. Enfin, vous êtes une femme de tête, guettez les occasions, inspirez-vous des circonstances, parlez, agissez, remuez; tous les moyens sont bons, argent, promesses, prières, menaces: pourvu qu'elle sorte, tout est là.»

«Hé! cher monsignor, que voulez-vous que je fasse? je n'ai ni crédit, ni pouvoir, ni\ldots{}» (elle s'arrêta fort à propos au moment où elle allait dire ni argent) «ni auxiliaire. J'avais autrefois un domestique dévoué; il a disparu le 6 octobre sans me dire adieu.»

«Et en emportant vos bijoux?»

«Dieu! non, le pauvre garçon! L'Anglais qui demeure là-haut l'accusait d'avoir volé un fusil: c'est peut-être ce qui lui a fait prendre la maison en horreur. Quand je l'avais ici, ce bon Cocomero, je savais tout; il pénétrait jusque dans le palais Feraldi pour m'apporter les nouvelles. Le butor qui l'a remplacé n'est capable de rien; autant vaudrait un sourd-muet aveugle et manchot.»

«Qu'à cela ne tienne! Voulez-vous que je vous laisse un homme?»

«Oui, certes.»

«La police est dans les attributions du cardinal-vicaire. J'ai du crédit dans les bureaux; je puis mettre un sbire à votre disposition.»

«Donnez, monsignor, donnez!»

«Attendez! Il y a six ou sept mois, j'ai enrôlé un drôle qui m'avait tout l'air d'avoir fait quelque mauvais coup; mais à tout péché miséricorde: c'est la devise de la police. Il m'a prié instamment de le placer hors de Rome; je lui ai offert Albano, Lariccia ou Velletri; il a demandé en grâce qu'on l'envoyât d'un autre côté; il est à Civita-Vecchia, il surveille les libéraux; ses chefs sont contents. de lui; je vous l'expédierai aujourd'hui même.»

«Mais s'il refusait de revenir à Rome?»

«Je voudrais bien voir qu'il essayât de refuser quelque chose! On est toujours sûr du dévouement d'un homme lorsqu'on a de quoi le faire pendre. Adieu, madame, je vais travailler pour vous: aidez-moi. Mes baise-mains à mademoiselle votre fille!»

«Elle dort, la pauvre innocente, tandis que nous nous occupons de son bonheur!»

Nadine écoutait à la porte.
