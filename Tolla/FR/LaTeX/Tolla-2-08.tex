\addchap{\RomanNumeralCaps 8.}

Rouquette trouva un carrosse attelé dans la cour du palais Coromila. Manuel et son frère, lestés d'une tasse de chocolat, se promenaient en fumant, tandis qu'on remplissait un fourgon de bagages. Le colonel dormait comme Noé après al première vendange~: il avait fait ses adieux la veille pour avoir le droit de se lever à midi. Tous les gens de la maison vinrent, chapeau bas, baiser la main de leurs maîtres. Le prince leur distribua un gros sac d'argent. Rouquette, qu'ils examinaient comme une curiosité d'histoire naturelle, aurait voulu leur distribuer des coups de bâton. On partit à cinq heures précises. Jusqu'à Civita-Vecchia, Manuel bâilla, fuma, soupira, et regarda par la portière; son frère lut le premier chant de \emph{Don Juan} dans le texte anglais; Rouquette dormit. Les quatre domestiques que l'on emmenait à Londres émerveillèrent les alouettes par l'éclat de leurs boutons neufs. En entrant dans la ville, les postillons firent claquer si superbement leurs fouets, qu'on crut voir entrer le duc de Toscane, dont l'arrivée était annoncée pour ce jour-là. La garnison prit les armes, les tambours battirent aux champs, et le gardien des portes refusa obstinément d'examiner les passeports. Les deux frères traversèrent au galop cet enthousiasme officiel: ils trouvèrent sur le port leur intendant, qui était venu la veille pour assurer les places et disposer les logements sur le bateau. Rouquette courut à la police, se nomma et demanda François le Napolitain. Il eut quelque peine à reconnaître son protégé. François le Napolitain, ci-devant Cocomero, avait rasé ses favoris et laissé croître ses cheveux. Ce changement de décoration, joint à la peur du bagne voisin, dont le spectacle l'avait horriblement maigri, lui avait fait une autre figure, aussi longue que la première était large. Depuis le 6 octobre et l'accident de Menico, François n'avait jamais dormi que d'un \oe{}il: aussi ses chefs louaient-ils sa vigilance. Il faisait le guet autour de la ville, gardait toutes les issues à la fois, et dépistait merveilleusement les nouveau-venus, tant il avait peur de voir arriver un couteau suivi du bras de Dominique. Malgré les témoignages de satisfaction qu'il avait souvent obtenus, il ne recherchait pas les occasions de comparaître devant les autorités policières: il avait peur de ses chefs, de ses camarades et de lui-même. Lorsqu'il se vit en présence de monsignor Rouquette, secrétaire intime de son éminence le cardinal vicaire, il serra instinctivement les mâchoires de peur qu'on n'entendît claquer ses dents.

\enlargethispage{\baselineskip}

«J'ai besoin de toi,» lui dit Rouquette. La figure de Cocomero s'épanouit. «Tu vas partir ce soir pour Rome.» La figure de Cocomero s'allongea. «Tu iras \emph{via Frattina,} n$\textsuperscript{o}$ 15; tu demanderas M$\textsuperscript{me}$ la générale Fratief.»

Cocomero tomba à genoux: «Grâce! cria-t-il,» «grâce, monsignor! Je suis, ou du moins je serai un pauvre père de famille! Ne me perdez pas: je vous servirai toute ma vie!»

«Je ne veux pas te perdre, je veux t'employer. Je sais tout.» Rouquette ne savait rien; mais \emph{je sais tout} est un talisman presque infaillible, et il y a bien peu d'hommes assez irréprochables pour entendre sans trembler ce bienheureux \emph{je sais tout}.

«Et, monsignor,» balbutia Cocomero, «vous croyez qu'il n'y a pas d'imprudence à m'envoyer dans \emph{cette} maison? Est-ce que l'Anglais du fusil n'y est plus?»

«Tiens, tiens!» pensa Rouquette. Il reprit à haute voix: «L'Anglais du fusil y est encore; mais tu es si changé, qu'il ne te re connaîtra pas. Parlons un peu du fusil de l'Anglais.»

Cocomero joignit piteusement les mains.

Le confesseur improvisé poursuivit: «Maître Cocomero, car je sais tous tes noms, fidèle valet de M$\textsuperscript{me}$ Fratief, on ne vole pas un fusil pour aller faire la chasse aux moineaux!»

«Plus bas! monsignor, au nom du ciel! Menico m'avait provoqué: il m'avait roué de coups deux fois de suite, dans la cour du palais Coromila et devant la porte de ses maîtres, ces scélérats de Feraldi. Ma patience était à bout: j'ai demandé pardon à Dieu, j'ai fait quatre neuvaines, et puis\ldots{} on est vif, et un malheur est bientôt arrivé.»

«Mais c'est un trésor que cet homme-là!» pensa Rouquette. «Il déteste les Feraldi, il a déjà servi la Fratief, il sait le métier d'espion, et il loge une balle à cent pas dans la tête d'un homme. Je veux faire sa fortune.» Il continua tout haut, d'un ton digne et sévère: «Vous êtes un grand coupable, mais vous pouvez réparer vos crimes. Choisissez entre l'expiation honorable que je vous propose et les peines honteuses que la loi suspend sur votre tête. Vous partirez pour Rome par la voiture du soir. Vous irez demain, à la brune, prendre les ordres de la respectable M$\textsuperscript{me}$ Fratief; vous exécuterez aveuglément tout ce que cette sainte femme vous commandera. Vous n'avez rien à craindre de la justice, tant que vous serez exact à remplir les nouveaux devoirs que le gouvernement du saint-père vous impose. Si vous croyez être en butte à quelque vengeance particulière, défendez-vous, sans jamais oublier la prudence. Pour subvenir à vos besoins, vous toucherez tous les mois une somme de vingt écus chez l'intendant des princes Coromila Borghi. Voici vos gages du mois de mai, et deux écus pour votre voyage. Allez, et souvenez-vous que vous êtes dans ma main.»

\enlargethispage{\baselineskip}

Cocomero, prosterné comme devant un saint, s'empara d'une des basques de l'habit de Rouquette, qu'il couvrit des plus tendres baisers et des larmes les plus reconnaissantes. Rouquette s'enfuit jusqu'au bateau en riant comme un augure qui vient d'en voir un autre.

Le voyage se fit en ligne directe, à toute vapeur, en moins de quarante-huit heures. La mer était belle, Manuel ne fut pas malade, et Rouquette lui donna deux longues leçons de français sans lui parler du couvent de Saint-Antoine. En débarquant à l'hôtel, Lello chercha au fond d'une malle le portrait de Tolla. La chère petite image était presque laide: les exhalaisons salines de la mer avaient altéré les couleurs. Il se consola comme il put en griffonnant une longue lettre à sa maîtresse. Ni son frère ni Rouquette ne lui demandèrent à qui il écrivait; mais quand il parla de faire venir un barbier pour raser ses moustaches qui avaient repoussé d'un millimètre, on le plaisanta si vertement qu'il se rendit. Son frère appelait le barbier l'exécuteur des hautes \oe{}uvres de Tolla. Rouquette demandait depuis quand les nobles Romains étaient taillables à merci. On fit acheter une paire de moustaches postiches qu'on posa sur un coussin avec cette inscription: \emph{Offrande à la beauté}. Rouquette crayonna une femme ornée de moustaches; il écrivit au-dessous: \emph{Tolla parée des présents de Lello}. La cheminée de sa chambre était surmontée d'un amour de plâtre: on lui mit un rasoir entre les bras et l'on grava sur le socle: Cruel enfant! Pour obtenir la paix, Manuel remit l'opération à des temps meilleurs, mais il confessa noblement sa faute dans la première lettre qu'il écrivit à Tolla.

Le séjour de Paris, où les trois voyageurs s'arrêtèrent jusqu'au 10 juin, ne refroidit pas l'amour de Manuel. Paris n'a que des séductions banales pour un étranger qui ne sait pas le français et qui court du matin au soir derrière un \emph{cicerone\footnote{expression désuète: guide}} de place, demi-valet, demi-drogman. La manufacture des Gobelins, la colonne Vendôme, les caveaux du Panthéon, et même le musée historique de Versailles, sont aussi incapables d'éteindre les passions que de les allumer. Manuel écrivait sans mentir qu'il avait les yeux à Paris et le c\oe{}ur à Rome. Lorsque son frère lui montrait aux Champs-Élysées une délicieuse toilette d'été, il répondait naïvement: «Oui, cela irait bien à Tolla.» Rouquette ne rencontrait jamais une jolie femme sans la lui faire remarquer. «J'aime mieux Tolla,» répondait-il; «d'abord elle est aussi belle, puis elle m'aime, enfin elle parle italien.» «Essayons du grand monde,» dit Rouquette. On porta une douzaine de lettres de recommandation, qui attirèrent cinq ou six invitations à dîner: il y avait déjà beaucoup de familles à la campagne. Manuel s'ennuya partout: son frère, qui parlait Français, et Rouquette, qui avait de l'esprit, l'éclipsèrent totalement. Il en prit son parti en rêvant à Tolla. Sa pensée voyageait incessamment entre la chère fenêtre et le parloir de Saint-Antoine. Ce gros garçon, qui n'avait jamais eu deux idées à la fois, fut pensif comme un philosophe et distrait comme un algébriste, en foi de quoi ses compagnons de voyage l'avaient surnommé le \emph{hanneton}.

Son principal et presque unique souci durant les trois premières semaines fut le silence de Tolla. Tous les jours, son domestique de place s'en allait rue Jean-Jacques-Rousseau et revenait les mains vides. Il accusa d'abord la poste de Paris, qui lui paraissait un chaos épouvantable: il ne comprenait pas qu'une administration qui transporte ses facteurs en omnibus pût distribuer les lettres sans en perdre la moitié. Ses soupçons se portèrent ensuite sur son oncle et sur la poste romaine, qui fut de tout temps sujette à caution. Enfin il surveilla Rouquette et son frère, sans parvenir à les prendre en faute. Au bout de vingt-deux jours, son banquier lui remit un mot de Tolla qui éclaircit tout le mystère. Elle lui avait écrit onze fois, ni plus ni moins, sous le nom de Manuel Miracolo, et les onze lettres attendaient bureau restant, casier M, que Miracolo vînt les prendre. Manuel y courut, suivi de son interprète à dix francs par jour. L'employé lui montra onze lettres à l'adresse de Manuel Miracolo, et lui demanda son passeport. Lello s'étonna que, sur la terre de la liberté, un étranger eût besoin d'un passeport pour obtenir sa correspondance. Dans la ville de Rome, où les facteurs ne vont pas en omnibus, donne les lettres à qui veut les prendre. Si vous vous appropriez le bien d'autrui, l'administration le met sur votre conscience. Manuel montra un passeport au nom de Coromila. On le renvoya à un autre employé qui présidait à la lettre C, mais qui n'avait rien à son adresse. A force d'aller d'un guichet à l'autre, il comprit, son domestique aidant, qu'il faudrait un ordre exprès du directeur général des postes pour rendre à la lettre les trésors d'amour que la lettre M avait usurpés. Il se défiait trop de Rouquette pour lui faire part de son embarras et lui demander assistance. Son inséparable interprète le conduisit chez un écrivain public qui expliqua l'affaire comme il la comprit, et lui recommanda expressément de faire viser la pétition par son ambassadeur. Manuel se transporta sans retard à la nonciature apostolique et mit tous les bureaux dans le secret. Un si beau zèle ne pouvait rester sans récompense: les lettres lui furent remises au bout de dix jours, quand son frère, son oncle, Rouquette, Rome et Paris en eurent appris l'histoire.

Tolla était bien triste. Si ses lettres n'étaient pas mouillées de larmes, c'est que son mouchoir avait préservé le papier. Sa retraite n'avait pas imposé silence à ses ennemis. Les uns disaient que Manuel l'avait mise au couvent par mépris pour sa mère et pour ne la point laisser aux mains d'une intrigante. Les autres prétendaient que Manuel n'était pour rien dans l'affaire, et qu'elle avait été enfermée, par ordre du pape, comme une fille perdue. Un sbire, dont on ignorait le nom, s'était vanté publiquement d'avoir pris part à cette exécution. On faisait circuler des copies d'une lettre de monsignor Rouquette, où il était dit en propres termes: «Vous pouvez assurer aux Feraldi que Manuel n'est pas pour eux.» A l'appui de cette menace, la générale affirmait que Manuel était venu la voir trois heures avant de quitter Rome. Les gens sensés avaient beau dire que le fait était invraisemblable, puisqu'on l'avait vu partir à cinq heures du matin: les habitants de la via Frattina déclaraient qu'à deux heures un homme en habit laïque avait réveillé tout le quartier en frappant au numéro 15. Le séjour du couvent n'était pas trop aimable: les religieuses étaient bonnes, encore qu'un peu curieuses; mais les murs étaient bien gris, la cellule bien étroite, et pas de jardin! Amarella avait d'abord pris le couvent en patience, mais au bout de quelques jours son humeur s'était aigrie. M$\textsuperscript{me}$ Feraldi venait tous les soirs à la grille, avec Victor et Menico. Il y avait un parloir pour les domestiques et les s\oe{}urs converses, mais personne n'y était encore entré pour Amarella. Le comte était accablé d'affaires, Philippe allait chercher sa mère à Florence, l'abbé La Marmora venait deux fois par semaine. Tolla recommandait à Manuel de fréquenter les sacrements. «Cela est facile à dire,» répondait Manuel; «mais où trouver des prêtres dans cette ville de païens? A peine si en un mois j'en ai rencontré quatre, et tous Français! J'essaierais bien de me confesser en français, avec ce peu que j'ai appris; mais comment faire? il m'est impossible de parler français sans rire. Je prie matin et soir, et; je remets lès sacrements à mon retour. Les sacrements ne sont bons qu'à Rome.»

«Veux-tu savoir l'emploi de mes journées?» écrivait Tolla. «Je me lève à neuf heures; à dix, je vais à la messe; je reste à l'église jusqu'à midi, à prier Dieu pour toi. A midi, je dîne avec les religieuses. A une heure un quart, on sonne la cloche du silence, et chacune est obligée d'aller dormir dans sa chambre. A trois heures, le silence est rompu, et les religieuses descendent au ch\oe{}ur. Je me lève un peu plus tard, et je me mets à écrire jusqu'à ce qu'on vienne me prendre pour la lecture spirituelle et le rosaire, qui se dit dans une grande salle où elles sont toutes à travailler. A six heures, je vais à la grille voir ma mère et les personnes qu'elle amène avec elle. Après leur départ, je remonte à ma chambre, ou je me promène sur une terrasse qui est auprès; j'y reste tant que les s\oe{}urs sont à matines, c'est-à-dire une heure environ après \emph{l'Ave Maria.} Je descends alors à l'église, où je prie toute seule pendant un bon quart d'heure, puis je viens souper dans ma chambre. A neuf heures, on sonne le silence; tout le monde se couche, et l'on n'entend plus souffler dans la maison. Je m'enferme avec Amarella, qui dort dans un cabinet auprès de moi, et nous restons, elle à travailler, moi à lire, jusqu'à minuit. Nous faisons nos neuvaines et nos autres oraisons, puis je me mets au lit, et, jusqu'à ce que le sommeil me vienne, je pense aux jardins, aux forêts, aux belles fleurs et aux grands arbres, aux chevaux, au bal, à la musique, à l'amour, à la vie, car je ne vis pas.» «Moi,» répliquait Lello, «je me lève à dix heures; c'est un peu tard. Je déjeune à onze, je sors à midi pour voir les monuments, je dîne à cinq; puis vite au théâtre! Et après le spectacle, une petite promenade sur le boulevard des Italiens, où l'on voit une multitude de braves filles, mises à la dernière mode et attendant la Providence! C'est un spectacle horrible à voir, et qui inspire plus de dégoût que de désir.»

Il faut connaître les m\oe{}urs et les idées romaines pour comprendre tout ce que le dernier trait de cette peinture ajouta aux ennuis de Tolla. Rome n'est pas une ville d'innocence, tant s'en faut, mais c'est une ville de bon exemple: la police n'y souffre aucun scandale. Jamais un jeune homme n'y rencontre ces dangers ambulants qui fourmillent dans les rues de Paris. La débauche y est discrète, et le vice y a des allures cléricales. Tolla fut plus étonnée qu'une Parisienne à qui l'on dépeint les m\oe{}urs des îles Marquises. Son imagination chaste, mais active, se figura les boulevards des Italiens comme une porte de l'enfer, théâtre éclairé par des langues de feu, où l'on représentait jour et nuit le grand mystère de la tentation de saint Antoine.

Cependant Manuel ne se mettait jamais au lit sans baiser la pâle miniature de sa chère Tolla.

Lorsqu'on partit pour Londres, la question n'avait pas fait un pas. Manuel se fortifiait dans son amour et Tolla dans sa retraite. Mme Fratief était aux abois; elle allait faire une tentative sur Amarella, par acquit de conscience, Rouquette ne savait plus à quoi se prendre; il prévoyait bien que les plaisirs brumeux de l'Angleterre et les augustes réjouissances du couronnement ne produiraient pas plus d'effet que les séductions de Paris. Dans cet épuisement de toutes ses ressources, il essaya de regagner la confiance de Manuel. Il adoucit ses plaisanteries contre Tolla; il témoigna même un certain respect pour ce grand exemple de constance. Il laissa entendre que, s'il n'avait aucune pitié pour les amours follets et les romans d'une heure qui font les délices des pensionnaires et le désespoir des familles, il savait admirer l'héroïsme d'une passion persévérante. Sous la même inspiration, le colonel écrivit coup sur coup deux longues lettres à son ne Le gros homme adoucissait sa voix, il reprochait à Manuel son manque de confiance et frappait timidement à son c\oe{}ur pour se faire ouvrir. Sans sortir des banalités d'une correspondance de famille, il se vantait d'avoir une indulgence de père; rien ne pourrait lui ôter de la mémoire, qu'il avait fait sauter le petit Lello sur ses genoux. C'était pour lui, bien plus que pour son frère, qu'il avait renoncé aux douceurs du mariage et accepté les ennuis de la vie de garçon. Il s'était toujours promis de lui laisser tout son bien, à telles enseignes que le testament était fait et cacheté. Pourquoi donc l'objet d'une prédilection si marquée témoignait-il si peu de reconnaissance? On n'exigeait de lui aucun sacrifice, on ne demandait que de la sincérité.

Ce texte, un peu vague fut commenté savamment par Rouquette. «Vous avez tort,» dit-il, «de vous cacher de votre oncle: c'est un homme dont vous avez tout à espérer et rien à craindre. À votre place, je lui raconterais naïvement toute l'histoire, puisqu'il la sait, et je lui demanderais son consentement, quitte à m'en passer.»

«Me l'accordera-t-il, mon cher Rouquette?»

«Pourquoi non? Cependant, entre nous, je crois qu'il a le couvent de Saint-Antoine sur le c\oe{}ur. On a dit à Rome que vous aviez enfermé M$\textsuperscript{lle}$ Feraldi afin de la protéger contre votre oncle. Quelle injure pour un pauvre homme qui vous aime et qui vous a fait son héritier! Que voulez-vous qu'il pense, lorsqu'il voit que vous aimez mieux martyriser votre maîtresse que de la laisser vivre tranquillement dans la même ville que lui?»

«Il est vrai, mon bon Rouquette, Tolla souffre le martyre.»

«Vous le saviez! On vous a donc parlé de tous les maux qu'elle endure dans cet horrible couvent?»

«Elle m'en a écrit quelque chose.»

«Et vous a-t-elle parlé de sa santé?»

«Quoi! serait-elle malade?»

«Vous a-t-elle dit que l'ennui la dévorait jusqu'aux os? que la fièvre\ldots{}»

«Parlez, Rouquette, au nom du ciel! ne me cachez rien de ce que vous savez.»

«On dit qu'elle ne dort pas, qu'une fièvre lente la consume, qu'elle est maigre à faire peur, que ses beaux yeux se creusent, que ses couleurs se flétrissent et qu'on ne la reconnaît plus. Sa femme de chambre ne peut plus tenir au régime du couvent et menace de la quitter; que deviendra-t-elle, seule avec ses chagrins?»

«Pas un mot de plus, mon ami! je me prendrais moi-même en horreur. J'ai fait, sans le savoir, le métier d'un bourreau; mais ne croyez pas que je l'aie mise à Saint-Antoine par défiance de mon oncle. J'avais d'autres raisons: je craignais que l'amitié d'un certain jeune homme ne profitât de mon absence pour se métamorphoser en amour.»

«Quelle idée, mon cher Lello! La nature vous a-t-elle fait pour être supplanté par personne?»

«Non, mais\ldots{}»

«D'ailleurs je vous réponds, moi qui me connais en femmes, que celle-là est incapable de trahir. Vous savez si je la regarde avec des yeux prévenus: vous m'avez toujours vu la juger très librement, trop librement peut-être, car je commence seulement à apprécier ses vertus. Eh bien! croyez-en ma parole, Tolla ne vous trahira jamais.»

Manuel écrivit à Tolla qu'il lui permettait de quitter le cloître, si elle s'y trouvait toujours aussi mal. Bientôt il la pria de retourner chez ses parents. Sous la dictée de Rouquette, la simple prière se changea en ardent désir, puis en \emph{amoroso comando}. Enfin il déclara que la présence de sa maîtresse dans ce maudit couvent le mettait au désespoir. «Si tu persistais,» disait-il, «tu m'attirerais tant de chagrins, que mes forces physiques n'y tiendraient pas.»
\footnote{
L'italien: commandement d'amour
}
Cependant Tolla persistait.

«J'ai déjà trop enduré,» répondait-elle, «pour ne pas aller jusqu'au bout. Si je t'obéissais, j'exposerais tout le fruit de mes souffrances. Demande-moi ce que tu voudras, excepté le sacrifice de notre avenir: tu ne trouveras sou mise à tes volontés et même à tes caprices.»

«Qui donc te pousse à me faire sortir d'ici? Cette idée ne vient pas de toi. Veux-tu savoir ce qu'elle vaut? Demande-toi si ceux qui te l'ont inspirée désirent notre union, ou s'ils cherchent à l'empêcher? Tu sais où tendent tous leurs efforts. Irons-nous leur rendre le succès facile en suivant leurs conseils? Est-ce dans notre intérêt qu'ils parlent, ou dans le leur? Voudrais-tu qu'après avoir tout fait pour ne leur point laisser d'armes contre nous, j'allasse, leur en fournir par un changement de conduite?»

«Mes parents approuvent ma persévérance, la marquise Trasimeni m'engage à continuer, le docteur Ely m'a dit qu'on m'admirait dans les plus honorables maisons de Rome; l'abbé La Marmora jure que je suis perdue, si je passe le seuil de la porte; l'abbé Fortunati, qui de sa vie n'a dit ni oui ni non, avoue que l'idée d'entrer au couvent a été une inspiration du ciel. J'y reste donc Je l'ai juré, et moi je tiens mes promesses; ta main seule ou celle de la mort pourra m'en arracher.»

Pendant ces débats, le frère de Manuel épousa une Anglaise assez jolie et une dot véritablement belle. Manuel, abstraction faite de la dot, reconnut que sa belle-s\oe{}ur ne soutiendrait pas la comparaison avec Tolla. C'est dans la semaine qui suivit ce mariage que la chambre des lords revêtit sa robe de velours cramoisi doublé d'hermine pour assister au couronnement de la reine, une des plus belles fêtes de ce siècle. Manuel, confondu dans les rangs de la légation napolitaine, vit toute la cérémonie. Il mit son célèbre habit de cour à cinq heures du matin, et l'ôta à trois heures après minuit. Il serait mort de faim dans l'intervalle, s'il n'avait eu la précaution d'apporter des gâteaux dans ses poches. Cette mémorable journée et toutes les belles choses qui passèrent sous ses yeux ne lui firent pas oublier Tolla, bien au contraire. N'entendait-il pas crier: Vive Victoria! Et le nom de Victoria ne brillait-il pas en lettres de feu au milieu de toutes les illuminations? Le lendemain de la fête, plus amoureux que jamais, il écrivit au colonel, sous la dictée de Rouquette, quatre pages d'aveux et de prières. Lorsqu'il eut cacheté l'enveloppe, Rouquette l'embrassa paternellement: «Bravo!» lui dit-il, «vous agissez en bon neveu et en homme d'esprit. Cette petite lettre est grosse de plusieurs millions. Vous serez aussi riche que votre frère.»

«Maintenant, mon cher Rouquette, je vais attendre la réponse de mon oncle à Paris. Londres m'ennuie: je ne comprends pas les enseignes des boutiques, et je trouve que les Anglais ne sont pas polis.» Manuel n'avait pas plus compris la magnifique politesse des Anglais que les enseignes des boutiques.

«Ma foi!» dit Rouquette, «pour un rien j'irais à Paris avec vous. Votre frère est dans sa lune de miel, et il regarde le genre humain du haut en bas, comme les habitants de toutes les lunes. Il se passera de moi aussi facile ment qu'un perdreau d'un coup de fusil. Allons à Paris! nous continuerons nos leçons de français.»

Le 8 juillet, ils s'installaient pour la seconde fois à l'hôtel Meurice. Rouquette, pour être plus agile, dépouilla le \emph{monsignor}, et s'appela sur ses cartes le comte de Rouquette. Manuel, qui n'avait pas plus compris la cuisine anglaise que le reste, fut ravi de retrouver les dîners de l'hôtel et les déjeuners du café de Paris. Il allait au théâtre tous les soirs pour apprendre la langue. Rouquette n'avait qu'un regret, c'était de ne pouvoir l'y conduire deux fois par jour. Il espérait toujours que Tolla serait détrônée par une cantatrice ou une comédienne, et il savait par expérience que les passions de théâtre sont celles qui mènent le plus loin, parce que la vanité y vient en aide à l'amour. Malheureusement, au mois de juillet, les Italiens étaient en voyage et l'Opéra en réparation. À la Comédie-Française, tous les chefs d'emploi étaient en congé, et les banquettes regardaient jouer les doublures. Manuel était réduit au drame et au vaudeville. Il avait un faible pour le vaudeville, quoiqu'il lui arrivât rarement de saisir la plaisanterie du premier bond: il riait après tout le monde, et sa gaieté retardait de quelques minutes sur celle du parterre. Quelquefois même il digérait un bon mot jusqu'au lendemain, et surprenait Rouquette par un éclat de rire homérique qui partait comme une fusée au milieu du déjeuner.

Trois jours après leur arrivée, les deux inséparables s'étaient fourvoyés aux Folies-Dramatiques. Manuel, du haut de l'avant-scène, lorgna très attentivement une jeune première blonde et blanche que l'affiche désignait sous le nom de Cornélie, et que l'auteur avait honorée d'un rôle de trente-cinq lignes. Il profita du premier entracte pour questionner l'ouvreuse, et il apprit à son grand étonnement que M$\textsuperscript{lle}$ Cornélie Sarrazin était sage. Elle vivait chez son père, ne sortait qu'avec sa mère, et montrait avec orgueil deux petites mains rouges comme des pivoines; \textemdash{} d'ailleurs bonne fille, son c\oe{}ur n'avait pas parlé, mais rien ne prouvait qu'il fût sourd-muet de naissance. Cette nouveauté piqua la curiosité de Manuel, et il regretta que pour cinq francs l'ouvreuse ne lui en eût pas conté plus long. Heureusement M$\textsuperscript{lle}$ Cornélie, qui ne jouait que dans la première pièce, se débarbouilla sommairement de son blanc et de son rouge, et vint s'asseoir au balcon avec sa mère. Manuel grillait d'aller contempler de près cette vertu paradoxale et cette mère d'une sévérité provisoire. Son gracieux compagnon l'y conduisit comme par la main. Rouquette, en homme qui a fréquenté le théâtre et qui sait son répertoire, ouvrit la conversation par un compliment et un sac de raisins glacés. Les bonbons firent accepter le compliment; la toilette des deux amis fit agréer les bonbons: on refuse quelquefois les bonbons d'un poète, jamais ceux d'un millionnaire. M$\textsuperscript{me}$ Sarrazin apprécia du premier coup d'\oe{}il les bijoux insolents dont Manuel était Les mères d'actrices sont les personnes qui se connaissent le mieux en bijoux après les bijoutiers. Elle ne lui demanda pas s'il était de Paris: il faut être bien étranger pour venir au mois de juillet, paré comme une châsse à l'avant-scène des Folies. Rouquette présenta son ami, après s'être présenté lui-même, le tout en un tour de main: on ne doute jamais des gens qui ne doutent de rien. Il se garda bien de faire à Manuel les honneurs de M$\textsuperscript{lle}$ Cornélie; il affecta de travailler pour son compte et de se mettre en première ligne pour que Manuel eût le plaisir de le distancer. Le hasard voulut que la jolie blonde parlât un peu l'italien; elle l'avait appris à sa première année de Conservatoire, lorsqu'elle espérait avoir de la voix; elle en savait juste autant que Manuel de français. Lello fut ravi de rencontrer une femme capable de le comprendre: il lui sembla qu'il retrouvait l'Italie. Après le spectacle, M$\textsuperscript{me}$ Sarrazin se laissa reconduire jusqu'à sa porte: elle occupait un quatrième étage à l'entrée du faubourg Saint-Denis. Chemin faisant, on prit des glaces devant le café de l'Ambigu.

En retournant à l'hôtel, Manuel plaisanta beaucoup sur les vertus de théâtre qui daignent s'asseoir devant un café entre deux inconnus. Rouquette défendit Cornélie, il soutint que ce sans-gêne et cette facilité apparente ne prouvaient rien, que les artistes avaient des m\oe{}urs à part, et qu'on pouvait être une bonne fille sans avoir une mauvaise conduite. Bref, il paria pour la vertu, Manuel contre, et le lendemain à quatre heures ils montèrent l'escalier de M$\textsuperscript{me}$ Sarrazin. Manuel avait pris un bouquet chez M$\textsuperscript{me}$ Prevost: il s'en repentit en entrant au salon. La mère raccommodait un bas, la fille en tricotait un autre; M. Sarrazin fourbissait une canne gigantesque; il était tambour-major dans la garde nationale. Le meuble en velours d'Utrecht jaune sentait la vertu d'une lieue. «Mes fleurs sont ridicules, pensa Lello; «si j'avais su, j'aurais apporté des cornichons.» Il examina avec stupéfaction les lithographies qui pendaient à la muraille. C'était une galerie de papiers enluminés représentant \emph{Mélanie, Victorine, Henriette, Julie, le Marié} et \emph{la Mariée.} Le \emph{Marié} ressemble au monsieur que tout paysan voudrait être; il a des bagues à tous les doigts et une grosse chaîne autour du Il promène un sourire aimable autour de lui, et tient un bouquet dans une main, une boîte de bonbons dans l'autre. «Me voilà!» dit avec douleur le pauvre Manuel. Il lut au bas de l'image \emph{le Marié,} et en italien \emph{lo Sposo.} Évidemment cette lithographie était une personnalité. \emph{Victorine,} qu'un hasard malicieux avait suspendue à côté du \emph{Marié,} a les yeux plus grands que la bouche, un pot de fleurs dans la main droite, un éventail dans la gauche; la prodigalité de l'artiste lui a dessiné une rose sur le dos de la main. Un poète que le monde n'a pas connu a écrit au bas de cette image un distique que Manuel ne lut pas sans confusion:

\begin{quote}
Soyez constant dans vos amours,\\
Et vous serez heureux toujours.
\end{quote}

Pendant qu'il se livrait à cet examen, il entendit M$\textsuperscript{me}$ Sarrazin qui causait mobilier avec Rouquette et qui disait: «Ma fille économise pour acheter une armoire à glace, parce que l'armoire à glace est un meuble comme il faut.» «Bon!» fit-il en lui-même; «j'enverrai une armoire à glace, et je ne reviendrai plus.» Sur ces entrefaites, il entra quelques visites. Ce fut d'abord une amie de Cornélie, plus avancée qu'elle dans la science de la vie, car elle avait un cachemire des Indes, puis un jeune peintre un peu débraillé, puis un auditeur au conseil d'état ganté de neuf, puis un jeune journaliste, puis un vaudevilliste qui commençait à se faire jouer, puis un joli sous-chef du ministère de l'intérieur, enfin un jeune premier de la Gaîté. Ces six jeunes gens se partageaient, en attendant mieux, l'amitié de Cornélie. Le jeune premier était un ancien camarade du Conservatoire; le feuilletoniste \emph{la soignait} dans articles; le sous-chef la protégeait au ministère; le peintre allait faire son portrait pour la prochaine exposition; l'auditeur, sans être très riche, avait des parents assez généreux pour qu'on pût de temps en temps lui demander un service de cinq louis; le vaudevilliste achevait pour Cornélie une pièce en trois actes, destinée à mettre en relief toutes les perfections de sa petite personne. Au premier acte, elle était paysanne et montrait ses jambes; au second, elle était marquise et montrait ses épaules; au troisième, elle jetait son bonnet par-dessus les moulins, et montrait ses cheveux. Cornélie témoignait à tous ses amis une reconnaissance impartiale. Il n'y avait point de préférés, partant point de jaloux, et ces rivaux, qui ne se saluaient pas dans la rue, vivaient chez elle en bonne harmonie. Manuel entendit pour la première fois une conversation parisienne, vive, fringante, entremêlée de propos de coulisses, d'anecdotes du monde et de charges d'atelier, saupoudrée de calembours, pailletée de bons mots et assaisonnée de scandales dont personne ne se scandalisait. Il fut tout ébaubi de cette joûte assise, de ce tournoi d'esprit, de ces lances rompues et de cette petite fête courtoise donnée par six chevaliers en redingote à une reine d'amour en peignoir. Il comprit le discours de son oncle sur les séductions de Paris, et il se promit de ne point retourner à Rome avant d'avoir soupé en si curieuse compagnie. Il en eut bientôt la joie. Deux jours après, M$\textsuperscript{me}$ Sarrazin, qui avait reçu une armoire à glace anonyme, invita tout son monde à un piquenique. Le sous-chef envoya un saumon, le journaliste un pâté, le comédien un buisson d'écrevisses, l'auteur dramatique un parthénon en gelée d'ananas, le peintre un feu d'artifice complet qu'on aurait tiré dans le salon, si le propriétaire l'avait permis; l'auditeur fournit les truffes, Rouquette les vins, et Manuel l'argenterie. Trois ou quatre amies de Cornélie honorèrent de leur présence cette fête de famille. M. Sarrazin y présida en vrai tambour-major, avec la dignité bouffonne qui n'appartient qu'à cette institution. Manuel se grisa du vin de Rouquette et surtout des regards de M$\textsuperscript{lle}$ Cornélie. La table enlevée, on dansa tant qu'il resta des cordes au piano. Avant de se séparer, tous les convives prirent rendez-vous pour le surlendemain: on irait à Versailles voir jouer les grandes eaux et dîner à l'hôtel des Réservoirs. «Quand je pense,» disait Manuel, «que j'ai failli quitter la France sans connaître l'hôtel des Réservoirs et sans avoir vu les grandes eaux!»

Il mettait un pantalon blanc pour aller à Versailles, lorsque son domestique de place, qui ne l'accompagnait plus dans ses promenades, lui apporta la lettre suivante:
\begin{quote}

Du monastère de Saint-Antoine.

Rome, \oldstylenums{5} juillet \oldstylenums{1838}.

Où êtes-vous, Lello? Où sont vos promesses, votre amour et mes espérances? Moi, je suis toujours au couvent, dans la même cellule et dans le même ennui. Savez-vous combien il y a de temps que vous ne m'avez écrit? Vos lettres étaient ma seule consolation. Que Dieu vous pardonne le mal que vous me faites, et qu'il vous préserve de souffrir jamais autant Je n'ose vous dépeindre l'état de mon âme: j'empoisonnerais tous vos plaisirs. De ma santé, je ne vous en parle pas; vous comprenez que mon c\oe{}ur est trop malade pour que le corps puisse se bien porter. J'avais pris pour deux mois de courage; mais il y a plus de deux mois que vous êtes parti, et ma provision est épuisée. Mon ami, souvenez-vous de temps en temps, en courant à vos plaisirs, que vous m'avez aimée pendant quelques jours et que je vous adorerai toute ma vie.

\hspace*\fill---Tolla\end{quote}

«Venez-vous?» cria Rouquette à travers la porte. «La voiture est en bas: il ne faut pas faire attendre ces dames.»

«Je suis à vous, mon cher. Donnez-moi seulement cinq minutes: une petite affaire à expédier.» Il écrivit:
\begin{quote}

Paris, \oldstylenums{16} juillet \oldstylenums{1838}.

Ma chère Tolla,

Tu connais bien mal mon c\oe{}ur, si tu crois que c'est l'amour des plaisirs frivoles qui m'a entraîné loin de toi et qui me retient sur cette terre d'exil. Sache que le but secret de mon voyage était d'obtenir le consentement de mon oncle. On peut demander dans une lettre ce qu'on n'oserait pas solliciter de vive vois. Tu te souviens bien que j'ai toujours désiré que notre bonheur obtînt la sanction de ma famille, et tu es trop tendre fille pour blâmer un sentiment si délicat. Nous ne devons pas, pour satisfaire notre caprice, déclarer la guerre à nos parents.

Après une lettre affectueuse de mon oncle, dont les tendres reproches m'ont déchiré le c\oe{}ur, je me suis décidé à lâcher le grand mot. En effet, notre situation était trop pénible: nous aimer en ayant l'air de ne nous point connaître! D'ailleurs les méchantes langues avaient trop beau jeu contre nous.

Tu dois comprendre combien je désire et je crains tout à la fois la réponse mon oncle. Dieu veuille toucher son c\oe{}ur et nous le rendre favorable! Rien ne manquerait plus à notre félicité. Si sa réponse n'est pas telle que je le désire, il faudra essayer de tous les moyens pour changer sa volonté. Je ne retournerai pas à Rome que la question ne soit résolue. En attendant, je souffre le martyre, le doute me tue; plains moi.

\end{quote}

Rouquette frappa à la porte: «Il y a dix minutes que les cinq minutes sont écoulées!»

«Une seconde encore, mon bon ami. Je suis aussi pressé que vous.» Il continua:
\begin{quote}

C'est maintenant, ma Tolla, qu'il faut redoubler nos prières et mettre en Dieu toutes nos espérances. S'il a décidé que nous serions heureux, il saura bien attendrir le c\oe{}ur de mon oncle. Tournons-nous vers cette Vierge sainte qui aime tant à consoler les affligés; qui sait si elle ne voudra pas faire quelque chose pour nous? J'importune non-seulement saint Joseph, comme tu me l'as recommandé, mais tous les autres saints du paradis. Je voudrais qu'ils fussent plus nombreux, pour avoir plus d'avocats auprès du juge suprême. Enfin jetons-nous dans les bras de la Providence, et espérons. Je t'aime.

\hspace*\fill---Lello\end{quote}

«Oui, je l'aime!» dit Manuel en allumant une bougie pour cacheter sa lettre, «et il y a bien quelque mérite à garder mon amour intact au milieu des plaisirs de Paris. Elle craint, pauvre enfant, que je ne l'oublie; mais j'ai pensé vingt fois à elle pendant cet infernal souper! Rien ne triomphera de ma passion, parce que ma passion, c'est moi-même, et que je suis plus fort que tout. Il y a pourtant de pauvres sires qui une bouteille de vin de Champagne ou le sourire d'une jolie fille fait oublier leur maîtresse! Mon amour est comme la salamandre, il traverse le feu sans y brûler ses ailes!»

La promenade à Versailles fut suivie de beaucoup d'autres. M$\textsuperscript{me}$ Sarrazin s'aperçut que Manuel connaissait fort mal Paris et les environs: elle lui fit voir du pays. C'était une bonne femme, aimée au théâtre et dans son quartier, et dévouée sans préjugés au bonheur de sa fille. Elle avait toujours dit à Cornélie: «Mon enfant, l'autorité maternelle a ses limites, et je n'ai pas la prétention ridicule de te garder en sevrage jusqu'à l'âge de trente ans. D'ailleurs je le voudrais, la loi ne me le permettrait pas. Vois donc à te pourvoir. Si tu trouves un mari opulent, j'en serai bien aise, il me fera une pension alimentaire. Malheureusement les Folies-Dramatiques n'ont pas la vogue pour les mariages, et l'on n'y en a pas vu beaucoup cette année. Avec la dot que je te donne, à savoir le talent et la beauté, il est rare qu'on trouve à se marier définitivement. Passe encore si tu étais à l'Opéra! L'empereur de Russie paie tous les ans deux ou trois grands seigneurs pour qu'ils épousent des danseuses. Mais tu es aux Folies; règle-toi là-dessus. Moi, si jamais je te vois amoureuse d'un homme jeune, bien élevé et riche, je commencerai par te faire une bonne morale (si je t'ennuie, tu ne m'écouteras pas); puis j'irai trouver ce monsieur, je lui dirai tous les sacrifices que j'ai faits pour ton éducation, et s'il a bon c\oe{}ur, il me laissera ma fille, ou du moins il me remboursera mes dépenses.»

Le \oldstylenums{8} août \oldstylenums{1838}, trois semaines environ après le voyage à Versailles, Manuel apprit à n'en pouvoir douter que M$\textsuperscript{me}$ Sarrazin avait dépensé pour l'éducation de sa fille vingt mille francs et quelques centimes. La chute de M$\textsuperscript{lle}$ Cornélie ne fit pas plus de bruit que celle d'une pomme. Chose incroyable! aucun des six adorateurs de la jolie blonde ne tint rigueur à Manuel. Il crut même s'apercevoir qu'ils lui serraient la main avec gratitude. Il ne devina jamais que Cornélie, dans le cours des trois années précédentes, avait souscrit à ces messieurs six billets à ordre, payables à un moment précis, dont son bonheur avait avancé l'échéance. Rouquette avait un traité à part: il ne laissa pas oublier qu'il s'était inscrit avant Lello.

\enlargethispage{\baselineskip}

M. Sarrazin conserva l'habitude de marcher tête levée, excepté lorsqu'il passait sous la porte Saint-Denis.

Rouquette choisit le jour où Cornélie pendait la crémaillère dans un appartement de six mille francs pour envoyer à Manuel la réponse de son oncle. Il la gardait en portefeuille depuis une semaine.

Manuel hésita un instant avant de briser le cachet. Évidemment la lettre contenait un \emph{oui} ou un \emph{non.} Un \emph{non} lui fermait le paradis du mariage; un \emph{oui} le chassait du paradis terrestre qu'il venait de meubler à grands frais. Un \emph{non} le séparait de Tolla; un \emph{oui} l'arrachait à Cornélie. Cependant je dois dire à sa louange que son dernier v\oe{}u fut pour un \emph{oui.}

La lettre disait \emph{non.} Le colonel n'avait point cherché de périphrases. Il écrivait à son neveu: «Je te permets toutes les folies, excepté une. Jette ton argent par les fenêtres, je t'en donnerai d'autre; ne jette pas ton nom: nous n'avons que celui-là. Je t'ai dit souvent que je n'avais rien à te refuser, je le répète encore. Veux-tu un million? Mais si tu cherches une corde pour te pendre, je n'en suis pas marchand. Remarque bien que tu peux te marier sans mon consentement: ce n'est donc pas une permission que tu me demandes, c'est un conseil. Or le diable en personne ne saurait me contraindre à t'en donner un mauvais. Fais ce que tu voudras: tu es maître absolu de tes actions, comme moi de mes écus. Je ne te défends pas d'épouser la fille qui t'a choisi et qui te fait la cour depuis plus d'une année; mais je t'avertis que si tu persistes, tu peux te dispenser de m'écrire; je ne te répondrais pas. Sur ce, je t'embrasse. Faut-il ajouter: \emph{pour la dernière fois?}»

«Diable d'homme!» se dit Lello. «Il parle avec autant d'assurance que s'il avait raison. Je vais mal souper ce soir. Rouquette!»

Rouquette n'était jamais loin. Il parcourut la lettre, et la trouva conforme au brouillon qu'il avait envoyé. «Eh bien?» Demanda-t-il.

«C'est moi qui vous dis: ‹eh bien?›»

«Eh bien! votre oncle a tort. Il ne rend pas justice aux vertus de M$\textsuperscript{lle}$ Feraldi.»

«N'est-il pas vrai, Rouquette? Tant de vertu, de beauté, de noblesse\ldots{}»

«Je ne parle pas de sa noblesse: on m'assure que la généalogie du docteur Feraldi était un peu véreuse. Quant à la beauté, elle en a eu autant que femme du monde: maintenant, nous ne savons pas ce qui lui en reste. Je passe légèrement sur la question financière. Elle vous apporte en dot une vigne de deux cent mille francs; c'est un joli denier. De plus elle assure par contrat un héritage de quatre ou cinq millions au prince votre frère: toute la fortune du colonel! Mais elle a des vertus. Or les vertus sont hors de prix par le temps qui court; vous le savez bien, vous qui venez d'en acheter une.»

«Mauvais plaisant!\ldots{} Rouquette, vous devriez intercéder auprès de mon oncle!»

«Bien obligé! Je trouve que j'ai assez d'ennemis.»

«Alors faites-moi un brouillon.»

«Pour dire que vous vous soumettez?»

«Non, pour expliquer que je ne peux pas me soumettre.»

«À quoi bon? Il jetterait ma prose au feu dès la première ligne.»

«Il faudrait pourtant lui faire savoir que je suis engagé d'honneur envers le comte Feraldi.»

«Une idée! Priez M. Feraldi de lui conter toute l'affaire. C'est lui qui est le plus intéressé à la conclusion de ce mariage, car vous conviendrez qu'il y gagne plus que vous. D'ailleurs n'est-il pas avocat? Il ne refusera pas de plaider sa propre cause. Faut-il vous faire un brouillon pour le comte?»

«Faites, mon ami; je ne lui ai jamais écrit, et je ne saurais pas comment m'y prendre.» Manuel se promena de long en large dans sa chambre, tandis que Rouquette écrivait:
\begin{quote}

Paris, \oldstylenums{11} août \oldstylenums{1838}.

Très cher comte,

Je n'avais jamais pris la liberté de vous écrire, sachant comme votre profession vous occupe, et combien le temps des hommes d'affaires est précieux; mais une cruelle nécessité me force à vous imposer l'ennui de me lire.

Depuis mon départ de Rome, mon unique préoccupation a été de faire approuver à mes parents mon mariage avec mademoiselle votre fille. Après deux mois d'hésitations, je me suis armé de courage, et j'ai écrit à mon oncle. Je lui ai tout confessé, je lui ai fait connaître la violence de mon amour et l'ancienneté de nos engagements, j'ai dépeint à ses yeux les vertus qui sont la plus belle richesse de Vittoria, j'ai décrit avec une scrupuleuse exactitude l'état de nos sentiments, j'ai conjuré mon oncle de ne pas séparer deux c\oe{}urs si bien unis. J'ai attendu longtemps sa réponse; plût à Dieu qu'elle ne fût jamais arrivée! Non-seulement mon oncle se refuse formellement à ma demande, mais il déclare en terminant qu'il m'embrasse pour la dernière fois.

Vous pouvez vous figurer mes angoisses au milieu de ce conflit d'affections. Je ne voudrais pas renoncer au bonheur, mais le devoir me commande de respecter la volonté de ma famille. Je voudrais dompter mes passions, mais quand je songe aux vertus de l'ange que j'adore, la force me manque.

Dans ce cruel embarras, je me tourne vers vous, et je remets notre sort entre vos mains. Puisque le destin me condamne ou à obtenir ce consentement ou à faire le terrible sacrifice, je viens vous prier à mains jointes de plaider ma cause auprès de mon oncle et d'obtenir, par une intervention amicale, ce que j'ai eu la douleur de m'entendre refuser. Si, par un malheur que je n'ose prévoir, vos prières échouaient comme les miennes, croyez, monsieur, que j'ai trop à c\oe{}ur la réputation de mademoiselle votre fille pour continuer les relations d'intimité qui existaient entre nous; mais je conserverai pour elle et pour votre famille une estime éternelle.

Je me fais un devoir de vous déclarer que je n'ai mis dans le secret que mon frère et mon oncle. Tout est resté entre nous, et l'honneur de la jeune fille a été soigneusement sauvegardé. J'espère que ma résolution sera approuvée de vous et de votre vertueuse fille, à qui je vous autorise à montrer cette lettre. Je vous prie de présenter mes compliments, et suis pour la vie votre très affectionné serviteur et ami,

\hspace*\fill---Manuel Coromila-Borghi.\end{quote}

Quand Manuel eut copié cette lettre, Rouquette réclama son brouillon pour le brûler. Il le mit sous enveloppe et l'envoya à M$\textsuperscript{me}$ Fratief.

Lello écrivit ensuite à Tolla une lettre touchante:
\begin{quote}

Mon c\oe{}ur saigne, disait-il. Dieu! quelle sentence cruelle! D'un côté la passion qui me consume, de l'autre le devoir qui m'égorge. J'entends ta voix qui me crie: Fais ton devoir, quoi qu'il en coûte; le devoir est la loi de Dieu. Oui, ma Tolla, tu es assez vertueuse pour me parler ainsi. Tu aimes tes parents, tu sais qu'il est impossible de rien refuser à ces êtres chers et respectables qui nous ont tenus tout enfants sur leurs genoux; tu approuveras la résolution que j'ai prise. Si tu écoutes le monde, il me blâmera peut-être; si tu fais parler ta conscience, elle me donnera raison.

Un espoir nous reste. J'ai écrit à ton père, je l'ai conjuré de s'entremettre pour nous auprès de mon oncle: peut-être obtiendra-t-il quelque chose. Si cette dernière branche de salut nous échappe, hélas! je suis forcé de t'oublier. Le pourrai-je? Dieu, qui exige de nous ce sacrifice, nous donnera la force de l'accomplir; mais si mon c\oe{}ur doit te retirer sa tendresse, jamais il n'oubliera l'image d'un ange orné de tant de belles vertus, et tu auras une place éternelle dans l'estime de ton très affectueux ami,

\hspace*\fill---Lello.\end{quote}
\begin{quote}

P.-S. De la réponse de ton père dépendra notre bonheur.

\end{quote}

Manuel monta en voiture avec Rouquette, porta ses lettres à la grande poste, et se fit conduire au nouvel appartement de sa maîtresse. L'arrivée des deux amis interrompit le jeune peintre, qui ébauchait un petit portrait de Cornélie.
