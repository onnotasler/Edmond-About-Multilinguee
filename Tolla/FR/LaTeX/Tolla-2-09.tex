\addchap{\RomanNumeralCaps 9.}

Amarella n'était pas entrée au couvent pour le plaisir de prier Dieu et d'accompagner sa maîtresse: elle pensait qu'on peut prier partout, et son dévouement pour Tolla n'allait pas jusqu'à l'abnégation. Elle avait la captivité en horreur, comme tous les êtres remuants; elle était friande du grand air, comme tous ceux qui sont nés au village; elle aimait à se faire voir, comme toutes les femmes. Ajoutez que, comme toutes les Romains des deux sexes, elle avait la passion de la loterie. La loterie est un jeu légal et pontifical, une partie engagée entre le saint-père et ses sujets: les fidèles y gagnent quelquefois, le pape toujours. Amarella faisait comme tous les domestiques, mercenaires, mendiants et frères quêteurs de la capitale du monde chrétien: elle économisait onze sous par semaine pour avoir le droit de prendre un billet, de rêver trois numéros, et d'attendre, confortablement logée dans château en Espagne, le tirage du jeudi et la ruine de ses espérances. En entrant à Saint-Antoine, elle avait renoncé à la loterie, au grand air, à la liberté et à l'admiration des hommes, le tout pour plaire à Menico. Menico lui avait dit en la prenant par la taille: «Si tu étais une brave fille, tu irais tenir compagnie à mademoiselle. Crains-tu de t'ennuyer? Je te promets que vous recevrez des visites: le parloir n'est pas fait pour les chiens. As-tu peur que tous les garçons ne se marient en votre absence et qu'il n'en reste plus pour toi? Sois tranquille: j'en connais un qui attendra patiemment et qui fera v\oe{}u, si tu l'exiges, de ne pas regarder une femme avant votre retour.» Ces promesses tant soit peu jésuitiques, appuyées de quelques caresses, avaient trompé la subtile Amarella. Elle sacrifia trois mois de sa liberté, avec la confiance aveugle d'un joueur qui risque son seul habit sur la carte qu'il croit bonne. Ce Menico si longtemps poursuivi était à ses yeux quelque chose de plus qu'un homme: c'était un \emph{terne} qu'elle avait nourri deux ans.

Lorsque les portes du cloître se fermèrent sur elle et qu'elle vit Dominique pleurer côte à côte avec Lello, elle sentit naître au fond de son c\oe{}ur quelque sympathie pour sa maîtresse: une conformité d'âge, de chagrin et d'espérance l'unissait à Tolla, et peu s'en fallut qu'elle ne lui fît confidence de son amour. Quinze jours se passèrent sans qu'elle reçût une visite de Dominique: elle s'imagina qu'il était retenu au palais Feraldi par quelque indisposition légère ou par la nature sédentaire de ses fonctions. Elle attendit une seconde quinzaine, et s'arma d'une patience rageuse: «Peut-être veut-il m'éprouver,» pensait-elle. Mais lorsqu'elle sut, par une indiscrétion innocente de Tolla, que Dominique venait tous les jours au couvent avec la comtesse, lorsqu'elle fut forcée de reconnaître qu'elle avait été sa dupe, elle se prit d'une haine effroyable, non contre lui, mais contre Tolla. La jalousie lui fit voir une rivale dans sa maîtresse; elle la soupçonna d'avoir usé d'une indigne coquetterie pour voler un c\oe{}ur plébéien dont elle n'avait que faire; elle se rappela les naïves confidences de Menico sur la route de Lariccia, les larmes de Tolla lorsqu'on l'avait cru mort, et le fameux baiser qu'elle lui avait donné le jour de l'Assomption: elle était trop aveuglée pour comprendre que le prétendu amour de Dominique était une adoration religieuse, et que Tolla ne s'en apercevait pas plus que les madones peintes et dorées n'entendent les prières qu'on murmure à leurs pieds. Dans un premier mouvement de colère, elle courut à sa chambre et fit ses paquets, bien décidée à abandonner Tolla à ses ennuis; puis elle se ravisa, remit tout en place et redescendit dans la cour en souriant à un autre projet de vengeance.

Dès ce jour, elle commença contre sa maîtresse une guerre sourde: «Attends!» dit-elle, «je ferai de ton c\oe{}ur une pelote à épingles!» Lorsque Tolla avait reçu quelque bonne nouvelle, Amarella accourait partager sa joie; ce n'était jamais sans y verser une goutte de poison: «Il vous aime,» dit-elle, «il veut donner au monde un grand exemple de constance. Qui l'aurait cru? Mademoiselle voit bien qu'il vaut mieux que sa réputation. Je le savais, moi, qu'il ne vous tromperait pas comme toutes les autres.» Si Tolla était triste, si cette pauvre âme, à force de creuser l'avenir, avait trouvé quelques raisons de désespoir, Amarella se faisait un visage de gaieté et d'insouciance, elle étourdissait la maison de son rire argentin et sonore, elle venait s'asseoir auprès de sa maîtresse et lui faire une peinture charmante du bonheur qu'elle n'espérait plus: «Pourquoi vous chagriner, mademoiselle? Les beaux jours viendront. Qui sait si dans deux mois vous n'entrerez pas à l'église, habillée comme une reine, en robe de velours blanc avec des boutons de perles, et une couronne d'oranger dans les cheveux? Dans un an, nous baptiserons un beau petit Lello, rouge comme une écrevisse: il me semble déjà que je l'entends crier! Dans vingt mois, il sera blanc comme du lait, frais comme une rose et ferme comme une pomme. Les dents lui viendront deux à deux; il essaiera ses mains mignonnes; il voudra parler et faire de longues phrases, mais il ne saura dire que \emph{mamma} et \emph{babbo}; il prendra son élan pour courir, mais il ne saura pas mettre une jambe devant l'autre, et il embrouillera ses deux petits pieds comme s'il en avait cinq ou six. Vous vous agenouillerez près de lui sur le tapis, vous le tiendrez par la ceinture de sa robe\ldots{}»

«Vous pleurez, mademoiselle? Sotte que je suis! je vous ai fait de la peine. J'oubliais que si M. Coromila vous abandonne, vous avez fait v\oe{}u de rester au couvent et de renoncer au bonheur d'être mère! Allons, mademoiselle, ne vous désolez pas; cela ne sera rien: peut-être n'êtes-vous pas tout à fait trahie. Voulez-vous que je vous chante une jolie chanson?»

\begin{quote}
Io ti voglio ben assai,\\
Ma tu\ldots{}
\end{quote}

«Tais-toi!» criait Tolla, et elle éclatait en sanglots.

«Chut! ma chère demoiselle; les religieuses vont vous entendre. Vous avez juré de renfermer votre amour en vous-même.»

Tolla retenait ses pleurs et dévorait son mouchoir pour s'empêcher de crier. Elle tint toutes ses promesses, et sans les bavardages calculés d'Amarella, personne dans le couvent n'aurait deviné ses douleurs. Les religieuses de Saint-Antoine étaient jeunes pour la plupart: quelques-unes avaient moins de vingt ans. Elles observaient scrupuleusement la règle de leur ordre, et surtout leur v\oe{}u d'obéissance: elles ne pouvaient ni changer de robe, ni laisser une bouchée de la portion qu'on leur servait, sans en demander la permission. Séparées du monde avant de l'avoir connu, elles se berçaient dans la monotonie des habitudes monastiques, et se croyaient heureuses parce qu'elles étaient résignées. Tolla enviait la tranquillité de leur âme, comme les vivants sont quelquefois jaloux des morts. Elle respectait leur ignorance, cachait son amour, s'efforçait de rire lorsqu'elle était triste, et de manger lorsqu'elle avait le c\oe{}ur gros; sinon, toute la table aurait voulu savoir pourquoi elle n'avait pas d'appétit. Amarella se plut à mettre tout le couvent dans les secrets de sa maîtresse: elle ne doutait pas qu'un tel scandale ne retombât sur la tête de Tolla. L'effet ne répondit pas à son attente: les s\oe{}urs n'eurent que de la pitié et de la tendresse pour cette pâle victime d'un mal qu'elles ne connaissaient point. Peut-être quelqu'une des plus jeunes envia-t-elle à son tour les souffrances de la belle pensionnaire; mais jeunes et vieilles observèrent une discrétion unanime, et donnèrent le rare exemple d'une communauté religieuse possédant un secret sans le commenter.

Le 23 août, après quatre mois de captivité volontaire, sans une seule visite de Dominique, Amarella avait épuisé toutes les ressources de la haine et ne savait plus à quel démon se vouer. On lui dit qu'un homme l'attendait au parloir: elle y courut en se demandant quel remords de conscience pouvait lui ramener Dominique; mais ce n'était pas Dominique qui l'avait fait appeler: c'était un gros homme blond, bien rasé, bien frisé, bien nourri, bien fleuri et d'une physionomie toute paternelle. Ce digne personnage, qu'elle reconnut à l'accent pour un Napolitain, lui apprit que sa belle conduite et son dévouement évangélique avaient touché le c\oe{}ur d'une très noble et très riche étrangère, que cette dame, Russe de nation, mais catholique de religion, voulait à tout prix l'attacher à son service, prête à doubler ses gages, s'il le fallait. Amarella, prise entre la crainte de lâcher sa vengeance et l'envie de regagner sa liberté, demanda quelques jours de réflexion. Elle allégua que la famille Feraldi lui avait promis une dot de cent écus, si elle restait avec mademoiselle.

«Qu'à cela ne tienne,» répondit l'inconnu. «La personne qui m'envoie est au moins aussi généreuse que vos Feraldi. Réfléchissez au plus vite; je reviendrai demain.»

Le même jour, le comte Feraldi reçut les deux lettres de Manuel en date du 11 août. Après avoir lu la sienne, il n'hésita pas à ouvrir celle qui portait l'adresse de Tolla La comtesse écouta cette lecture d'un \oe{}il sec et stupide: elle croyait entendre l'arrêt de mort de sa fille. Victor était assis, serrant les poings et mordant ses lèvres. Cette consternation se changea en fureur lorsqu'on vit accourir le docteur Ely, l'abbé Fortunati et Philippe Trasimeni; chacun d'eux avait reçu, sans savoir comment, une copie de la lettre au comte. Un exemplaire de la même lettre avait été placardé à la porte du palais Feraldi, et Menico, qui l'avait arraché, l'apporta en pleurant. Les parents et les amis de Tolla tinrent conseil en tumulte: Menico jurait d'assommer le colonel et tous ses domestiques; Philippe et Victor voulaient partir le soir même pour Paris; le docteur assurait qu'en lisant une seule de ces lettres Tolla mourrait sur le coup; la comtesse offrait de se jeter aux pieds du vieux Coromila; l'abbé parlait d'en appeler au pape; le comte avait perdu la tête et ne savait auquel entendre. Il allait, venait, se laissait tomber sur une chaise, se levait en sursaut, froissait dans ses mains les deux lettres de Manuel, et répétait machinalement le post-scriptum de la dernière: «De la réponse de ton père dépendra notre bonheur!» Tout était désordre, affliction et contradiction; chacun parlait au hasard sang écouter ni les autres ni soi-même. Au milieu de la confusion générale, Menico prit sur lui d'aller chercher l'oncle de la comtesse, le cardinal Pezzato. L'entrée de ce beau vieillard en cheveux blancs apaisa le tumulte et rassit les esprits les plus exaltés. Les jeunes gens fermèrent la bouche, et tous les conseils violents se turent en présence de l'auguste octogénaire, qui avait été ministre de Pie VII et de Léon XII. Le cardinal se fit lire les deux lettres par Victor Feraldi, dont la voix tremblait d'émotion et de colère. Il déclara sans hésiter que la prière de Manuel était absurde, et que le comte ne pouvait pas décemment demander au colonel la main de son neveu; mais comme le jeune Coromila s'était engagé par serment à épouser Vittoria Feraldi, comme il avait invoqué le nom de Dieu à l'appui de ses promesses, l'affaire était du ressort de la police ecclésiastique, et il fallait recourir au cardinal-vicaire.

\enlargethispage{\baselineskip}

L'intervention de la police dans les affaires de conscience est un des traits caractéristiques de l'administration pontificale; les papes ne croient pas gouverner des hommes, mais des âmes. Leurs tribunaux participent de la nature du confessionnal: le juge est doux, discret, familier, curieux, indulgent pour les fautes confessées, prêt à tout pardonner hormis la fierté et la résistance, inhabile à distinguer un péché d'un délit et un mauvais chrétien d'un mauvais citoyen, confiant dans les verrous, ennemi de la violence, incapable de verser le sang d'un criminel et capable d'oublier un innocent en prison. La police est plus taquine que rigoureuse et plus humiliante qu'oppressive; le gouvernement est un despotisme velouté, onctueux, décent, modeste, et patient parce qu'il se croit éternel. Le prince Odescalchi, cardinal-vicaire, ne fut point surpris de la demande du cardinal Pezzato: il trouva tout simple que, pour empêcher un jeune fou de violer ses serments et d'offenser la majesté divine, on eût recours à l'autorité du vicaire de Jésus-Christ. D'ailleurs le prince Odescalchi était allié à la famille Feraldi: sa s\oe{}ur avait épousé en \oldstylenums{1817} un cousin germain du comte. Enfin la vertu, le malheur et la beauté de Tolla lui inspiraient un vif intérêt. Sans accorder une entière confiance aux accusations qui s'élevaient contre son secrétaire intime, il fit écrire à Rouquette que son congé était expiré, et qu'il eût à revenir au plus tôt s'il tenait à sa place. Sans vouloir contraindre en rien la volonté du colonel Coromila, il promit de le mander en sa présence et de ne rien négliger pour obtenir son consentement. Il pria le comte de lui adresser une note courte et précise en forme de supplique, contenant en quatre pages le résumé de ses relations avec Manuel; il demanda qu'on lui remît les lettres, la bague et le portrait, et qu'on y joignît un extrait de tous les passages de la correspondance où le nom de Dieu était positivement invoqué. Le cardinal Pezzato se rendit en toute hâte au palais Feraldi, et rédigea avec le comte la supplique suivante:
\begin{quote}

Prince éminentissime,

Le comte Alexandre Feraldi se voit contraint d'implorer l'intervention officieuse de votre éminence révérendissime en faveur d'une noble, innocente, vertueuse enfant, qui a eu l'honneur d'être tenue sur les fonts de baptême par la propre s\oe{}ur de votre éminence, mariée au cousin germain de l'exposant.

Cette enfant, fille unique et l'aînée des deux enfants du suppliant, comblée des plus rares talents par les bontés de la Providence, a reçu l'éducation la plus chrétienne, la plus noble et la plus vertueuse qu'on puisse trouver dans notre Italie. Les certificats ci-joints et la liste des prix et des accessits qu'elle a remportés à l'institut impérial et royal de Marie-Louise à Lucques feront voir à votre éminence si elle a répondu aux soins de ses parents. Rentrée dans sa famille, toute la sollicitude de son père et de sa mère s'est employée à lui trouver un établissement avantageux et honorable. Plusieurs partis se sont offerts, qui ont été repoussés l'un après l'autre, parce qu'aucun ne semblait digne d'elle. En dernier lieu, un des fils de la très noble et très riche famille Morandi, d'Ancône, se mit sur les rangs, et pressa de tout son pouvoir la conclusion de cette affaire, comme il résulte des lettres originales que l'on soumet à votre éminence.

Ce fut alors que Manuel, cadet de la très illustre famille Coromila-Borghi, qui, en rencontrant la jeune fille dans les réunions de la noblesse, avait pris pour elle des sentiments affectueux, se présenta à l'exposant et à sa femme dans la compagnie d'un très honorable cavalier, le marquis Trasimeni, et, déclarant avoir connaissance de l'affaire qui allait se conclure avec Morandi, demanda que l'on rompît toutes les négociations, si l'on croyait que la jeune fille pût être plus heureuse avec lui, car il était décidé à la prendre pour femme. Les époux Feraldi ne manquèrent pas d'opposer à Manuel Coromila toutes les difficultés imaginables relativement au consentement de son père, sans lequel les comtes Feraldi n'auraient jamais permis une telle union. Il prit sur lui d'obtenir ce consentement, n'y ayant rien qui pût y faire un légitime obstacle, puis que la jeune fille n'était ni de la basse classe ni de la bourgeoisie, mais d'un rang à avoir pour tantes la s\oe{}ur de votre éminence et la fille du prince Barberini.

Après s'être entendu dire que sa démarche le rendait garant du consentement de son père et responsable de l'avenir de la jeune fille, il renouvela ses déclarations et ses serments, ajoutant que, vu le déplorable état de la santé de son père, il attendrait qu'il fût rétabli pour lui demander son assentiment. Rassuré par ces paroles, le comte Feraldi lui déclara que la dot de sa fille devait être de vingt mille sequins en argent, mais que pour reconnaître autant qu'il était en lui l'honneur d'une telle alliance, il doublerait la somme, et donnerait quarante mille sequins en biens allodiaux situés dans l'île de Capri, libres de toute hypothèque, dépendance ou redevance, et faisant partie du domaine patrimonial de sa famille: lesdits biens évalués quarante mille sequins dans une estimation faite quinze ans auparavant à l'occasion d'un partage. Afin que Manuel Coromila, dans une affaire de si grand poids, pût se décider en toute connaissance de cause, on lui confia les lettres du comte Morandi. Il les rapporta le lendemain, et renouvela, après les avoir froidement examinées, tous les engagements qu'il avait pris. Ce fut après cette seconde et formelle déclaration que l'on fit dire au comte Morandi que sa demande, si honorable qu'elle fût, ne pouvait être agréée. Durant toutes ces négociations, la jeune fille, en bonne chrétienne, alluma des cierges devant toutes les images miraculeuses, se recommanda aux prières des communautés les plus saintes, fit et fit faire des neuvaines et des tridui en nombre incroyable, pour intéresser le ciel au succès de l'affaire.

Au mois de février, Dieu rappela à lui le prince Coromila, et Manuel, majeur d'âge, fut maître de ses actions. Des devoirs de reconnaissance et de respect le liaient à son oncle le colonel, et lui commandaient à tout prix d'obtenir son consentement. Sollicité d'entreprendre à cette fin les démarches nécessaires, il répondit qu'il le ferait aussitôt après le mariage de son frère aîné, et il annonça son départ pour l'Angleterre. Les époux Feraldi n'eurent pas de peine à deviner dans quelle intention la famille Coromila poussait Manuel à ce voyage. Cependant ils ne voulaient pas croire qu'on se proposât de conduire ce jeune homme au parjure et leur fille innocente au sacrifice. Ils mandèrent Manuel Coromila, et après l'avoir adjuré de penser sérieusement à ce qu'il avait fait et à ce qui pourrait advenir par la suite, ils lui déclarèrent, en présence de la jeune fille elle-même, que si la mort de son père avait changé ses idées ou s'il prévoyait que ce voyage pût les modifier, il était encore temps de retirer sa parole, et qu'on le déliait de toutes les obligations qu'il avait contractées, mais si, majeur et libre comme il l'était, il réitérait ses promesses, qu'il se souvînt bien que son engagement devenait irrévocable, nonobstant toute injuste opposition de sa famille. Il répondit à cette déclaration par les promesses les plus formelles, les protestations les plus ardentes, et les plus terribles serments de ne changer jamais.

Pour s'engager irrévocablement, et pour fermer la bouche à tous ceux qui voudraient, par de faux rapports, le prévenir contre la jeune fille, il voulut qu'elle se renfermât durant son absence dans un couvent cloîtré, et il pria lui-même leur commun directeur, le digne abbé La Marmora, d'aller l'y confesser tous les huit jours. La vertueuse Vittoria, soumise aux volontés de celui qui avait juré de devenir son époux, passa des brillants salons de la capitale à la vie austère d'un cloître. Ses prières et ses vertus excitèrent l'admiration et gagnèrent l'amitié de toute cette communauté religieuse: votre éminence révérendissime peut aisément s'en assurer.

Cependant les lettres de Manuel Coromila se succédaient à chaque courrier. Ces lettres attestent ses engagements et les sacrifices de la jeune fille. Elles sont pleines de serments, non pas de ces serments légers qui s'échappent au hasard au milieu d'un vague parlage d'amour, mais de serments solennels, entourés des idées les plus sérieuses et des sentiments les plus religieux. Votre éminence révérendissime remarquera en plus de dix endroits l'invocation expresse de ce Dieu redoutable qui ne veut pas que son nom devienne un instrument de fraude et d'imposture. Ces lettres prouvent d'une manière éclatante la pureté des sentiments dont deux c\oe{}urs sont enflammés. Le conseil réciproque de fréquenter les sacrements., la confiance dans la bonté de Dieu, l'invocation de la Vierge et des saints, chose bien rare dans des écrits de ce genre, font de toute cette correspondance une lecture agréable et édifiante, propre à toucher les c\oe{}urs honnêtes et religieux; \textendash{} tout cela jusqu'à la lettre du 16 juillet inclusivement.

Tout à coup, et hors de toute attente, l'exposant reçoit une lettre en date du 11 courant, où Manuel, changeant brusquement de langage, invite l'exposant lui-même, père de la malheureuse jeune fille, à intervenir auprès du colonel Coromila pour obtenir le consentement qu'il refuse. Si cette démarche (inutile, absurde et inconvenante) reste sans résultat, Manuel déclare qu'il se croira délié de tous ses engagements, alléguant qu'une passion et un amour doivent céder aux devoirs impérieux de la famille. Si l'on ne mettait dans la balance qu'une simple passion et un amour aveugle, cette maxime serait incontestable et sacrée; mais, dans l'espèce, il s'agit de tout autre chose, puisqu'à l'amour et à la passion se joignent des devoirs directs et positifs, résultant d'obligations réelles contractées par une personne majeure, sans qu'elle y ait été amenée ni par contrainte, ni par prière, ni par séduction. Ajoutez à cela les devoirs de stricte justice résultant des dommages irréparables causés à une noble et vertueuse fille âgée de plus de vingt ans, qui a renoncé à un établissement avantageux, qui s'est laissé compromettre aux yeux de toute l'Italie, qui a vécu quatre mois enfermée dans un cloître, qui est d'une santé assez délicate pour succomber à la perte de ses légitimes espérances, qui enfin a fait v\oe{}u de prendre le voile et de renoncer à son avenir temporel, si elle était abandonnée; ajoutez la sainteté terrible de serments formels, réitérés à haute voix et par écrit, avec l'invocation expresse du nom de Dieu, et votre éminence reconnaîtra que Manuel n'est pas, comme il le suppose, mis en demeure d'opter entre sa passion et ses devoirs envers son oncle, mais entre ces devoirs de simple reconnaissance et les lois inviolables de la justice, de l'honneur, de la conscience et de la religion.

Éminence révérendissime, il faut que le colonel Coromila n'ait pas été informé de tous les faits énoncés ci-dessus, car il est certain que, s'il en avait connaissance, un cavalier si accompli et un chrétien si exemplaire emploierait son autorité à tout autre chose qu'à commander le parjure et le sacrilège. Si les discours de la malice et de l'envie n'avaient pas égaré sa conscience, il serait le premier à favoriser un projet formé au milieu des prières, et que la prière a sanctifié jusqu'à ce jour. Rome entière le cite comme un homme juste et craignant Dieu. Pour obtenir le consentement qu'il refuse, il ne faut ni supplications ni menaces, il faut seulement lui apprendre la vérité: on aura gagné son c\oe{}ur lorsqu'on aura dessillé ses yeux.

Le comte Feraldi a l'âme trop haute pour aller lui-même plaider devant le colonel la cause de sa fille; mais il serait un mauvais père s'il ne cherchait pas à lui faire connaître les engagements sacrés de Manuel.

C'est pourquoi le suppliant se jette aux pieds de votre éminence révérendissime. Plein de confiance dans l'efficacité d'une intervention qu'il espère sans oser la demander, il a le très haut honneur, en baisant votre pourpre sacrée, d'être, avec la plus profonde vénération, de votre éminence révérendissime,

Le très humble, très dévoué et très obéissant serviteur,

\hspace*\fill---,,Alexandre Feraldi.\textquotedblleft{}\end{quote}

Voilà comme on écrit à un cardinal-vicaire. La supplique, copiée en belle ronde sur papier jésus in-folio, fut portée le soir même au prince Odescalchi, avec l'extrait de la correspondance et toutes les lettres de Lello, que la comtesse emprunta à sa fille pour les relire. On n'osa lui demander ni le portrait, ni l'anneau, de peur d'éveiller ses soupçons.

Le lendemain matin, le colonel se rendit à jeun chez le cardinal Odescalchi. Il devinait fort bien ce qu'on pouvait avoir à lui dire, et pourquoi on le faisait lever avant midi; mais il n'était ni inquiet, ni intimidé. Il s'enfonçait dans les coussins de sa voiture avec la pesante assurance d'un homme qui ne craint rien au monde que l'apoplexie. «Parbleu!» disait-il entre ses dents, «il est heureux que Manuel ait quelques millions et quelques ancêtres: s'il s'appelait Nicolas, fils de Mathieu, propriétaire de deux bons bras, les cafards l'auraient déjà marié malgré moi et malgré lui. On l'aurait fait espionner par quelques agents de la morale publique, on aurait donné le mot à sa maîtresse, et au plus beau moment d'un rendez-vous, il aurait vu sortir d'une armoire un prêtre, deux gendarmes et un enfant de ch\oe{}ur. Cela se fait tous les jours, et les filles ne réclament jamais contre ces brutalités de la police. Il faut que le pauvre diable pris en flagrant délit choisisse, séance tenante, entre le mariage, prison des âmes, et le château Saint-Ange, prison des corps. S'il accepte l'eau bénite du prêtre, les gendarmes servent de témoins au mariage; s'il se décide en faveur du cachot, le prêtre sert de témoin à l'arrestation; dans les deux cas, la vertu est vengée, le coupable est puni: prisonnier pour toujours ou marié à perpétuité! Mais, grâce à Dieu! ces plaisanteries-là ne sont pas faites pour nous, et quand la morale publique se livre à ces fredaines, elle choisit d'autres plastrons que les Coromila. Que va-t-il me dire, ce vieil Odescalchi? Il ferait aussi bien de se mêler de ses affaires. Parce que sa s\oe{}ur a eu la sottise d'épouser un Feraldi, veut-il que tous les princes romains se mettent dans le Feraldi jusqu'au cou? C'est l'histoire du renard à qui l'on a coupé la queue; mais à renard, renard et demi! Est-ce qu'il se serait mis en tête de me faire un sermon? Fi donc! les cardinaux ne prêchent pas; ils laissent cela aux capucins. D'ailleurs, quoi qu'il pense de moi, il ne m'en dira pas seulement la moitié; c'est un de nos privilèges, à nous autres gens de qualité: on ne nous montre jamais une vérité toute nue. Les prêtres nous vénèrent, les cardinaux nous respectent, les papes nous ménagent, et je parie que Dieu lui-même, au jugement dernier, cherchera quelque circonlocution pour nous apprendre que nous sommes damnés!»

Il sauta gaillardement hors de sa voiture; mais en entrant dans le cabinet du cardinal il prit un air digne et confit. Il lut attentivement la supplique du comte et l'extrait des lettres de Manuel, haussa deux ou trois fois les épaules, et murmura quelques réflexions morales sur la légèreté de la jeunesse; puis il rendit toutes les pièces au prince Odescalchi.

«Éminence,» dit-il, «je vous remercie de m'avoir éclairé sur cette affaire.»

«Je n'ai fait que mon devoir, excellence.»

«Éminence, le comte Feraldi me paraît un fort honnête homme, et je l'estime infiniment.»

«Vous lui rendez justice, excellence.»

«La jeune fille est très intéressante.»

«Très intéressante assurément.»

«Et mon neveu est un enfant terrible.»

«Je n'aurais pas osé le dire, mais\ldots{}»

«C'est moi qui le dis! Je ne sais pas masquer la vérité. Il est évident que Manuel a aimé cette jeune fille, qu'il s'en est fait aimer, qu'il a promis de l'épouser.»

«Oui, excellence.»

«Maintenant il ne l'aime plus.»

«Je le crains.»

«J'en suis sûr. S'il l'aimait encore, il ne chercherait pas de mauvaises raisons pour rompre avec elle. Il l'épouserait sans s'inquiéter de ce qu'on pourra dire, et sans en demander la permission à personne. Lorsqu'on aime (votre éminence excusera la liberté de mon langage), on oublie les amis, les parents, les lois, et tous les devoirs de convenance et de reconnaissance; on court au but sans regarder en arrière. Ceux qui songent à quêter des permissions, à ménager des amitiés, à apaiser des mécontentements, sont des chercheurs de prétextes qui n'aiment pas ou qui n'aiment plus.»

«Mais,» reprit le cardinal, «si l'amour est un sentiment passager\ldots{}»

«Je devine,» interrompit le colonel, «ce que votre éminence va me dire, et j'admire la justesse de sa réflexion. Oui, si l'amour est un sentiment passager, qui nous vient quand il lui plaît, qui s'en va quand bon lui semble, il n'en est pas de même des promesses, des serments et des actes sérieux et définitifs que nous faisons son influence: l'amour passe, les obligations restent. Mon neveu est impardonnable.»

Le cardinal chercha dans le dossier les deux dernières lettres de Manuel. «Avez-vous lu,» demanda-t-il, «ces deux lettres où il rejette sur vous toute la responsabilité de sa trahison?»

«Et voilà,» reprit vivement le colonel, «ce que je ne lui pardonnerai jamais! Il peut se marier sans mon consentement: il est majeur, son père est mort, sa fortune est indépendante, personne n'a le droit de lui demander compte de ses actions; quelle mouche le pique, et pour quoi cette rage d'obtenir ma signature? Pourquoi? je le sais, et c'est un secret que je puis confier à votre éminence. Manuel me demande mon consentement parce qu'il sait qu'une puissance supérieure me défend de le lui accorder.»

«Et quelle voix pourrait parler plus haut que l'honneur, la justice et la conscience?»

«La dernière volonté d'un mort.» Le colonel se rapprocha du fauteuil du cardinal, et lui dit d'un ton mystérieux et solennel: «Dieu seul et moi, nous avons entendu les paroles suprêmes de mon frère bien-aimé, feu le prince Coromila. Ce père excellent, ce chrétien sublime, avant d'entrer au sein de la béatitude éternelle, m'a laissé des ordres précis touchant la gloire et la prospérité de sa famille. Il était instruit des relations clandestines, sans doute innocentes, qui existaient entre son fils et la jeune Vittoria. Il les désapprouvait absolument pour des raisons qu'il n'a jamais exprimées, et lui, qui sont ensevelies dans sa tombe. Ce que je sais, et ce que Manuel n'ignore pas, c'est que le prince m'a défendu de bénir cette union, et que son dernier soupir a été contraire à la famille Feraldi.»

«Mais le nom des Feraldi est sans tache, leur noblesse remonte à quatre siècles, leur fortune\ldots{}»

«Prenez garde, éminence. Je suis de votre avis, et vous argumentez contre un mort!»

Le cardinal se leva; le colonel suivit son exemple. «Excellence,» dit le prince Odescalchi, «je suis heureux de voir que, comme tous les honnêtes gens, vous blâmiez la conduite de votre neveu. Je porterai cette consolation à la famille Feraldi; mais je regretterai éternellement que lorsqu'il suffirait d'une parole pour ramener ce jeune homme à ses devoirs, des raisons de l'autre monde vous empêchent de la dire.»

«Mes paroles, éminence, n'ont pas tout le crédit que vous daignez leur attribuer: il n'y a que les paroles magiques qui aient la vertu de changer les c\oe{}urs. Mon neveu n'aime plus Vittoria: si je lui accordais mon consentement, il susciterait lui-même quelque nouvel obstacle; il serait capable de dire qu'il lui faut le consentement de son père. Je m'intéresse, comme vous, à la situation du malheureux comte, et pour lui épargner, ainsi qu'à votre éminence, des démarches inutiles, je crois devoir vous confesser une dernière faute de Manuel. II aime ailleurs. Malgré les sages avis de monsignor Rouquette, dont les vertus vous sont bien connues, il s'est épris d'une fille de théâtre qui lui coûte à l'heure qu'il est près de deux cent mille francs, la dot de M$\textsuperscript{lle}$ Feraldi! C'est à vous de décider, maintenant que vous tout, s'il n'y a pas un peu de cruauté à laisser derrière les grilles d'un couvent une pauvre fille dont l'amant se perd dans les plaisirs.»

Le colonel sorti, le prince Odescalchi écrivit au comte: «Je n'ai rien obtenu; venez ce soir à l'Ave-Maria avec son éminence le cardinal Pezzato; nous tiendrons conseil.» Menico, qui attendait dans une antichambre, reçut le billet des mains du camérier du prince, et courut à toutes jambes le porter au palais Feraldi. La famille de Tolla, assistée de la marquise et de Philippe, fondit en larmes à la lecture de cette sentence. «C'est ma faute!» criait en pleurant la pauvre comtesse. «Je n'aurais pas dû le recevoir ici avant le consentement de sa famille.»

«C'est moi qui l'ai amené,» disait Philippe. «J'ai cru, comme un sot, que son oncle était un bonhomme.»

«Je suis plus coupable que toi,» ajoutait la marquise. «Je savais, moi, que le colonel ne permettrait jamais ce mariage, et cependant je n'ai rien dit!»

«Ah!» murmurait fièrement Victor Feraldi, «le colonel Coromila veut garder son neveu pour lui! Nous verrons.»

«Je jure,» dit Philippe, «qu'il ne le gardera pas longtemps, car je le tuerai entre ses bras, s'il reste encore deux lames d'acier en ce monde.»

La marquise se leva doucement, et alla prendre son châle et son chapeau qu'elle avait ôtés en entrant. «Attendez-moi,» dit-elle, «je vais parler au chevalier Coromila.»

Elle prononça ces paroles du ton dont un condamné à mort dit à son bourreau: «Je suis prêt.» Son fils et ses amis la laissèrent partir une question, sans une parole, sans un geste. Philippe connaissait son aversion pour le colonel, M$\textsuperscript{me}$ Feraldi en pressentait les causes; chacun devinait dans cette démarche simple et sans apparat le dévouement sublime des martyrs.

Elle entra au palais Coromila quelques minutes après le colonel. Le gros homme allait se mettre à table. L'annonce d'une visite si peu attendue lui coupa l'appétit. Il dissimula son trouble sous une politesse de corps de garde, et présenta un siège à la marquise en la saluant du nom de belle dame.

\enlargethispage{3em}

«Pierre Coromila,» lui dit-elle, «vous devinez qu'il faut des motifs bien puissants pour que je vienne, après plus de vingt années, ré veiller mes chagrins et vos remords.»

«Diantre!» pensa le colonel, «est-ce que la belle Assunta serait lasse d'être veuve, et voudrait-elle?\ldots{} Hé! hé! les Coromila sont très demandés depuis quelque temps.» Il reprit à haute voix: «J'espérais, madame la marquise, que mon ami Trasimeni aurait enseveli vos chagrins comme il a enterré mes remords. Cependant, s'il vous plaît de revenir sur le passé, en parlerons ensemble. Je comprends tous les goûts, sans excepter l'amour de l'histoire ancienne; d'ailleurs je n'ai jamais rien su refuser à la beauté. Or vous êtes toujours belle, Assunta, aussi belle et peut-être plus que le jour de notre premier baiser.»

La marquise fut prise d'une petite toux sèche, et les pommettes de ses joues se colorèrent pour un instant: le séjour de Florence ne l'avait pas guérie. «Ce n'est pas de moi,» dit-elle, «que je viens vous parler, c'est de Tolla.»

«Encore!» s'écria involontairement le colonel. Il reprit avec douceur: «Madame, je sors de chez le cardinal-vicaire; il m'a dit sur cette malheureuse affaire tout ce que vous pouvez avoir à me dire; je vous en prie, ne me forcez pas de vous répéter tout ce que je lui ai répondu.»

«Soyez tranquille: j'éviterai les répétitions et je vous dirai ce que personne autre que moi n'a le droit de vous dire. Vous savez avec quelle résignation j'ai subi le sort que vous m'avez imposé; je me suis sacrifiée, sans une plainte, à votre égoïsme et à l'ambition de votre famille.»

«Vous avez trouvé un consolateur.»

«Taisez-vous, mon pauvre Pierre: quand on n'a pas l'honneur du soldat, on ne doit pas en afficher la brutalité. Je vous ai rendu votre parole et toutes vos lettres, comme on rend les titres d'une créance à un débiteur insolvable. J'ai traîné ma vie, près d'un quart de siècle, dans la même ville que vous, triste au milieu des heureux, morte au milieu des vivants, sans qu'un seul de mes regards vous ait reproché votre conduite et mes souffrances; mais si j'ai supporté patiemment toutes les tortures, je ne sais pas assister les bras croisés au supplice d'une autre, et je me révolte. Vous avez prononcé ce matin, devant le cardinal-vicaire, l'arrêt de mort de Tolla.»

«Elle n'en mourra pas, madame. Tous ceux que nous avons tués se portent à merveille.»

«Vous trouvez!» Il est impossible de rendre l'accent de douleur, d'amertume et de découragement avec lequel elle prononça cette parole. Tout autre que le colonel aurait frémi, comme en écoutant le râle d'une mourante. Il se contenta de ricaner, et répondit en appuyant lourdement sur sa plaisanterie: «Vous êtes fraîche comme une rose.»

La marquise ne se contint plus. «Lâche!» dit-elle, «tu ne m'as point pardonné de n'être pas morte sur le coup, et ce peu de vie qui me reste est une offense à ta vanité! Tu trouves que mon agonie a été trop longue, et que j'aurais dû me hâter un peu, pour ta gloire. Eh bien! console-toi: Tolla ne résistera pas si longtemps. Je la vois dépérir, et je te promets qu'elle s'éteindra bientôt, à l'honneur de Manuel, dans la prison où lui-même l'a cloîtrée. On connaîtra que les Coromila ne sont point dégénérés et qu'ils ont fait des progrès dans l'art de tuer les femmes; mais après ce beau triomphe, je te conseille de cacher soigneusement ton cher Lello: Philippe a du c\oe{}ur, il est le digne fils d'un honnête homme, il aime Tolla comme sa s\oe{}ur, il la vengera.»

«Si Philippe est le digne fils de son père,» répliqua aigrement le colonel, «il épousera M$\textsuperscript{lle}$ Feraldi, au lieu de la venger. Qui sait si le fabricateur souverain n'a pas inventé les Trasimeni pour consoler les victimes des Coromila?»

\enlargethispage{\baselineskip}

Quand la marquise fut sortie, le colonel se sentit soulagé, mais non satisfait. Les dernières paroles de M$\textsuperscript{me}$ Trasimeni lui restaient sur le c\oe{}ur, et il craignait pour la réputation et pour la vie de Manuel. Avant de se rendre aux prières de son maître d'hôtel et à l'appel de son déjeuner, il écrivit à Rouquette et donna des ordres à Cocomero. Il disait à Rouquette:

«Je remets en vos mains la vie de Lello; ne le quittez sous aucun prétexte. Le cardinal Odescalchi va probablement vous rappeler: faites la sourde oreille. Si vous perdez votre place, je vous indemniserai largement: la maison Rothschild a cinquante mille francs pour le jeune Feraldi et son ami Philippe iront chercher querelle à notre enfant: tirez-le de leurs mains. Lisez tous les jours la liste des étrangers débarqués à Paris; au premier danger, partez pour l'Angleterre, et ne dites à personne où vous allez. En attendant, et pour plus de prudence, fréquentez le tir de Lepage et la salle de Bertrand.» Il déclara à Cocomero qu'il fallait, pour l'honneur de la famille Coromila, que M$\textsuperscript{me}$ Feraldi sortît au plus tôt de Saint-Antoine.

«Que faire, excellence?»

«Tu me le demandes? animal! C'est à toi de le trouver: je te paie pour avoir de ton l'esprit. Délibère avec la dame russe, associée.»

«Elle n'est pas mon associée, excellence. C'est\ldots{}»

«Je ne tiens pas à savoir ce que c'est. As-tu parlé à la femme de chambre?»

«Oui, excellence, hier soir. Elle sortira si on lui fait une dot.»

«Promets-lui mille écus, et qu'elle sorte aujourd'hui même. Tu me l'amèneras sans tarder.»

Ce chiffre de mille écus fit réfléchir Amarella. Pour six cents francs, elle serait sortie sans marchander; elle trouva que mille écus, pour enjamber le seuil d'une porte, étaient un maigre salaire. Les paysans sont ainsi faits: offrez-leur cinq francs d'un bahut, ils vous frappent dans la main; offrez-en cinquante, ils en veulent dix mille: c'est le dernier prix. N'essayez pas de discuter, ils ne le laisseront pas à moins: vous leur avez persuadé que le bahut contenait un trésor. Le pauvre. Cocomero devint un habitué du parloir de Saint-Antoine. Le 1$\textsuperscript{er}$r octobre, après trente-sept jours de discussion, il n'avait pas gagné un pouce de terrain.

Le comte Feraldi employa tout ce temps à une lutte désespérée contre le mauvais vouloir de Manuel. Trop sûr que l'obstination de l'oncle résisterait à toutes les remontrances, il s'était rejeté sur le neveu, et ne se lassait pas de lui écrire; mais Manuel était bien conseillé. M. Feraldi sortait du cabinet du cardinal-vicaire, de l'oratoire de la marquise ou du parloir de sa fille, avec des arguments qu'il croyait sans réplique; Manuel, entre deux verres de vin de Champagne, dans un cabinet du Café Anglais ou dans le boudoir de Cornélie, trouvait une réplique triomphante à tous ces arguments. Si le comte lui rappelait qu'il avait promis d'aimer Tolla jusqu'à la mort, il répondait imperturbablement que jusqu'à la mort il aimerait Tolla.

«Mais, reprenait le comte, «vous avez ajouté: ‹Je jure de n'avoir pas d'autre femme que Vittoria Feraldi.›» «En ai-je donc épousé une autre?» demandait Manuel. «Vous avez dit et écrit à Tolla: ‹Je t'épouserai.›» «Et je suis prêt à le faire, dès que j'aurai obtenu le consentement de mes parents.» «Vous avez déclaré que si vos parents s'obstinaient à refuser leur consentement, vous sauriez vous en passer.» «Sans doute, après avoir épuisé tous les moyens de conciliation; mais je suis loin de les avoir épuisés; peut-être même sont-ils inépuisables.» Si le comte essayait de rappeler le beau sacrifice de Tolla et le courage qu'elle avait eu de s'enfermer dans un cloître, Manuel énumérait victorieusement tous les efforts qu'il avait faits pour l'en arracher. Le comte se plaignait de la scandaleuse publicité qu'on avait donnée à la lettre du 11 août; Manuel blâmait l'indiscrétion de ceux qui avaient fait lire sa correspondance à son oncle. Dans le cours de cette discussion, où Manuel poussa la mauvaise foi jusqu'à l'impertinence, la douceur et la modération du comte ne se démentirent pas un instant. Il réfutait un mensonge par jour sans exprimer un doute sur la sincérité de Lello; il traitait d'erreurs et de malentendus les faussetés les plus notoires; il prédisait que les légers nuages qui s'étaient élevés entre son gendre et lui se dissiperaient au premier souffle; il évitait par politesse, mais aussi par prudence, de trop mettre Lello dans son tort; il n'avait garde de faire allusion à la conduite qu'il menait à Paris. Ses lettres, écrites dans la douleur la plus profonde et l'indignation la plus légitime, commençaient toutes par très cher Manuel Coromila, et finissaient par votre très affectionné serviteur et ami. Manuel, de son côté, écrivait très cher comte, et signait \emph{vostro affettuosissimo servo ed amico\footnote{
L'italien: votre très affectueux serviteur et ami
}}. Tolla n'entendit parler ni des lettres ni des réponses.

Elle n'en était pas plus heureuse. Manuel ne lui avait écrit, du 16 juillet au 1$\textsuperscript{er}$r octobre, que la lettre du 11 août, que ses parents s'étaient bien gardés de lui faire lire: elle était donc restée deux mois et demi sans nouvelles de son amant. Sa passion avait résisté à une si cruelle épreuve: elle aimait avec désespoir, mais elle aimait. Elle écrivait sans se lasser à celui qui ne lui répondait plus. Jamais on n'entendit une plainte sortir de sa bouche: sa douleur tranquille et résignée édifiait tout le couvent: les religieuses apprenaient à son école l'art sublime de souffrir sans murmure et d'adorer le bien-aimé jusque dans ses rigueurs. Les plus austères expliquaient dans un sens mystique le triste roman qui se dénouait sous leurs yeux: elles le commentaient comme certaines âmes naïvement ferventes ont commenté le Cantique des cantiques de Salomon. «Puissions-nous,» disaient-elles, «aimer notre divin époux comme elle aime son Lello!» Les salons de Rome, naguère hostiles à Tolla, commençaient à se tourner contre ses ennemis. Ses malheurs et son courage étaient cités partout, et l'on ne parlait plus d'autre chose. En l'absence de toute autre préoccupation, dans un pays où la politique est obscure et souterraine, où les journaux sont aussi insignifiants que des almanachs, où les procès se jugent clandestinement dans une cave, où le théâtre est sans liberté et partant sans intérêt, l'attention publique, qui se prend où elle peut, s'attacha au couvent de Saint-Antoine. Les Romains ont l'âme bonne et les pleurs faciles; leur sensibilité un peu banale n'est pas tempérée par cette ironie dont nous sommes si fiers: ils ont plus d'abandon, plus d'ouverture, plus de chaleur et moins d'esprit que nous. Rome entière applaudit, comme dans un théâtre, à la belle conduite du jeune Morandi, qui vint pour la troisième fois demander au comte la main de Tolla. Morandi fut pendant huit jours l'orgueil de l'Italie: jusqu'au moment où il repartit pour Ancône sans avoir obtenu autre chose que les remerciements et les larmes de la famille Feraldi, il marcha d'ovations en ovations. Les paysans qui venaient au marché ou les maçons qui s'en allaient à l'ouvrage lui criaient à tue-tête: \emph{«Bravo, ser pajno!»\footnote{
L'italien: «Bien, monsieur le monsieur!»
}} Ces témoignages éclatants de l'opinion firent rentrer sous terre tous les ennemis de Tolla. Ceux qu'une petite jalousie avait soulevés contre elle lui accordèrent sa grâce dès le jour où elle inspira plus de pitié que d'envie. La générale, dont les sentiments ne pouvaient changer, parce que ses intérêts étaient toujours les mêmes, se crut ce pendant obligée de faire une visite à M$\textsuperscript{me}$ Feraldi: elle vint avec Nadine apporter quelques grimaces de condoléance dans ce palais où ses calomnies avaient fait couler tant de larmes. Tels étaient le frémissements de l'émotion publique, qu'ils traversèrent les murailles du couvent et parvinrent jusqu'aux oreilles de Tolla. Malgré les précautions admirables de ses parents et les ordres exprès du docteur Ely, qui déclarait qu'une mauvaise nouvelle pouvait la tuer, la pitié indiscrète de quelques amis, une allusion maladroite à la trahison de Manuel, un blâme sévère exprimé contre Rouquette, la mirent sur la trace de la vérité: la haine ingénieuse d'Amarella fit le reste. Cette créature, née mauvaise, et que la passion avait rendue pire, alla jusqu'à faire entendre à sa maîtresse qu'il existait des preuves écrites de son abandon. Rien n'est plus propre à faire juger des angoisses et de la résignation de Tolla, que cette lettre choisie au milieu de toutes celles qu'elle écrivit à Manuel.
\begin{quote}

Rome, \oldstylenums{16} septembre \oldstylenums{1838}.

Il y a deux mois aujourd'hui que je n'ai reçu une ligne de toi: d'où vient cela, mon Lello? Ils disent que cela vient de ce que tu ne m'aimes plus. Ton nom et celui de monsignor Rouquette sont dans toutes les bouches, suivis des épithètes les plus infâmes. On raconte mille traits qui te déshonorent; on dit que tu te fais un jeu de tromper les filles et de les faire mourir; on énumère la liste de celles que tu as perdues: juge si j'ai de quoi souffrir, moi qui connais ton c\oe{}ur, qui sais tes serments et qui suis sûre que tu n'y manqueras point! Chaque fois qu'il me vient une visite à la grille, j'ai peur. Ils voulaient me persuader que tu étais infidèle: j'ai répondu que je ne le croirais jamais. Et si vous en voyiez la preuve écrite? m'a-t-on demandé. J'ai dit que cela était impossible, mais que si je voyais un aussi méchant écrit, je répondrais qu'il n'est pas de toi, ou qu'on t'a forcé, et que ta bouche démentira ta main, enfin que je ne me croirais trahie que lorsque tu me l'auras dit toi-même. Je l'ai juré: quoi que je voie, quoi que j'entende, je ne croirai rien avant ton retour. À tout ce qu'ils me disent, je réponds: C'est impossible, \textemdash{} et je les fais taire. Cependant tu ne m'écris pas; pourquoi me faire cette peine? Est-ce que tu crains de m'apprendre la réponse de ton oncle? Je l'ai devinée, va, et j'en ai pris mon parti. Je te réconcilierai avec lui quand je serai ta femme. Mais tu m'as écrit; on aura intercepté tes lettres; il est impossible que tu ne m'aies pas écrit: une mortelle ennemie qui t'aurait supplié comme je l'ai fait aurait obtenu au moins quelques lignes. Si tu voyais ta Tolla, mon bon Lello, elle te ferait pitié. Je ne ris plus, je dors bien peu, et ce peu est si agité que je m'éveille à chaque instant. Tout le jour, je pleure aux pieds de la sainte Vierge en la suppliant de me venir Je me lève aussi la nuit pour prier Dieu, et mes prières sont toujours trempées de larmes: quelquefois les sanglots m'étouffent. Ah! reviens vite, tu veux que je vive! J'ai souffert assez, je n'en peux plus, je sens que mes forces sont à bout: si l'on mourait de tristesse, il y a longtemps que tu n'aurais plus de Tolla. Mais sois tranquille, la force pourra me manquer, non le courage; on désespérera de ma vie avant que je doute de ton honneur, et j'emporterai jusqu'au fond de la tombe ma foi dans tes promesses et ma confiance en toi.

\end{quote}

L'amant de M$\textsuperscript{lle}$ Cornélie (c'est Manuel que je veux dire) avait tant d'occupations qu'il laissait à Rouquette le soin de dépouiller sa correspondance.
