\addchap{X.}

Le 1$\textsuperscript{er}$ octobre, Cocomero s'introduisit assez avant dans la confiance d'Amarella. Il lui apporta une copie de cette terrible lettre du \oldstylenums{11} août qu'il avait reproduite lui-même, sous la dictée de Nadine, à plus de vingt exemplaires. Amarella, ravie d'avoir en main de quoi assassiner sa maîtresse, ouvrit son c\oe{}ur à l'aimable Napolitain. «Ne croyez pas,» lui dit-elle, «que ce soit l'intérêt qui me retienne ici; c'est une plus noble passion, la haine. Quand vous m'avez vue refuser successivement tant d'offres magnifiques, vous avez peut-être supposé que je ne songeais qu'à me faire donner une plus grosse dot, et que mon ambition que croissait avec vos promesses. Non, mon cher monsieur; mais que ferai-je d'une dot si je ne trouve pas un mari?»

«Vous en trouverez de reste. L'argent attire les épouseurs comme le grain les moineaux, et l'on ne voit pas, dans toute l'histoire romaine, qu'une fille bien dotée ait jamais coiffé sainte Catherine.»

«Oui, si je voulais prendre un mari à la douzaine! Mais quand \emph{on veut du bien} à quel qu'un!»

Les Italiens ont tout un dictionnaire à l'usage de l'amour. \emph{Vouloir du bien,} c'est aimer passionnément. On ne dit pas l'amant, mais le \emph{voisin} d'une femme mariée: le cardinal un tel avoisine, \emph{avvicina}, telle comtesse, qui loge à une lieue de son palais.

Amarella raconta longuement qu'elle voulait du bien à un jeune homme qui ne lui voulait Elle apprit à Cocomero le nom de son ingrat, les services qu'elle lui avait rendus, et comment elle lui avait sauvé la vie un soir qu'il avait été frappé dans l'ombre par un lâche assassin. Cocomero salua. Elle se déchaîna ensuite contre sa maîtresse, qu'elle accusait d'être la complice de Dominique. «Enfin,» dit-elle, «depuis quatre mois je ne me nourris que d'amour et de haine; je ne vis plus que pour épouser Menico et me venger de Tolla.»

«Eh! chère enfant, que ne le disiez-vous? Vos désirs sont légitimes, et ils seront satisfaits, s'il y a une justice. Quoi de plus naturel que de faire du bien à ceux qu'on aime et du mal à ceux qu'on déteste? Dieu lui-même n'agit pas autrement: il a fondé le paradis pour ses amis et l'enfer pour ses ennemis. Mais pourquoi n'avoir pas parlé plus tôt? Il y a un grand mois que je vous aurais vengée et mariée.»

«Mariée à Dominique?»

«A lui-même.»

«Vous êtes donc un ange du ciel?»

«Pas tout à fait.»

«Un sbire de la police?»

«Peut-être.»

«Vous pouvez le forcer de me prendre pour femme?»

«Est-ce la première fois que la police pontificale se mêle de mariages?»

«Ne me trompez pas, je vous en prie; cette \ldots{} affaire se ferait-elle bientôt?»

«Il est quatre heures; avant minuit, vous aurez reçu le sacrement.»

«Et que faudra-t-il que je fasse?»

«Presque rien: vous irez porter cette lettre à votre maîtresse.»

«C'est ma vengeance.»

«Vous lui direz que puisque tout espoir est perdu pour elle, et qu'elle ne reste plus au couvent que pour son plaisir, vous ne souciez pas de lui tenir éternellement compagnie.»

«Soyez tranquille, je lui dirai cela, et bien autre chose. Après?»

«Vous sortirez immédiatement de Saint Antoine, et vous viendrez habiter le logement que je vous ai préparé \emph{via de Pontifici, \oldstylenums{24}.} N'oubliez pas de laisser ici votre nouvelle adresse: il faut que Menico sache où vous de Il aime Tolla, dites-vous?»

«J'en suis sûre.»

«C'est lui qui vous a décidée à vous renfermer avec elle?»

«Lui seul.»

«Il viendra ce soir vous prier de retourner au couvent. Il faut qu'il vous trouve au lit. Vous disputerez, vous résisterez, vous ferez traîner la discussion jusqu'à minuit. On frappera violemment à votre porte: vous crierez d'effroi, vous craindrez d'être compromise, vous le cacherez dans un cabinet. Je me charge du reste.»

«Vous serez là?»

«Non, il ne faut pas que je paraisse. C'est le cardinal-vicaire qui fera les frais de la cérémonie. Je lui apprendrai à neuf heures, par un avis anonyme, que vous avez quitté le cloître pour courir à un rendez-vous. Le cardinal est un saint homme, ennemi juré de l'immoralité: il enverra le prêtre et les gendarmes.»

«Et \ldots{} j'aurai la belle dot que vous m'avez promise?»

«Ce soir même je vous donnerai mille écus; vous me signerez un reçu de deux mille.»

«Vous offriez hier de me donner les deux mille écus!»

«Oui, mais je n'offrais pas de vous donner Menico.»

Marché fait, Amarella monta en courant chez sa maîtresse. Tolla était assise, la tête penchée, les bras pendants, sur une chaise basse, devant une petite table de bois noir. Elle avait commencé une lettre à Lello, sans avoir le courage de la finir. Depuis plus d'une semaine, elle était en proie à un malaise étrange: son appétit diminuait tous les jours, et quelques efforts qu'elle fît sur elle-même, souvent elle sortait de table sans avoir rien pris. Elle sentait tous les ressorts de son être se détendre: sa fière volonté, sa pétulante énergie, s'enfuyaient lentement, comme le vin découle d'un cristal fêlé. Tous ses sens, autrefois si alertes et si heureux, étaient lents, émoussés et tristes: le soleil lui paraissait terne, l'air froid, la musique sourde. Son embonpoint si sobre, si juste et si chaste avait fondu comme rayon de cire: ses joues s'étaient creusées, et les jolies fossettes étaient devenues de grands trous. La pâleur de son visage semblait moins fraîche et moins lumineuse: sa peau n'était plus ce réseau transparent sous lequel on voyait courir la vie. Ses grands yeux avaient pris une beauté morne et désespérée: ils ne lançaient que des sourires pâles et des éclairs éteints. Ses mains étaient si faibles, qu'un instant avant l'entrée d'Amarella elle avait laissé tomber sa plume, comme un fardeau trop lourd. À ses pieds, un mouchoir taché de sang traînait à terre: elle avait saigné du nez plus de vingt fois en une semaine. Amarella contempla cette douleur et cet abattement comme un habile ouvrier regarde son ouvrage au moment d'y mettre la dernière main. Elle fut impitoyable; elle raconta sans ménagement tout ce qu'elle savait de la trahison de Lello; elle ajouta à ce qu'elle avait appris tous les détails que son imagination put lui suggérer: elle le peignit consolé, joyeux, entouré de maîtresses, et lisant, pour égayer quelque orgie, les lettres lamentables de Tolla. Ses paroles étaient chargées d'une pitié accablante; elle écrasait sa maîtresse sous d'odieuses consolations, et à travers les fausses larmes qu'elle se forçait de répandre, on voyait percer le triomphe et l'insolence de ses regards. Sa conclusion fut de prendre congé et de donner la lettre.

Tolla resta plus d'une heure en présence de cette dépêche de mort, qu'elle regardait sans la lire, qu'elle lisait sans la comprendre, qu'elle comprit enfin, mais dans un tel trouble d'esprit, qu'elle n'en aperçut pas toute la portée. Elle la tournait dans ses mains, et jouait avec elle comme un enfant avec un couteau. Elle ne s'avisa même pas que l'écriture n'était point celle de son amant, et lorsqu'on vint lui dire, à six heures, que sa mère l'attendait au parloir, on la surprit à baiser machinalement l'autographe de Cocomero.

La comtesse, rassurée par la résignation apparente de sa fille, lui avoua tout, les lettres de Lello, les démarches du cardinal et de la marquise, les refus du colonel, les réponses dictées par Rouquette et la perte des dernières espérances. «Mon enfant,» lui dit-elle, «Amarella a raison; il faut sortir du couvent.» Ce mot provoqua une crise violente: Tolla fondit en larmes. Sa mémoire, son jugement, sa passion, ses forces, se réveillèrent à la fois. Elle cria: «C'est impossible! Il n'est pas capable de me trahir. Ces lettres sont écrites pour son oncle; il veut le gagner par un semblant de soumission. Tu n'as rien compris; tu ne le connais pas: moi seule je le connais. Ne le juge pas! Il est fidèle, je réponds de lui. Il est impossible que dans l'espace de quatre mois un c\oe{}ur si tendre et si religieux soit devenu un monstre. Ses lettres respirent les meilleurs sentiments: elles sentent bon comme l'encens des églises! Il me dit de prier Dieu, les saints, la vierge Marie; il prie lui-même du matin au soir. Est-ce qu'il oserait parler à Dieu, s'il ne m'aimait plus? D'ailleurs, il sait mon v\oe{}u: crois-tu qu'il soit assez cruel pour me condamner au couvent pour toute la vie? Que deviendrais-je s'il m'abandonnait? Que ferais-je de mon c\oe{}ur? Dieu n'en voudrait pas: il exige qu'on soit toute à lui. Ma pauvre mère! que tu as dû souffrir pendant ces deux mois! C'est pour toi que j'aurais voulu être heureuse: la vue de mon bonheur t'aurait fait tant de bien! Voilà maintenant que je te prépare une triste vieillesse. Cependant crois-tu qu'il ait pu oublier tout ce qu'il m'a promis?» Là-dessus, elle cita avec une volubilité fébrile des paroles, des discours et des lettres entières de Manuel; puis elle retomba dans un abattement doux et tranquille; elle pria sa mère de lui renvoyer Amarella pour quelques jours; elle demanda que son confesseur vînt la voir le lendemain mardi; elle voulait communier le mercredi, jour consacré à saint Joseph. À huit heures, elle prit congé de sa mère, qui se félicitait intérieurement de la voir si calme après tant d'agitations. Elle remonta à sa chambre en tenant la rampe de l'escalier. Comme elle traversait la \emph{loge\footnote{
L'italien galerie couverte
}} qui conduisait à sa cellule, elle se tourna vers la basilique de Sainte-Marie-Majeure en murmurant une prière. À cet instant, ses genoux fléchirent, un éblouissement la contraignit de fermer les yeux, et elle crut entendre une voix d'en haut qui lui disait: «Pourquoi pleures-tu? N'as-tu pas une tendre mère dans le ciel?»

Elle dormit d'un sommeil agité, et s'éveilla le lendemain avec un grand mal de tête. Elle se leva, se traîna péniblement jusqu'à son petit miroir, et s'effraya en voyant combien ses traits étaient altérés. Sa faiblesse, et un frisson qui ne dura pas plus de dix minutes, la forcèrent de rentrer au lit. Quand les religieuses vinrent savoir de ses nouvelles, elle avait le pouls violent, le visage rouge, la peau sèche, la gorge enflammée, les entrailles brûlantes: le progrès fut si prompt et si imprévu, qu'on n'eut pas le temps de la renvoyer à sa famille, comme le prescrivait la règle du couvent. La comtesse, mandée en toute hâte, accourut avec s médecin. Le docteur Ely reconnut tous les symptômes de la fièvre typhoïde, et pratiqua immédiatement une saignée. Il s'efforça de rassurer la comtesse en affirmant que, de toutes les formes de la maladie, la forme inflammatoire était celle qui laissait le plus d'espérances: il se garda de lui dire que le mal était presque toujours incurable lorsqu'il était engendré par des causes morales. M$\textsuperscript{me}$ Feraldi aurait voulu qu'on transportât sa fille, soigneusement enveloppée, jusqu'à son palais: elle accusait l'air du couvent d'être malsain. Le docteur rapportait le mal à d'autres causes, telles que le chagrin, les privations et la nostalgie. Tolla avait souffert au-delà de ses forces, elle avait vécu de jeûnes et d'abstinence, et, depuis la veille du \oldstylenums{1}$\textsuperscript{er}$ mai, elle s'était exilée du printemps, du grand air et de la liberté.

Pendant sept jours entiers elle vécut sans sommeil, sans repos, agitée par des rêves pénibles, accablée par un mal de tête insupportable qui pesait sur toutes ses pensées. Lorsque le délire la quittait, elle consolait sa mère. Elle ne douta pas un instant que sa maladie ne fût mortelle. Dès le second jour, elle voulut écrire une lettre pour Lello. «Si j'attendais plus longtemps,» dit-elle, «je ne pourrais plus lui faire mes adieux.» En l'absence de la comtesse, une jeune religieuse écrivit sous dictée la lettre suivante:
\begin{quote}

Te souviens-tu, Lello, que nous sommes convenus autrefois de ne jamais nous mettre au lit sans avoir fait la paix ensemble? Réconcilions-nous, mon ami: ja vais dormir longtemps. Je me suis couchée hier matin avec une grosse fièvre, il paraît que c'est la fièvre typhoïde. Le cher docteur assure qu'on n'en meurt presque jamais; moi, je sens bien que je n'en guérirai pas. C'est ma faute: j'ai passé trop de nuits en prières, j'ai jeûné trop souvent. J'aurais dû savoir qu'on ne joue pas impunément avec la santé. Ne cherche pas d'autres causes à ma mort: c'est le châtiment d'une longue imprudence. Ma mère s'imagine que l'air du couvent m'a fait mal, mais le docteur affirme que non: je te dis cela pour te prouver que tu n'as pas de reproches à te faire; tu auras assez de tes chagrins! Voilà tous nos projets bien changés! Nous n'irons ni à Venise, ni à Lariccia, ni à Capri. Quand je comparaîtrai en présence du bon Dieu, j'espère qu'il me pardonnera de t'avoir aimé plus que lui. Toi, tu vas vivre longtemps; je prierai mon ange gardien qu'il ajoute mes années aux tiennes. Sois heureux pour tout le bonheur que tu m'as donné. Quand tu me disais: «Tolla mia!» je voyais les cieux ouverts. Tu m'as promis de ne pas te marier si tu venais à me perdre: c'est une promesse qui était bonne autrefois, dans le temps où nous nous croyions éternels; maintenant je te commande de l'oublier. Tu ne désobéiras pas à ma volonté dernière? Choisis une femme douce et pieuse, qui ne te défende pas de prier pour moi. Si tu as une fille, tâche d'obtenir qu'on l'appelle Tolla: de cette façon, tu te souviendras de mon nom toute ta vie. Je crois que nous aurions eu de beaux enfants et que je les aurais bien élevés. Adieu. Quand tu recevras cette lettre, donne un baiser à mon pauvre petit portrait: c'est tout ce qui restera sur la terre de ta fidèle.

\hspace*\fill---Tolla.\end{quote}

Cette lettre, signée de la propre main de Tolla, fut portée discrètement à la poste: elle partit le soir même par la voie de terre, à l'insu de la famille Feraldi. Le comte et Victor se désespéraient de ne pouvoir pénétrer dans le couvent. A la fin de septembre, M. Feraldi, poursuivi par l'idée qu'on réservait Lello pour un riche mariage, avait fait une dé marche officielle tendant à enchaîner sa liberté. Sur sa réclamation, contrôlée par le cardinal-vicaire, \emph{il deputato dei matrimonj\footnote{
L'italien: le chef du bureau des mariages
}} avait mis \emph{l'advertatur\footnote{
Latin: l'annonce
}} au nom de Manuel. «Si nous ne pouvons pas le contraindre à épouser Tolla, disait le comte, au moins nous l'empêcherons d'en épouser une autre.» Mais la mort allait déjouer les calculs de cette prudence paternelle et rendre au jeune Coromila toute sa liberté.

Victor, las de verser des larmes inutiles et de rôder jour et nuit autour du couvent de Saint-Antoine, disparut dans la soirée du \oldstylenums{4} octobre. On perdit sa trace à Civita-Vecchia, et sa mère devina en frémissant qu'il s'était embarqué pour la France. Rome entière s'associait aux douleurs de la famille Feraldi. Mille personnes attendaient à la porte du couvent la sortie du médecin. Toutes les communautés entreprirent des neuvaines; les \emph{Sepolte vive\footnote{
L'italien: Enterrement vivant
}} se condamnèrent à la pénible pénitence de l'ascension du Calvaire; les \emph{Capucines\footnote{
L'italien: Frères mineurs capucins
}} envoyèrent en grande pompe la célèbre image de saint Joseph qui a sauvé tant de malades; plusieurs églises offrirent des reliques miraculeuses; la générale Fratief fit parvenir au docteur Ely son \emph{Codex} de famille et la recette du lézard vert. La ville était en prières, comme si chaque fa mille avait eu un enfant en danger de mort. Pour suppléer Amarella, qui ne se retrouvait point, quatre religieuses voilées se tenaient à toute heure dans la cellule de la malade; autant de s\oe{}urs converses attendaient au dehors. Les pauvres s\oe{}urs embrassèrent avec passion les fatigues et les dégoûts d'un état si nouveau pour elles. Condamnées par leurs v\oe{}ux à la sainte oisiveté des prières perpétuelles, elles étaient trop heureuses de pouvoir mettre au jour ces trésors de charité active que toute femme porte dans son c\oe{}ur: c'était à qui passerait les nuits. De temps en temps une des gardes-malades s'échappait de la chambre pour pleurer librement: qui n'aurait pas pleuré en voyant mourir tant de jeunesse et de beauté?

Le \oldstylenums{8} octobre, la maladie entra dans une période nouvelle: les maux de tête se dissipèrent, la soif devint moins vive, les douleurs d'entrailles furent presque insensibles; mais le pouls était misérable, la stupeur profonde, l'accablement extrême, la respiration étouffée: la pauvre créature râlait à faire peine. Le \oldstylenums{10}, on lui administra le saint viatique, et la foule suivit en longue procession le carrosse doré qui lui apportait Dieu. Le samedi \oldstylenums{12}, on signala un mieux sensible, et un rayon de joie éclaira la ville. Quelques hommes en veste vinrent crier sous les fenêtres du colonel: «Sauvez Tolla!» Le colonel partit le soir même pour Albano. Tolla profita du répit que lui laissait la mort pour rompre les derniers liens qui l'attachaient à cette terre. Elle fit porter son anneau de fiançailles à la madone de Sant' Agostino, qui possède le plus riche écrin qui soit au monde; elle renvoya au palais Coromila le portrait de Manuel; mais le porteur, qui était Dominique, eut l'imprudence de le laisser voir, et le peuple le brûla, au milieu du Corso, sans respect pour le génie de l'artiste et la beauté de la peinture. Le lendemain, toute lueur d'espoir s'éteignit; la mourante reçut l'extrême-onction, et la comtesse fut entraînée loin de sa fille, qu'elle ne devait plus revoir. Tolla, étendue sans mouvement, ne recevait plus aucune impression du monde extérieur. Étrangère à tout ce qui l'entourait, elle n'entendait ni les prières de la communauté, ni les bénédictions de l'abbé La Marmora, ni les sanglots du bon vieux docteur qui l'avait amenée à la vie et qui ne pouvait l'arracher à la mort. Elle avait demandé à saint Joseph qu'il daignât la recevoir un mercredi: son dernier v\oe{}u fut exaucé, et ce fut le mercredi \oldstylenums{17} octobre, au premier coup de \emph{l'Ave Maria,} qu'elle entra dans le repos des justes. Sa vie s'exhala dans un soupir si faible, qu'il fut à peine entendu des personnes qui entouraient son lit. La supérieure, en rendant compte de l'événement au cardinal-vicaire, disait: «Ce n'est pas une mort, c'est le doux passage d'une âme pure dans le sein de Dieu.»

\enlargethispage{\baselineskip}

Le couvent qu'elle avait sanctifié par son martyre envoya jusqu'à trois ambassades chez le comte pour implorer la faveur de conserver ses reliques: déjà le peuple parlait d'elle comme d'une sainte; mais le comte Feraldi crut qu'il était de son honneur et de sa vengeance de la conduire pompeusement au tombeau de sa famille. Il eut assez de crédit pour obtenir ce qui ne s'accorde pas une fois en dix ans: le droit de la transporter découverte, sur un lit de velours blanc, et de lui épargner l'horreur du cercueil. On enveloppa cette chère dépouille dans le peignoir de mousseline qu'elle portait au jardin le jour où elle formait de si doux projets avec Lello. La marquise Trasimeni, malade et bien maigrie, vint elle-même arranger ses cheveux et lui faire la coiffure qu'elle aimait. Tous les jardins de Rome se dépouillèrent pour lui envoyer des fleurs: on eut de quoi choisir. Le convoi quitta l'église de Saint-Antoine-Abbé le jeudi soir, à sept heures et demie, pour se rendre aux Saints-Apôtres, où les Feraldi ont leur sépulture. Le corps était précédé d'une longue file de confréries blanches et noires, portant chacune sa bannière. La lumière rouge des torches se jouait sur le visage de la belle morte et semblait l'animer de nouveau. Un détachement de vingt-quatre grenadiers accompagnait le cortège pour rendre honneur à la famille Feraldi et protéger le palais Coromila. Lorsqu'on traversa le Corso, un sourd frémissement parcourut le peuple, et quelques torches vinrent tomber devant la porte du colonel; les soldats se hâtèrent de les éteindre. La procession funèbre se replia vers l'arc des Carbognani, prit la rue des Vierges, et entra dans l'église des Saints-Apôtres. La place était envahie par une foule épaisse, serrée et muette; pas un cri ne vint troubler la douleur des parents et des amis de Tolla, qui pleuraient en semble au palais Feraldi.

Au moment où le-convoi arrivait à la porte de l'église, une chaise de poste accourue au galop de quatre chevaux fut arrêtée par Dominique. Un jeune homme endormi dans la voiture s'éveilla, vit le cortège, poussa un cri, sauta par la portière, et s'enfuit en courant comme un fou; c'était Manuel Coromila.

Voici ce qui s'était passé à Paris. Le \oldstylenums{11} octobre, Cornélie célébra avec tous ses amis le retour de la belle saison d'hiver. On rit un peu, on joua beaucoup, et l'on but énormément. Rouquette gagna cinq cents louis, et Manuel une migraine. Le lendemain à midi, Rouquette était sorti, Manuel couché; le garçon de l'hôtel apporta deux lettres. Manuel le renvoya à Rouquette, mais Rouquette était loin, et l'une des deux lettres était très pressée. Manuel l'ouvrit sans prendre garde à l'adresse, et il lut:
\begin{quote}

Mon seul vrai prince,

Je me plais à croire que le fils des Coromila repose sur ses lauriers comme un jambon. Ça lui apprendra à boire plus que sa jauge. Arrange-toi pour qu'il dorme trente-six heures; je le connais, c'est dans ses moyens. Je t'attendrai ce soir, ou plutôt demain à une demi-heure du matin, et je te prouverai que le proverbe est une vieille bête, et qu'on peut être heureux au jeu sans être malheureux en amour. Brûle ma lettre: s'il allait la trouver, il aboie rai comme un \emph{doge\footnote{
L'italien: duc, premier magistrat
}}.

\hspace*\fill---Cornélie.\end{quote}

La seconde lettre était le dernier adieu de Tolla. Manuel déposa la première chez Rouquette, après y avoir écrit de sa main: «En quelque lieu que je vous trouve, je vous tuerai comme un chien.» Il commanda qu'on fît ses paquets, puis courut faire viser son passe-port et assurer sa place. Il partit le soir même par la malle de Marseille. En traversant une des cours de l'hôtel des Postes, il entendit prononcer indistinctement le nom de Feraldi; il avait des bourdonnements étranges dans les oreilles. Au même instant, il heurta en courant un jeune homme qui ressemblait à Victor; il se crut en butte à la persécution des remords. À Marseille, il trouva un vapeur qui chauffait pour Civita-Vecchia; à Civita, il se jeta dans la première voiture qu'on lui offrit; il fit tout ce long voyage en six jours, pleurant, priant, et jurant d'épouser Tolla s'il la trouvait vivante. La fatigue et la douleur avaient altéré ses traits; cependant il fut reconnu et suivi par Dominique.

Dominique s'était laissé marier sans résistance; la prison l'aurait séparé de Tolla. Cinq minutes après la sortie du prêtre, il usa de ses nouveaux pouvoirs pour envoyer sa femme à Velletri, où elle avait des parents. Quand la santé de Tolla fut désespérée, il acheta un couteau et le fit bénir par le pape: c'était pour tuer Manuel. Les couteaux du petit peuple de Rome ont la forme des couteaux catalans; ils sont munis d'un anneau de fer pour qu'on puisse les suspendre à une ficelle; la lame est arrêtée solidement par un gros ressort; mais elle n'est pas plus pointue qu'un fleuret moucheté. La police enjointe aux couteliers, sous peine des galères, de laisser un morceau de fer arrondi à la pointe de chaque couteau. Dominique démoucheta le sien en le frottant sur une pierre. Il alla ensuite acheter une douzaine de chapelets: les marchands qui les vendent se chargent de les faire bénir. Ils les enferment dans une boîte et les envoient au Vatican: le pape les bénit en gros. Dominique glissa subtilement son arme sous les chapelets, et deux jours après il la retrouva sanctifiée par la main de Grégoire XVI. C'est en compagnie de ce couteau bénit qu'il se mit à la poursuite de Manuel. Il le joignit au milieu du pont Saint-Ange, et arriva fort à point pour le voir sauter dans le Tibre. Il s'y lança après lui et le ramena sur le bord. «Puisque vous voulez mourir, lui dit-il, je vous condamne à vivre. Vous ne méritez pas d'aller la rejoindre poursuivais pour vous tuer, mais je me garderai bien de le faire, maintenant que je sais que vous êtes capable de remords. Allez vous mettre au lit, et dormez si vous pouvez. Le service est pour demain à onze heures; toute la société y sera: vous ne pouvez pas y manquer, c'est vous qui donnez la fête!»

La messe des morts fut célébrée par le cardinal Pezzato. La ville entière accourut admirer pour la dernière fois cette fleur de vertu et de beauté. Son visage était calme et souriant; la mort avait effacé tous les ravages de la maladie: Tolla fut encore un jour la plus jolie fille de Rome. Tous les poètes de l'état romain publièrent des sonnets en son honneur; vingt artistes demandèrent la permission de prendre son portrait, prévoyant qu'ils auraient à peindre des anges. Les pieuses femmes qui vinrent baiser ses pieds nus mirent en pièces le velours de la draperie. Les soldats qui gardaient le catafalque étaient aveuglés par les larmes; aucun chrétien ne sortit de l'église sans s'essuyer les yeux; Nadine Fratief pleura mieux que personne.

Dix-huit ans se sont écoulés depuis le dé nouement de ce drame historique, qui commença au milieu d'un bal et finit autour d'une tombe.

Parmi les personnages que j'ai mis en scène, quelques-uns vivent encore. Lello ne s'est jamais marié; il habite son palais de Venise, en paix avec tout le monde, excepté avec lui-même. Philippe et Victor lui ont laissé la vie, comme Dominique, de peur de le délivrer de ses remords. Le colonel, dont nul regret n'interrompit jamais la digestion, est mort il y a deux ans d'une attaque d'apoplexie. Après son souper, il glissa sous la table, comme à son ordinaire, et ne se releva plus. Tous les ivrognes conviennent qu'il a fait une fin digne de sa vie. Rouquette se porte bien: il s'était enfui de l'hôtel Meurice un quart d'heure avant l'arrivée de Victor Feraldi. On ne l'a jamais revu à Rome, et son ambition a renoncé aux dignités ecclésiastiques. La passion des aventures, qui ne s'éteindra jamais en lui, l'a jeté dans les affaires: il a été longtemps un des chevaliers errants de la spéculation. L'argent des Coromila a prospéré entre ses mains, et vous l'entendrez citer à la Bourse parmi les plus honnêtes gens, je veux dire parmi les plus riches. Depuis que sa fortune est faite, il a des principes, et même un peu de religion. Il médit de Voltaire, entretient une danseuse, et songe, dit-on, à fonder un couvent.

La générale à reconnu avec surprise que Manuel n'avait jamais songé à Nadine. première fois qu'elle le fit sonder par la chanoinesse de Certeux, il répondit en haussant les épaules: «J'y penserai dans quelques années, quand j'aurai besoin d'une nourrice!» Après cette découverte, la mère et la fille ont parcouru le monde entier, lanterne en main, à la recherche d'un homme: elles n'ont pas encore trouvé.

\enlargethispage{\baselineskip}

La marquise Trasimeni ne survécut pas longtemps à Tolla; elle tomba avec les dernières feuilles. Philippe quitta le service: il prit Menico pour domestique et pour ami. Les malheurs de Tolla exercèrent une fâcheuse influence sur son esprit: il se mit à douter de bien des choses auxquelles il avait cru; il fréquenta les étrangers, lut la Bible, et devint en peu de temps un assez mauvais catholique. La proclamation de la république romaine ne le surprit pas: il l'espérait activement depuis plusieurs années. Il fut élu à l'assemblée constituante, et mourut le \oldstylenums{3} juillet \oldstylenums{1849} sur les remparts de Rome. Menico finit avec lui. Amarella, veuve sans avoir jamais été femme, prête à usure aux petites gens de Velletri: l'argent la console de tout. Cocomero est un des plus beaux fleurons de la police napolitaine. Lorsqu'il retourna dans son pays, il portait les marques du couteau de Dominique.

Victor Feraldi a six enfants, dont quatre filles; l'aînée habite avec ses grands parents: elle s'appelle Tolla. Le comte est la seule personne qui se soit vengée de la trahison de Manuel. En \oldstylenums{1841}, trois ans après la mort de sa fille, il réunit comme il put les lettres des deux amants et les fit imprimer à Paris, avec un court exposé des faits\footnote{Vittoria, istoria del secolo XIX, in-\oldstylenums{8}$\textsuperscript{e}$ de vingt feuilles; Paris, \oldstylenums{1841}.
}. Le récit, qui occupe environ vingt-cinq pages, se termine ainsi: «Puisse cette véridique histoire servir d'utile exemple aux parents, aux jeunes gens mal conseilles et aux jeunes filles sans expérience!»

Le jour même où ce livre pénétra en Italie, le colonel Coromila fit acheter et détruire l'édition entières, mais la tradition, à défaut de l'histoire, a perpétué le souvenir des malheurs de Tolla. L'église des Saints-Apotres et le tombeau de la pauvre amoureuse deviennent à certains jours de l'année un but de pèlerinage, et plus d'une jeune Romaine ajoute à ses litanies du soir: «Sainte Tolla, vierge et martyre, priez pour nous!»

\hspace*\fill---Edmond About.
